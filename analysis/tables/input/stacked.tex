\begin{table}[]
    \caption{Effect of increase in federal MW to \$9 in January 2020}
    \label{tab:stacked}
    \centering

    \begin{table}[]
        \begin{tabular}{@{}lccc@{}}
        \toprule
                                      & \multicolumn{3}{c}{Change log rents} \\ \cmidrule(l){2-4} 
                                      & $w=3$      & $w=6$     & $w=9 $      \\ \midrule
        Change residence minimum wage & #4#        & #4#       &  #4#        \\
                                      & (#4#)      & (#4#)     & (#4#)       \\
        Change workplace minimum wage & #4#        & #4#       & #4#         \\
                                      & (#4#)      & (#4#)     & (#4#)       \\ \midrule
        P-value equality              & #4#        & #4#       & #4#         \\
        R-squared                     & #4#        & #4#       & #4#         \\
        Observations                  & #0,#       & #0,#      & #0,#        \\ \bottomrule
        \end{tabular}
    \end{table}
    
    \begin{minipage}{.95\textwidth} \footnotesize
        \vspace{2mm}
        Notes: The table shows estimates of the stacked model under different balancing
        periods $w$.
        To construct the samples we proceed as follows. First, we identify all ZIP code months
        that have a change in the binding MW. Then, we call a CBSA-month as treated if in 
        that month has at least one ZIP code that had a change in the binding MW. For each 
        of the treated CBSA-months we assign a unique event ID. For each event, we take a 
        window $w$, and we keep all months within that window for the ZIP codes that are within
        the treated CBSA. If a ZIP code has missing data for some month within the window, 
        we drop the entire ZIP code. 
    \end{minipage}
\end{table}
