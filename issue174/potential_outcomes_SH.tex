%%%%%%%%%%%%%%%%%%%%%%%%%%%%%%%%%%%%%%%%%%%%%%%%%%%%%%%%%%%%%%%%%%%%%%%%%%%%%%%%
%%%%%                             SETTINGS                                  %%%%
%%%%%%%%%%%%%%%%%%%%%%%%%%%%%%%%%%%%%%%%%%%%%%%%%%%%%%%%%%%%%%%%%%%%%%%%%%%%%%%%
\documentclass{article}

\usepackage{inputenc}
\usepackage[margin = 1.25in]{geometry}

\usepackage{setspace}
\onehalfspacing

\usepackage{hyperref}
\usepackage{dsfont}
\usepackage{amsmath,amssymb,dsfont,amsthm}

\usepackage{graphicx}
\usepackage{caption}
\usepackage{subcaption}
\usepackage{booktabs}

\usepackage[inline]{enumitem}
\usepackage{pdflscape}

% Default paths for figures
\usepackage{epstopdf}

\usepackage[backend = bibtex,
            style = authoryear,
            maxnames = 6,
            maxcitenames = 3,
            doi = false,
            eprint = false]{biblatex}
\addbibresource{biblio.bib}

% Bibliography
\DeclareCiteCommand{\citeyear}
	{}
	{\bibhyperref{\printdate}}
	{\multicitedelim}
	{}

% Math-related commands
\newtheorem{assu}{Assumption}
\newtheorem{prop}{Proposition}
\newtheorem{definition}{Definition}

\newcommand{\Z}{\mathcal{Z}}
\newcommand{\MW}{\underline{W}}
\newcommand{\mw}{\underline{w}}
\newcommand{\wkp}{\text{wkp}}
\newcommand{\res}{\text{res}}
\newcommand{\pre}{\text{Pre}}
\newcommand{\post}{\text{Post}}
\DeclareMathOperator{\Var}{Var}
\DeclareMathOperator{\Corr}{Corr}
\DeclareMathOperator{\Cov}{Cov}
\DeclareMathOperator{\E}{E}

%%%%%%%%%%%%%%%%%%%%%%%%%%%%%%%%%%%%%%%%%%%%%%%%%%%%%%%%%%%%%%%%%%%%%%%%%%%%%%%%
%%%%%                               TITLE                                   %%%%
%%%%%%%%%%%%%%%%%%%%%%%%%%%%%%%%%%%%%%%%%%%%%%%%%%%%%%%%%%%%%%%%%%%%%%%%%%%%%%%%

\title{Identification of MW effects for \\
       ``Minimum Wage as a Place-Based Policy''}

\author{SH}

\date{\today}

%%%%%%%%%%%%%%%%%%%%%%%%%%%%%%%%%%%%%%%%%%%%%%%%%%%%%%%%%%%%%%%%%%%%%%%%%%%%%%%%
%%%%                              STRUCTURE                                 %%%%
%%%%%%%%%%%%%%%%%%%%%%%%%%%%%%%%%%%%%%%%%%%%%%%%%%%%%%%%%%%%%%%%%%%%%%%%%%%%%%%%

\begin{document}

\maketitle

Consider the causal model for rents given by
$$r_{it}=r_{it}(\{\mw_{zt}\}_{z\in\Z})$$
where $\mw_{zt}=\ln\MW_{zt}$ for $\MW_{zt}$ the statutory MW (in dollars).
We refer to $\mw_{zt}$ as the ``dose'' of treatment received by unit $i$ from 
unit $z$ in period $t$.
The set $\Z$ contains all ZIP codes in a closed metropolitan area.

We assume that the effects of MW across locations can be summarized in the
residence and workplace MW measures, so that
\begin{equation}\label{eq:causal_model}
    r_{it} = r_{it}(\mw_{it}^{\res}, \mw_{it}^{\wkp}) .
\end{equation}

We follow \textcite{AngristImbens1995} and define the treatment effects of 
interest as follows.
Focus on the workplace MW, and condition on a level of the residence MW.
A unit's causal response is 
$\partial r_{it}(\mw_{it}^{\res}, \mw_{it}^{\wkp})/\partial \mw_{it}^{\wkp} .$
The average causal response for the treated with dose $w$ of the workplace MW is
\begin{equation*}
    ACRT^{\wkp}(w | \mw^{\res}, w) = \frac{\partial \E\left[r_{it}(\mw^{\res}, l)
    |                           \mw^{\res}, w\right]}{\partial l} \Big|_{l=w}
\end{equation*}
and the average causal response to dose $w$ for any group is
\begin{equation*}
    ACR^{\wkp}(w | \mw^{\res}) = \frac{\partial \E\left[r_{it}(\mw^{\res}, w) \right] }{\partial w} .
\end{equation*}
We are also interested in analogous parameters for the residence MW:
\begin{equation*}
    ACRT^{\res}(w | w, \mw^{\wkp}) = \frac{\partial \E\left[r_{it}(l, \mw^{\wkp})
    |                               w, \mw^{\wkp}\right]}{\partial l} \Big|_{l=w}
\end{equation*}
and
\begin{equation*}
    ACR^{\res}(w | \mw^{\wkp}) = \frac{\partial \E\left[r_{it}(w, \mw^{\wkp}) \right] }{\partial w} .
\end{equation*}

We study the effects of the following policy.
\begin{definition}[Policy change]\label{def:policy_change}
    In period $t-1$, all locations are bound by a common MW level, 
    $\MW_{i,t-1}=\MW_0$ for all $i$.
    In period $t$, the MW increases to $\MW_1>\MW_0$ in some 
    (directly treated) ZIP codes $i\in\Z_1\subset\Z$ for non-empty $\Z_1$.
    The MW levels in $t$ are then 
    $\MW_{i,t}=\MW_0$ for $i\notin\Z_1$ and
    $\MW_{i,t}=\MW_1$ for $i\in\Z_1$.
    Thus, for each $i$ the residence MW levels are
    $$
    \mw_{i,t}^{\res} = 
    \begin{cases}
        \ln \MW_1 & \text{ if } i\in\Z_1 \\
        \ln \MW_0 & \text{ if } i\notin\Z_1 
    \end{cases}
    \quad\text{ and }\quad
    \mw_{i,t-1}^{\res} = \ln \MW_0 ,
    $$
    and the workplace MW levels are computed by
    $$
    \mw_{i,t}^{\wkp} = \sum_{z\in\Z} \pi_{iz} \ln \MW_{i,t}
    $$
    where $\{\pi_{iz}\}$ are (unchanging) commuting shares.
\end{definition}

We are envisioning a policy change in which a subset of ZIP codes in a metropolitan 
area raise the level of the MW from a previously uniform level.

\begin{assu}[Parallel trends] \label{assu:PT}
    Consider the policy change in Definition \ref{def:policy_change}.
    We assume that, for all dose levels $w>\mw_0 = \ln \MW_0$,
    \begin{equation}\label{eq:PT_workplace}
        \E\left[r_{it}(\mw_0, \mw_0) - r_{i,t-1}(\mw_0, \mw_0) \big| \mw_{it}^{\wkp} = w \right] 
        = \E\left[r_{it}(\mw_0, \mw_0) - r_{i,t-1}(\mw_0, \mw_0) \big| \mw_{it}^{\wkp} = \mw_0 \right] 
    \end{equation}
    and that
    \begin{equation}\label{eq:PT_residence}
        \E\left[r_{it}(\mw_0, w) - r_{i,t-1}(\mw_0, w) \big| \mw_{it}^{\res} = \mw_1 \right] 
        = \E\left[r_{it}(\mw_0, w) - r_{i,t-1}(\mw_0, w) \big| \mw_{it}^{\res} = \mw_0 \right] .
    \end{equation}
\end{assu}

Equation \eqref{eq:PT_workplace} imposes that, conditional on not having changed
the residence MW, the counterfactual evolution of rents is the same in ZIP codes
that received any dose of the workplace MW $w$.
Equation \eqref{eq:PT_residence} imposes that, conditional on a given value of
the workplace MW, the counterfactual evolution of rents across those ZIP codes
directly treated and those not directly treated is the same.

Focusing on the workplace MW, under Assumption \ref{assu:PT}
Proposition 3 in \textcite{CallawayEtAl2021} implies that
$$
\frac{\partial \E\left[r_{it}(\mw_0, w) | \mw_{it}^{\res} = \mw_0, \mw_{it}^{\wkp} = w\right]}
     {\partial w} 
    = ACRT^{\wkp}(w | \mw_0, w) + \frac{\partial ATT^{\wkp}(w | \mw_0, l)}{\partial w} \Big|_{l = w}
$$
where $ATT^{\wkp}(w | \mw_0, w) = \E\left[r_{it}(\mw_0, w) - r_{it}(\mw_0, \mw_0) \big| \mw_{it}^{\res} = \mw_0, \mw_{it}^{\wkp} = w\right]$.
The slope of average rents with respect to the minimum wage identifies
the average causal response of interest plus a selection bias term that arises
due to differences in treatment effects across dosage leves.

Unlike the workplace MW, the residence MW experienced a discrete jump.
Then, under Assumption \ref{assu:PT} as well, 
Proposition 3 in \textcite{CallawayEtAl2021} implies that
\begin{equation*}
\begin{split}
\E\left[\Delta r_{it}(\mw_1, w) | \mw_{it}^{\res} = \mw_1, \mw_{it}^{\wkp} = w\right] - \E\left[\Delta r_{it}(\mw_0, w) | \mw_{it}^{\res} = \mw_0, \mw_{it}^{\wkp} = w\right] \\
 = ACRT^{\res}(\mw_1 | \mw_1, w) + ATT^{\res}(\mw_0 | \mw_1, w) - ATT^{\res}(\mw_0 | \mw_0, w)
\end{split}
\end{equation*}
where $ATT^{\res}(\mw_1 | \mw_1, w) = \E\left[r_{it}(\mw_1, w) - r_{it}(\mw_0, w) \big| \mw_{it}^{\wkp} = w\right]$.
For a fixed level of the workplace MW, a difference in differences between
those directly treated (with $\mw^{\res}_{i,t}=\mw_1$) and
those indirectly treated (with $\mw^{\res}_{i,t-1}=\mw_0$) identifies
the causal response of interest plus a selection bias term.

As \textcite{CallawayEtAl2021} point out, the usual parallel trends assumption 
is not enough to identify the treatment effects because there may be 
selection of units to a different dosage level.
To remove these selection bias terms we could rely on a stronger version of
the parallel trends assumption 
\parencite[see Assumption 5 in][]{CallawayEtAl2021}.
However, as the authors explain in Remark 5, an alternative is to directly
assume that the selection bias terms are zero.
In this setting, we are willing to maintain this assumption.
The reason is that we are assuming an unchanging commuting structure, and
so units cannot endogenously affect the amount of dose they receive.
We think that conditioning on the level of the other MW measure makes this
assumption plausible.

What comparisons are we making to get these estimates?
To obtain the workplace MW we exploit the slope of the relationship
between the change in the workplace MW and the change in rents, 
conditioning on units that were not directly treated.
For the residence MW we compare directly treated and directly untreated units
that have a similar level of the workplace MW.

\end{document}
