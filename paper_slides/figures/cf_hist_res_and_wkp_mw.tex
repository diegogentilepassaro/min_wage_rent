\begin{figure}[h!]
    \centering
    \caption{Distribution of counterfactual increases in residence and workplace, 
                urban ZIP codes}
    \label{fig:cf_hist_res_and_wkp_mw}
    \begin{subfigure}{0.5\textwidth}
        \caption*{Residence MW}
        \includegraphics[width = 1\textwidth]{counterfactuals/output/d_mw_res.png}
    \end{subfigure}%
    \begin{subfigure}{0.5\textwidth}
        \caption*{Workplace MW}
        \includegraphics[width = 1\textwidth]{counterfactuals/output/d_mw_wkp_tot_17.png}
    \end{subfigure}

    \begin{minipage}{.95\textwidth} \footnotesize
        \vspace{3mm}
        Notes:
        Data are from the LODES and the minimum wage panel described in Section 
        \ref{sec:mw_construction}.
        The figures show the distribution of changes in the residence and 
        workplace MW measures generated by a counterfactual increase to \$9 
        in the federal MW in January 2020, holding constant other MW policies 
        in their December 2019 levels.
        The unit of observation is the urban ZIP code, where we define a ZIP code 
        as urban if at least 80\% of its population is classified as urban by
        the 2010 Census.
    \end{minipage}
\end{figure}
