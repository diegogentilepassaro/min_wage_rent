\begin{table}[hbt!] \centering
    \caption{Estimates of the effect of the MW on rents by share of MW workers, baseline sample}
    \label{tab:heterogeneity}
    \begin{tabular}{@{}lccc@{}}
        \toprule
            & \multicolumn{3}{c}{Change log rents}                                                  \\ \cmidrule(l){2-4} 
            & \shortstack{Baseline\\(1)} 
            & \shortstack{MW shares\\(2)}                                             
            & \shortstack{Public housing\\(3)}                                                      \\ \midrule
        Change residence minimum wage                                     &  #4#   &  #4#  &  #4#   \\
                                                                          & (#4#)  & (#4#) & (#4#)  \\
        Change residence minimum wage $\times$ High share of MW workers   &        &  #4#  &        \\
                                                                          &        & (#4#) &        \\
        Change residence minimum wage $\times$ Public housing             &        &       &  #4#   \\
                                                                          &        &       & (#4#)  \\
        Change workplace minimum wage                                     &  #4#   &  #4#  &  #4#   \\
                                                                          & (#4#)  & (#4#) & (#4#)  \\
        Change workplace minimum wage $\times$ High share of MW residents &        &  #4#  &        \\
                                                                          &        & (#4#) &        \\
        Change workplace minimum wage $\times$ Public housing             &        &       &  #4#   \\
                                                                          &        &       & (#4#)  \\
        County-quarter economic controls                                  &  Yes   &  Yes  &   Yes  \\
        R-squared                                                         &  #4#   &  #4#  &   #4#  \\
        Observations                                                      &  #0,#  &  #0,# &   #0,# \\ \bottomrule
    \end{tabular}

    \begin{minipage}{.95\textwidth} \footnotesize
        \vspace{2mm}
        \textit{Notes}: 
        Rents and MW data are from the baseline estimation sample described in Section 
        \ref{sec:data_final_panel}.
        Public housing data are from \ref{hudHousing}.
        In all columns we report the results of regressions of the log of median rents 
        per square foot on our MW-based measures.
        Column (1) reproduces estimates our baseline results from Table \ref{tab:static}.
        Column (2) presents estimates of our baseline model in which the residence MW 
        measure is interacted with an indicator that proxies for having a high share 
        of MW workers (``High share of MW workers'') and the workplace MW measure is 
        fully interacted with an indicator that proxies for having a high share of MW 
        residents (``High share of MW residents'').
        We define the indicators as follows.
        First, using LODES data we create indicator for being above the within-state 
        median across basline ZIP codes in the following variables: (i) share of workers 
        with less than a high school diploma, (ii) share of workers who earn earn less 
        than \$1251, (iii) share of residents with less than a high school diploma, (iv) 
        share of residents who earn earn less than \$1251.
        Second, we define the indicator ``High share of MW workers'' as 1 if the ZIP
        code is above median of both (i) and (ii), and the indicator ``High share of MW 
        residents'' if the ZIP code is above median of both (iii) and (iv).
        Column (3) presents estimates of our baseline model in which the residence and 
        the workplace MW measures are interacted with an indicator that equals 1 if 
        the ZIP code has any public housing units.
        Both columns (2) and (3) fully interact the MW measures and the fixed effects
        with the indicator variables.
        The time fixed effects are interacted with the indicators and the economic controls 
        are not.
    \end{minipage}
\end{table}
