\begin{table}[hbt!]
    \centering
    \caption{AVerage effect of an increase in federal MW to \$7.97 and to \$15 
             in January 2020, urban ZIP codes}
    \label{tab:counterfactuals_other}

    \begin{tabular}{@{}lccccc@{}}
        \toprule
                         & N & \shortstack{Change in\\res.\ MW}
                             & \shortstack{Change in\\wkp.\ MW}
                             & \shortstack{Share of\\housing exp.}  
                             & \shortstack{Share\\Pocketed}                      \\ \midrule
        \textit{Panel A: Fed.\ MW to \$7.97}         &      &       &       &     &      \\
        $\quad $Effect in ZIP codes with...          &      &       &       &     &      \\
        $\quad \quad$previous MW $\leq\$7.97\quad$   & #0,# &  #3# & #3#  & #3# &  #3#   \\
        $\quad \quad$previous MW $>\$7.97\quad$      & #0,# &  #0# & #3#  & #3# & #3#    \\[.3em]
        \textit{Panel B: Fed.\ MW to \$15}           &      &       &       &     &      \\
        $\quad $Effect in ZIP codes with...          &      &       &       &     &      \\
        $\quad \quad$previous MW $\leq\$15\quad$     & #0,# &  #3# & #3#  & #3# &  #3#   \\
        $\quad \quad$previous MW $>\$15\quad$        & #0,# &  #0# & #3#  & #3# & #3#    \\ \bottomrule
    \end{tabular}
    
    \begin{minipage}{.95\textwidth} \footnotesize
        \vspace{2mm}
        Notes: 
        Data are from LODES and the minimum wage panel described in Section 
        \ref{sec:mw_construction}.
        The table shows averages of the estimated ZIP-code specific shares of the 
        additional income pocketed by landlords (``Share pocketed''), 
        defined as the ratio of the increase in income to the increase in rents.
        We also report the average 
        change in residence MW, change in workplace MW,
        and share of ZIP-code specific housing expenditure 
        (``Avg.\ share of housing exp.'') defined as explained in XXXX.
        Increases in income and rents are simulated following the procedure 
        described in Section \ref{sec:counterfactual}.
        Panel A assumes a 10\% increase in the federal MW, and
        Panel B assumes an increase in the federal MW to \$15.
        We assume the following parameter values:
        $\beta = \betaCounterfactual$, $\gamma = \gammaCounterfactual$, and $\varepsilon = \epsilonCounterfactual$.
        We carry out our computations only for urban ZIP codes, defined as 
        those that belong to a CBSA with at least 80\% of urban population
        according to the 2010 census.
        The figure excludes ZIP codes located in CBAs for which the average
        estimated change in log total wages was below 0.1\% in the respective
        counterfactual scenario.
    \end{minipage}
\end{table}
