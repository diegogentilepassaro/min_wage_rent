\begin{table}[hbt!]
    \centering
    \caption{Effect of an increase in federal MW to \$9 in January 2020, urban ZIP codes}
    \label{tab:counterfactuals_fed_9usd}

    \begin{tabular}{@{}lccccc@{}}
        \toprule
                            & 
                            & \multicolumn{2}{c}{Average change in...} 
                            & \multirow{2}{*}{\thead{Avg.\ share of\\housing exp.}}   
                            &  \multirow{2}{*}{\thead{Avg.\ share\\pocketed}} \\ \cmidrule(lr){3-4}
                            & N & Res.\ MW & Wkp.\ MW \\ \midrule
        Effect in ZIP codes with...          &      &       &       &     &      \\
        $\quad$previous MW $\leq\$9\quad$    & #0,# &  #3# & #3#  & #3# &  #3#   \\
        $\quad$previous MW $>\$9\quad$       & #0,# &  #3# & #3#  & #3# & #3#    \\ \bottomrule
    \end{tabular}
    
    \begin{minipage}{.95\textwidth} \footnotesize
        \vspace{2mm}
        Notes: 
        Data are from LODES and the minimum wage panel described in Section 
        \ref{sec:mw_construction}.
        The table shows averages of the estimated ZIP-code specific shares of the 
        additional income pocketed by landlords (``Avg.\ share pocketed''), 
        defined as the ratio of the increase in income to the increase in rents.
        We also report the average share of ZIP-code specific housing expenditure
        (``Avg.\ share of housing exp.''), defined as explained in XXXX.
        Increases in income and rents are simulated following the procedure 
        described in Section \ref{sec:counterfactual}, 
        where we assume an increase in the federal MW to \$9.
        We assume the following parameter values: 
        $\beta = \betaCounterfactual$, $\gamma = \gammaCounterfactual$, and $\varepsilon = \epsilonCounterfactual$.
        We carry out our computations only for urban ZIP codes, defined as 
        those that belong to a CBSA with at least 80\% of urban population
        according to the 2010 census.
        The table excludes ZIP codes located in the 60 CBAs for which the average
        estimated change in log total wages was below 0.1\%.
    \end{minipage}
\end{table}
