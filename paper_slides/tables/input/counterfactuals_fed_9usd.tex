\begin{table}[hbt!]
    \centering
    \caption{Effect of an increase in federal MW to \$9 in January 2020, urban ZIP codes}
    \label{tab:counterfactuals_fed_9usd}

    \begin{tabular}{@{}lccccc@{}}
        \toprule
                            &   & \multicolumn{2}{c}{Average change in...}
                                & \multicolumn{2}{c}{Avg.\ landlord share}       \\ \cmidrule(lr){3-4}\cmidrule(lr){5-6}
                            & N & Res.\ MW & Wkp.\ MW
                            & $s = $ #2#  & $s = $ #2#                           \\ \midrule
        Effect in ZIP codes with...          &      &       &       &     &      \\
        $\quad$previous MW $\leq\$9\quad$    & #0,# &  #3# & #3#  & #3# &  #3#   \\
        $\quad$previous MW $>\$9\quad$       & #0,# &  #3# & #3#  & #3# & #3#    \\ \bottomrule
    \end{tabular}
    
    \begin{minipage}{.95\textwidth} \footnotesize
        \vspace{2mm}
        Notes: 
        Data are from LODES and the minimum wage panel described in Section 
        \ref{sec:mw_construction}.
        The table shows averages of the ZIP code-specific landlord share ($\rho_i$),
        defined as the ratio of the increase in income to the increase in rents.
        Increases in income and rents are simulated following the procedure
        described in Section \ref{sec:counterfactual}, where we assume 
        an increase in the federal MW to \$9.
        We assume the following parameter values: 
        $\beta = 0.0546$, $\gamma = -0.0207$, $\varepsilon = 0.1083$, and 
        $s\in\{0.25, 0.45\}$.
        We carry out our computations only for urban ZIP codes, defined as 
        those that belong to a CBSA with at least 80\% of urban population
        according to the 2010 census.
        We exclude 5 ZIP codes for which the estimated landlord share was 
        below $-1$.
    \end{minipage}
\end{table}