\begin{table}[hbt!] \centering
    \caption{Estimates of the effect of the MW on rents in levels and first differences,
             baseline sample}
    \label{tab:autocorrelation}
    \begin{tabular}{@{}lcc@{}}
        \toprule
            & \multicolumn{2}{c}{Log rents $r_{it}$}                    \\ \cmidrule(l){2-3} 
            & \shortstack{Levels\\(1)} 
            & \shortstack{First Differences\\(2)}                       \\ \midrule
        Residence minimum wage $\mw^{\res}_{it}$    &  #4#   &  #4#              \\
                                                    & (#4#)  & (#4#)             \\
        Workplace minimum wage $\mw^{\wkp}_{it}$    &  #4#   &  #4#              \\
                                                    & (#4#)  & (#4#)             \\ \midrule
        County-quarter economic controls            &  Yes   &  Yes              \\
        P-value autocorrelation test                &        &  $<0.0001$        \\
        R-squared                                   &  #4#   &  #4#              \\
        Observations                                &  #0,#  &  #0,#             \\ \bottomrule
    \end{tabular}

    \begin{minipage}{.95\textwidth} \footnotesize
        \vspace{2mm}
        \textit{Notes}: 
        Data are from the baseline estimation sample described in Section 
        \ref{sec:data_final_panel}.
        Both columns report the results of regressions of the log of 
        median rents per square foot on our MW-based measures.
        Column (1) presents estimates of a model in levels, including 
        ZIP code and year-month fixed effects.
        Column (2), presents estimates of a model in first differences, 
        including year-month fixed effects 
        (note that the ZIP code fixed effect drops out).
        For the model in first differences, we also report the results of an 
        AR(1) auto-correlation test.
        We proceed as in \textcite[][Section 10.6.3]{wooldridge2010}.
        First, we compute the residuals of the model estimated in column (2), 
        and we regress those residuals on their lag.
        Let the auto-correlation coefficient of this model be $\rho$.
        The model in levels is efficient assuming no auto-correlation in the 
        error term, which would imply that the residuals of the 
        first-differenced model are auto-correlated with $\rho = -0.5$.
        The row ``P-value autocorrelation test'' reports the $p$-value of 
        a Wald test of that hypothesis.
        Standard errors in parentheses are clustered at the state level.
    \end{minipage}
\end{table}
