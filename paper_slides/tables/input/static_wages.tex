\begin{table}[hbt!]
    \centering
    \caption{Estimates of the effect of minimum wage on income, full sample}
    \label{tab:static_wages}

    \begin{tabular}{@{}lccccc@{}}
        \toprule
                                & \multicolumn{4}{c}{Log total wages}
                                & \multicolumn{1}{c}{Log dividends}                        \\ \cmidrule(lr){2-5}\cmidrule(lr){6-6}
                                & (1)       & (2)      & (3)      & (4)       & (5)        \\ \midrule
        Workplace MW            & #4#       & #4#      & #4#      & #4#       & #4#        \\
                                & (#4#)     & (#4#)    & (#4#)    & (#4#)     & (#4#)      \\
        Workplace MW $\times$ Std.\ 
            share of MW workers &           &          &          & #4#       &            \\
                                &           &          &          & (#4#)     &            \\ \midrule
        Economic controls       & No        & Yes      & Yes      & Yes       & Yes        \\
        CBSA $\times$ year FE   & No        & No       & Yes      & Yes       & Yes        \\
        Within R-squared        & #4#       & #4#      & #4#      & #4#       & #4#        \\
        Observations            & #0,#      & #0,#     & #0,#     & #0,#      & #0,#       \\ \bottomrule
    \end{tabular}

    \begin{minipage}{.95\textwidth} \footnotesize
        \vspace{2mm}
        Notes: 
        Income data are from the IRS, commuting data are from LODES, and MW
        data are from the panel described in Section \ref{sec:data_mw_panel}.
        The table shows different estimations of the effect of the workplace MW
        on income measures.
        The sample includes all ZIP codes with valid income data for the years 
        2014--2019.
        The workplace MW and the economic controls are defined as the yearly 
        average of the respective variables used in our baseline estimates of 
        Section \ref{sec:results_main}.
        Columns (1) through (3) show estimates of a regression of log total wages
        on the workplace MW and ZIP code and year fixed effects.
        Column (2) adds time-varying economic controls from the QCEW.
        Column (3) interacts the year fixed effects with indicators for each
        Core-Based Statistical Area (CBSA).
        Column (4) interacts the workplace MW with the standardized share of MW 
        workers discussed in Section \ref{sec:data_income_housing}
        Column (5) repeats the estimation in column (3) but using the log of 
        total dividends as dependent variable.
        Standard errors in parentheses are clustered at the state level.
    \end{minipage}
\end{table}

