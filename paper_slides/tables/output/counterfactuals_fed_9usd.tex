\begin{table}[hbt!]
    \centering
    \caption{Average effect of an increase in federal MW to \$9 in January 2020, urban ZIP codes}
    \label{tab:counterfactuals_fed_9usd}

    \begin{tabular}{@{}lccccc@{}}
        \toprule
                         & N & \shortstack{Change in\\res.\ MW}
                             & \shortstack{Change in\\wkp.\ MW}
                             & \shortstack{Share of\\housing exp.}  
                             & \shortstack{Share\\Pocketed}                      \\ \midrule
        Effect in ZIP codes with...          &      &       &       &     &      \\
        $\quad$previous MW $\leq\$9\quad$    & 5,882 &  0.161 & 0.153  & 0.226 &  0.093   \\
        $\quad$previous MW $>\$9\quad$       & 1,070 &  0.000 & 0.017  & 0.244 & 0.153    \\ 
        Total incidence                      & 6,919 &      &      &     & 0.083    \\ \bottomrule
    \end{tabular}
    
    \begin{minipage}{.95\textwidth} \footnotesize
        \vspace{2mm}
        Notes: 
        Data are from LODES and the minimum wage panel described in Section 
        \ref{sec:mw_construction}.
        The table shows averages of the estimated ZIP-code specific shares of the 
        additional income pocketed by landlords (``Share pocketed''), 
        defined as the ratio of the increase in income to the increase in rents. 
        We also report the average change in residence MW, change in workplace MW,
        and share of ZIP-code specific housing expenditure 
        (``Avg.\ share of housing exp.'') defined as explained in \label{sec:measure_housing_expenditure}.
        Increases in income and rents are simulated following the procedure 
        described in Section \ref{sec:counterfactual},
        where we assume an increase in the federal MW to \$9.
        We assume the following parameter values: 
        $\beta = \betaCf$, $\gamma = \gammaCf$, 
        and $\varepsilon = \epsilonCf$.
        We carry out our computations only for urban ZIP codes, defined as 
        those that belong to a CBSA with at least 80\% of urban population
        according to the 2010 census.
        In the last row, we also report the total incidence of the counterfactual policy
        as explained in \ref{sec:emp_cf}.
        The table excludes ZIP codes located in 60 CBAs for which the average
        estimated change in log total wages was below 0.1\%.
    \end{minipage}
\end{table}


