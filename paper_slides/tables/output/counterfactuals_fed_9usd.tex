\begin{table}[hbt!]
    \centering
    \caption{Median effect of an increase in federal MW to \$9 in January 2020
             by treatment status, urban ZIP codes}
    \label{tab:counterfactuals_fed_9usd}

    \begin{tabular}{@{}lccccc@{}}
        \toprule
                         & N & \shortstack{Change in\\res.\ MW}
                             & \shortstack{Change in\\wkp.\ MW}
                             & \shortstack{Share of\\housing exp.}  
                             & \shortstack{Share\\Pocketed}                      \\ \midrule
        Effect in ZIP codes with...          &      &       &       &     &      \\
        $\quad$previous MW $\leq\$9\quad$    & 5,741 &  0.216 & 0.204  & 0.214 &  0.088   \\
        $\quad$previous MW $>\$9\quad$       & 1,057 &  0.000 & 0.013  & 0.233 & 0.146    \\ 
        Total incidence                      & 6,798 &      &      &     & 0.085    \\ \bottomrule
    \end{tabular}
    
    \begin{minipage}{.95\textwidth} \footnotesize
        \vspace{2mm}
        Notes: 
        Data are from LODES \parencite{CensusLODES}, 
        Small Area Fair Market Rents \parencite{hudSAFMR},
        ZIP code aggregate statistics \parencite{IRS}, and
        the minimum wage panel described in Section \ref{sec:mw_construction}.
        The table shows the median of the estimated ZIP-code specific shares of 
        the additional income pocketed by landlords (``Share pocketed''), 
        defined as the ratio of the increase in income to the increase in rents. 
        We also report the median change in residence MW, change in workplace MW,
        and share of ZIP-code specific housing expenditure 
        (``Share of housing exp.'') defined in Appendix 
        \ref{sec:measure_housing_expenditure}.
        Increases in income and rents are simulated following the procedure 
        described in Section \ref{sec:counterfactual},
        where we assume an increase in the federal MW to \$9.
        We assume the following parameter values: 
        $\beta = \betaCf$, $\gamma = \gammaCf$, and $\varepsilon = \epsilonCf$.
        We carry out our computations only for urban ZIP codes, defined as 
        in Table \ref{tab:stats_zip_samples}.
        In the last row, we also report the total incidence of the counterfactual 
        policy as explained in Section \ref{sec:emp_cf}.
        The table excludes ZIP codes located in $\cbsaLowIncFedNine$ CBAs for 
        which the average estimated change in log total wages was below 0.1\%.
    \end{minipage}
\end{table}


