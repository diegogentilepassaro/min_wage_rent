\begin{table}[hbt!] \centering
    \caption{Estimates of the effect of the MW on rents including one lag of the 
             dependent variable, stacked sample}
    \label{tab:stacked_w6}
    \begin{tabular}{l*{4}{c}}
        \toprule
        & \multicolumn{1}{c}{\shortstack{Change wkp.\ MW\\$\Delta\mw_{it}^{\wkp}$}}
            & \multicolumn{3}{c}{\shortstack{Change in log rents\\$\Delta r_{it}$}} \\ \cmidrule(lr){2-2}\cmidrule(lr){3-5}
                                            & (1)   & (2)   & (3)   & (4)            \\ \midrule
        Change residence MW 
                    $\Delta\mw_{it}^{\res}$  &  0.5393  &  0.0076  &       &  -0.0291     \\
                                            & (0.0326) & (0.0129) &       & (0.0207)    \\
        Change workplace MW 
                    $\Delta\mw_{it}^{\wkp}$ &       &       &  0.0244  & 0.0681      \\
                                            &       &       & (0.0225) & (0.0352)    \\ \midrule
        Sum of coefficients                &       &       &       &  0.0390     \\
                                            &       &       &       & (0.0237)    \\ \midrule
        County-quarter economic controls   &  Yes  & Yes   & Yes   & Yes      \\
        P-value equality                   &       &       &       & 0.0759      \\
        R-squared                          &  0.9766  &  0.0817  &  0.0817  & 0.0817      \\
        Observations                       & 110,239  & 110,239  & 110,239  & 110,239     \\\bottomrule
    \end{tabular}

    \begin{minipage}{.95\textwidth} \footnotesize
        \vspace{2mm}
        Notes: 
        Data are from Zillow \parencite{ZillowData}, 
        the statutory MW panel described in Section \ref{sec:mw_construction}, 
        LODES origin-destination statistics \parencite{CensusLODES},
        and the QCEW \parencite{QCEW}.
        The table mimicks the estimates in Table \ref{tab:static} using a 
        ``stacked'' sample.
        To construct the sample we proceed as follows.
        First, we define a CBSA-month as treated if in that month there is at 
        least one ZIP code that had a change in the binding MW.
        For each of the selected CBSA-months we assign a unique event ID. 
        Second, for each event we take a window $w = 6$, and we keep all months 
        within that window for the ZIP codes that belong to the treated CBSA.
        If a ZIP code has missing data for some month within the window, we drop 
        the entire ZIP code from the respective event.
        For each column, we estimate the same regression as the analogous column 
        in Table \ref{tab:static} but include event indicators $\times$ year-month
        fixed effects.
    \end{minipage}
\end{table}
