\begin{table}[hbt!]
    \centering
    \caption{Effect of an increase in federal MW to \$9 in January 2020, urban ZIP codes}
    \label{tab:counterfactuals_other}

    \begin{tabular}{@{}lccccc@{}}
        \toprule
                            &   & \multicolumn{2}{c}{Average change in...}
                                & \multicolumn{2}{c}{Avg.\ landlord share}             \\ \cmidrule(lr){3-4} \cmidrule(l){5-6}
                            & N & Res.\ MW & Wkp.\ MW
                            & $s = $ 0.25  & $s = $ 0.45                                 \\ \midrule
        \textit{Panel A: Fed.\ MW to \$7.97}         &      &       &       &     &      \\
        $\quad $Effect in ZIP codes with...          &      &       &       &     &      \\
        $\quad \quad$previous MW $\leq\$7.97\quad$   & 15,086 &  0.095 & 0.088  & 0.074 &  0.134   \\
        $\quad \quad$previous MW $>\$7.97\quad$      & 7,419 &  0 & 0.002  & 0.126 & 0.227    \\[.3em]
        \textit{Panel B: Fed.\ MW to \$15}          &      &       &       &     &      \\
        $\quad $Effect in ZIP codes with...          &      &       &       &     &      \\
        $\quad \quad$previous MW $\leq\$15\quad$     & 22,248 &  0.494 & 0.480  & 0.075 &  0.135   \\
        $\quad \quad$previous MW $>\$15\quad$        & 260 &  0 & 0.054  & 0.126 & 0.226    \\ \bottomrule
    \end{tabular}
    
    \begin{minipage}{.95\textwidth} \footnotesize
        \vspace{2mm}
        Notes: 
        The table shows averages of the ZIP code-specific landlord share ($\rho_i$),
        defined as the ratio of the increase in income to the increase in rents.
        Increases in income and rents are simulated following the procedure
        described in Section \ref{sec:counterfactual}. 
        Panel A assumes a 10\% increase in the federal MW, and
        Panel B assumes an increase in the federal MW to \$15.
        We assume the following parameter values:
        $\beta = 0.0546$, $\gamma = -0.0207$, $\varepsilon = 0.1083$, and 
        $s\in\{0.25, 0.45\}$.
        We carry out our computations only for urban ZIP codes, defined as 
        those that belong to a CBSA with at least 80\% of urban population
        according to the 2010 census.
        We exclude 5 ZIP codes for which the estimated landlord share was 
        below $-1$.
    \end{minipage}
\end{table}