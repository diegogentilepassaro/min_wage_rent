\begin{table}[hbt!] \centering
	\caption{Stacked model}
	\label{tab:stacked}
	\begin{tabular}{l*{4}{c}}
		\toprule
		& \multicolumn{1}{c}{Change wrk.\ MW}
		& \multicolumn{3}{c}{Change log rents}                            \\ \cmidrule(lr){2-2}\cmidrule(lr){3-5}
		& (1)   & (2)   & (3)   & (4)      \\ \midrule
		Change residence minimum wage      &  0.5210  &  0.0116  &       &  -0.0267     \\
		& (0.5210) & (0.0116) &       & (-0.0267)    \\
		Change workplace minimum wage      &       &       &  0.0327  & 0.0121      \\
		&       &       & (0.0156) & (0.0327)    \\ \midrule
		Sum of coefficients                &       &       &       &  0.0121     \\
		&       &       &       & (0.0156)    \\
		&       &       &       &          \\ \midrule
		County-quarter economic controls   &  No  & No   & No   & No      \\
		P-value equality                   &       &       &       & 0.0332      \\
		R-squared                          &  0.0736  &  0.0332  &  0.0736  & 0.0254      \\
		Observations                       & 0  & 0  & 0  & 0     \\\bottomrule
	\end{tabular}
    
    \begin{minipage}{.95\textwidth} \footnotesize
        \vspace{2mm}
        Notes: Data on rents are from Zillow (YYY). Data on statutory minimum wage levels
        are from \textcite{VaghulZipperer2016, BerkeleyLaborCenter}, and on 
        commuting shares from LODES (YYYY).
        To construct the estimation samples we proceed as follows.
        First, we define a CBSA-month as treated if in that month there is at least one ZIP 
        code that had a change in the binding MW.
        For each of the selected CBSA-months we assign a unique event ID. 
        Second, for each event we take a window $w = 6$, and we keep all months within that 
        window for the ZIP codes that belong to the treated CBSA.
        If a ZIP code has missing data for some month within the window, we drop the entire 
        ZIP code from the respective event. 
    \end{minipage}
\end{table}
