\begin{table}[hbt!] \centering
	\caption{Autocorrelation}
	\label{tab:autocorrelation}
    \begin{tabular}{@{}lcc@{}}
		\toprule
        & \multicolumn{2}{c}{Log rents}                                 \\ \cmidrule(l){2-3} 
        & \shortstack{Level}           & \shortstack{First Difference}  \\ \midrule
		                                   &  (1)   &  (2)              \\ \midrule
		Residence minimum wage             &  0.0862   &  -0.0204              \\
		                                   & (0.2144)  & (0.0169)             \\
		Workplace minimum wage             &  -0.0660   &  0.0545              \\
		                                   & (0.2159)  & (0.0283)             \\ \midrule
		County-quarter economic controls   &  Yes   &  Yes              \\
		P-value autocorrelation test       &        &  0.0000              \\
		Observations                       &  132,897  &  131,383             \\\bottomrule
	\end{tabular}

    \begin{minipage}{.95\textwidth} \footnotesize
        \vspace{2mm}
        \textit{Notes}: In column (1) we report results from a regression of log median SFCC rents 
        on residence and workplace MW levels with ZIP code and year-month fixed effects. 
        In column (2), we report the same regression but in first differences (note that the ZIP code 
        fixed effects drop out). For the model in first differences, we also report an AR 1 auto-correlation 
        test. We proceed as in \parencite[][section 10.6.3]{wooldridge2010}. We compute the residuals of 
        the model estimated in column (2), and we regress those residuals on their lag. We call that 
        coefficient $\rho$. The model in levels is efficient assuming no auto-correlation of the 
        error term, which would imply that the residuals of the first difference model are 
        auto-correlated with $\rho = -0.5$. We report the p-value of testing that hypothesis.
    \end{minipage}
\end{table}
