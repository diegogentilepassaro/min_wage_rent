\begin{table}[hbt!]
    \centering
    \caption{Estimates of the effect of minimum wage on income, urban ZIP codes}
    \label{tab:static_wages}

    \begin{tabular}{@{}lccccc@{}}
        \toprule
                                & \multicolumn{4}{c}{Log total wages}
                                & \multicolumn{1}{c}{Log dividends}                        \\ \cmidrule(lr){2-5}\cmidrule(lr){6-6}
                                & (1)       & (2)      & (3)      & (4)       & (5)        \\ \midrule
        Workplace MW            & 0.1488       & 0.1112      & 0.1083      & 0.1310       & 0.0262        \\
                                & (0.0704)     & (0.0405)    & (0.0390)    & (0.0917)     & (0.0841)      \\ \midrule
        Sample                  & All       & All      & All      & Baseline  & All        \\
        Economic controls       & No        & Yes      & Yes      & Yes       & Yes        \\
        CBSA $\times$ year FE   & No        & No       & Yes      & Yes       & Yes        \\
        Within R-squared        & 0.0165       & 0.1395      & 0.0266      & 0.0376       & 0.0018        \\
        Observations            & 274,271      & 247,962     & 247,852     & 12,943      & 235,193       \\ \bottomrule
    \end{tabular}

    \begin{minipage}{.95\textwidth} \footnotesize
        \vspace{2mm}
        Notes: 
        Income data are from the IRS, commuting data are from LODES, and minimum wage 
        data are from the panel described in Section \ref{sec:mw_construction}.
        The table shows different estimations of the effect of the workplace MW
        on a ZIP code's income aggregates.
        The unit of observation is the ZIP code by year.
        The workplace MW and the economic controls are defined as the yearly 
        average of the respective variables used in our baseline estimates of 
        Section \ref{sec:results_main}.
        Columns (1) through (4) show estimates of a regression of log total wages
        on the workplace MW and ZIP code and year fixed effects.
        Column (2) adds time-varying economic controls from the QCEW.
        Column (3) interacts the year fixed effects with indicators for each
        Core-Based Statistical Area (CBSA).
        Column (4) repeats the estimates of column (3) on the baseline sample
        of ZIP codes used in Section \ref{sec:results_main}.
        Column (5) repeats the estimation in column (3) but using the log of 
        total dividends as dependent variable.
        Standard errors in parentheses are clustered at the state level.
    \end{minipage}
\end{table}

