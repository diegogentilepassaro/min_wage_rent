
\subsection{Overview}
\begin{frame}
	\frametitle{Overview}
	
	Goals of the model:
	\begin{itemize}
		\item Motivate a new MW measure: the experienced MW.
		\item Motivate our empirical strategy: use commuting patterns to account for 
		spillovers of MW policies.
	\end{itemize}
	
	\pause
	\vspace{2mm}
	The model is \textit{not} intended to:
	\begin{itemize}
		\item Describe within-city residential sorting or local goods markets.
		\item Perform welfare analysis of MW policies.
	\end{itemize}

\end{frame}

%%: Let me show a simple example that illustrates the workings of the model

\begin{frame}
	\frametitle{Simple Illustration}
	
	No commuting across ZIP codes:
	\input{../input/example_1}
	
	\pause
	\vspace{2mm}
	All residents of A work in B and vice-versa:
	\input{../input/example_2}
	
\end{frame}

\subsection{Set-up}
\begin{frame}
	\frametitle{Set-up}
	
	%% Intuition: workers get income, eat tradable good with exogenous price, and then
	%% decide to consumer local good and housing.
	%% ==> We model the housing market
	
	\begin{itemize}
		\item $\mathbb{Z}$: set of ZIP codes, indexed by $i$ (residence) and $z$ 
		(workplace).
		
		\pause
		\vspace{2mm}
		\item $L_{i z}$: measure of workers who reside in $i$ and work in $z$.
		\begin{itemize}
			\item Total measure of workers $\mathcal{L}=\sum_{i \in \mathbb{Z}} 
			\sum_{z \in \mathbb{Z}} L_{i z} $ is assumed exogenous.
		\end{itemize}
	
		\pause
		\vspace{2mm}
		\item $y_{i z} = y_{i z} \left(\underline{w}_i, \underline{w}_z\right)$: 
		disposable income, where:
		\begin{itemize}
			\item $\underline{w}_i$ and $\underline{w}_z$ are residence and 
			workplace MWs.
		\end{itemize}
	
		\pause
		\vspace{2mm}
		\item $H_{i z} \left(r_i, y_{i z} \right)$: housing demand, where $r_i$ 
		represents housing rents.
		
		\pause
		\vspace{2mm}
		\item $D_i \left(r_i \right)$: housing supply.
	\end{itemize}
	
\end{frame}


\begin{frame}
	\frametitle{Assumptions}
	
	Disposable income $y_{i z}$:
	\begin{itemize} \small
		\item decreasing in residence MW, $\frac{d y_{i z}}{d \underline{w}_i} < 0$.
		
		\vspace{.5mm}
		\item increasing in workplace MW, $\frac{d y_{i z}}{d \underline{w}_z} > 0$.
	\end{itemize}

	\pause
	\vspace{2mm}
	Housing demand $H_{i z} \left(r_i, y_{i z} \right)$:
	\begin{itemize} \small
		\item decreasing in rents, $\frac{d H_{i z}}{d r_i} < 0$;
		
		\vspace{.5mm}
		\item increasing in disposable income, $\frac{d H_{i z}}{d y_{i z}} > 0$.
	\end{itemize}
	
	\pause
	\vspace{2mm}
	Housing supply: 
	\begin{itemize} \small
		\item $D_i \left(r_i \right)$ is increasing in rents, $\frac{d D_i}{d 
			r_i} > 0$.
	\end{itemize}
\end{frame}

\subsection{Equilibrium and Comparative Statics}
\begin{frame}
	\frametitle{Equilibrium}
	
	The rental market of ZIP code $i$ is in equilibrium if
	
	$$ \sum_{z\in\mathbb{Z}} L_{i z} H_{i z} (r_i, y_{i z}) =  D_i (r_i) .$$
	
	We denote equilibrium rents as $r^*_i = f\left( 
	\{\underline{w}_j\}_{j\in\mathbb{Z}} \right)$. 
	%% A function of MWs in the entire metropolitan area.
	
	% As lhs is decreasing and rhs is increasing, curves cross only once.
	% To guarantee r^*_i > 0, assume D_i(r_i = 0) is high enough
	
	\vspace{5mm}
	\pause
	We are interested in the effects of MW policies on rents.
	\vspace{1mm}
	\begin{itemize} \small
		\item What are the consequences of not accounting for both residence and 
		workplace MWs?
		\vspace{.5mm}
		\item Under what conditions one can reduce the dimensionality in the rents 
		function?
	\end{itemize}

\end{frame}


\begin{frame}
	\frametitle{The Differential Effect of Residence and Workplace MWs}
	
	\small
	\textbf{Proposition 1:} \textit{Consider a new MW policy. Under the assumptions of
	\begin{itemize}
		\item fixed distribution of workers across residence and workplace locations;
		\pause
		\item disposable income is increasing in workplace MW and decreasing in residence 
		MW;
	\end{itemize}
	\pause
	we have that workplace-MW hikes \textbf{increase} rents, and residence-MW hikes, 
	\textbf{conditional} on workplace MWs, \textbf{decrease} rents.}
	
	\pause
	\vspace{1.5mm}
	\textit{Furthermore, assuming that
		\begin{itemize} 
			\item elasticities of rents to income and of income to MWs do not vary by 
			workplace;
		\end{itemize}
	\pause
	we can write the change in \textbf{log rents} as a function of the change in two 
	MW-based measures: \textbf{experienced log MW} and \textbf{statutory log MW}.}
	
\end{frame}

\begin{frame}[label = proof_main]
	\frametitle{Proof of Proposition 1}
	
	%% Joke: mention Simon and Blume
	First part of the proposition can be shown applying implicit function theorem. 
	\hyperlink{proof_prop_1}{\beamerbutton{Details}}
	
	\pause
	\vspace{2mm}
	As for the second part, we can write
	
	$$
	d \ln r_i
	= \underbrace{\frac{\xi_i^y \epsilon_i^z}
		   {\eta_i - \sum_z \pi_{i z} \xi_{i z}^r}}_{\beta_i > 0}
		   \underbrace{\sum_z \pi_{i z} d \ln \underline{w}_z}_{
		   		\substack{\text{Exp. log MW}\\\text{at residence}}}
	+ \underbrace{\frac{\xi_i^y \epsilon_i^i}
		   {\eta_i - \sum_z\pi_{i z}\xi_{i z}^r}}_{\gamma_i < 0}
		   \underbrace{d \ln \underline{w}_i}_{
		   		\substack{\text{Stat. log MW}\\\text{at residence}}}
	$$
	
	\vspace{-1mm}
	where 
	\begin{itemize} \small
		\item $\pi_{i z}$: \textit{share} of $i$'s residents 
		working in $z$;
		
		\item $\xi_{i z}^r$ and $\xi_{i}^y$: 
		elasticities of $H_{i z}$ wrt $r$ and $y$ 
		($\xi_{i}^y$ assumed constant over $z$);
		%%% evaluated at average housing demand in ZIP code;
		
		\item $\epsilon_{i}^i$ and $\epsilon_{i}^z$: elasticities of 
		$y_{i z}$ wrt $\underline{w}_i$ and $\underline{w}_z$
		(both assumed constant over $z$);
		
		\item $\eta_i$: elasticity of \textit{housing supply}.
	\end{itemize}
	
\end{frame}

\begin{frame}
	\frametitle{Implications of the Model}
	
	Per equation above, we are interested in the parameters
	$$ \beta_i = \frac{\xi_i^y \epsilon_i^z}
	{\eta_i - \sum_z \pi_{i z} \xi_{i z}^r} > 0 
	\quad \quad \quad 
	\gamma_i = \frac{\xi_i^y \epsilon_i^i}
	{\eta_i - \sum_z\pi_{i z}\xi_{i z}^r} < 0 .	$$
	
	%% Empirical model gets to average beta. Mention heterogeneity
	%% Also mention importance of gamma for distributional consequences of MW changes
	
	\vspace{3mm}
	\pause
	Imagine an econometrician who omits one of the measures in the above model:
	\begin{itemize}
		\item It can be shown that the resulting elasticity will be between $\gamma_i$ 
		and $\beta_i$.
	\end{itemize}
\end{frame}

\subsection{Discussion}
\begin{frame}
	\frametitle{Discussion}
	
	\begin{itemize}
		\item Including both the statutory and experienced log MW should allow estimation 
		of the differential effect of MWs on rents from workplace and residence changes.
		\vspace{1mm}
		\begin{itemize}
			\item Assumption of fixed $\pi_{i z}$ shares appears plausible in short-run.
			
			{\color{gray} \footnotesize \parencite{MonteEtAl2018, CegnizEtAl2019, 
			PerezPerez2020}} 
			% MonteEtAl2018 studies the shift in distribution of residence workplace at 
			%%%   the decadal level // Use "share who work in same county"
			% PerezPerez2020 finds no effect of MWs on resident shares and low effect on 
			%%%   employment shares // Model at level of counties
			
			\vspace{1mm}
			\item Assumption that local MWs decrease disposable income is consistent with 
			literature on price effects of MWs.
			
			{\color{gray} \footnotesize \parencite{Allegretto2018, Leung2020}}
			% Allegretto2018 on restaurants following MW hike in San Jose, CA, in 2013
			% Leung2020 effect of MW on supermarkets (Nielsen data)
		\end{itemize}
		
		\vspace{2.5mm}
		\item Can test the model's rationale by including only the statutory or 
		experienced log MW in empirical models and comparing with main estimates.
	\end{itemize}
	
\end{frame}

