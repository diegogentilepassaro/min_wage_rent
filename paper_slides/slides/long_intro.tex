
\begin{frame}
	\frametitle{Motivation}
	
	Research on minimum wage (MW) has mostly focused on employment.
	% States: people work and reside under same MW.
	
	\vspace{1.5mm}
	However, MW policies are \textit{place-based}, so one should expect broader effects 
	in the local economy:
	\begin{enumerate}[$\Rightarrow$]
		\item Housing market.
	\end{enumerate}

	\pause
	\vspace{3mm}
	Because
	\begin{itemize}
		\vspace{.5mm} \item people tend to work and reside in different locations; and 
		%\parencite{MonteEtAl2018}
		\vspace{.5mm} \item MW levels tend to vary within metropolitan areas;
		%\parencite{DubeLindner2021}
	\end{itemize}
	% MonteEtAl2018: Change in the distribution of share who work in same county
	% DubeLindner2021: 42 local MWs in the US
	\vspace{.5mm} 
	accurate welfare analysis of MW increases requires understanding the consequences of 
	the spatial re-distribution of income they induce.
	%% People tend to live and work under different MW levels!
	
	%A canonical version of the monocentric city model suggests that wage increases will 
	%fully pass-through to rents.
\end{frame}

\begin{frame}
	\frametitle{This paper}
	We investigate the effect of MW policies on rents between Jan 2010 and Dec 
	2019:
	\begin{itemize}
		\vspace{.5mm} \item Estimate elasticity of median rents to workplace and 
		residence MWs;
		
		\vspace{.5mm} \item Estimate pass-trough of MW increases to rents.
	\end{itemize}
	
	\vspace{3mm}
	\pause
	To do so, we:
	\begin{itemize}
    	\vspace{.5mm} \item Exploit high-frequency rents data from Zillow at a fine 
    	geography (ZIP code);
    	
    	\vspace{.5mm} \item Propose a novel measure of exposure to MW changes based on 
    	commuting shares;
    	
    	\vspace{.5mm} \item Leverage variation in MW levels \textit{within} metropolitan 
    	areas to estimate effect of workplace and residence MWs changes.
	\end{itemize}
\end{frame}

\begin{frame}
	\frametitle{Preview of Findings}
	
	We find that:
	\begin{itemize}
		\vspace{.5mm} \item
		The elasticity of rents to workplace MW is 0.072--0.108;
		% Controlling for residence MW
		
		\vspace{.5mm} \item
		If residence MW also increases, the elasticity is 0.034--0.061;
		% Controlling for residence MW
		
		\vspace{.5mm} \item
		Failing to account for commuting patterns results in an elasticity of 
		0.026--0.058 \textit{only at residence};
		
		\vspace{.5mm} \item
		The pass-through of MW to rents is at least 22\%.
	\end{itemize}
	
	\pause
	\vspace{4mm}
	Overall, our results highlight the importance of accounting for variation of MWs 
	within metropolitan areas and commuting patterns of workers when evaluating MW 
	policies.
\end{frame}


%\begin{frame}
%	\frametitle{Literature}
%	
%	Botija.
%	
%\end{frame}
