\begin{frame}
	\frametitle{Conclusions}

	\begin{itemize}
		\item Unlike employment effects, accounting for commuting patterns is key to 
		study MW effects on the housing market.
		
		\vspace{.5mm} 
		\begin{enumerate}[$\Rightarrow$]
			\item We propose a novel experienced MW measure accounting for the difference 
			between workplace and residence.
			
			\vspace{.5mm} \item We find richer spatial patterns in the estimated 	
			effects.
		\end{enumerate}
		
		\pause
		\vspace{1.5mm} \item A 10\% increase in the experienced MW translates to a 
		0.7-1.1\% increase in rents.
		
		If statutory MW also increases by 10\%, the increase in rents would be 
		0.3-0.6\%.
		
		\pause
		\vspace{1.5mm} \item Ignoring the experienced MW would lead to a smaller effect 
		only at residence.
		
		\pause
		\vspace{1.5mm} \item Landlords pocket an average of at least 22 percent of the 
		extra income generated by the MW increase.
	\end{itemize}
\end{frame}

\begin{frame}
	\frametitle{Next Steps}
	\begin{itemize}
		\item Explore heterogeneity of estimated elasticities by ZIP code 
		characteristics.
		
	    \vspace{3mm} \item Micro-found our model to compute welfare changes of MW 
	    workers, firms, and landlords.
	    
	    \vspace{3mm} \item Use estimated model to compute rent changes under 
	    counterfactual MW policies:
	    \vspace{.5mm}
	    \begin{itemize}
	    	\item Effect of raising federal MW to \$15.
	    	\item Effect of local MWs within metropolitan areas.
	    \end{itemize}	    
	    
%	    \textbf{Idea}: two types of households fixed exogenously along with fixed 
%	    residence location, and fixed housing supply. Have transport costs, and allow for 
%	    workplace location choice. Introduce local firms that produces a local good using 
%	    both MW and non-MW workers.
\end{itemize}

	
\end{frame}