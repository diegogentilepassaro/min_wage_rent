\subsection{Static Model}
\begin{frame}
	\frametitle{Static Model}
	\begin{table} \centering
		\scalebox{.74}
			{{
\def\sym#1{\ifmmode^{#1}\else\(^{#1}\)\fi}
\begin{tabular}{l*{4}{c}}
\hline\hline
          &\multicolumn{1}{c}{$\Delta \underline{w}_{ict}^{\text{exp}}$}&\multicolumn{3}{c}{$\Delta \ln r_{ict}$}                \\\cmidrule(lr){2-2}\cmidrule(lr){3-5}
          &\multicolumn{1}{c}{(1)}         &\multicolumn{1}{c}{(2)}         &\multicolumn{1}{c}{(3)}         &\multicolumn{1}{c}{(4)}         \\
\hline
$\Delta \ln \underline{w}_{ict}$&   0.8718\sym{***}&   0.0257\sym{*}  &                  &  -0.0320\sym{*}  \\
          & (0.0296)         & (0.0137)         &                  & (0.0163)         \\
[1em]
$\Delta \underline{w}_{ict}^{\text{exp}}$&                  &                  &   0.0320\sym{**} &   0.0662\sym{**} \\
          &                  &                  & (0.0151)         & (0.0278)         \\
\hline
$\Delta \ln \underline{w}_{ict}$ + $\Delta \underline{w}_{ict}^{\text{exp}}$&                  &                  &                  &                  \\
          &                  &                  &                  &                  \\
\vspace{-2mm}&                  &                  &                  &                  \\
Wage controls&      Yes         &      Yes         &      Yes         &      Yes         \\
Employment controls&      Yes         &      Yes         &      Yes         &      Yes         \\
Establishment-count controls&      Yes         &      Yes         &      Yes         &      Yes         \\
P-value equality&                  &                  &                  &    0.027         \\
R-squared &    0.947         &    0.021         &    0.021         &    0.021         \\
Observations&  131,196         &  131,196         &  131,196         &  131,196         \\
\hline\hline
\end{tabular}
}
}
		\begin{minipage}{.95\textwidth} \scriptsize
			\vspace{2mm}
			Notes: All regressions include month FE.
			Economic controls correspond to the Financial, IT, and Professional 
			and Business Services sectors in QCEW. Standard errors clustered at 
			state level.
		\end{minipage}
	\end{table}
\end{frame}

\subsection{Dynamic Models}
\begin{frame}[label = dyn_stat_only]
	\frametitle{Dynamic Model: Statutory log MW Only}
	\begin{figure} \centering
		\includegraphics[width = 0.6\textwidth]
			{../../analysis/first_differences_expmw/output/dynamic_statutory_only_6.eps}
	\end{figure}
	
	\vspace{-2mm}
	\hyperlink{dyn_experienced_only}{\beamerbutton{Experienced log MW only}}
\end{frame}

\subsection{Dynamic Models}
\begin{frame}[label = dyn_both]
	\frametitle{Dynamic Model: Experienced and Statutory MW}
	\begin{figure} \centering
		\includegraphics[width = 0.6\textwidth]
		{../../analysis/first_differences_expmw/output/dynamic_exp_and_statutory_6.eps}
	\end{figure}

	\vspace{-2mm}
	\hyperlink{dyn_both_alt}{\beamerbutton{Dynamics of $\gamma$}}
\end{frame}


\begin{frame}[label = dyn_comp]
	\frametitle{Dynamic Model: Comparison}
	\begin{figure} \centering
		\includegraphics[width = 0.6\textwidth]
			{../../analysis/first_differences_expmw/output/fd_model_comparison_expmw_6.eps}
	\end{figure}

	\vspace{-2mm}
	\hyperlink{window_size_perturbations}{\beamerbutton{Window size perturbations}}
\end{frame}

\subsection{Pass-through estimates}
\begin{frame}
	\frametitle{Assessing the magnitude of the effects: Rationale} 
	
	How much more income is generated by MW changes?
	
	\pause
	\vspace{2mm}
	\textbf{One idea}: use elasticity of average wages to MW $\Rightarrow$ $\epsilon$.
	
	\pause 
	\vspace{3mm}
	We use two different estimates of $\epsilon$:
	\begin{itemize}
		\item We estimate it in our sample using QCEW data.
		\item We use an estimate from \textcite{CegnizEtAl2019}.
	\end{itemize}

	\pause 
	\vspace{3mm}
	Compute the pass-through dividing different rent elasticities by $\epsilon$.
\end{frame}

\begin{frame}
	\frametitle{Assessing the magnitude of the effects: Results}
	\begin{table} \centering
		\scalebox{.75}
		{{
\def\sym#1{\ifmmode^{#1}\else\(^{#1}\)\fi}
\begin{tabular}{l*{4}{c}}
\hline\hline
            &\multicolumn{1}{c}{(1)}&\multicolumn{1}{c}{(2)}&\multicolumn{1}{c}{(3)}&\multicolumn{1}{c}{(4)}\\
            &\multicolumn{1}{c}{\shortstack{r}}&\multicolumn{1}{c}{\shortstack{QCEW \\ regression}}&\multicolumn{1}{c}{\shortstack{QCEW Regression + \\ Experienced MW}}&\multicolumn{1}{c}{\shortstack{Dube et al. (2019) + \\ Experienced MW}}\\
\hline
Effect on Rents&   0.123&   0.026&   0.031&   0.031\\
Effect on Wages&   0.214&   0.045&   0.058&   0.112\\
Pass-Through&   0.575&   0.577&   0.530&   0.277\\
\hline\hline
\end{tabular}
}
}
	\end{table}
	
\end{frame}

\subsection{Robustness}
\begin{frame}
	\frametitle{Robustness exercises}

	Data:
	\begin{itemize} \small
		\item Use fully unbalanced panel.
		% late entrant ZIP codes show similar dynamics
	
	    \vspace{.5mm} \item Use fully balanced panel starting July 2015.
	    % entry of ZIP codes to sample does not drive the result
	    
	    \vspace{.5mm} \item Use re-weighted data to match socio-demographics of top-100 
	    CBSA.
	    % Results are even larger in magnitude.
	\end{itemize}
	
	\pause
	\vspace{2.5mm}
	Identification:
	\begin{itemize} \small
	   	\item Pre-trends tests.
	   	% $\Rightarrow$ dynamics consistent with causal effect
		
		\vspace{.5mm} \item Check effects of MW on housing supply.
		% We find nothing
		
		\vspace{.5mm} \item Allowing for feedback \textit{a la} Arellano-Bond: sequential 
		exogeneity.
		% weaker identifying assumption but stringer structure)
		% Same results
		
		\vspace{.5mm} \item Change economic control sets.
		% Results not driven by controlling for particular economic shocks
		
		\vspace{.5mm} \item Allow for ZIP code-level heterogeneity in time paths.
		% results not driven by time paths differences between treated and untreated % 
		%%units.
	\end{itemize}

\end{frame}
