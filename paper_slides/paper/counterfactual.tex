%%%%%%%%%%%%%%%%%%%%%%%%%%%%%%%%%%%%%%%%%%%%%%%%%%%%%%%%%%%%%%%%%%%%%%%%%%%%%%%%%
%%%%%                             DISCUSSION                                 %%%%
%%%%%%%%%%%%%%%%%%%%%%%%%%%%%%%%%%%%%%%%%%%%%%%%%%%%%%%%%%%%%%%%%%%%%%%%%%%%%%%%%

In this section, we use our results to explore the consequences of an increase
in the federal MW.
Because our data ends in December 2019, we study the consequences of a 
counterfactual increase in the federal minimum wage in January 2020 to \$9. 
In an appendix, we present two other counterfactual scenarios (i) an increase of 10\%, 
and (ii) an increase to \$15. 

% Discuss pass-through estimates. Stress that they depend on geography of prevailing
% MWs across the commmuting zone.

The incidence of MW policies on rents depend on the prevailing MW levels and on
the entire commuting structure. If a new federal MW is established some ZIP 
codes will experience a change in their binding residence MW and some will not. In turn, 
some ZIP codes will have changes in both their residence and workplace MW while other 
will only be affected through changes in the workplace.

% Discuss welfare briefly. Maybe conjecture on long-run effects (low-wage workers 
% relocating to areas with low MW and commuting to areas with high MW).

% Discuss policy implications.

\subsection{Empirical Approach}

Consider an increase of the federal MW to \$9 in January 2020. As a consequence, there 
will be changes in income, and therefore changes in the housing expenditure.
In \ref{sec:results} we estimated the effects of MW changes on rents. Now, we are interested 
the incidence of those changes, and in particular we want to know how much of each extra 
dollar, put on the table due to the MW increase, is captured by landlords. Following the 
notation of our model in Section \ref{sec:model}, define this ZIP code-specific pass-through 
as

\begin{equation*}\label{eq:pass_through}
    \rho_i := \frac{\Delta H_i}{\Delta Y_i} 
            = \frac{H^{\text{Post}}_i R^{\text{Post}}_i - H^{\text{Pre}}_i R^{\text{Pre}}_i}{\Delta Y_i}
\end{equation*}

where $H_i = \sum_{iz}\Delta H_{iz}$ denotes total rented space in $i$, 
$\text{Pre}$ and $\text{Post}$ denote moments before and after the MW change,
$R_i$ denotes rents per square foot, and 
$Y_i$ denotes total income.

Changes in the rented square footage are unobserved so we assume 
$h^{\text{Pre}}_i = h^{\text{Post}}_i = h_i$ so that

\begin{equation*}
	\rho_i = \frac{h^{\text{Post}}_i R^{\text{Post}}_i - h^{\text{Pre}}_i R^{\text{Pre}}_i}{\Delta Y_i} = 
	         h_i \frac{\Delta R_i}{\Delta Y_i}
\end{equation*}
If instead $\Delta h_i > 0$ then our estimate of $\rho_i$ is a lower bound.

From our empirical results we have that

\[
    \Delta \ln R_i = \beta \MW_i^{\text{wkr}} + \gamma \MW_i^{\text{res}}
\]

Now, we also have to define the change in log income as

\[
    \Delta \ln Y_i = \varepsilon \MW_i^{\text{wkr}}
\]

where income is only a function of the MW at workplace. This is such because
the MW at residence has no influence on the total wages unless a worker works 
where he lives and that is already accounted for in the workplace MW measure. Note,
that the negative effect of residence MW on rents, conditional on workplace MW, are
going through changes in other prices. 

We estimate $\varepsilon$ with IRS data on the total wage bill at a ZIP code year using
the following level model\footnote{We use a level model because it is not feasible to 
take monthly first differences with yearly data. The level model on the average month 
within a year variables is identified under the same assumptions as our baseline model.} 

\begin{equation}
    \ln Y_{iy} = \alpha_i + \alpha_y + \varepsilon \overline{\MW_{iy}^{\text{wkr}}} + 
                 \overline{\mathbf{X}^{'}_{iy}}\eta + \psi_{iy}
\end{equation}

where $y$ indexes a year and the line over a variable indicates that it represents 
the average month within a year.

With knowledge of $\varepsilon$, algebra implies that

\begin{equation}
	\begin{split}
		\rho_i & = h_i \left[ 
		\frac{\exp \left(\Delta \ln R_i + \ln R_i \right) - R_i }{\exp \left( \Delta \ln Y_i + \ln Y_i \right) - Y_i }
		\right] \\
		& = s_i \left[
		\frac{\exp \left( \beta \MW_i^{\text{wkr}} + \gamma \MW_i^{\text{res}} \right) - 1 }{\exp \left( \varepsilon \MW_i^{\text{wkr}} \right) - 1 }
		\right]
	\end{split}
\end{equation}

where $s_i = \frac{h_i R_i}{Y_i}$ is the share of $i$'s expenditure in housing. 
We will show results for different assumptions on $s_i$.

%%%%%%%%%%%%%%%%%%%%%%%%%%%%%%%%%%%%%%%%%%%%%%%%%%%%%%%%%%%%%%%%%%%%%%%%%%%%%%%%
\subsection{Results}

We use our estimates on data for US urban ZIP codes to compute $\rho_i$ after a 
counterfactual increase of the federal MW to \$9.

\subsubsection{Counterfactual changes in residence and workplace MWs}

\subsubsection{Predicted changes in rents and wage income}

\subsubsection{Incidence}




