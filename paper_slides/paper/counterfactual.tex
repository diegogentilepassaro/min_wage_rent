%%%%%%%%%%%%%%%%%%%%%%%%%%%%%%%%%%%%%%%%%%%%%%%%%%%%%%%%%%%%%%%%%%%%%%%%%%%%%%%%%
%%%%%                             DISCUSSION                                 %%%%
%%%%%%%%%%%%%%%%%%%%%%%%%%%%%%%%%%%%%%%%%%%%%%%%%%%%%%%%%%%%%%%%%%%%%%%%%%%%%%%%%

In this section, we use our results to explore the consequences of an increase
in the federal MW.
Because our data ends in December 2019, we study the consequences of a 
counterfactual increase in the federal minimum wage in January 2020 to \$9. 
In an appendix, we present two other counterfactual scenarios (i) an increase of 10\%, 
and (ii) an increase to \$15. 

The incidence of MW policies on rents depend on the prevailing MW levels and on
the entire commuting structure. If a new federal MW is established some ZIP 
codes will experience a change in their binding residence MW and some will not. In turn, 
some ZIP codes will have changes in both their residence and workplace MW while other 
will only be affected through changes in the workplace.

% Discuss welfare briefly. Maybe conjecture on long-run effects (low-wage workers 
% relocating to areas with low MW and commuting to areas with high MW).

% Discuss policy implications.

\subsection{Empirical Approach}\label{sec:emp_cf}

Consider an increase of the federal MW to \$9 in January 2020. As a consequence, there 
will be changes in income, and therefore changes in the housing expenditure.
In \ref{sec:results} we estimated the effects of MW changes on rents. Now, we are interested 
the incidence of those changes, and in particular we want to know how much of each extra 
dollar, put on the table due to the MW increase, is captured by landlords. Following the 
notation of our model in Section \ref{sec:model}, define this ZIP code-specific pass-through 
as

\begin{equation*}\label{eq:pass_through}
    \rho_i := \frac{\Delta H_i}{\Delta Y_i} 
            = \frac{H^{\text{Post}}_i R^{\text{Post}}_i - H^{\text{Pre}}_i R^{\text{Pre}}_i}{\Delta Y_i}
\end{equation*}

where $H_i = \sum_{iz}\Delta H_{iz}$ denotes total rented space in $i$, 
$\text{Pre}$ and $\text{Post}$ denote moments before and after the MW change,
$R_i$ denotes rents per square foot, and 
$Y_i$ denotes total income.

Changes in the rented square footage are unobserved so we assume 
$h^{\text{Pre}}_i = h^{\text{Post}}_i = h_i$ so that

\begin{equation*}
	\rho_i = \frac{h^{\text{Post}}_i R^{\text{Post}}_i - h^{\text{Pre}}_i R^{\text{Pre}}_i}{\Delta Y_i} = 
	         h_i \frac{\Delta R_i}{\Delta Y_i}
\end{equation*}
If instead $\Delta h_i > 0$ then our estimate of $\rho_i$ is a lower bound.

From our empirical results we have that

\[
    \Delta \ln R_i = \beta \MW_i^{\text{wkr}} + \gamma \MW_i^{\text{res}}
\]

Now, we also have to define the change in log income as

\[
    \Delta \ln Y_i = \varepsilon \MW_i^{\text{wkr}}
\]

where income is only a function of the MW at workplace. This is such because
the MW at residence has no influence on the total wages unless a worker works 
where he lives and that is already accounted for in the workplace MW measure. Note,
that the negative effect of residence MW on rents, conditional on workplace MW, are
going through changes in other prices. 

We estimate $\varepsilon$ with IRS data on the total wage bill at a ZIP code year using
the following level model\footnote{We use a level model because it is not feasible to 
take monthly first differences with yearly data. The level model on the average month 
within a year variables is identified under the same assumptions as our baseline model.}

\begin{equation}\label{eq:wage_level_model}
    \ln Y_{iy} = \alpha_i + \alpha_y + \varepsilon \overline{\MW_{iy}^{\text{wkr}}} + 
                 \overline{\mathbf{X}^{'}_{iy}}\eta + \psi_{iy}
\end{equation}

where $y$ indexes a year and the line over a variable indicates that it represents 
the average month within a year.

With knowledge of $\varepsilon$, algebra implies that

\begin{equation}\label{eq:rho}
	\begin{split}
		\rho_i & = h_i \left[ 
		\frac{\exp \left(\Delta \ln R_i + \ln R_i \right) - R_i }{\exp \left( \Delta \ln Y_i + \ln Y_i \right) - Y_i }
		\right] \\
		& = s_i \left[
		\frac{\exp \left( \beta \MW_i^{\text{wkr}} + \gamma \MW_i^{\text{res}} \right) - 1 }{\exp \left( \varepsilon \MW_i^{\text{wkr}} \right) - 1 }
		\right]
	\end{split}
\end{equation}

where $s_i = \frac{h_i R_i}{Y_i}$ is the share of $i$'s expenditure in housing. 
We will show results for different assumptions on $s_i$.

%%%%%%%%%%%%%%%%%%%%%%%%%%%%%%%%%%%%%%%%%%%%%%%%%%%%%%%%%%%%%%%%%%%%%%%%%%%%%%%%
\subsection{Results}

We use our estimates on data for US urban ZIP codes to compute $\rho_i$ after a 
counterfactual increase of the federal MW to \$9. We estimate only on urban ZIP codes
because our rental data is not representative of rural areas.

\subsubsection{Counterfactual increases in residence and workplace MWs}\label{sec:cf_res_and_wkp_changes}

We compute the counterfactual statutory MW in January 2020 at a given ZIP code by taking 
the max between (i) the state, county, and local MW in December 2019, and (ii) the 
counterfactual federal MW. Then we compute the counterfactuals for the residence MW by 
taking the natural logarithm, and for the workplace MW by using the structure of 
counterfactual statutory MWs and the constant commuting shares that we used throughout.

The counterfactual increases are displayed in Figure \ref{fig:cf_res_and_wkp_changes}. 
In panel (a), we see that around 40\% of the ZIP codes do not have a change in their 
residence MW, and around the same share have a change of about 24\% which corresponds 
to going from the previous federal MW of \$7.25 to \$9. The other fifth of the ZIP 
codes have changes in between with most of them clustered around changes of about 
5-10\%. The average change among urban ZIP codes of XXXXXX.

%%% DGP: reminder to complete number.

As for the workplace MW, we can appreciate in panel (b) that XXXXXXXX

%%% DGP: interpret panel (b) after we can see the correct figure.

\subsubsection{Predicted changes in rents and wage income}\label{sec:cf_rents_and_wage_changes}

With the counterfactual increases in residence and workplace MW and the 
coefficients estimated in \ref{sec:results_main}, we can compute the predicted 
counterfactual changes in rents as explained in \ref{sec:emp_cf}. Figure 
\ref{fig:cf_rents_and_wage_changes} panel (a) displays a histogram of those 
changes. 

%%% DGP: reminder to interpret panel (a) after we have the correct figure.

In Appendix Table \ref{tab:static_wages}, we show estimates of $\varepsilon$ for different 
specifications of the model given in \ref{eq:wage_level_model}. In panel (b), we display 
the counterfactual changes in the total wage bill as given by the estimate of $\varepsilon$ 
of column (3). Our estimates are in the same ballpark that the recent literature \parencite{CegnizEtAl2019}.

%%% DGP: reminder to interpret panel (b) after we have the correct figure.

Finally, in Figure \ref{fig:map_chicago_cf_changes}, we show a visualization example 
of our counterfactual changes and rent predictions for the case of the 
Chicago-Naperville-Elgin CBSA. 

%%% DGP: Shouldn't we show the map of $\rho$ (assuming $s_i = 0.35$) or total wagebill as well?

\subsubsection{Predicted pass-through to landlords}\label{sec:rho}

With the counterfactual changes in residence and workplace MWs and predictions of 
the changes in rents and the total wage bill we can directly apply Equation \ref{eq:rho} 
to obtain estimates of the pass-through to landlords. In principle, we could compute $\rho$ 
for each ZIP code. Unfortunately, we do not observe $s_i$, the share of ZIP code $i$'s 
expenditure in housing. For that reason, we will use average changes in the residence and 
workplace MW measures for two groups of urban ZIP codes, (i) those that before the counterfactual 
change had a residence MW of at most \$9, and (ii) those that had it already of more than\$9. The 
first group, is representative of ZIP codes where the new federal MW is binding and as a consequence 
the residence MW changed, therefore producing changes in the workplace measure. The second group, 
represents ZIP codes that are only affected by the counterfactual change through changes in the 
workplace measures.
For each group we compute estimates of $\rho$ for two hypothetical ZIP codes with $s_i$ equal to 
$0.25$ and $0.45$ respectively. In Table \ref{tab:counterfactuals}, we display those estimates.

%%% DGP: reminder to interpret table.

More generally, and holding constant $s_i$, one can think of the $\rho$ for different values
of the gap between the residence and workplace measures $\Delta \MW_i^{wrk} - \Delta \MW_i^{res}$.
In Figure \ref{fig:rho_by_decile_MW_gap} we display estimates of $\rho$ for each decile of that gap.

%%% DGP: Reminder to interpret figure.


