%%%%%%%%%%%%%%%%%%%%%%%%%%%%%%%%%%%%%%%%%%%%%%%%%%%%%%%%%%%%%%%%%%%%%%%%%%%%%%%%%
%%%%%                             DISCUSSION                                 %%%%
%%%%%%%%%%%%%%%%%%%%%%%%%%%%%%%%%%%%%%%%%%%%%%%%%%%%%%%%%%%%%%%%%%%%%%%%%%%%%%%%%

In this section, we use our empirical results to explore the consequences of 
a counterfactual increase in the federal MW.
Because our data ends in December 2019, we study the consequences of a 
counterfactual increase in the federal minimum wage in January 2020 from 
\$7.25 to \$9, assuming all other MW policies stay constant. 
In an appendix, we present simulations for two more counterfactual policies: 
a 10 percent increase and an increase from \$7.25 to \$15 in the federal MW.

The results in Section \ref{sec:results} suggest that landlords will benefit
from the increase. 
In this section, we estimate the share of each dollar generated by the MW 
increase that is absorbed by rent increases.
A federal MW increase will alter the statutory MW in some places, and this will
spill over onto other places via commuting.
Thus, the pass-through will vary by ZIP code depending on how exposed they
are to the policy which, in turn, is determined by the prevailing commuting
structure and MW levels.

% Discuss policy implications.

\subsection{Empirical Approach}\label{sec:emp_cf}

Consider an increase of the federal MW to \$9 in January 2020.
The policy will shift income spatially, and therefore changes in 
the housing expenditure.
In \ref{sec:results} we estimated the effects of MW changes on rents.
Now, we are interested the incidence of those changes on different economic
actors.
In particular, we want to know how much of each extra dollar put on the table 
due to the MW increase is captured by landlords.

Following the notation of our model in Section \ref{sec:model}, define the 
ZIP code-specific landlord share as
\begin{equation*}\label{eq:share_landlords}
    \rho_i := \frac{\Delta H_i}{\Delta Y_i} 
            = \frac{H^{\post}_i R^{\post}_i - H^{\pre}_i R^{\pre}_i}{\Delta Y_i}
\end{equation*}
$\pre$ and $\post$ denote moments before and after the MW change,
where $H_i = \sum_{iz}\Delta H_{iz}$ denotes total rented space in $i$, 
where $H_i = \sum_{iz}\Delta Y_{iz}$ denotes total wage income in $i$, 
and, as before,
$R_i$ denotes rents per square foot.

Changes in rented square footage (if any) are unobserved.
Therefore, we assume that
$H^{\pre}_i = H^{\post}_i = H_i$ 
so the share becomes
\begin{equation}\label{eq:share_landlords}
    \rho_i = \frac{H^{\post}_i R^{\post}_i - H^{\pre}_i R^{\pre}_i}{\Delta Y_i} = 
                H_i \frac{\Delta R_i}{\Delta Y_i}
\end{equation}
If $\Delta H_i > 0$ instead (as one would expect on average), 
our estimates of $\rho_i$ will be a lower bound.

Abstracting away from fixed effects, controls, and the error term, 
equation \eqref{eq:fd} implies
\begin{equation}\label{eq:cf_rents_model}
    \Delta r_i = \beta \mw_i^{\wkp} + \gamma \mw_i^{\res} .
\end{equation}
Our assumptions thus hold 
constant common shocks affecting all ZIP codes,
economic trends reflected in the controls, and
idiosyncratic shocks that show up in the error term.
We define the change in log wage using a first-differenced model as well:
\begin{equation}\label{eq:cf_wages_model}
    \Delta y_i = \varepsilon \mw_i^{\wkp}
\end{equation}
where $y_i=\ln Y_i$.
This is such because we are considering the effect of the MW on nominal wages
which should not be affected by changing prices.

We estimate $\varepsilon$ using IRS data aggregated at the ZIP code level.
While estimating the spillover effect of the MW on wages across ZIP codes is 
not the main goal of the paper, 
estimates of this parameter are not readily available in the literature.
There are of course estimates of the effect of the MW on income of workers
inside the same juridisction.
Appendix \ref{sec:mw_on_income} discusses the details of our estimation 
strategy.
As we discuss in the next section, our results are in line with existing 
literature.
We also find, consistent with equation \eqref{eq:cf_wages_model}, no effect 
of the residence MW on income.

Assuming that we know the value of $\varepsilon$, we can put substitute
\eqref{eq:cf_rents_model} and \eqref{eq:cf_wages_model} into equation
\eqref{eq:share_landlords} to obtain
\begin{equation}\label{eq:rho}
    \begin{split}
        \rho_i & = H_i \left[ 
        \frac{\exp \left(\Delta \ln R_i + \ln R_i \right) - R_i }{\exp \left( \Delta \ln Y_i + \ln Y_i \right) - Y_i }
        \right] \\
        & = s_i \left[
            \frac{\exp \left( \beta \mw_i^{\wkp} + \gamma \mw_i^{\res} \right) - 1 }
                {\exp \left( \varepsilon \mw_i^{\wkp} \right) - 1 }
            \right]
    \end{split}
\end{equation}
where $s_i = \frac{H_i R_i}{Y_i}$ is the share of $i$'s expenditure in housing.
Because $s_i$ is not observable, we assume a range of plausible values for it
in our results.

%%%%%%%%%%%%%%%%%%%%%%%%%%%%%%%%%%%%%%%%%%%%%%%%%%%%%%%%%%%%%%%%%%%%%%%%%%%%%%%%
\subsection{Results}

We use our estimates on data for US urban ZIP codes to compute $\rho_i$ after a 
counterfactual increase of the federal MW to \$9. We estimate only on urban ZIP codes
because our rental data is not representative of rural areas.

\subsubsection{Counterfactual increases in residence and workplace MWs}\label{sec:cf_res_and_wkp_changes}

We compute the counterfactual statutory MW in January 2020 at a given ZIP code by taking 
the max between (i) the state, county, and local MW in December 2019, and (ii) the 
counterfactual federal MW. Then we compute the counterfactuals for the residence MW by 
taking the natural logarithm, and for the workplace MW by using the structure of 
counterfactual statutory MWs and the constant commuting shares that we used throughout.

The counterfactual increases are displayed in Figure \ref{fig:cf_res_and_wkp_changes}. 
In panel (a), we see that around 40\% of the ZIP codes do not have a change in their 
residence MW, and around the same share have a change of about 24\% which corresponds 
to going from the previous federal MW of \$7.25 to \$9. The other fifth of the ZIP 
codes have changes in between with most of them clustered around changes of about 
5-10\%. The average change among urban ZIP codes of XXXXXX.

%%% DGP: reminder to complete number.

As for the workplace MW, we can appreciate in panel (b) that XXXXXXXX

%%% DGP: interpret panel (b) after we can see the correct figure.

\subsubsection{Predicted changes in rents and wage income}\label{sec:cf_rents_and_wage_changes}

With the counterfactual increases in residence and workplace MW and the 
coefficients estimated in \ref{sec:results_main}, we can compute the predicted 
counterfactual changes in rents as explained in \ref{sec:emp_cf}. Figure 
\ref{fig:cf_rents_and_wage_changes} panel (a) displays a histogram of those 
changes. 

%%% DGP: reminder to interpret panel (a) after we have the correct figure.

%%% DGP: reminder to interpret panel (b) after we have the correct figure.

Finally, in Figure \ref{fig:map_chicago_cf_changes}, we show a visualization example 
of our counterfactual changes and rent predictions for the case of the 
Chicago-Naperville-Elgin CBSA. 

%%% DGP: Shouldn't we show the map of $\rho$ (assuming $s_i = 0.35$) or total wagebill as well?

\subsubsection{Predicted pass-through to landlords}\label{sec:rho}

With the counterfactual changes in residence and workplace MWs and predictions of 
the changes in rents and the total wage bill we can directly apply Equation \ref{eq:rho} 
to obtain estimates of the pass-through to landlords. In principle, we could compute $\rho$ 
for each ZIP code. Unfortunately, we do not observe $s_i$, the share of ZIP code $i$'s 
expenditure in housing. For that reason, we will use average changes in the residence and 
workplace MW measures for two groups of urban ZIP codes, (i) those that before the counterfactual 
change had a residence MW of at most \$9, and (ii) those that had it already of more than\$9. The 
first group, is representative of ZIP codes where the new federal MW is binding and as a consequence 
the residence MW changed, therefore producing changes in the workplace measure. The second group, 
represents ZIP codes that are only affected by the counterfactual change through changes in the 
workplace measures.
For each group we compute estimates of $\rho$ for two hypothetical ZIP codes with $s_i$ equal to 
$0.25$ and $0.45$ respectively. In Table \ref{tab:counterfactuals}, we display those estimates.

%%% DGP: reminder to interpret table.

More generally, and holding constant $s_i$, one can think of the $\rho$ for different values
of the gap between the residence and workplace measures $\Delta \MW_i^{wrk} - \Delta \MW_i^{res}$.
In Figure \ref{fig:rho_by_decile_MW_gap} we display estimates of $\rho$ for each decile of that gap.

%%% DGP: Reminder to interpret figure.


