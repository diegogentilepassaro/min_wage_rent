%%%%%%%%%%%%%%%%%%%%%%%%%%%%%%%%%%%%%%%%%%%%%%%%%%%%%%%%%%%%%%%%%%%%%%%%%%%%%%%%%
%%%%%                             DISCUSSION                                 %%%%
%%%%%%%%%%%%%%%%%%%%%%%%%%%%%%%%%%%%%%%%%%%%%%%%%%%%%%%%%%%%%%%%%%%%%%%%%%%%%%%%%

We use our empirical results to explore the consequences of counterfactual 
MW policies.
We present two scenarios.
First, we study the consequences of a counterfactual increase in the federal MW
from \$7.25 to \$9.
Second, we explore the incidence of increasing the local MW of Chicago City in 
the Chicago-Naperville-Elgin CBSA from \$13 to \$14.
For each policy, we estimate the share of each dollar generated by the new 
policy that is absorbed by rents in each location, and the share of each 
dollar that accrues to landlords overall.
These exercises illustrate the consequences of different MW policies in the 
housing market.

\subsection{Empirical Approach}\label{sec:emp_cf}

Consider an increase in the MW.
The policy will shift income spatially, and therefore affect housing demand
across places.
The extent of such shift will depend on the nature of the change in the MW.
A federal increase will affect all jurisdictions whose previous MW levels was 
surpassed, potentially influencing most regions in the country indirectly 
through commuting.
The impact of a local increase will be contained to nearby ZIP codes where 
affected workers reside.
In this section we derive formulas to estimate the incidence of these 
policies across ZIP codes.

Following the notation in Section \ref{sec:model}, define the ZIP code-specific 
share pocketed by landlords as
\begin{equation*}
    \rho_i := \frac{\Delta H_i R_i}{\Delta Y_i} 
            = \frac{H^{\post}_i R^{\post}_i - H^{\pre}_i R^{\pre}_i}{\Delta Y_i} 
\end{equation*}
where
``$\pre$'' and ``$\post$'' denote moments before and after the MW change,
$H_i R_i = \sum_{iz} H_{iz} R_{iz}$ denotes total housing expenditure in $i$, and
$Y_i = \sum_{iz} Y_{iz}$ denotes total wage income in $i$.

Changes in rented square footage (if any) are unobserved.
Therefore, we assume that
$H^{\pre}_i = H^{\post}_i = H_i$ 
and the share becomes
\begin{equation}\label{eq:share_pocketed}
    \rho_i = \frac{H^{\post}_i R^{\post}_i - H^{\pre}_i R^{\pre}_i}{\Delta Y_i} = 
                H_i \frac{\Delta R_i}{\Delta Y_i} .
\end{equation}
If $\Delta H_i > 0$ instead then our estimates of $\rho_i$ will be a lower bound.

We predict rent changes for all ZIP codes using the model in \eqref{eq:fd}.
Because we are interested only in the partial effect of the policy, we hold 
constant common shocks affecting all ZIP codes,
local economic trends reflected in the controls, and
idiosyncratic shocks that show up in the error term.
Then,
\begin{equation}\label{eq:cf_rents_model}
    \Delta r_i = \beta \Delta \mw_i^{\wkp} + \gamma \Delta \mw_i^{\res} .
\end{equation}
We define the change in log total wages using a first-differenced model as well:
\begin{equation}\label{eq:cf_wages_model}
    \Delta y_i = \varepsilon \Delta \mw_i^{\wkp} ,
\end{equation}
where $y_i=\ln Y_i$.
The residence MW is excluded because we are considering the effect of the MW on 
nominal wages.
We estimate $\varepsilon$ using IRS data aggregated at the ZIP code level.

While gauging the spillover effect of the MW on wages across ZIP codes is 
not the main goal of the paper, estimates of this parameter are not readily 
available in the literature.
Appendix \ref{sec:mw_on_income} discusses the details of our estimation 
strategy, and compares our result with estimates of the effect of the MW 
on income of workers in the same jurisdiction.
We also show that our results are heterogeneous depending on the share of 
MW workers residing in a location, although for ease of interpretation of the
results we use the simpler model in \ref{eq:cf_wages_model} in our exercises.

Assuming that we know the value of $\varepsilon$, we can substitute
\eqref{eq:cf_rents_model} and \eqref{eq:cf_wages_model} into equation
\eqref{eq:share_pocketed} to obtain
\begin{equation*}\label{eq:rho}
    \begin{split}
        \rho_i & = H_i \left[ 
        \frac{\exp \left(\Delta r_i + r_i \right) - R_i }
             {\exp \left(\Delta y_i + y_i \right) - Y_i }
        \right] \\
        & = s_i \left[
            \frac{\exp \left( \beta \Delta \mw_i^{\wkp} + \gamma \Delta \mw_i^{\res} \right) - 1 }
                {\exp \left( \varepsilon \Delta \mw_i^{\wkp} \right) - 1 }
            \right]
    \end{split}
\end{equation*}
where $s_i = \left(H_i R_i\right)/Y_i$ is the share of $i$'s expenditure in 
housing.
As discussed in Section \ref{sec:data_income_housing},
we estimate this share as the ratio of the 2-bedroom SAFMR rental value, 
$\tilde R_i$, and monthly average wage per household, $\tilde Y_i$.

We also compute the total incidence of the policy on ZIP codes $i\in\Z_1$
for some subset $\Z_1\subseteq\Z$ as follows:
\begin{equation*}\label{eq:tot_incidence}
    \rho_{\Z_1} = 
        \frac{\sum_{i\in\Z_1} \tilde R_i \left(\exp \left( \beta \Delta \mw_i^{\wkp} 
                                    + \gamma \Delta \mw_i^{\res} \right) - 1\right) }
            {\sum_{i\in\Z_1} \tilde Y_i \left( \exp \left( \varepsilon \Delta \mw_i^{\wkp} \right) 
                                    - 1\right) } .
\end{equation*}
Hence, total incidence is defined as the ratio of the total change in rents
per household in $\Z_1$ to the total change in wage income per household 
in $\Z_1$.

%%%%%%%%%%%%%%%%%%%%%%%%%%%%%%%%%%%%%%%%%%%%%%%%%%%%%%%%%%%%%%%%%%%%%%%%%%%%%%%%
\subsection{Results}\label{sec:results_cf}

We use our estimates to compute the shares 
$\{\{\rho_i\}_{i\in\Z_1},\rho_{\Z_1}\}$ for 2 counterfactual scenarios:
an increase of the federal MW from \$7.25 to \$9 and 
an increase in the Chicago City MW from \$13 to \$14.

In the federal case, we let $\Z_1$ be the set of ZIP codes located in urban 
CBSAs (as defined in Table \ref{tab:stats_zip_samples}) and exclude ZIP codes 
that are part of a CBSA where the average estimated increase in log total wages 
is less than 0.1\%.%
\footnote{\label{foot:restriction_on_zipcodes}
The goal of this restriction is to exclude metropolitan areas located 
in jurisdictions with a MW level above the new counterfactual federal level.
Because all those ZIP codes only experience a very small increase in the 
workplace MW, the estimated share pocketed will be equal to the estimated
housing expenditure share times the constant 
$\left(\exp(\beta x)-1\right)/\left(\exp(\varepsilon x)-1\right)$,
where $x$ is the value of the workplace MW increase.
These estimates, however, are not economically meaningful because the increase
in income due to the policy is negligible.}
In the local case, we focus on all ZIP codes in the Chicago-Naperville-Elgin CBSA,
which are the most exposed to this policy.

\subsubsection{Counterfactual increases in residence and workplace MW levels}
\label{sec:cf_res_and_wkp_changes}

We compute the counterfactual statutory MW in January 2020 at a given ZIP code 
by taking the max between (i) the state, county, and local MW in December 2019, 
and (ii) the assumed value for the federal or city MW in January 2020.%
\footnote{To be more precise, we take the maximum between the MWs of different
jurisdictions at the level of the block.
Then, we aggregate up to ZIP codes using the correspondence table in Appendix 
\ref{sec:blocks_to_uspszip}.
We do so to account for the fact that the new MW policy may be partially 
binding in some ZIP codes.}
Then, we compute the counterfactual values of the residence MW and the workplace
MW following the procedure outlined in Section \ref{sec:data_mw_measures}.
Like in our baseline estimates, we use commuting shares for all workers in
2017.

\paragraph{Federal increase}

The distributions of counterfactual increases in the MW measures are displayed 
in Appendix Figure \ref{fig:cf_hist_res_and_wkp_mw}.
Out of the $\zipcodesFedNine$ ZIP codes that satisfy our criteria, 
$\zipNoIncFedNine$ (or $\zipNoIncPctFedNine$\%) experience no increase in 
the residence MW at all.
The residence MW increases in $\zipIncFedNine$ ZIP codes  (or $\zipIncPctFedNine$\%), 
$\zipBoundFedNine$ of which were bound by the previous federal MW, and 
so the residence MW increases by $\ln(9)-\ln(7.25)\approx 0.2162$ in them.
Correspondingly, we observe mass points in the distribution of the residence MW,
with the two largest ones at $0$ and $0.2162$.
Since many people reside and work under the same statutory MW, these two mass
points are still visible in the histogram of the workplace MW.
However, we observe more places experiencing moderate increases in this measure.
The median change in the residence MW is $\MedChangeMWResFedNine$ log points, and 
in the workplace MW is $\MedChangeMWWkpFedNine$ log points.

Panel A of Appendix Figure \ref{fig:map_chicago_cf_wkp_res} maps the changes 
in the residence and workplace MW in the Chicago-Naperville-Elgin CBSA.
Unlike in Figure \ref{fig:map_mw_chicago_jul2019}, we observe the MW increasing 
from the outside of Cook County and spilling over inside it.

\paragraph{Local increase}

In our second counterfactual experiment we increase the Chicago City MW 
from \$13 to \$14 on January 2020, keeping constant other MW policies.
Importantly, under this assumption the difference between the Chicago
and Cook County MW levels increases by \$1.

In this case, there are $\zipIncChiFourteen$ ZIP codes whose 
residence MW are affected by this change and $\zipNoIncChiFourteen$ 
that remain directly unaffected.
Panel B of Appendix Figure \ref{fig:map_chicago_cf_wkp_res} shows the changes 
in both MW measures after this policy.
As expected, we observe large increases in the workplace MW in the city, 
which become smaller as one moves away from it.

\subsubsection{The share of extra wage income pocketed by landlords}
\label{sec:cf_rents_and_wage_changes}

We couple the counterfactual increases in residence and workplace MW with 
estimates of $\beta$, $\gamma$, and $\varepsilon$.
Following the results in Table \ref{tab:static}, we take 
$\beta = \betaCf$ and 
$\gamma = \gammaCf$.
Based on the results discussed in Appendix \ref{sec:mw_on_income}, we take
$\varepsilon = \epsilonCf$.
We follow the procedure outlined in the previous subsection to estimate the 
incidence of the counterfactual policy.

\paragraph{Federal increase}

Panel A of Figure \ref{fig:cf_hist_shares} displays a histogram 
of the estimated ZIP code-specific shares of the additional income pocketed by 
landlords $\{\rho_i\}_{i\in\Z_1}$.
The median estimated share equals $\rhoMedianFedNine$, which implies that at the 
median landlords capture roughly $\rhoMedianCentsFedNine$ cents of each dollar.
The distribution of the shares is skewed to the right.
However, we observe a long left-tail with a few negative values which arise due 
to declines in rents in locations where the increase in the residence MW is much 
larger than the increase in the workplace MW.

Panel A of Figure \ref{fig:map_chicago_cf_shares} maps the estimated shares 
pocketed by landlords in the Chicago-Naperville-Elgin CBSA.
Panel A of Appendix Figure \ref{fig:map_chicago_cf_rents_wages} shows
estimated increases in rents and wage income.
We estimate a larger share pocketed in Cook County.
The reason is that these ZIP codes experience the new policy only through
their workplace MW and, as a result, rents increase relatively more that 
wage income.
We also observe a larger incidence on landlords in the south of Cook County,
where the housing expenditure share is larger 
(as reflected in Appendix Figure \ref{fig:map_hous_exp_chicago}).

The top rows of Panel A in Table \ref{tab:counterfactuals} show the medians of 
the key estimated objects for two groups:
ZIP codes where the residence MW did not change, and 
ZIP codes where it did.
ZIP codes in the first group have both MW measures affected by the policy,
and as a result rent increases are moderated by the negative effect on local
prices.
The median incidence for this group is $\rhoMedCentsDirFedNine$ cents of each 
dollar.
Locations in the second group are only affected by the counterfactual policy 
through changes in the workplace MW, and as a result rent changes are 
relatively larger.
The median incidence for this group is $\rhoMedCentsIndirFedNine$ cents of each 
dollar.
The bottom row of Panel A in Table \ref{tab:counterfactuals} shows our estimate
of total incidence of the policy.
The share accruing to landlords in this sample of ZIP codes is given by 
$\totIncidenceCentsFedNine$ cents of each dollar.
The share is lower than the median values reported earlier because landlords 
capture more in locations with lower rent increases.

More generally, one can think of the average landlord share for different 
values of the gap between the residence MW and the workplace MW, 
i.e., $\Delta \mw_i^{\wkp} - \Delta \mw_i^{\res}$.
Figure \ref{fig:rho_by_decile_MW_gap} displays the average estimated $\rho$ for 
each decile of that gap.
We observe a nearly monotonic and positive relation.
The share is lower in ZIP codes that had a low increase in the workplace MW 
relative to the residence MW, highlighting how the share pocketed depends on
the incidence of the statutory MW increase on the MW measures.

\paragraph{Local increase}

Panel B of Figure \ref{fig:cf_hist_shares} shows the distribution of the 
estimated ZIP code-specific shares pocketed by landlords in the 
Chicago-Naperville-Elgin CBSA.
The median share pocketed by landlords throughout is $\rhoMedianChiFourteen$, 
slightly higher than in the federal MW increase.
Panel B of Figure \ref{fig:map_chicago_cf_shares} maps the shares.
Panel B of Appendix Figure \ref{fig:map_chicago_cf_rents_wages} shows the 
estimated changes in rents and total wages.
Unlike the previous exercise, the share pocketed by landlords is now higher 
right outside of Chicago City.
In those locations many commuters to the city reside, and thus the workplace
MW changes the most.
This translates into higher income increases and lower price increases, implying
a large share pocketed.%
\footnote{It is worth emphasizing that we estimate large increases in wage income
inside the city due to the fact that our model in \eqref{eq:cf_wages_model}
excludes heterogeneity based on the share of MW workers.
In a setting where this equation accounts for the share of MW workers we would 
not expect a strong effect on wages inside the city.} 

Panel B Table \ref{tab:counterfactuals} displays median values of the
share pocketed for ZIP codes that experienced an increase in the residence MW 
(inside the city), and for those where the residence MW did not change.
We observe similar patterns to Panel A.
The total incidence is now $\totIncidenceCentsChiFourteen$ cents of each dollar.

%%%%%%%%%%%%%%%%%%%%%%%%%%%%%%%%%%%%%%%%%%%%%%%%%%%%%%%%%%%%%%%%%%%%%%%%%%%%%%%%
\subsection{Discussion}\label{sec:discussion_cf}

Overall, we observe that landlords capture a significant portion of the income 
generated by MW policies.
We also found strong spatial heterogeneity of the incidence of the policy 
depending on commuting patterns.
The share pocketed by landlords tends to be larger in ZIP codes located in 
jurisdictions where the MW policy did not change,
particularly those located close to the MW change as many of their residents
work under the new MW level.

Because of the housing market, 
the impact of the MW will be less equalizing in terms of the distribution of
real incomes than nominal incomes.
There are many reasons for this.
First, poorer areas tend to have a higher share of expenditure in housing.
Second, as we discussed in Section \ref{sec:data_income_housing},
low-wage households are more likely to rent.
Finally, in the case of high-income cities enacting MW policies, affected 
low-wage individuals are more likely to live outside the city where rent
increases will be larger.
