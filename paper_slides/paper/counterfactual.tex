%%%%%%%%%%%%%%%%%%%%%%%%%%%%%%%%%%%%%%%%%%%%%%%%%%%%%%%%%%%%%%%%%%%%%%%%%%%%%%%%%
%%%%%                             DISCUSSION                                 %%%%
%%%%%%%%%%%%%%%%%%%%%%%%%%%%%%%%%%%%%%%%%%%%%%%%%%%%%%%%%%%%%%%%%%%%%%%%%%%%%%%%%

In this section, we use our empirical results to explore the consequences of 
a counterfactual increase in the MW. 
We present two scenarios.
First, we study the consequences of a counterfactual increase in the federal MW
from \$7.25 to \$9.
Second, we explore the incidence of increasing the local MW in Cook County in 
the Chicago-Naperville-Elgin CBSA from \$13 to \$14.
This exercise allows us to estimate the share of each dollar generated by a MW 
increase that is absorbed by rents, and the share of each dollar that accrues to
landlords overall.
In an appendix, we present simulations for two more counterfactual policies: 
a 10 percent increase and an increase from \$7.25 to \$15 in the federal MW.

% Discuss policy implications.

\subsection{Empirical Approach}\label{sec:emp_cf}

Consider a MW increase in January 2020.
The policy will shift income spatially, and therefore affect housing demand
across places.
The extent of such shift will depend, among other factors, on the nature of the 
MW change.
A federal increase will affect all jurisdictions whose previous MW levels was 
surpassed, potentially affecting most regions in the country indirectly 
through commuting.
The impact of local increase will be contained to nearby ZIP codes where 
affected workers reside.
In this section we derive some formulas to estimate the incidence of these 
policies across different ZIP codes.

Following the notation of our model in Section \ref{sec:model}, define the 
ZIP code-specific share pocketed by landlords as
\begin{equation*}
    \rho_i := \frac{\Delta H_i R_i}{\Delta Y_i} 
            = \frac{H^{\post}_i R^{\post}_i - H^{\pre}_i R^{\pre}_i}{\Delta Y_i} 
\end{equation*}
where
``$\pre$'' and ``$\post$'' denote moments before and after the MW change,
$H_i R_i = \sum_{iz} H_{iz} R_{iz}$ denotes total housing expenditure in $i$, and
$Y_i = \sum_{iz} Y_{iz}$ denotes total wage income in $i$.

Changes in rented square footage (if any) are unobserved.
Therefore, we assume that
$H^{\pre}_i = H^{\post}_i = H_i$ 
so the share becomes
\begin{equation}\label{eq:share_pocketed}
    \rho_i = \frac{H^{\post}_i R^{\post}_i - H^{\pre}_i R^{\pre}_i}{\Delta Y_i} = 
                H_i \frac{\Delta R_i}{\Delta Y_i} .
\end{equation}
If $\Delta H_i > 0$ instead, our estimates of $\rho_i$ will be a lower bound.

We predict rent changes for all ZIP codes using our model in \eqref{eq:fd}.
Because we are interested only on the partial effect of the policy, we
hold constant common shocks affecting all ZIP codes,
local economic trends reflected in the controls, and
idiosyncratic shocks that show up in the error term.
Then,
\begin{equation}\label{eq:cf_rents_model}
    \Delta r_i = \beta \mw_i^{\wkp} + \gamma \mw_i^{\res} .
\end{equation}
We define the change in log total wages using a first-differenced model as well:
\begin{equation}\label{eq:cf_wages_model}
    \Delta y_i = \varepsilon \mw_i^{\wkp} ,
\end{equation}
where $y_i=\ln Y_i$.
The residence MW is excluded because we are considering the effect of the MW on 
nominal wages.
We estimate $\varepsilon$ using IRS data aggregated at the ZIP code level.
While estimating the spillover effect of the MW on wages across ZIP codes is 
not the main goal of the paper, 
estimates of this parameter are not readily available in the literature.
There are of course estimates of the effect of the MW on income of workers
inside the same jurisdiction.
Appendix \ref{sec:mw_on_income} discusses the details of our estimation 
strategy.
As we discuss there, our results are consistent with existing literature.

%
% SH: If we add a column to `tab:static_wages` using the residence MW we can
%     add the sentence below.
%
% We also find, consistent with equation \eqref{eq:cf_wages_model}, no effect 
% of the residence MW on income.
%

Assuming that we know the value of $\varepsilon$, we can substitute
\eqref{eq:cf_rents_model} and \eqref{eq:cf_wages_model} into equation
\eqref{eq:share_pocketed} to obtain
\begin{equation}\label{eq:rho}
    \begin{split}
        \rho_i & = H_i \left[ 
        \frac{\exp \left(\Delta r_i + r_i \right) - R_i }
             {\exp \left(\Delta y_i + y_i \right) - Y_i }
        \right] \\
        & = s_i \left[
            \frac{\exp \left( \beta \mw_i^{\wkp} + \gamma \mw_i^{\res} \right) - 1 }
                {\exp \left( \varepsilon \mw_i^{\wkp} \right) - 1 }
            \right]
    \end{split}
\end{equation}
where $s_i = \left(H_i R_i\right)/Y_i$ is the share of $i$'s expenditure in 
housing.
We estimate this share as the ratio of the 2-bedroom SAFMR rental value, 
$\tilde R_i$, and monthly average wage per household, $\tilde Y_i$,
so that $\tilde s_i = \tilde R_i/\tilde Y_i$.%
\footnote{This computation assumes that total housing expenditure and total
wages are proportional to their averages under the same constant of 
proportionality.}

We also compute the total incidence of the policy on ZIP codes $i\in\Z_1$
for some subset $\Z_1\subseteq\Z$ as follows:
\begin{equation}\label{eq:tot_incidence}
    \rho_{\Z_1} = 
        \frac{\sum_{i\in\Z_1} \tilde R_i \left(\exp \left( \beta \mw_i^{\wkp} 
                                    + \gamma \mw_i^{\res} \right) - 1\right) }
            {\sum_{i\in\Z_1} \tilde Y_i \left( \exp \left( \varepsilon \mw_i^{\wkp} \right) 
                                    - 1\right) } .
\end{equation}
Hence, total incidence is defined as the ratio of the total change in rents
per household to the total change in income per household across ZIP codes 
in $\Z_1$.

%%%%%%%%%%%%%%%%%%%%%%%%%%%%%%%%%%%%%%%%%%%%%%%%%%%%%%%%%%%%%%%%%%%%%%%%%%%%%%%%
\subsection{Results}\label{sec:results_cf}

We use our estimates to compute the shares 
$\{\{\rho_i\}_{i\in\Z_1},\rho_{\Z_1}\}$ for 2 counterfactual scenarios:
an increase of the federal MW to \$9, and 
an increase in the Cook County MW to \$14.

In the federal case, we let $\Z_1$ be the set of ZIP codes located in urban 
CBSAs (as defined in Table \ref{tab:stats_zip_samples}), excluding ZIP codes 
that are part of a CBSA where the average estimated increase in log total wages 
is less than 0.1\%.%
\footnote{\label{foot:restriction_on_zipcodes}
The goal of this restriction is to exclude metropolitan areas located 
in jurisdictions with a MW level above the new counterfactual federal level.
Because all those ZIP codes only experience a very small increase in the 
workplace MW, the estimated share pocketed will be equal the estimated
expenditure share share times the constant 
$\lim_{x\to 0} \left(\exp(\beta x)-1\right)/\left(\exp(\varepsilon x)-1\right)$.
These estimates, however, are not economically meaningful because the increase
in income due to the policy is negligible.}
In the local case,...[ADD ANY RESTRICTION USED FOR THIS SCENARIO].

\subsubsection*{Counterfactual increases in residence and workplace MW levels}
\label{sec:cf_res_and_wkp_changes}

We compute the counterfactual statutory MW in January 2020 at a given ZIP code 
by taking the max between (i) the state, county, and local MW in December 2019, 
and (ii) the assumed value for the federal or county MW in January 2020.%
\footnote{To be more precise, we take the maximum between the MWs of different
jurisdictions at the level of the block.
Then, we aggregate up to ZIP codes using the correspondence table in Appendix 
\ref{sec:blocks_to_uspszip}.
We do so to account for the fact that the new MW policy may be partially 
binding in some ZIP codes.}
Then, we compute the counterfactual values of the residence MW and the workplace
MW following the procedure outlined in Section \ref{sec:data_mw_measures}.
Like in our baseline estimates, we use commuting shares for all workers in
2017.

\paragraph{Federal Increase}

The distributions of counterfactual increases in the MW measures are displayed 
in Appendix Figure \ref{fig:cf_hist_res_and_wkp_mw}.
Out of the $\zipcodesFedNine$ ZIP codes that satisfy our criteria, 
$\zipNoIncFedNine$ (or $\zipNoIncPctFedNine$\%) experience no increase in 
the residence MW at all.
The residence MW increases in $\zipIncFedNine$ (or $\zipIncPctFedNine$\%) ZIP 
codes, $\zipBoundFedNine$ of which were bound by the previous federal MW, and 
so the residence MW increases by $\ln(9)-\ln(7.25)$ in them.
Correspondingly, we observe mass points in the distribution of the residence MW,
with the two largest ones at $0$ and $\ln(9)-\ln(7.25)$.
As a lot of people reside and work under the same statutory MW, these two mass
points are still visible in the histogram of the workplace MW.
However, we observe more places experiencing moderate increases in this measure.
The average change in the workplace MW is of $\AvgChangeMWWkpFedNine$ log points.

Appendix Figure \ref{fig:map_chicago_cf_wkp_res} maps the changes in the 
residence and workplace MW in the Chicago-Naperville-Elgin CBSA.
Unlike in Figure \ref{fig:map_mw_chicago_jul2019}, we observe the MW increasing 
from the outside of Cook County and spilling over inside it.

\paragraph{Local Increase}

[ADD RESULTS FOR CITY LEVEL MW INCREASE]

\subsubsection*{The share of extra income pocketed by landlords}
\label{sec:cf_rents_and_wage_changes}

We couple the counterfactual increases in residence and workplace MW with 
estimates of $\beta$, $\gamma$, and $\varepsilon$.
Following the results in Table \ref{tab:static}, we take 
$\beta = \betaCf$ and 
$\gamma = \gammaCf$.
Based on the results discussed in Appendix \ref{sec:mw_on_income}, we take
$\varepsilon = \epsilonCf$.
We follow the procedure outlined in the previous subsection to estimate the 
incidence of the counterfactual policy.

\paragraph{Federal Increase}

The top plot in Figure \ref{fig:cf_hist_rents_wages_shares} displays a histogram 
of the estimated ZIP code-specific shares of the additional income pocketed by 
landlords $\{\rho_i\}_{i\in\Z_1}$.
The bottom row of plots in Figure \ref{fig:cf_hist_rents_wages_shares} displays 
a histogram of the estimated changes in log rents and log total wages.
The median estimated share pocketed by landlords equals $\rhoMedianFedNine$, 
which implies that landlords capture roughly $\rhoMedianCentsFedNine$ cents of 
each dollar of the income increase generated by the new MW policy.
We also observe a few negative values for the share pocketed by landlords, which
arise due to declines in rents in places where the increase in the residence MW
is much larger than the increase in the workplace MW.
Figure \ref{fig:map_chicago_cf_rents_wages_shares} maps the estimated shares 
pocketed by landlords, along with the changes in rents and total wages, in the 
Chicago-Naperville-Elgin CBSA.
Because the new statutory MW is binding outside of Cook County, we estimate 
a larger share pocketed inside.
The reason is that these ZIP codes experience the new policy only through
their workplace MW and, as a result, rents increase relatively more.
We also observe a larger incidence on landlords in the south of Cook County,
where the housing expenditure share is larger 
(see Appendix Figure \ref{fig:map_hous_exp_chicago}).
As apparent from the figure, our three-parameter model captures rich patterns
of the policy.

The top rows of Table \ref{tab:counterfactuals_fed_9usd} show the medians of 
the key estimated objects for two groups: 
(i) ZIP codes that before the counterfactual increase had a statutory MW of 
at most \$9, and 
(ii) ZIP codes where the statutory MW was already of more than \$9.
ZIP codes in the first group have both MW measures affected by the policy,
and as a result rent increases are moderated by the negative effect on local
prices.
The median incidence for this group is $\rhoMedCentsDirFedNine$ cents of each 
dollar.
Locations in the second group are only affected by the counterfactual policy 
through changes in the workplace MW, and as a result rent changes are 
comparatively larger.
The median incidence for this group is $\rhoMedCentsIndirFedNine$ cents of each 
dollar.
Appendix Table \ref{tab:counterfactuals_other} shows analogous estimates for
two alternative policies: a 10\% increase in the federal MW, and an increase
in the federal MW to \$15.
We find very similar shares pocketed by landlords in both cases.

The bottom row of Table \ref{tab:counterfactuals_fed_9usd} shows our estimate
of total incidence of the policy.
The share accruing to landlords in this sample of ZIP codes is given by 
$\totIncidenceCentsFedNine$ cents of each dollar.
The share is lower than the median values reported earlier, consistent with
the fact that landlords capture more in locations with lower rent increases.

More generally, one can think of the average landlord share for different 
values of the gap between the residence MW and the workplace MW, 
i.e., $\Delta \mw_i^{\wkp} - \Delta \mw_i^{\res}$.
Figure \ref{fig:rho_by_decile_MW_gap} displays the average estimated $\rho$ for 
each decile of that gap.
We observe a clear monotonic and positive relation.
The share is lower in ZIP codes that had a low increase in the workplace MW 
relative to the residence MW, consistent with the idea that a high increase 
in the residence MW lowers rents.

\paragraph{Local increase}

[ADD RESULTS ON SHARE POCKETED FOR CITY LEVEL INCREASE + COMPARISONS
WITH FEDERAL EXAMPLE]
