%%%%%%%%%%%%%%%%%%%%%%%%%%%%%%%%%%%%%%%%%%%%%%%%%%%%%%%%%%%%%%%%%%%%%%%%%%%%%%%%%
%%%%%                             DISCUSSION                                 %%%%
%%%%%%%%%%%%%%%%%%%%%%%%%%%%%%%%%%%%%%%%%%%%%%%%%%%%%%%%%%%%%%%%%%%%%%%%%%%%%%%%%

In this section, we use our empirical results to explore the consequences of 
a counterfactual increase in the federal MW.
Because our data ends in December 2019, we study the consequences of a 
counterfactual increase in the federal minimum wage in January 2020 from 
\$7.25 to \$9, assuming all other MW policies stay constant. 
In an appendix, we present simulations for two more counterfactual policies: 
a 10 percent increase and an increase from \$7.25 to \$15 in the federal MW.

The results in Section \ref{sec:results} suggest that landlords will benefit
from the increase. 
In this section, we estimate the share of each dollar generated by the MW 
increase that is absorbed by rent increases.
A federal MW increase will alter the statutory MW in some places, and this will
spill over onto other places via commuting.
Thus, the pass-through will vary by ZIP code depending on how exposed they
are to the policy which, in turn, is determined by the prevailing commuting
structure and MW levels.

% Discuss policy implications.

\subsection{Empirical Approach}\label{sec:emp_cf}

Consider an increase of the federal MW to \$9 in January 2020.
The policy will shift income spatially, and therefore changes in 
the housing expenditure.
In \ref{sec:results} we estimated the effects of MW changes on rents.
Now, we are interested the incidence of those changes on different economic
actors.
In particular, we want to know how much of each extra dollar put on the table 
due to the MW increase is captured by landlords.
We call this quantity the ``landlord share.''

Following the notation of our model in Section \ref{sec:model}, define the 
ZIP code-specific landlord share as
\begin{equation*}\label{eq:share_landlords}
    \rho_i := \frac{\Delta H_i}{\Delta Y_i} 
            = \frac{H^{\post}_i R^{\post}_i - H^{\pre}_i R^{\pre}_i}{\Delta Y_i}
\end{equation*}
$\pre$ and $\post$ denote moments before and after the MW change,
where $H_i = \sum_{iz}\Delta H_{iz}$ denotes total rented space in $i$, 
where $Y_i = \sum_{iz}\Delta Y_{iz}$ denotes total wage income in $i$, 
and, as before,
$R_i$ denotes rents per square foot.

Changes in rented square footage (if any) are unobserved.
Therefore, we assume that
$H^{\pre}_i = H^{\post}_i = H_i$ 
so the share becomes
\begin{equation}\label{eq:share_landlords}
    \rho_i = \frac{H^{\post}_i R^{\post}_i - H^{\pre}_i R^{\pre}_i}{\Delta Y_i} = 
                H_i \frac{\Delta R_i}{\Delta Y_i}
\end{equation}
If $\Delta H_i > 0$ instead (as one would expect on average), 
our estimates of $\rho_i$ will be a lower bound.

Abstracting away from fixed effects, controls, and the error term, 
equation \eqref{eq:fd} implies
\begin{equation}\label{eq:cf_rents_model}
    \Delta r_i = \beta \mw_i^{\wkp} + \gamma \mw_i^{\res} .
\end{equation}
Our assumptions thus hold 
constant common shocks affecting all ZIP codes,
economic trends reflected in the controls, and
idiosyncratic shocks that show up in the error term.
We define the change in log wage using a first-differenced model as well:
\begin{equation}\label{eq:cf_wages_model}
    \Delta y_i = \varepsilon \mw_i^{\wkp}
\end{equation}
where $y_i=\ln Y_i$.
This is such because we are considering the effect of the MW on nominal wages,
which should not be affected by changing prices.

We estimate $\varepsilon$ using IRS data aggregated at the ZIP code level.
While estimating the spillover effect of the MW on wages across ZIP codes is 
not the main goal of the paper, 
estimates of this parameter are not readily available in the literature.
There are of course estimates of the effect of the MW on income of workers
inside the same juridisction.
Appendix \ref{sec:mw_on_income} discusses the details of our estimation 
strategy.
As we discuss in the next section, our results are in line with existing 
literature.

%
% SH: If we add a column to `tab:static_wages` using the residence MW we can
%     add the sentence below.
%
% We also find, consistent with equation \eqref{eq:cf_wages_model}, no effect 
% of the residence MW on income.
%

Assuming that we know the value of $\varepsilon$, we can substitute
\eqref{eq:cf_rents_model} and \eqref{eq:cf_wages_model} into equation
\eqref{eq:share_landlords} to obtain
\begin{equation}\label{eq:rho}
    \begin{split}
        \rho_i & = H_i \left[ 
        \frac{\exp \left(\Delta r_i + r_i \right) - R_i }
             {\exp \left(\Delta y_i + y_i \right) - Y_i }
        \right] \\
        & = s_i \left[
            \frac{\exp \left( \beta \mw_i^{\wkp} + \gamma \mw_i^{\res} \right) - 1 }
                {\exp \left( \varepsilon \mw_i^{\wkp} \right) - 1 }
            \right]
    \end{split}
\end{equation}
where $s_i = \left(H_i R_i\right)/Y_i$ is the share of $i$'s expenditure in 
housing.
Because $s_i$ is not observable, we assume a range of plausible values for it
in our results.

%%%%%%%%%%%%%%%%%%%%%%%%%%%%%%%%%%%%%%%%%%%%%%%%%%%%%%%%%%%%%%%%%%%%%%%%%%%%%%%%
\subsection{Results}\label{sec:results_cf}

We use our estimates on data for US urban ZIP codes to compute the set of
of landlord shares ${\rho_i}$ after a counterfactual increase of the federal 
MW to \$9.
We compute the shares only for urban ZIP codes because our rental data are not 
representative of rural areas.

\subsubsection*{Counterfactual increases in residence and workplace MW levels}
\label{sec:cf_res_and_wkp_changes}

We compute the counterfactual statutory MW in January 2020 at a given ZIP code 
by taking the max between (i) the state, county, and local MW in December 2019, 
and (ii) the assumed value for the federal MW in January 2020.%
\footnote{To be more precise, we take the maximum at the level of the block and 
then aggregate up to ZIP codes using the correspondence table in Appendix 
\ref{sec:blocks_to_uspszip}.
We do so to account for the fact that the federal MW may be partially binding
in some ZIP codes.}
Then we compute the counterfactual values of the residence MW and the workplace
MW following the procedure outlined in Section \ref{sec:sec:mw_construction}.
Following our baseline results, we use commuting shares for all workers in
2017.

The distributions of counterfactual increases are displayed in Appendix
Figure \ref{fig:cf_hist_res_and_wkp_mw}.
We observe that around 40\% of the ZIP codes do not experience a change 
in their residence MW, and a similar percentage of ZIP codes experiences a
change of about 24\%.
These mass points correspond to the cases of ZIP codes where the previous 
federal MW was binding, and ZIP codes where the statutory MW was already 
above \$9, respectively.
The remaining ZIP codes experiences changes in between with most of them
clustered around changes of about 4 to 10\%.
Changes in the workplace MW measure look like a smoothed-out version of
the changes in the residence MW.
As a lot of people reside and work under the same statutory MW, the two mass
points are still visible.
However, they concertrate fewer ZIP codes, and thus we observe more places 
experiencing moderate increases in this measure.
Appendix Figure \ref{fig:map_chicago_cf_wkp_res} maps the changes in the 
residence and workplace MW in the Chicago-Naperville-Elgin CBSA.

\subsubsection*{Predicted changes in rents, wage income, and the landlord share}
\label{sec:cf_rents_and_wage_changes}

We couple the counterfactual increases in residence and workplace MW with 
estimates of $\beta$, $\gamma$ and $\varepsilon$.
Following the results in Table \ref{tab:static}, we take $\beta = 0.0546$ and 
$\gamma=-0.0207$.
Based on the results discussed in Appendix \ref{sec:mw_on_income}, we take
$\varepsilon = 0.1083$.
We assume that the ZIP code-specific share of housing expenditure is homogeneous,
and we present estimates for $s\in\left[0.25,0.45\right]$.
We follow the procedure outlined in the previous subsection to estimate the 
landlord share $\rho_i$.

The top plot in Figure \ref{fig:cf_hist_rents_wages_shares} displays a histogram 
of the estimated landord shares $\{\rho_i\}$.
The bottom row of plots in Figure \ref{fig:cf_hist_rents_wages_shares} displays 
a histogram of the estimated changes in log rents and log total wages.
The plot of the landlord share is computed by applying equation \eqref{eq:rho} 
on the changes in log rents and log total wages using the parameter values
described above.
The median estimated landlord share equals 0.108, which implies that landlords
capture roughly 10 cents on the dollar of the income increase generated by the 
federal MW increase.
We observe a mass point at 0.176 cents which corresponds to ZIP codes with
no change in the residence MW.
We also observe a few negative values for the landlord share, which
arise due to declines in rents in places where the increase in the residence MW
is much larger than the increase in the workplace MW.
Figure \ref{fig:map_chicago_cf_rents_wages_shares} maps the estimated 
landlord shares, along with the changes in rents and total wages, in the 
Chicago-Naperville-Elgin CBSA.
Because the new statutory MW is binding outside of Cook county, 
we estimate a larger increase in the landlord share inside of it.
As apparent from the figure, our three-parameter model captures rich patterns
of the policy.%
\footnote{The fact that the estimated landlord share is homogeneous inside Cook 
county arises because these ZIP codes experience a very similar change in the MW 
measures, and the ZIP code-specific share of expenditure is assumed constant.
Allowing for heterogeneity in any of these features would break this pattern.}

Table \ref{tab:counterfactuals_fed_9usd} shows the average estimated landlord 
share for two groups: 
(i) ZIP codes that before the counterfactual increase had a statutory MW of 
at most \$9, and 
(ii) ZIP codes were the statutory MW was already of more than \$9.
ZIP codes in the first group have both MW measures affected by the policy.
ZIP codes in the second group are only affected by the counterfactual change 
through changes in the workplace measures.
For expenditure shares $s\in[0.25, 0.45]$ we find that the average landlord
share is $[0.075, 0.136]$ in the first group and $[0.126, 0.227]$ in the 
second.
It is clear that, for a given $s$, ZIP codes that are exposed both directly
and indirectly have a lower increase in rents, and thus a smaller share
on the dollar of new income generated by the policy that accrus to landlords.
Appendix Table \ref{tab:counterfactuals_other} shows analogous estimates for
two alternative policies: a 10\% increase in the federal MW, and an increase
in the federal MW to \$15.
We find very similar average landlord shares in both cases.

More generally, and holding constant $s=0.35$, one can think of the average 
landlord share for different values of the gap between the residence and 
workplace MW measures 
$\Delta \mw_i^{\wkp} - \Delta \mw_i^{\res}$.
In Figure \ref{fig:rho_by_decile_MW_gap} we display estimates of $\rho$ for 
each decile of that gap.
The landlord share is lower in ZIP codes that had a low increase in the 
workplace MW relative to the residence MW, consistent with the idea
that a high increase in the residence MW lowers rents.
