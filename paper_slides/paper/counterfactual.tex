%%%%%%%%%%%%%%%%%%%%%%%%%%%%%%%%%%%%%%%%%%%%%%%%%%%%%%%%%%%%%%%%%%%%%%%%%%%%%%%%%
%%%%%                             DISCUSSION                                 %%%%
%%%%%%%%%%%%%%%%%%%%%%%%%%%%%%%%%%%%%%%%%%%%%%%%%%%%%%%%%%%%%%%%%%%%%%%%%%%%%%%%%

In this section, we use our empirical results to explore the consequences of 
a counterfactual increase in the federal MW.
Because our data ends in December 2019, we study the consequences of a 
counterfactual increase in the federal minimum wage in January 2020 from 
\$7.25 to \$9, assuming all other MW policies stay constant. 
In an appendix, we present simulations for two more counterfactual policies: 
a 10 percent increase and an increase from \$7.25 to \$15 in the federal MW.
We estimate the share of each dollar generated by the MW increase that is 
absorbed by rent increases, and the share of each dollar that accrues to 
landlords overall.

% Discuss policy implications.

\subsection{Empirical Approach}\label{sec:emp_cf}

Consider an increase of the federal MW to \$9 in January 2020.
The policy will shift income spatially, and therefore affect housing demand
across places.
In Section \ref{sec:results} we estimated the effects of MW changes on rents.
Now, we are interested in the incidence of those changes on different economic
actors.
In particular, we want to know how much of each extra dollar put on the table 
due to the MW increase is captured by landlords.

Following the notation of our model in Section \ref{sec:model}, define the 
ZIP code-specific share pocketed by landlords as
\begin{equation*}
    \rho_i := \frac{\Delta H_i R_i}{\Delta Y_i} 
            = \frac{H^{\post}_i R^{\post}_i - H^{\pre}_i R^{\pre}_i}{\Delta Y_i} 
\end{equation*}
where
``$\pre$'' and ``$\post$'' denote moments before and after the MW change,
$H_i = \sum_{iz} H_{iz}$ denotes total rented space in $i$, 
$Y_i = \sum_{iz} Y_{iz}$ denotes total wage income in $i$ 
and, as before,
$R_i$ denotes rents per square foot.

Changes in rented square footage (if any) are unobserved.
Therefore, we assume that
$H^{\pre}_i = H^{\post}_i = H_i$ 
so the share becomes
\begin{equation}\label{eq:share_pocketed}
    \rho_i = \frac{H^{\post}_i R^{\post}_i - H^{\pre}_i R^{\pre}_i}{\Delta Y_i} = 
                H_i \frac{\Delta R_i}{\Delta Y_i} .
\end{equation}
If $\Delta H_i > 0$ instead (as one would expect on average), 
our estimates of $\rho_i$ will be a lower bound.

We predict rent changes to all ZIP codes using our model in \eqref{eq:fd}.
Because we are interested only on the partial effect of the policy, we
hold constant common shocks affecting all ZIP codes,
local economic trends reflected in the controls, and
idiosyncratic shocks that show up in the error term.
Then,
\begin{equation}\label{eq:cf_rents_model}
    \Delta r_i = \beta \mw_i^{\wkp} + \gamma \mw_i^{\res} .
\end{equation}
We define the change in log total wages using a first-differenced model as well:
\begin{equation}\label{eq:cf_wages_model}
    \Delta y_i = \varepsilon \mw_i^{\wkp} ,
\end{equation}
where $y_i=\ln Y_i$.
The residence MW is excluded because we are considering the effect of the MW on 
nominal wages.

We estimate $\varepsilon$ using IRS data aggregated at the ZIP code level.
While estimating the spillover effect of the MW on wages across ZIP codes is 
not the main goal of the paper, 
estimates of this parameter are not readily available in the literature.
There are of course estimates of the effect of the MW on income of workers
inside the same jurisdiction.
Appendix \ref{sec:mw_on_income} discusses the details of our estimation 
strategy.
As we discuss in the next section, our results are in line with existing 
literature.

%
% SH: If we add a column to `tab:static_wages` using the residence MW we can
%     add the sentence below.
%
% We also find, consistent with equation \eqref{eq:cf_wages_model}, no effect 
% of the residence MW on income.
%

Assuming that we know the value of $\varepsilon$, we can substitute
\eqref{eq:cf_rents_model} and \eqref{eq:cf_wages_model} into equation
\eqref{eq:share_pocketed} to obtain
\begin{equation}\label{eq:rho}
    \begin{split}
        \rho_i & = H_i \left[ 
        \frac{\exp \left(\Delta r_i + r_i \right) - R_i }
             {\exp \left(\Delta y_i + y_i \right) - Y_i }
        \right] \\
        & = s_i \left[
            \frac{\exp \left( \beta \mw_i^{\wkp} + \gamma \mw_i^{\res} \right) - 1 }
                {\exp \left( \varepsilon \mw_i^{\wkp} \right) - 1 }
            \right]
    \end{split}
\end{equation}
where $s_i = \left(H_i R_i\right)/Y_i$ is the share of $i$'s expenditure in 
housing.
We estimate this share as the ratio of the 2-bedroom SAFMR rental value, 
$\tilde R_i$, and monthly average wage per household, $\tilde Y_i$.%
\footnote{This computation assumes that total housing expenditure and total
wages are proportional to their averages under the same constant of 
proportionality.}

We also compute the total incidence of the policy on ZIP codes $i\in\Z_1$ as
\begin{equation}\label{eq:tot_incidence}
    \rho = 
        \frac{\sum_{i\in\Z_1} \tilde R_i \left(\exp \left( \beta \mw_i^{\wkp} 
                                    + \gamma \mw_i^{\res} \right) - 1\right) }
            {\sum_{i\in\Z_1} \tilde Y_i \left( \exp \left( \varepsilon \mw_i^{\wkp} \right) 
                                    - 1\right) } .
\end{equation}
Hence, the total incidence is defined as the ratio of the total change in rents
to the ratio of the total change in income across ZIP codes in $\Z_1$

%%%%%%%%%%%%%%%%%%%%%%%%%%%%%%%%%%%%%%%%%%%%%%%%%%%%%%%%%%%%%%%%%%%%%%%%%%%%%%%%
\subsection{Results}\label{sec:results_cf}

We use our estimates on data for US urban ZIP codes to compute the set of
shares pocketed ${\rho_i}$ after a counterfactual increase of the federal 
MW to \$9.
We compute the shares only for urban ZIP codes because our rental data are not 
representative of rural areas.
We also exclude from our results ZIP codes that are part of a CBSA where the
average estimated increase in log total wages is less than 0.1\%.%
\footnote{\label{foot:restriction_on_zipcodes}
The goal of this restriction is to exclude metropolitan areas located 
in jurisdictions with a MW level above the new counterfactual federal level.
Because all those ZIP codes experience a similar very small increase in 
the workplace MW, the estimated share pocketed will be almost identical and 
close to 
$\lim_{x\to 0} s \left(\exp(\beta x)-1\right)/\left(\exp(\varepsilon x)-1\right)$.
These estimates, however, are not economically meaningful because the increase
in income due to the policy is negligible.}

\subsubsection*{Counterfactual increases in residence and workplace MW levels}
\label{sec:cf_res_and_wkp_changes}

We compute the counterfactual statutory MW in January 2020 at a given ZIP code 
by taking the max between (i) the state, county, and local MW in December 2019, 
and (ii) the assumed value for the federal MW in January 2020.%
\footnote{To be more precise, we take the maximum between the MWs of different
jurisdictions at the level of the block.
Then, we aggregate up to ZIP codes using the correspondence table in Appendix 
\ref{sec:blocks_to_uspszip}.
We do so to account for the fact that the new federal MW may be partially binding
in some ZIP codes.}
Then, we compute the counterfactual values of the residence MW and the workplace
MW following the procedure outlined in Section \ref{sec:mw_construction}.
Like in our baseline estimates, we use commuting shares for all workers in
2017.

The distributions of counterfactual increases are displayed in Appendix
Figure \ref{fig:cf_hist_res_and_wkp_mw}.
Out of the $\zipcodesFedNine$ ZIP codes that satisfy our criteria, $\zipNoIncFedNine$ (or $\zipNoIncPctFedNine$\%) 
experience no increase in the residence MW at all.
$\zipBoundFedNine$ ZIP codes ($\zipBoundPctFedNine$\%) were bound by the previous MW, and so the residence MW
increases by $\ln(9)-\ln(7.25)$.
The rest of the ZIP codes are somewhere in between.
As a lot of people reside and work under the same statutory MW, the two mass
points are still visible.
However, we observe more places experiencing moderate increases in this measure.
The average change in the workplace MW is of 13.2 log points.
As an example, Appendix Figure \ref{fig:map_chicago_cf_wkp_res} maps the changes 
in the residence and workplace MW in the Chicago-Naperville-Elgin CBSA.

\subsubsection*{The share of extra income pocketed by landlords}
\label{sec:cf_rents_and_wage_changes}

We couple the counterfactual increases in residence and workplace MW with 
estimates of $\beta$, $\gamma$, and $\varepsilon$.
Following the results in Table \ref{tab:static}, we take 
$\beta = \BothBetaBase$ and 
$\gamma = \BothGammaBase$.
Based on the results discussed in Appendix \ref{sec:mw_on_income}, we take
$\varepsilon = \epsilonCf$.
We follow the procedure outlined in the previous subsection to estimate the 
share of extra income pocketed by landlords $\rho_i$.
We assume that the share of housing expenditure is homogeneous, and 
we present estimates for $s\in\{0.25, 0.45\}$.
Because $\rho_i$ is linear in $s$, interpolating between these values of 
$s$ is straightforward.

The top plot in Figure \ref{fig:cf_hist_rents_wages_shares} displays a histogram 
of the estimated shares of the additional income pocketed by landlords 
$\{\rho_i\}$.
The bottom row of plots in Figure \ref{fig:cf_hist_rents_wages_shares} displays 
a histogram of the estimated changes in log rents and log total wages.
The plot of the landlord share is computed by applying equation \eqref{eq:rho} 
on the changes in log rents and log total wages using the parameter values
described above.
The median estimated share pocketed by landlords equals $\rhoMedianFedNine$, which implies 
that landlords capture roughly $\rhoMedianCentsFedNine$ cents on the dollar of the income increase 
generated by the new MW policy.
We observe a mass point at around 0.172 which corresponds to ZIP codes with
no change in the residence MW and small changes in the workplace MW.%
\footnote{As explained in footnote \ref{foot:restriction_on_zipcodes}, the 
share pocketed by landlords in these ZIP codes will tend to
$s \left(\exp(\beta)-1\right)/\left(\exp(\varepsilon)-1\right) 
= 0.35 \left(\exp(\BothBetaBase)-1\right)/
     \left(\exp(0.1083)-1\right)
= 0.1716$.}
    %%
    %% SH : Fill \varepsilon
    %%
We also observe a few negative values for the landlord share, which
arise due to declines in rents in places where the increase in the residence MW
is much larger than the increase in the workplace MW.
Figure \ref{fig:map_chicago_cf_rents_wages_shares} maps the estimated shares 
pocketed by landlords, along with the changes in rents and total wages, in the 
Chicago-Naperville-Elgin CBSA.
Because the new statutory MW is binding outside of Cook County 
(see Appendix Figure \ref{fig:map_chicago_cf_wkp_res}), 
we estimate a larger share pocketed inside.
The reason is that these ZIP codes experience the new policy only through
their workplace MW and, as a result,
rents increase relatively more there.
As apparent from the figure, our three-parameter model captures rich patterns
of the policy.%
\footnote{The estimated share pocketed by landlords is homogeneous inside 
Cook County because these ZIP codes experience a very similar change in the 
MW measures, and the ZIP code-specific share of expenditure is assumed constant.
This pattern would break if we either allow for heterogeneity in the model for 
rents or income or use expenditure share that vary geographically.}

Table \ref{tab:counterfactuals_fed_9usd} shows the average estimated landlord 
share for two groups: 
(i) ZIP codes that before the counterfactual increase had a statutory MW of 
at most \$9, and 
(ii) ZIP codes were the statutory MW was already of more than \$9.
ZIP codes in the first group have both MW measures affected by the policy.
ZIP codes in the second group are only affected by the counterfactual change 
through changes in the workplace measures.
For expenditure shares $s\in[0.25, 0.45]$ we find that the average landlord
share is $[0.075, 0.136]$ in the first group and $[0.126, 0.227]$ in the 
second.
It is clear that, for a given $s$, ZIP codes that are exposed both directly
and indirectly experience a lower increase in rents, and thus a smaller share
on the dollar of new income generated by the policy accrues to landlords.
Appendix Table \ref{tab:counterfactuals_other} shows analogous estimates for
two alternative policies: a 10\% increase in the federal MW, and an increase
in the federal MW to \$15.
We find very similar average shares pocketed by landlords in both cases.

More generally, and holding constant $s=0.35$, one can think of the average 
landlord share for different values of the gap between the residence MW and 
the workplace MW, $\Delta \mw_i^{\wkp} - \Delta \mw_i^{\res}$.
Figure \ref{fig:rho_by_decile_MW_gap} displays the average estimated $\rho$ for 
each decile of that gap.
The first decile equals -0.0194, whereas the ninth decile equals 0.0135.
We observe a clear monotonic and increasing relation, with a slope that 
steepens after the seventh.
The share is lower in ZIP codes that had a low increase in the workplace MW 
relative to the residence MW, consistent with the idea that a high increase 
in the residence MW lowers rents.
