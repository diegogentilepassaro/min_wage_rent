%%%%%%%%%%%%%%%%%%%%%%%%%%%%%%%%%%%%%%%%%%%%%%%%%%%%%%%%%%%%%%%%%%%%%%%%%%%%%%%%%
%%%%%                             DISCUSSION                                 %%%%
%%%%%%%%%%%%%%%%%%%%%%%%%%%%%%%%%%%%%%%%%%%%%%%%%%%%%%%%%%%%%%%%%%%%%%%%%%%%%%%%%

In this section, we use our results to explore the consequences of an increase
in the federal MW.
Because our data ends in December 2019, we study the consequences of a 
counterfactual increase in the federal minimum wage in January 2020. We
evaluate 3 cases: (i) an increase of 10\%, (ii) and increase to \$9, and 
(iii) an increase to \$15. 

% Discuss pass-through estimates. Stress that they depend on geography of prevailing
% MWs across the commmuting zone.

The effects of a MW increase on rents depend both on the residence and
the workplace MW, and as such there will be a gradient of counterfactual 
rental changes that will depend on the entire structure of commuting in the nearby 
ZIP codes and that could be very heterogeneous depending on the prevailing combination
of workers and residents ahead of the change.

% Discuss welfare briefly. Maybe conjecture on long-run effects (low-wage workers 
% relocating to areas with low MW and commuting to areas with high MW).

% Discuss policy implications.

%%%%%%%%%%%%%%%%%%%%%%%%%%%%%%%%%%%%%%%%%%%%%%%%%%%%%%%%%%%%%%%%%%%%%%%%%%%%%%%%
\subsection{Empirical Approach}

Following notation of the model in Section \ref{sec:model}, define the ZIP code-
specific pass-through as

\begin{equation}\label{eq:pass_through}
    \rho_i := \frac{\Delta H_i}{\Delta Y_i} 
            = \frac{H^{\text{Post}}_i R^{\text{Post}}_i - H^{\text{Pre}}_i R^{\text{Pre}}_i}{\Delta Y_i}
\end{equation}
where $H_i = \sum_{iz}\Delta H_{iz}$ denotes total rented space in $i$, 
$\text{Pre}$ and $\text{Post}$ denote moments before and after the MW change,
$R_i$ denotes rents per square foot, and 
$Y_i$ denotes total income.



%%%%%%%%%%%%%%%%%%%%%%%%%%%%%%%%%%%%%%%%%%%%%%%%%%%%%%%%%%%%%%%%%%%%%%%%%%%%%%%%
\subsection{Results}


