%%%%%%%%%%%%%%%%%%%%%%%%%%%%%%%%%%%%%%%%%%%%%%%%%%%%%%%%%%%%%%%%%%%%%%%%%%%%%%%%%
%%%%%                             CONCLUSION                                 %%%%
%%%%%%%%%%%%%%%%%%%%%%%%%%%%%%%%%%%%%%%%%%%%%%%%%%%%%%%%%%%%%%%%%%%%%%%%%%%%%%%%%

In this paper, we ask whether minimum wage changes affect housing rental prices. To answer this 
question we use rental listings from Zillow and MW changes collected from 
\textcite{vaghul2016historical}, \textcite{cengiz2019effect} and our ourselves, to construct a panel 
at the zipcode-month level. We exploit state, county, and city-level changes in the MW to identify 
the causal impact of increasing the MW. To do that, we leverage on a panel difference-in-differences 
approach that exploits the staggered implementation and the intensity of hundreds of MW increases 
across thousands of zipcodes. Our results indicate that minimum wage increases have a small but 
significant positive impact on rents that is robust to many alternative explanations. Across most 
specifications, a $10\%$ percent increase in MW causes on average an increase of $0.03 \%$ percent of 
the rental prices. The effect is largely concentrated in the first two months of the MW change. We go 
beyond the average MW effect and we look at the heterogeneity of effects across zipcodes. We show 
that rents disproportionately increase in zipcodes where: (i) it is more likely to find MW workers as 
residents, (ii) there is higher unemployment rate, and (iii) a larger share of African-American 
residents. Our results highlights that place-based policies aimed at the labor market can also have 
significant impacts on other related markets. In particular, MW provisions are usually thought as a 
way to guarantee economic means to low income workers but they may also be benefiting landlords in 
ways that are unintended. In this sense, studying how place-based policies affect the housing market 
becomes an important step to better understand income inequality across U.S. neighborhoods.

