%%%%%%%%%%%%%%%%%%%%%%%%%%%%%%%%%%%%%%%%%%%%%%%%%%%%%%%%%%%%%%%%%%%%%%%%%%%%%%%%%
%%%%%                         EMPIRICAL STRATEGY                             %%%%
%%%%%%%%%%%%%%%%%%%%%%%%%%%%%%%%%%%%%%%%%%%%%%%%%%%%%%%%%%%%%%%%%%%%%%%%%%%%%%%%%
    
%%%%%%%%%%%%%%%%%%%%%%%%%%%%%%%%%%%%%%%%%%%%%%%%%%%%%%%%%%%%%%%%%%%%%%%%%%%%%%%%%
\subsection{First-differences model}

Consider the following two-way fixed effects model relating rents and the 
minimum wage:
\begin{equation*} \label{eq:did_lev}
    r_{it} = \alpha_i + \hat{\delta}_t 
           + \gamma \mw^{\res}_{it} + \beta \mw^{\wkp}_{it}
           + \mathbf{X}^{'}_{it}\eta
           + \varepsilon_{it} \ 
\end{equation*}    
where
$i$ and $t$ index ZIP codes and time periods (months),
$r_{it}$ represents the log of rents per square foot,
$\mw^{\res}_{it}$ is the ZIP code's residence MW, defined as 
$\ln \MW_{it}$,
$\mw^{\wkp}_{it}$ is the ZIP code's workplace MW, defined as 
$\sum_{z\in\Z(i)} \pi_{iz}\ln \MW_{zt}$,
$\alpha_i$ and $\hat{\delta}_t$ are fixed effects, and 
$\mathbf{X}_{it}$ is a vector of county-level time-varying controls.
Time runs from February 2010 ($\underline{T}$) to December 2019 ($\overline{T}$).
The parameters of interest are $\gamma$ and $\beta$, which we interpret as the 
elasticity of rents to the residence and workplace MW, respectively.

By taking first differences on the previous equation we obtain
\begin{equation}\label{eq:fd}
    \Delta r_{it} = \delta_t
                  + \gamma \Delta \mw^{\res}_{it} + \beta \Delta \mw^{\wkp}_{it}
                  + \Delta \mathbf{X}^{'}_{it}\eta
                  + \Delta \varepsilon_{it} ,
\end{equation}
where $\delta_t = \hat{\delta}_t - \hat{\delta}_{t-1}$.
We spell out the model in first differences because it is reasonable to expect 
unobserved shocks to rental prices to be persistent over time. 
Appendix Table XX shows evidence of AR(1) auto-correlation in the error term.
% SH: 
%    Add Woldrige cite in the notes of table \textcite[][chapter 10]{wooldridge2010}
The main results of the paper are obtained under the model in \ref{eq:fd}. 
However, to compare to previous results we also estimate versions of the model
that exclude some of the MW measures.



%%%%%%%%%%%%%%%%%%%%%%%%%%%%%%%%%%%%%%%%%%%%%%%%%%%%%%%%%%%%%%%%%%%%%%%%%%%%%%%%%
\subsection{Identification and Causality}

We start by noting that, in order to separately identify the effect of 
residence and workplace MW changes, we need these variables to have independent
variation.
While this requirement is standard, it is not obvious that it holds in our
application.
For instance, if there were a single national minimum wage level or if everybody 
lived and worked in the same location, then we would have
$\Delta \mw^{\res}_{it} = \Delta \mw^{\wkp}_{it}$ for all $(i,t)$.
In the next section we show that there is substantial independent variation
in the MW measures.

Being able to compute $\gamma$ and $\beta$ does not mean that they can be given
a causal interpretation.
For this, we require a \textit{strict exogeneity} of both MW variables.
Formally,
\begin{equation*}\label{eq:strict_exogeneity}
    E\left[
        \begin{pmatrix}
            \Delta \mw^{\res}_{is} \\
            \Delta \mw^{\wkp}_{is} \\
        \end{pmatrix}
        \Delta \varepsilon_{it}
    \bigg| \delta_t, \Delta \mathbf{X}_{it} \right] =
    \begin{pmatrix}
        0 \\
        0 \\
    \end{pmatrix}
\end{equation*}
for all $s\in\{\underline{T}, ..., \overline{T}\}$.
That is, we require the unobserved shocks to rents to be uncorrelated with 
past and present values of changes in our MW measures 
conditional on time-period fixed effects and controls.

This assumption has two important implications.
First,
it implies no pre-trends in rents leading up to minimum wage changes 
(conditional on controls). We will test this implication more formally by 
including leads of the MW variables.
Second,
it rules out feedback effects from current values of rents on our MW variables, 
i.e., MW changes are assumed not to be influenced by past values of rents.
While we think this is a reasonable assumption---MW levels are usually not 
set by considering their effects on the housing market---, we allow this type of 
feedback effects in a specification described in the upcoming subsection.
Finally, we note that our identifying assumption allows for arbitrary 
correlation between ZIP code effects $\alpha_i$ and both MW variables
(e.g., our empirical strategy is robust to the fact that richer districts tend 
to vote for MW policies).


We worry that unobserved shocks, such as local business cycles, may 
systematically affect both rents and minimum wage changes, which is why
we include period-fixed effects and time-varying controls.
The period fixed effects should capture common trends in the housing market.
In some specifications we allow this trends to vary by CBSA.
To control for variation arising from unobserved trends in local markets we 
include economic controls from the QCEW.%
\footnote{These data are aggregated at the county level, and represent a second best given the 
unavailability of controls at the ZIP code level.}
Specifically, we use average weekly wages, employment and establishment counts 
for the sectors ``Professional and business services,'' ``Information,'' and 
``Financial activities.''
We assume that these sectors are not affected by the minimum wage.
In fact, according to \textcite[][table 5]{MinWorkersReportBLS}, in 2019 
such industries accounted for 3.5, 1, and 1.2 percent of the total number of MW workers, 
respectively.%
\footnote{In Appendix \ref{sec:app_econ_control} we show suggestive evidence 
that they are not affected by the MW by using them as dependent variable in our 
models.}
%%% SH:
%%%    REVISE APPENDIX
We also try models where we control for ZIP code-specific linear and quadratic 
trends, which should account for time-varying heterogeneity not captured by our 
economic controls that follows this pattern.

We can test the identification assumption using models that include leads and lags
of the MW variables:
\begin{equation} \label{eq:leads_lags}
    \Delta r_{it} = \delta_t
                  + \sum_{r=-s}^{s} \gamma_s \Delta \mw^{\res}_{is} 
                  + \sum_{r=-s}^{s} \beta_s \Delta \mw^{\wkp}_{is}
                  + \Delta \mathbf{X}^{'}_{it}\eta
                  + \Delta \varepsilon_{it} ,
\end{equation}
where $s$ is the number of months of a symmetric window around the MW change. 
We use $s=6$ but our results are very similar for windows of 3 or 9 months.
Because the MW measures are strongly correlated, adding leads and lags of both 
leads to a decline in precision.
Thus, we try models with leads and lags of only one of the MW measures as well.

Under the assumption that there are no anticipatory effects in the housing 
market, we interpret the absence of pre-trends as evidence against the presence 
of unobserved economic shocks driving our results.\footnote{We can also interpret the absence of pre-trends as a test 
	for anticipatory effects if we are willing to assume that the controls embedded in 
	$\mathbf{X}^{'}_{ct}$ capture all relevant unobserved heterogeneity arising from local 
	business cycles. While we find the interpretation given in the text more palatable, the 
	data does is consistent with both.}
    We think that, given the high frequency of our data and that we focus on short windows around MW 
    changes, the assumption of no anticipatory effects is reasonable. We further present evidence in 
    favor of this assumption by showing that MW changes do not predict the number of listings of 
    houses for sale in Zillow.\footnote{Ideally, we would run this regression on the number of rental 
        units. Unfortunately, as described in Section \ref{sec:data}, this information is not available.}
    Specifically, we can track the number of houses listed for sale in selected ZIP codes during the 
    period 2013-2019 for our preferred house type (SFCC). We use such series to run placebo 
    regressions where we estimate both the static and dynamic models using the change in (log) 
    listings as dependent variable. %Significant effects of MW changes, or pronounced pre-trends will 
    %indicate that policy changes actually affect the Zillow inventory composition and cast doubt on
    %the identifying assumption. We fail to find that.




The model in \ref{eq:fd} assumes a balanced panel of ZIP codes.
However, as explained in Section XX   %% Refer to last part of data section
we use a partially balanced panel.

%%%%%%%%%%%%%%%%%%%%%%%%%%%%%%%%%%%%%%%%%%%%%%%%%%%%%%%%%%%%%%%%%%%%%%%%%%%%%%%%%
\subsection{Identification beyond the MW measures}

Appendix \ref{sec:did_spillovers_id} shows that under parallel trends assumptions
the model in equation \ref{eq:fd} identifies parameters $\gamma$ and $\beta$.


%%%%%%%%%%%%%%%%%%%%%%%%%%%%%%%%%%%%%%%%%%%%%%%%%%%%%%%%%%%%%%%%%%%%%%%%%%%%%%%%%
\subsection{Alternative models}


Inspired by this approach, we use a stacked sample in Appendix Table XX.

Appendix Table XX shows that we obtain similar results under a weaker 
\textit{sequential exogeneity} assumption \parencite{ArellanoHonore2001}.
%% SH: 
%%    Discuss estimation procedure in notes of figure
