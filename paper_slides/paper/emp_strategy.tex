%%%%%%%%%%%%%%%%%%%%%%%%%%%%%%%%%%%%%%%%%%%%%%%%%%%%%%%%%%%%%%%%%%%%%%%%%%%%%%%%
%%%%%                         EMPIRICAL STRATEGY                            %%%%
%%%%%%%%%%%%%%%%%%%%%%%%%%%%%%%%%%%%%%%%%%%%%%%%%%%%%%%%%%%%%%%%%%%%%%%%%%%%%%%%
    
%%%%%%%%%%%%%%%%%%%%%%%%%%%%%%%%%%%%%%%%%%%%%%%%%%%%%%%%%%%%%%%%%%%%%%%%%%%%%%%%
\subsection{First-differences model}

Consider the following two-way fixed effects model relating rents and the 
minimum wage:
\begin{equation*} \label{eq:did_lev}
    r_{it} = \alpha_i + \tilde{\delta}_t 
           + \gamma \mw^{\res}_{it} + \beta \mw^{\wkp}_{it}
           + \mathbf{X}^{'}_{it}\eta
           + \varepsilon_{it} \ 
\end{equation*}    
where
$i$ and $t$ index ZIP codes and time periods (months),
$r_{it}$ represents the log of rents per square foot,
$\mw^{\res}_{it}$ is the ZIP code's residence MW, defined as 
$\ln \MW_{it}$,
$\mw^{\wkp}_{it}$ is the ZIP code's workplace MW, defined as 
$\sum_{z\in\Z(i)} \pi_{iz}\ln \MW_{zt}$,
$\alpha_i$ and $\hat{\delta}_t$ are fixed effects, and 
$\mathbf{X}_{it}$ is a vector of time-varying controls.
%% SH: I removed the c(i) from the controls. 
%%                Why? Sometimes we use zip code-specific trends
Time runs from February 2010 ($\underline{T}$) to December 2019 ($\overline{T}$).
The parameters of interest are $\gamma$ and $\beta$ which, following the model, 
we interpret as the elasticity of rents to the residence and workplace MW, 
respectively.

By taking first differences on the previous equation we obtain
\begin{equation}\label{eq:fd}
    \Delta r_{it} = \delta_t
                  + \gamma \Delta \mw^{\res}_{it} + \beta \Delta \mw^{\wkp}_{it}
                  + \Delta \mathbf{X}^{'}_{it}\eta
                  + \Delta \varepsilon_{it} ,
\end{equation}
where $\delta_t = \tilde{\delta}_t - \tilde{\delta}_{t-1}$.
We spell out the model in first differences because we expect unobserved shocks 
to rental prices to be serially autocorrelated over time, making the levels
model less efficient. 
%% SH:
%%    Make sure this sentence is correct.
Appendix Table XX shows strong evidence of AR(1) auto-correlation in the error 
term of the model in levels.
% SH: 
%    Add Woldrige cite in the notes of table \textcite[][chapter 10]{wooldridge2010}
The main results of the paper are obtained under the model in \eqref{eq:fd}. 
However, to compare to results in the literature we also estimate versions of 
the model that exclude either one of the MW measures.


%%%%%%%%%%%%%%%%%%%%%%%%%%%%%%%%%%%%%%%%%%%%%%%%%%%%%%%%%%%%%%%%%%%%%%%%%%%%%%%%
\subsection{Identification and Causality}

We start by noting that, in order to separately identify the effect of 
residence and workplace MW changes, we need these variables to have independent
variation.
While this is a standard requirement in applied work, it is not obvious that it 
holds in our application.
For instance, if there were a single national minimum wage level or if everybody 
lived and worked in the same location, then we would have
$\Delta \mw^{\res}_{it} = \Delta \mw^{\wkp}_{it}$ for all $(i,t)$.
In the next section we show that there is substantial independent variation
in the MW measures that allows us to separately identify their effects.

Being able to compute $\gamma$ and $\beta$ does not mean that they can be given
a causal interpretation.
For this, we require a \textit{strict exogeneity} assumption of both MW 
variables. 
Formally,
\begin{equation}\label{eq:strict_exogeneity}
    E\left[
        \begin{pmatrix}
            \Delta \mw^{\res}_{is} \\
            \Delta \mw^{\wkp}_{is} \\
        \end{pmatrix}
        \Delta \varepsilon_{it}
    \bigg| \delta_t, \Delta \mathbf{X}_{it} \right] =
    \begin{pmatrix}
        0 \\
        0 \\
    \end{pmatrix}
\end{equation}
for all $s\in\{\underline{T}, ..., \overline{T}\}$.
That is, we require the unobserved shocks to rents to be uncorrelated with 
past and present values of changes in our MW measures conditional on time-period 
fixed effects and controls.

This assumption has two important implications.
First,
it implies parallel trends in rents leading up to minimum wage changes 
(conditional on controls). We will test this implication more formally by 
including leads of the MW variables.
Second,
it rules out feedback effects from current values of rents on our MW variables, 
i.e., MW changes are assumed not to be influenced by past values of rents.
While we think this is a reasonable assumption---MW levels are often set at a 
much larger jurisdiction than the ZIP code and they are not usually
set considering their effects on the local housing market---, we allow this type of 
feedback effects in a specification described later.
Finally, we note that our identifying assumption allows for arbitrary 
correlation between ZIP code effects $\alpha_i$ and both MW variables
(e.g., our empirical strategy is robust to the fact that richer districts tend 
to vote for MW policies).
%% DGP: Should we make a cite or explain a bit why?

We worry that unobserved shocks, such as those caused by local business cycles, 
may systematically affect both rents and minimum wage changes.
To account for common trends in the housing market we include time-period 
fixed effects.
In some specifications we allow these trends to vary by different geographic 
jurisdictions. To control for variation arising from unobserved trends in local 
markets we include economic controls from the QCEW.%
\footnote{These data are aggregated at the county level, and represent a second 
best given the unavailability of local business cycle data at the ZIP code level.}
Specifically, we control for average weekly wage and establishment counts at the 
county-quarter, and for employment counts at the county-month, for the sectors 
``Professional and business services,'' ``Information,'' and 
``Financial activities.''
We assume that these sectors are not affected by the minimum wage.%
\footnote{In fact, according to \textcite[][table 5]{MinWorkersReportBLS}, in 
2019 the share of workers earning at or below the minimum wage in those 
industries was 0.8, 1.5, and 0.2, respectively.}
We also try models where we control for ZIP code-specific linear
trends, which should account for time-varying heterogeneity not controlled by 
our economic controls that follows a linear pattern.

We can test assumption \eqref{eq:strict_exogeneity} using models that include 
leads and lags of the MW variables:
\begin{equation} \label{eq:fd_leads_lags}
    \Delta r_{it} = \delta_t
                  + \sum_{r=-s}^{s} \gamma_s \Delta \mw^{\res}_{is} 
                  + \sum_{r=-s}^{s} \beta_s \Delta \mw^{\wkp}_{is}
                  + \Delta \mathbf{X}^{'}_{it}\eta
                  + \Delta \varepsilon_{it} .
\end{equation}
In this equation $s$ is the number of months of a symmetric window around the 
MW change.
We use $s=6$ as baseline but our results are very similar for windows of 3 or 9 
months.
%%% SH:
%%%     How should we go about the figure that showed the results for different
%%%     windows

%%% DGP: We should probably make a figure that has 3 panels? Related but more
%%% generally, should we report the coefficients obtained from the cumsum of the
%%% dynamic baseline model in some table? 

Because the MW measures are strongly correlated, adding leads and lags of both 
leads to a decline in precision.
%%% DGP: We have to document this claim. Should we just softly reference tab:static?
%%% It is not great becasue we haven't showed results but we can say, as we will show
%%% in section XXX ... 
Thus, we try models with leads and lags of only one of the MW measures as well.

Under the assumption that there are no anticipatory effects in the housing 
market, we interpret the absence of pre-trends as evidence against the presence 
of unobserved economic shocks driving our results.
We think that, given the high frequency of our data and the focus on short 
windows around MW changes, the assumption of no anticipatory effects seems 
plausible.%
\footnote{We can also interpret the absence of pre-trends as a test for 
anticipatory effects if we are willing to assume that the controls embedded in 
$\mathbf{X}_{it}$ capture all relevant unobserved heterogeneity arising from 
local business cycles.
While we find the interpretation given in the text more palatable, the data is 
consistent with both.}
We further present evidence in favor of this assumption by showing that our MW 
measures do not predict the number of listings of houses for sale in Zillow.%
%%% SH:
%%%    Update this analysis

%%% DGP: Good and easy task for Martin! But we should also consider removing this 
%%% sentence because sale listings are not great. Maybe we can present alternative 
%%% evidence, like BPS construction permits? 
\footnote{Ideally, we would run this regression on the number of rental units.
Unfortunately this information is not available in the Zillow data.
Specifically, we track the number of houses listed for sale in a sample of ZIP 
codes during the period 2013-2019 for our preferred house type (SFCC).}

As explained in Section \ref{sec:data_final_panel}, 
the model in \eqref{eq:fd} is estimated using a partially balanced panel.
This estimation is valid under a stronger version of 
\eqref{eq:strict_exogeneity}, where we also condition on sample selection.
However, if MW levels and rents tend to change in a particular way upon entry 
of a ZIP code to the data, then our results would be invalid.
Because of this we show that our results are similar when we estimate our model
on a fully balanced panel and under an unbalanced panel where we include 
time-period by time-of-entry fixed effects.

%%%%%%%%%%%%%%%%%%%%%%%%%%%%%%%%%%%%%%%%%%%%%%%%%%%%%%%%%%%%%%%%%%%%%%%%%%%%%%%%
\subsection{Identification beyond the MW measures}

As stated earlier, to be able to compute $\gamma$ and $\beta$ we require the 
residence and workplace MWs to vary independently within each period.
Furthermore, we require the strict exogeneity assumption to assure that our
estimates will be unbiased.
However, it is not totally clear what these assumptions mean for the underlying 
commuting shares and the dynamics of unobserved heterogeneity in trends for the 
most common type of treatment in our data: 
an increase in the statutory minimum wage of a city or state that affects 
a subset of the ZIP codes in a metropolitan area.

Following such a policy there will be ZIP codes where the residence MW goes up
and ZIP codes where it does not.
We call ZIP codes in the first group ``directly treated.''
Appendix \ref{sec:did_spillovers_id} shows that, for this policy, 
$\beta$ and $\gamma$ can be computed under the following assumptions: 
1) parallel trends between ZIP codes that are directly treated and ZIP codes 
that are not;
2) the existence of two groups of ZIP codes that are not treated directly and
have differential exposure to the policy via the commuting shares; and 
3) parallel trends between these two groups.


%%%%%%%%%%%%%%%%%%%%%%%%%%%%%%%%%%%%%%%%%%%%%%%%%%%%%%%%%%%%%%%%%%%%%%%%%%%%%%%%
%%% DGP: reminder that we need to potentially rethink thsi whole section!

\subsection{Heterogeneity and Representativeness of the Results}

If the mechanism proposed in Section \ref{sec:model} is correct, then we
expect the effect of the residence MW to be stronger where there are many MW 
workers.
The reason is that the production of non-tradable goods presumably uses more
low-wage work, and thus the increase in the MW would affect prices more.
Similarly, we expect the effect of the workplace MW to be stronger in locations
with lots of MW residents, as income would increase more strongly there.
We use different proxies for the shares of MW workers and residents, and 
estimate the following model:
\begin{equation}\label{eq:fd_heterogeneity}
    \Delta r_{it} = \delta_t
                  + \sum_{q\in\{1,2\}}\gamma_q I^{\wkp}_q \Delta \mw^{\res}_{it}
                  + \sum_{q\in\{1,2\}}\beta_q I^{\res}_q \Delta \mw^{\wkp}_{it}
                  + \Delta \mathbf{X}^{'}_{it}\eta
                  + \Delta \varepsilon_{it} ,
\end{equation}
where 
$I^{\wkp}_1$ and $I^{\wkp}_2$ are indicators for being below and above median 
in the share of low-wage workers, and
$I^{\res}_1$ and $I^{\res}_2$ are analogous indicators for the share of low-wage
residents.
We conduct similar heterogeneity exercises with socioeconomic variables, 
such as the share of non-white workers in a ZIP code.


%%% DGP: the following paragraph merits its own subsection with a bit more explanation.
Because our ZIP codes come from a selected sample, they may not represent
the causal effect for the average urban US ZIP code.
To obtain effects that are more representative we follow 
\textcite{Hainmueller2012} and estimate our main models re-weighting 
observations to match key moments of the distribution of characteristics of 
urban ZIP codes.

%%%%%%%%%%%%%%%%%%%%%%%%%%%%%%%%%%%%%%%%%%%%%%%%%%%%%%%%%%%%%%%%%%%%%%%%%%%%%%%%
\subsection{Alternative Strategies}\label{sec:alt_emp_strategies}
%% SH: Not sure what's the best name for this section

Recent literature on difference-in-differences methods has shown that usual
estimators do not correspond to the average treatment effect when the treatment 
roll-out is staggered and there is treatment-effect heterogeneity 
\parencite{deChaisemartinEtAl2022,RothEtAl2022}.
One solution to this issue usually relies on carefully defining the control
group of the treatment.
While our setting does not correspond exactly to the models discussed in this
literature, we worry about the validity of our estimator.
In an appendix, we take two steps to ease those concerns:
1) we estimate equation \eqref{eq:fd} allowing the time fixed effects to vary
by different jurisdictions; and
2) we construct a ``stacked'' implementation of our model in which we take
6 months of data around MW changes for ZIP codes in CBSAs where some ZIP codes 
did not received a direct MW change, and then estimate equation \eqref{eq:fd} on
this restricted sample including event-by-time fixed effects.
These strategies limit the comparisons that identify the coefficients of 
interest to ZIP codes within the same metropolitan area.

Regardless of the estimation strategy, our results still rely on the 
strict exogeneity assumption.
In an appendix we show that we obtain similar results under a model that 
includes the lagged dependent variable as control.
In such a model, $\beta$ and $\gamma$ have a causal interpretation under a 
weaker \textit{sequential exogeneity} assumption
\parencite{ArellanoBond1991, ArellanoHonore2001}.
This alternative assumption allows for feedback of rents onto MW changes.
