%%%%%%%%%%%%%%%%%%%%%%%%%%%%%%%%%%%%%%%%%%%%%%%%%%%%%%%%%%%%%%%%%%%%%%%%%%%%%%%%
%%%%%                         EMPIRICAL STRATEGY                            %%%%
%%%%%%%%%%%%%%%%%%%%%%%%%%%%%%%%%%%%%%%%%%%%%%%%%%%%%%%%%%%%%%%%%%%%%%%%%%%%%%%%

In this section we discuss our empirical strategy.
We start with an intuitive presentation of our identification argument, which
is formalized in an appendix.
Next, we specialize our discussion under the functional form suggested by
the model in Section \ref{sec:model}. 
We also discuss alternative estimation strategies, concerns related to
the sample of ZIP codes we use, and heterogeneity of estimated effects.

%%%%%%%%%%%%%%%%%%%%%%%%%%%%%%%%%%%%%%%%%%%%%%%%%%%%%%%%%%%%%%%%%%%%%%%%%%%%%%%%
\subsection{Intuitive Identification Argument}

Our data consist of rents, the residence and workplace MW measures, and 
economic controls.
We can learn the effect of the workplace MW from the slope of the relationship 
between the workplace MW and rents conditioning to places with a similar change 
in the residence MW.
We need to condition on the residence MW to remove confounding variation 
as it may affect locations through other channels, 
such as changes in prices of non-tradable consumption.
Likewise, a similar argument suggests that identifying the effect of the 
residence MW requires controlling for the workplace MW. 

For these slopes to correspond to causal effects, we need to make 
two assumptions.
The first one is a form of \textit{parallel trends}: among ZIP codes with 
the same residence MW, ZIP codes with higher and lower workplace MW levels would 
have had parallel trends in rents if not for the change in the workplace MW.
The second one is \textit{no selection on gains}: ZIP codes that receive
different levels of the workplace MW must experience a similar treatment effect
on average, conditional again on the residence MW.
If these assumptions hold, the (conditional) slope of the relationship between
the workplace MW and rents is actually the causal effect. 
We similarly need these assumptions to hold for the residence MW if we hope 
to give a causal interpretation to its coefficient.
Online Appendix \ref{sec:potential_outcomes} formalizes these assumptions in a 
potential outcomes framework following \textcite{CallawayEtAl2021}.
We discuss the plausibility of these assumptions later in this section.

%%%%%%%%%%%%%%%%%%%%%%%%%%%%%%%%%%%%%%%%%%%%%%%%%%%%%%%%%%%%%%%%%%%%%%%%%%%%%%%%
\subsection{Parametric Model}

Consider the two-way fixed effects model relating rents and the MW measures
given by
\begin{equation} \label{eq:func_form}
    r_{it} = \alpha_i + \tilde{\delta}_t 
           + \beta \mw^{\wkp}_{it} + \gamma \mw^{\res}_{it} 
           + \mathbf{X}^{'}_{it}\eta
           + \upsilon_{it} ,
\end{equation}    
where
$i$ and $t$ index ZIP codes and time periods (months), respectively,
$r_{it}$ represents the log of rents per square foot,
$\mw^{\wkp}_{it} = \sum_{z\in\Z(i)} \pi_{iz}\ln \MW_{zt}$ is the ZIP code's 
workplace MW,
$\mw^{\res}_{it} = \ln \MW_{it}$ is the ZIP code's residence MW,
$\alpha_i$ and $\tilde{\delta}_t$ are fixed effects, and 
$\mathbf{X}_{it}$ is a vector of time-varying controls.
Time runs from January 2015 $\left(\underline{T}\right)$ 
to December 2019 $\left(\overline{T}\right)$.
The parameters of interest are $\beta$ and $\gamma$ which, 
following the model in Section \ref{sec:model}, 
we interpret as the elasticity of rents to the residence MW and the workplace MW, 
respectively.

By taking first differences in equation \eqref{eq:func_form} we obtain
\begin{equation}\label{eq:fd}
    \Delta r_{it} = \delta_t
                  + \gamma \Delta \mw^{\res}_{it} + \beta \Delta \mw^{\wkp}_{it}
                  + \Delta \mathbf{X}^{'}_{it}\eta
                  + \Delta \upsilon_{it} ,
\end{equation}
where $\delta_t = \tilde{\delta}_t - \tilde{\delta}_{t-1}$.
We estimate the model in first differences because we expect unobserved shocks
to rental prices to be serially autocorrelated over time, making the levels
model less efficient.
Online Appendix Table \ref{tab:autocorrelation} shows strong evidence of serial 
auto-correlation in the error term of the model in levels.
While estimated coefficients are similar in levels and in first differences, 
standard errors are seven to nine times larger in the former.

A standard requirement for a linear model like \eqref{eq:fd} to be
estimable is a rank condition, which implies that both MW measures must have 
independent variation, conditional on the controls.
For instance, if there were a single national minimum wage level or if everybody 
lived and worked in the same location, then we would have
$\Delta \mw^{\res}_{it} = \Delta \mw^{\wkp}_{it}$ for all $(i,t)$.
If so, $\gamma$ and $\beta$ could not be separately identified.
We check in the data that the rank condition is satisfied.

The main results of the paper are obtained under the model in \eqref{eq:fd}. 
In order to compare with the literature we also estimate versions of the 
model that exclude either one of the MW measures.

%%%%%%%%%%%%%%%%%%%%%%%%%%%%%%%%%%%%%%%%%%%%%%%%%%%%%%%%%%%%%%%%%%%%%%%%%%%%%%%%
\subsection{Validity of Identification Assumptions}

The model in \eqref{eq:func_form} imposes a linear functional form.
This property rules out selection on gains, since then ZIP codes receiving a 
particular level of the MW measures will experience the same (constant) effect 
than ZIP codes that receive a different level.
This is one of the assumptions required for identification according to 
Online Appendix \ref{sec:potential_outcomes}.
We view this as a reasonable assumption.
For it to not hold, workers would need to anticipate not only future MW policies 
but also how future rental markets would be affected by them given the commuting 
structure, and select their residence so that rents react differently to the 
MW in different ZIP codes with similar levels of the MW measures.
We show in the results section that the (conditional) slope of log rents with 
respect to each of the MW measures appears linear, suggesting that the 
assumption of no selection on gains is plausible.

For estimates of $\beta$ and $\gamma$ from \eqref{eq:func_form} to have a causal 
interpretation we need another assumption: the error term $\Delta\upsilon_{it}$ 
must be \textit{strictly exogenous} with respect to the MW measures, and in 
particular with respect to the workplace MW.
This is in the spirit of parallel trends, the second assumption required for
identification in Online Appendix \ref{sec:potential_outcomes}.
This assumption implies that rents prior to a change in the workplace MW must
evolve in parallel.
We test for pre-trends adding leads and lags of the workplace MW in 
\eqref{eq:fd},%
\footnote{Specifically, we estimate
    \begin{equation*}
        \Delta r_{it} = \delta_t
                    + \gamma \Delta \mw^{\res}_{it} 
                    + \sum_{k=-s}^{s} \beta_k \Delta \mw^{\wkp}_{ik}
                    + \Delta \mathbf{X}^{'}_{it}\eta
                    + \Delta \upsilon_{it} ,
    \end{equation*}
    where $s=6$.
    Our results are very similar for different values of the window $s$.}
though we also experiment with adding leads and lags of the residence MW.
We only shift the workplace MW because estimating its effect is our focus, 
seeing the residence MW as a key control.
In fact, Online Appendix \ref{sec:potential_outcomes} suggests that we 
only need to condition on one of the MW measures for parallel trends of the 
other measure to hold.
Under the assumption of no anticipatory effects in the housing market, we 
interpret the absence of pre-trends as evidence against the presence 
of unobserved economic shocks driving our results.
Given the high frequency of our data and the focus on short windows around 
MW changes, the assumption of no anticipatory effects seems plausible.%
\footnote{We can also interpret the absence of pre-trends as a test for 
    anticipatory effects if we are willing to assume that the controls embedded 
    in $\mathbf{X}_{it}$ capture all relevant unobserved heterogeneity arising 
    from local business cycles.
    While we find the interpretation given in the text more palatable, the data 
    are consistent with both.}

Another implication of the strict exogenity assumption is that it allows for 
arbitrary correlation between $\alpha_i$ and both MW variables.
This means that our empirical strategy is robust to the fact that districts 
with more expensive housing tend to vote for MW policies.

We worry that unobserved shocks, such as those caused by local business cycles, 
may systematically affect both rent changes and MW changes, violating the 
strict exogeneity assumption.
To account for common trends in the housing market we include time-period 
fixed effects $\delta_t$, which in some specifications are allowed to vary by 
jurisdiction.
To control for variation in local labor markets trends we include economic 
controls from the QCEW in the vector $\mathbf{X}_{it}$.
Specifically, we control for average weekly wage and establishment counts at the 
county-quarter level, and for employment counts at the county-month level, 
for the sectors ``Professional and business services,'' ``Information,'' and 
``Financial activities.''%
\footnote{We assume that these sectors are not affected by the MW.
    In fact, according to the \textcite[][Table 5]{MinWorkersReportBLS}, in 
    2019 the percent of workers paid an hourly rate at or below the federal MW 
    in those industries was 0.8, 1.5, and 0.2, respectively.
    In comparison, 9.5 percent of workers in ``Leisure and hospitality'' were 
    paid an hourly rate at or below the federal MW.}
We also try models where we control for ZIP code-specific linear
trends, which should account for time-varying heterogeneity at the ZIP 
code-level that follows a linear pattern.

A second worry is that changes in the composition of rentals may drive the 
results.
For instance, if on the same month of the MW change more expensive listings go
to the market, then what looks like a rent increase may actually be changes in 
quality.%
\footnote{We thank an anonymous referee for pointing out this concern.}
We note that changes in housing size, which seem to be the key driver 
in price heterogeneity according to Online Appendix Figure 
\ref{fig:ahs_rent_sqft}, are controlled for because we use rents per square foot
as our outcome.
To more directly address this concern we present evidence using Zillow's 
observed rental index (ZORI), which is constructed using rental prices for the 
same housing unit in different moments in time.
Given that the index is averaged using three lags, a regression analysis
that relates period-$t$ MWs would be expected to affect the ZORI index
at $t-3$, and its first difference at $t-4$.
To adjust for this, in these models we use the 4th lead of the change in 
the MW measures.

%%%%%%%%%%%%%%%%%%%%%%%%%%%%%%%%%%%%%%%%%%%%%%%%%%%%%%%%%%%%%%%%%%%%%%%%%%%%%%%%
\subsection{Alternative Strategies}\label{sec:alt_emp_strategies}

Recent literature has shown that usual estimators in a difference-in-differences 
setting do not correspond to well-defined average treatment effects when the 
treatment roll-out is staggered and there is treatment-effect heterogeneity 
\parencite{deChaisemartinEtAl2022,RothEtAl2022}.
While our setting does not correspond exactly to the models discussed in this
literature, we worry about the validity of our estimator.
%\footnote{\textcite[][Section 3.4]{CallawayEtAl2021} discusses the properties 
%    of the TWFE estimator in the context of a single continuous treatment.}
To ease these concerns, in an appendix we construct a ``stacked'' implementation 
of equation \eqref{eq:fd} in which we take six months of data around MW changes 
for ZIP codes in CBSAs where some ZIP codes received a direct MW change and 
some did not, 
and then estimate the model on this restricted sample including event-by-time 
fixed effects.
This strategy limits the comparisons used to compute the coefficients of 
interest to ZIP codes within the same metropolitan area and event.

In a separate exercise we relax the strict exogeneity assumption.
We do so in an appendix as well, where we propose a model that includes 
lagged rents as an additional control.
In such a model, $\beta$ and $\gamma$ have a causal interpretation under a 
weaker \textit{sequential exogeneity} assumption
\parencite{ArellanoBond1991, ArellanoHonore2001}.
This alternative assumption requires innovations to rents to be uncorrelated 
only with past changes in the MW measures, and thus allows for feedback of 
rent shocks onto MW changes in future periods.
We estimate this model using an IV strategy in which the first lag of the 
change in rents is instrumented with the second lag.


%%%%%%%%%%%%%%%%%%%%%%%%%%%%%%%%%%%%%%%%%%%%%%%%%%%%%%%%%%%%%%%%%%%%%%%%%%%%%%%%
\subsection{Heterogeneity and Sample Selection Concerns}\label{sec:emp_start_heterogeneity}

We explore heterogeneity of our results with respect to pre-determined 
variables.
Given the mechanism proposed in Section \ref{sec:model}, we expect the effect 
of the residence MW to be stronger in locations where many workers earn 
close to the MW.
The reason is that the production of non-tradable goods presumably uses more
low-wage work, and thus the increase in the MW would affect prices more.
Similarly, we expect the effect of the workplace MW to be stronger in locations
with lots of MW workers as residents since income would increase more 
strongly there.
We then estimate the following model:
\begin{equation*}\label{eq:fd_heterogeneity}
    \Delta r_{it} = \Xi_t
                  + \tilde\gamma_0 \Delta \mw^{\res}_{it}
                  + \tilde\gamma_1 \iota_i \Delta \mw^{\res}_{it}
                  + \tilde\beta_0 \Delta \mw^{\wkp}_{it}
                  + \tilde\beta_1 \iota_i \Delta \mw^{\wkp}_{it}
                  + \Delta \mathbf{X}^{'}_{it}\tilde\eta
                  + \Delta \tilde\upsilon_{it} ,
\end{equation*}
where $\iota_i$ represents the standardized share of MW workers residing in $i$.
Because we cannot estimate the share of MW workers working in a given location,
we interact both the residence and workplace MW with the estimated share of 
MW residents.
We conduct a similar exercise using median household income and the share of 
public housing units.

As explained in Section \ref{sec:data_final_panel}, 
the model in equation \eqref{eq:fd} relies on a selected sample.
In an alternative estimation exercise we use an unbalanced panel with all 
ZIP codes with Zillow rental data in the SFCC category 
from February 2010 to December 2019, controlling for time period by 
quarterly date of entry fixed effects.
However, even all ZIP codes available in the Zillow data may be 
a selected sample of the set of urban ZIP codes.
To approximate the average treatment effect in urban ZIP codes we follow
\textcite{Hainmueller2012} and estimate our main models re-weighting 
observations to match key moments of the distribution of characteristics of 
those.
