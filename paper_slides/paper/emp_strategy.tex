%%%%%%%%%%%%%%%%%%%%%%%%%%%%%%%%%%%%%%%%%%%%%%%%%%%%%%%%%%%%%%%%%%%%%%%%%%%%%%%%
%%%%%                         EMPIRICAL STRATEGY                            %%%%
%%%%%%%%%%%%%%%%%%%%%%%%%%%%%%%%%%%%%%%%%%%%%%%%%%%%%%%%%%%%%%%%%%%%%%%%%%%%%%%%

%%%%%%%%%%%%%%%%%%%%%%%%%%%%%%%%%%%%%%%%%%%%%%%%%%%%%%%%%%%%%%%%%%%%%%%%%%%%%%%%
\subsection{Potential Outcomes Framework}

We begin our discussion assuming a general causal model for log rents as a 
function of the residence and workplace MW levels.
Appendix \ref{sec:potential_outcomes} formalizes the details of our discussion 
in a potential outcomes framework akin to \textcite{CallawayEtAl2021}.
We are interested in identifying average causal response parameters,
which we define as the derivative of average rents with respect to either
of the MW measures for a given treatment group.
The key variation in the data that we use to estimate these parameters is 
given by the slope of log rents with respect to one of the MW measures, 
conditional on the other one.

Appendix \ref{sec:potential_outcomes} makes it clear that two key assumptions 
are needed for the conditional slope of log rents with respect to the MW 
measures to identify the parameters of interest.
Since the assumptions are analogous for both MW measures, we focus on the 
workplace MW.
The first assumption is a form of \textit{parallel trends}: conditional on the 
residence MW, the average counterfactual evolution of rents under no dose for 
ZIP codes that receive some dose must equal that of ZIP codes that 
receive no dose.
And this must hold for any dose of the workplace MW.
The second assumption is \textit{no selection on gains}: ZIP codes that receive
different levels of the workplace MW must experience a similar treatment effect,
conditional again on the residence MW.

Will these assumptions hold?
The parallel-trends assumption is standard in a difference-in-differences 
setting.
We discuss it in the next section, where we also discuss pre-trends testing.
No selection on gains is a less standard assumption, and reflects the 
fact that we need to compare ZIP codes that receive different levels 
of treatment to obtain the average causal response from the slope of log 
rents with respect to either of the MW measures.
Because commuting shares are (assumed) fixed, any selection on gains must 
predate the change in the MW.
It is unlikely that workers would select in the past to cause the 
selection bias terms to have a particular shape across the distribution 
of residence and workplace MW levels.
They would need to anticipate not only future MW policies but also how 
future rental markets would be affected given the commuting structure.
We show in the next section that the (conditional) slope of log rents with 
respect to each of the MW measures appears linear, suggesting that the 
assumption of no selection on gains is plausible.

%%%%%%%%%%%%%%%%%%%%%%%%%%%%%%%%%%%%%%%%%%%%%%%%%%%%%%%%%%%%%%%%%%%%%%%%%%%%%%%%
\subsection{Parametric Model}

Consider the following two-way fixed effects model relating rents and the 
minimum wage:
\begin{equation} \label{eq:func_form}
    r_{it} = \alpha_i + \tilde{\delta}_t 
           + \gamma \mw^{\res}_{it} + \beta \mw^{\wkp}_{it}
           + \mathbf{X}^{'}_{it}\eta
           + \varepsilon_{it} \ 
\end{equation}    
where
$i$ and $t$ index ZIP codes and time periods (months), respectively,
$r_{it}$ represents the log of rents per square foot,
$\mw^{\res}_{it} = \ln \MW_{it}$ is the ZIP code's residence MW,
$\mw^{\wkp}_{it} = \sum_{z\in\Z(i)} \pi_{iz}\ln \MW_{zt}$ is the ZIP code's 
workplace MW,
$\alpha_i$ and $\hat{\delta}_t$ are fixed effects, and 
$\mathbf{X}_{it}$ is a vector of time-varying controls.
Time runs from January 2015 $\left(\underline{T}\right)$ 
to December 2019 $\left(\overline{T}\right)$.
The parameters of interest are $\gamma$ and $\beta$ which, 
following the model in Section \ref{sec:model}, 
we interpret as the elasticity of rents to the residence and workplace MW, 
respectively.
We show evidence in favor of the linearity assumption in Section \ref{sec:results}.

In the potential outcomes framework discussed in Appendix 
\ref{sec:potential_outcomes}, both identifying assumptions will hold if
the causal model for rents is given by \eqref{eq:func_form}.
Furthermore, because \eqref{eq:func_form} assumes constant effects, the 
average causal response \textit{on the treated} equals the average causal 
response for any level of the MW measures.

By taking first differences in equation \eqref{eq:func_form} we obtain
\begin{equation}\label{eq:fd}
    \Delta r_{it} = \delta_t
                  + \gamma \Delta \mw^{\res}_{it} + \beta \Delta \mw^{\wkp}_{it}
                  + \Delta \mathbf{X}^{'}_{it}\eta
                  + \Delta \varepsilon_{it} ,
\end{equation}
where $\delta_t = \tilde{\delta}_t - \tilde{\delta}_{t-1}$.
We estimate the model in first differences because we expect unobserved shocks
to rental prices to be serially autocorrelated over time, making the levels
model less efficient.
Appendix Table \ref{tab:autocorrelation} shows strong evidence of serial 
auto-correlation in the error term of the model in levels.
While estimated coefficients are similar in levels and in first differences, 
standard errors are seven to eight times larger in the former.

The main results of the paper are obtained under the model in \eqref{eq:fd}. 
To compare to results in the literature we also estimate versions of the 
model that exclude either one of the MW measures.

\subsubsection*{Identification of parametric model}

A standard requirement for a linear model like equation \eqref{eq:fd} is a 
rank condition.
Namely, the MW variables must have independent conditional variation for the 
model to be estimable.
For instance, if there were a single national minimum wage level or if everybody 
lived and worked in the same location, then we would have
$\Delta \mw^{\res}_{it} = \Delta \mw^{\wkp}_{it}$ for all $(i,t)$.
We check in the data that the MW measures experience independent conditional 
variation.

The second assumption required for identification is akin to the assumptions 
in the potential outcomes framework.
For $\beta$ and $\gamma$ to be identified the error term in \eqref{eq:fd} must 
be \textit{strictly exogenous} with respect to the MW measures. 
That is, we require the unobserved shocks to rents to be uncorrelated with 
past and present values of changes in our MW measures.
This assumption has two important implications.
First,
it implies that current rents cannot be influenced by future changes in the MW 
variables, similar to parallel trends.
We test this implication formally including leads of the MW variables.
Second,
it rules out feedback effects from current values of rents on our MW variables, 
i.e., MW changes are assumed not to be influenced by past values of rents.
While we think this is a reasonable assumption, we allow this type of feedback 
effects in a specification described later.
Finally, we note that this assumption allows for arbitrary correlation between 
ZIP code effects $\alpha_i$ and both MW variables 
(e.g., our empirical strategy is robust to the fact that districts with more
expensive housing tend to vote for MW policies).

We perform our pre-trends test using either one of the MW measures at a time.%
\footnote{For instance, for the workplace MW we estimate
\begin{equation*}
    \Delta r_{it} = \delta_t
                  + \gamma \Delta \mw^{\res}_{it} 
                  + \sum_{k=-s}^{s} \beta_k \Delta \mw^{\wkp}_{ik}
                  + \Delta \mathbf{X}^{'}_{it}\eta
                  + \Delta \varepsilon_{it} ,
\end{equation*}
where $s=6$.
Our results are very similar for different values of the window $s$.}
We do so because the potential outcomes framework in Appendix 
\ref{sec:potential_outcomes} suggests that we only need to condition on the
current level of one of the MW measures for the parallel trends assumption
of the other measure to hold (see Assumption \ref{assu:PT}).
A second reason is that the residence MW and workplace MW are strongly 
correlated.
Including leads and lags of both MW measures results in standard errors
that are between two and four times larger, diminishing the power of the 
pre-trends test.

We worry that unobserved shocks, such as those caused by local business cycles, 
may systematically affect both rents and minimum wage changes.
To account for common trends in the housing market we include time-period 
fixed effects.
In some specifications we allow these trends to vary by different geographic 
jurisdictions.
To control for variation arising from unobserved trends in local labor markets 
we include economic controls from the QCEW.%
\footnote{These data are aggregated at the county level, and represent a second 
best given the unavailability of local business cycle data at the ZIP code 
level.}
Specifically, we control for average weekly wage and establishment counts at the 
county-quarter level, and for employment counts at the county-month level, 
for the sectors ``Professional and business services,'' ``Information,'' and 
``Financial activities.''%
\footnote{We assume that these sectors are not affected by the MW.
In fact, according to the \textcite[][Table 5]{MinWorkersReportBLS}, in 
2019 the percent of workers earning at or below the MW in those 
industries was 0.8, 1.5, and 0.2, respectively.}
We also try models where we control for ZIP code-specific linear
trends, which should account for time-varying heterogeneity not controlled for 
by our economic controls that follows a linear pattern.

Under the assumption that there are no anticipatory effects in the housing 
market, we interpret the absence of pre-trends as evidence against the presence 
of unobserved economic shocks driving our results.
Given the high frequency of our data and the focus on short windows around 
MW changes, the assumption of no anticipatory effects seems plausible.%
\footnote{We can also interpret the absence of pre-trends as a test for 
anticipatory effects if we are willing to assume that the controls embedded in 
$\mathbf{X}_{it}$ capture all relevant unobserved heterogeneity arising from 
local business cycles.
While we find the interpretation given in the text more palatable, the data are 
consistent with both.}
% We further present evidence in favor of this assumption by showing that our MW 
% measures do not predict the number of listings of houses for sale in Zillow.%
% \footnote{Ideally, we would run this regression on the number of rental units.
% Unfortunately this information is not available in the Zillow data.
% Specifically, we track the number of houses listed for sale in a sample of ZIP 
% codes during the period 2013-2019 for our preferred house type (SFCC).}

% DGP: Adding back this sentence and footnote above seem like a good idea. Actually,
% Matt Pecenco suggested something like this after the dissertation presentation.

As explained in Section \ref{sec:data_final_panel}, 
the model in \eqref{eq:fd} is estimated using a balanced panel.
We also conduct an estimation exercise using an unbalanced panel with all 
ZIP codes with Zillow rental data in the SFCC category 
from February 2010 to December 2019, controlling for time period by 
quarterly date of entry fixed effects.

%%%%%%%%%%%%%%%%%%%%%%%%%%%%%%%%%%%%%%%%%%%%%%%%%%%%%%%%%%%%%%%%%%%%%%%%%%%%%%%%
\subsection{Alternative Strategies}\label{sec:alt_emp_strategies}
%% SH: Not sure what's the best name for this section

Recent literature has shown that usual estimators in a difference-in-differences 
setting do not correspond to well-define average treatment effects when the 
treatment roll-out is staggered and there is treatment-effect heterogeneity 
\parencite{deChaisemartinEtAl2022,RothEtAl2022}.
While our setting does not correspond exactly to the models discussed in this
literature, we worry about the validity of our estimator.%
\footnote{\textcite[][Section 3.4]{CallawayEtAl2021} discusses the properties 
of the TWFE estimator in the context of single continuous treatment.}
To try to ease these concerns, in an appendix we construct a ``stacked'' 
implementation of \eqref{eq:fd} in which we take 6 months of data around MW 
changes for ZIP codes in CBSAs where some ZIP codes did not receive a direct 
MW change, 
and then estimate the model on this restricted sample including event-by-time 
fixed effects.
This strategy limits the comparisons used to compute the coefficients of 
interest to ZIP codes within the same metropolitan area and event.

In a separate exercise we relax the strict exogeneity assumption.
We do so in an appendix, where we propose a model that includes the lagged 
rents variable as control.
In such a model, $\beta$ and $\gamma$ have a causal interpretation under a 
weaker \textit{sequential exogeneity} assumption
\parencite{ArellanoBond1991, ArellanoHonore2001}.
This alternative assumption requires innovations to rents to be uncorrelated 
only with past changes in the MW measures, and thus allows for feedback of 
rent shocks onto MW changes in future periods.
We estimate this model using an IV strategy in which the first lag of the change
in rents is instrumented with the second lag.


%%%%%%%%%%%%%%%%%%%%%%%%%%%%%%%%%%%%%%%%%%%%%%%%%%%%%%%%%%%%%%%%%%%%%%%%%%%%%%%%
\subsection{Sample Selection Concerns and Heterogeneity}\label{sec:emp_start_heterogeneity}

Because our ZIP codes come from a selected sample, they may not represent
the causal effect for the average urban ZIP code.
To approximate the average treatment effect in urban ZIP codes we follow
\textcite{Hainmueller2012} and estimate our main models re-weighting 
observations to match key moments of the distribution of characteristics of 
those.

As a separate exercise, we exploit heterogeneity of our model with respect
to pre-determined variables.
If the mechanism proposed in Section \ref{sec:model} is correct, then we
expect the effect of the residence MW to be stronger in locations with many 
residents earning the MW.
The reason is that the production of non-tradable goods presumably uses more
low-wage work, and thus the increase in the MW would affect prices more.
Similarly, we expect the effect of the workplace MW to be stronger in locations
with lots of MW workers as income would increase more strongly there.
We then estimate the following model:
\begin{equation}\label{eq:fd_heterogeneity}
    \Delta r_{it} = \delta_t
                  + \tilde\gamma_0 \Delta \mw^{\res}_{it}
                  + \tilde\gamma_1 \iota_i \Delta \mw^{\res}_{it}
                  + \tilde\beta_0 \Delta \mw^{\wkp}_{it}
                  + \tilde\beta_1 \iota_i \Delta \mw^{\wkp}_{it}
                  + \Delta \mathbf{X}^{'}_{it}\eta
                  + \Delta \varepsilon_{it} ,
\end{equation}
where $\iota_i$ represents the standardized share of MW workers in a ZIP code.
Because we cannot estimate the share of MW workers working in a given location,
we interact both the residence and workplace MW with the share of MW residents
according to the MW in the location.%
\footnote{We discuss our estimates of the share of MW workers who reside in each 
location in Section \ref{sec:data_income_housing}.}
We conduct a similar exercise using median household income and the share of 
public housing units.
