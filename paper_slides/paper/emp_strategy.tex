%%%%%%%%%%%%%%%%%%%%%%%%%%%%%%%%%%%%%%%%%%%%%%%%%%%%%%%%%%%%%%%%%%%%%%%%%%%%%%%%
%%%%%                         EMPIRICAL STRATEGY                            %%%%
%%%%%%%%%%%%%%%%%%%%%%%%%%%%%%%%%%%%%%%%%%%%%%%%%%%%%%%%%%%%%%%%%%%%%%%%%%%%%%%%
    
%%%%%%%%%%%%%%%%%%%%%%%%%%%%%%%%%%%%%%%%%%%%%%%%%%%%%%%%%%%%%%%%%%%%%%%%%%%%%%%%
\subsection{First Differences Model}

Consider the following two-way fixed effects model relating rents and the 
minimum wage:
\begin{equation*} \label{eq:did_lev}
    r_{it} = \alpha_i + \tilde{\delta}_t 
           + \gamma \mw^{\res}_{it} + \beta \mw^{\wkp}_{it}
           + \mathbf{X}^{'}_{it}\eta
           + \varepsilon_{it} \ 
\end{equation*}    
where
$i$ and $t$ index ZIP codes and time periods (months), respectively,
$r_{it}$ represents the log of rents per square foot,
$\mw^{\res}_{it}$ is the ZIP code's residence MW, defined as 
$\ln \MW_{it}$,
$\mw^{\wkp}_{it}$ is the ZIP code's workplace MW, defined as 
$\sum_{z\in\Z(i)} \pi_{iz}\ln \MW_{zt}$,
$\alpha_i$ and $\hat{\delta}_t$ are fixed effects, and 
$\mathbf{X}_{it}$ is a vector of time-varying controls.
%% SH: I removed the c(i) from the controls. 
%%                Why? Sometimes we use zip code-specific trends
Time runs from January 2015 $\left(\underline{T}\right)$ 
to December 2019 $\left(\overline{T}\right)$.
The parameters of interest are $\gamma$ and $\beta$ which, 
following the model in Section \ref{sec:model}, 
we interpret as the elasticity of rents to the residence and workplace MW, 
respectively.

By taking first differences on the previous equation we obtain
\begin{equation}\label{eq:fd}
    \Delta r_{it} = \delta_t
                  + \gamma \Delta \mw^{\res}_{it} + \beta \Delta \mw^{\wkp}_{it}
                  + \Delta \mathbf{X}^{'}_{it}\eta
                  + \Delta \varepsilon_{it} ,
\end{equation}
where $\delta_t = \tilde{\delta}_t - \tilde{\delta}_{t-1}$.
We estimate the model in first differences because we expect unobserved shocks
to rental prices to be serially autocorrelated over time, making the levels
model less efficient.
Appendix Table \ref{tab:autocorrelation} shows strong evidence of serial 
auto-correlation in the error term of the model in levels.
While the model in levels share coefficients signs with the one in first 
differences, standard errors are seven to eight times larger.

The main results of the paper are obtained under the model in \eqref{eq:fd}. 
However, to compare to results in the literature we also estimate versions of 
the model that exclude either one of the MW measures.


%%%%%%%%%%%%%%%%%%%%%%%%%%%%%%%%%%%%%%%%%%%%%%%%%%%%%%%%%%%%%%%%%%%%%%%%%%%%%%%%
\subsection{Identification}

We start by noting that, in order to separately identify the effect of 
residence and workplace MW changes, we need these variables to have independent
variation.
While this is a standard requirement in applied work, it is not obvious that it 
holds in our application.
For instance, if there were a single national minimum wage level or if everybody 
lived and worked in the same location, then we would have
$\Delta \mw^{\res}_{it} = \Delta \mw^{\wkp}_{it}$ for all $(i,t)$.

Being able to compute $\gamma$ and $\beta$ does not mean that they can be given
a causal interpretation.
For this, we require a \textit{strict exogeneity} assumption of both MW 
variables.
Formally, the key restriction is that
\begin{equation}\label{eq:strict_exogeneity}
    E\left[
        \begin{pmatrix}
            \Delta \mw^{\res}_{is} \\
            \Delta \mw^{\wkp}_{is} \\
        \end{pmatrix}
        \Delta \varepsilon_{it} \right] =
    \begin{pmatrix}
        0 \\
        0 \\
    \end{pmatrix}
\end{equation}
for all $s\in\{\underline{T}, ..., \overline{T}\}$.
%%
%% SH: We also need that this holds for controls
%%
That is, we require the unobserved shocks to rents to be uncorrelated with 
past and present values of changes in our MW measures.

This assumption has two important implications.
First,
it implies parallel trends in rents leading up to minimum wage changes 
(conditional on controls). We will test this implication more formally by 
including leads of the MW variables.
Second,
it rules out feedback effects from current values of rents on our MW variables, 
i.e., MW changes are assumed not to be influenced by past values of rents.
While we think this is a reasonable assumption---MW levels are often set at a 
much larger jurisdiction than the ZIP code and not considering their effects 
on local housing markets---we allow this type of feedback effects in a 
specification described later.
Finally, we note that our identifying assumption allows for arbitrary 
correlation between ZIP code effects $\alpha_i$ and both MW variables
(e.g., our empirical strategy is robust to the fact that districts with more
expensive housing tend to vote for MW policies).

We evaluate assumption \eqref{eq:strict_exogeneity} using models that 
include leads and lags of the MW variables:
\begin{equation} \label{eq:fd_leads_lags}
    \Delta r_{it} = \delta_t
                  + \sum_{k=-s}^{s} \gamma_k \Delta \mw^{\res}_{ik} 
                  + \sum_{k=-s}^{s} \beta_k \Delta \mw^{\wkp}_{ik}
                  + \Delta \mathbf{X}^{'}_{it}\eta
                  + \Delta \varepsilon_{it} .
\end{equation}
In this equation,
$k$ represents time relative to a MW change and 
$s$ is the number of months of a symmetric window around the event.
We use $s=6$ throughout the paper. However, our results are very similar for 
windows of 3 or 9 months.
%%% SH:
%%%     We decided that, for now, we won't show results for alternative windows.
%%%
Because the MW measures are strongly correlated, adding leads and lags of both 
results in a decline in precision.
Thus, we try models with leads and lags of only one of the MW measures as well.

We worry that unobserved shocks, such as those caused by local business cycles, 
may systematically affect both rents and minimum wage changes.
To account for common trends in the housing market we include time-period 
fixed effects.
In some specifications we allow these trends to vary by different geographic 
jurisdictions.
To control for variation arising from unobserved trends in local labor markets 
we include economic controls from the QCEW.%
\footnote{These data are aggregated at the county level, and represent a second 
best given the unavailability of local business cycle data at the ZIP code 
level.}
Specifically, we control for average weekly wage and establishment counts at the 
county-quarter level, and for employment counts at the county-month level, 
for the sectors 
``Professional and business services,'' ``Information,'' and 
``Financial activities.''
We assume that these sectors are not affected by the minimum wage.%
\footnote{In fact, according to \textcite[][Table 5]{MinWorkersReportBLS}, in 
2019 the percent of workers earning at or below the minimum wage in those 
industries was 0.8, 1.5, and 0.2, respectively.}
We also try models where we control for ZIP code-specific linear
trends, which should account for time-varying heterogeneity not controlled for 
by our economic controls that follows a linear pattern.

Under the assumption that there are no anticipatory effects in the housing 
market, we interpret the absence of pre-trends as evidence against the presence 
of unobserved economic shocks driving our results.
We think that, given the high frequency of our data and the focus on short 
windows around MW changes, the assumption of no anticipatory effects seems 
plausible.%
\footnote{We can also interpret the absence of pre-trends as a test for 
anticipatory effects if we are willing to assume that the controls embedded in 
$\mathbf{X}_{it}$ capture all relevant unobserved heterogeneity arising from 
local business cycles.
While we find the interpretation given in the text more palatable, the data are 
consistent with both.}
% We further present evidence in favor of this assumption by showing that our MW 
% measures do not predict the number of listings of houses for sale in Zillow.%
% \footnote{Ideally, we would run this regression on the number of rental units.
% Unfortunately this information is not available in the Zillow data.
% Specifically, we track the number of houses listed for sale in a sample of ZIP 
% codes during the period 2013-2019 for our preferred house type (SFCC).}

% DGP: Adding back this sentence and footnote above seem like a good idea. Actually,
% Matt Pecenco suggested something like this after the dissertation presentation.

As explained in Section \ref{sec:data_final_panel}, 
the model in \eqref{eq:fd} is estimated using a balanced panel.
We also conduct an estimation exercise using an unbalanced panel with all 
ZIP codes with Zillow rental data in the SFCC category 
from February 2010 to December 2019, controlling for time-period by 
year-quarter-of-entry fixed effects.


%%%%%%%%%%%%%%%%%%%%%%%%%%%%%%%%%%%%%%%%%%%%%%%%%%%%%%%%%%%%%%%%%%%%%%%%%%%%%%%%
\subsection{Interpreting the Identifying Assumptions}

It is not obvious what our identifying assumptions mean for the underlying 
commuting shares and the dynamics of unobserved heterogeneity in trends for the 
most common type of treatment in our data:
an increase in the statutory minimum wage of a city or state that affects 
a subset of the ZIP codes in a metropolitan area.
Following such a policy there will be ZIP codes where the residence MW goes up
and ZIP codes where it does not.
We call ZIP codes in the first group ``directly treated.''
Appendix \ref{sec:did_spillovers_id} shows that, for this policy, 
$\beta$ and $\gamma$ can be computed under the following assumptions: 
1) parallel trends between ZIP codes that are directly treated and ZIP codes 
that are not;
2) the existence of two groups of ZIP codes that are not treated directly and
have differential exposure to the policy via the commuting shares; and 
3) parallel trends between these two groups.

%%%%%%%%%%%%%%%%%%%%%%%%%%%%%%%%%%%%%%%%%%%%%%%%%%%%%%%%%%%%%%%%%%%%%%%%%%%%%%%%
\subsection{Alternative Strategies}\label{sec:alt_emp_strategies}
%% SH: Not sure what's the best name for this section

Recent literature on difference-in-differences methods has shown that usual
estimators do not correspond to the average treatment effect when the treatment 
roll-out is staggered and there is treatment-effect heterogeneity 
\parencite{deChaisemartinEtAl2022,RothEtAl2022}.
One solution to this issue usually relies on carefully defining the control
group of the treatment.
While our setting does not correspond exactly to the models discussed in this
literature, we worry about the validity of our estimator.
In an appendix, we take two steps to ease those concerns:
1) we estimate equation \eqref{eq:fd} allowing the time fixed effects to vary
by different jurisdictions; and
2) we construct a ``stacked'' implementation of our model in which we take
6 months of data around MW changes for ZIP codes in CBSAs where some ZIP codes 
did not receive a direct MW change, and then estimate equation \eqref{eq:fd} on
this restricted sample including event-by-time fixed effects.
These strategies limit the comparisons that identify the coefficients of 
interest to ZIP codes within the same metropolitan area.

Regardless of the estimation strategy, our results still rely on the 
strict exogeneity assumption.
In an appendix we show that we obtain similar results under a model that 
includes the lagged dependent variable as control.
In such a model, $\beta$ and $\gamma$ have a causal interpretation under a 
weaker \textit{sequential exogeneity} assumption
\parencite{ArellanoBond1991, ArellanoHonore2001}.
This alternative assumption requires \eqref{eq:strict_exogeneity} to hold
only for periods $s$ such that $s \leq t$, and thus allows for feedback of 
rent shocks onto MW changes in future periods.
We estimate this model using an IV strategy in which the first lag of the change
in rents is instrumented with the second lag.


%%%%%%%%%%%%%%%%%%%%%%%%%%%%%%%%%%%%%%%%%%%%%%%%%%%%%%%%%%%%%%%%%%%%%%%%%%%%%%%%
\subsection{Sample Selection Concerns and Heterogeneity}\label{sec:emp_start_heterogeneity}

Because our ZIP codes come from a selected sample, they may not represent
the causal effect for the average urban US ZIP code.
To obtain effects that are more representative we follow 
\textcite{Hainmueller2012} and estimate our main models re-weighting 
observations to match key moments of the distribution of characteristics of 
urban ZIP codes.

If the mechanism proposed in Section \ref{sec:model} is correct, then we
expect the effect of the residence MW to be stronger in locations with many 
MW workers.
The reason is that the production of non-tradable goods presumably uses more
low-wage work, and thus the increase in the MW would affect prices more.
Similarly, we expect the effect of the workplace MW to be stronger in locations
with lots of MW workers as income would increase more strongly there.
We then estimate the following model:
\begin{equation}\label{eq:fd_heterogeneity}
    \Delta r_{it} = \delta_t
                  + \tilde\gamma_0 \Delta \mw^{\res}_{it}
                  + \tilde\gamma_1 \iota_i \Delta \mw^{\res}_{it}
                  + \tilde\beta_0 \Delta \mw^{\wkp}_{it}
                  + \tilde\beta_1 \iota_i \Delta \mw^{\wkp}_{it}
                  + \Delta \mathbf{X}^{'}_{it}\eta
                  + \Delta \varepsilon_{it} ,
\end{equation}
where $\iota_i$ represents the standardized share of MW workers in a ZIP code.
Because we cannot estimate the share of MW workers working in a given location,
we interact both the residence and workplace MW with the share of MW residents.%
\footnote{We discuss our estimates of the share of MW workers in 
Section \ref{sec:data_income_housing}.}

We conduct similar heterogeneity exercises with the standardized median household 
income from the ACS \parencite{CensusACS} and the standardized share of public 
housing units in each ZIP code from the HUD \parencite{hudHousing}.
