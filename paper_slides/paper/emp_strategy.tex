%%%%%%%%%%%%%%%%%%%%%%%%%%%%%%%%%%%%%%%%%%%%%%%%%%%%%%%%%%%%%%%%%%%%%%%%%%%%%%%%%
%%%%%                         EMPIRICAL STRATEGY                             %%%%
%%%%%%%%%%%%%%%%%%%%%%%%%%%%%%%%%%%%%%%%%%%%%%%%%%%%%%%%%%%%%%%%%%%%%%%%%%%%%%%%%

In this section, we present the empirical strategy adopted to study the effect 
of the MW on rents and we discuss the assumptions required for identification.
    
%%%%%%%%%%%%%%%%%%%%%%%%%%%%%%%%%%%%%%%%%%%%%%%%%%%%%%%%%%%%%%%%%%%%%%%%%%%%%%%%%
\subsection{First-differenced model}

Consider the following two-way fixed effects model relating rents and the minimum wage:
\begin{equation*} \label{eq:did_lev}
    r_{it} = \alpha_i + \hat{\delta}_t 
           + \gamma \mw^{\res}_{it} + \beta \mw^{\wkp}_{it}
           + \mathbf{X}^{'}_{c(i)t}\eta
           + \varepsilon_{it} \ .
\end{equation*}    
In this equation, 
$r_{it}$ represents the log of rents per square foot, in ZIP code $i$at month $t$,
$\mw^{\res}_{it}$ is the ZIP code's residence MW, defined as $\ln \MW_{it}$,
$\mw^{\wkp}_{it}$ is the ZIP code's workplace MW, defined as $\sum_{z\in\Z(i)} \pi_{iz}\ln \MW_{zt}$,
$\alpha_i$ and $\hat{\delta}_t$ are ZIP code and month fixed effects, 
respectively, and 
$\mathbf{X}_{c(i)t}$ is a vector of county-level time-varying controls.
Time runs from $\underline{T} =$ February 2010 to $\overline{T} = $ December 2019.
The parameters of interest are $\gamma$ and $\beta$, which we interpret as the 
elasticity of rents to the residence and workplace MW, respectively.

By taking first differences on \ref{eq:did_lev} we obtain the following model In
first differences:
\begin{equation}\label{eq:fd}
    \Delta r_{it} = \delta_t 
                  + \gamma \Delta \mw^{\res}_{it} + \beta \Delta \mw^{\wkp}_{it}
                  + \Delta \mathbf{X}^{'}_{c(i)t}\eta
                  + \Delta \varepsilon_{it} \ .
\end{equation}
where $\delta_t = \hat{\delta}_t - \hat{\delta}_{t-1}$.
We spell out the model in first differences because it is reasonable to expect 
unobserved shocks to rental prices to be persistent over time. 
Appendix Table XX, shows evidence of AR(1) auto-correlation in the error term.
% SH: 
%    Add Woldrige cite in the notes of table \textcite[][chapter 10]{wooldridge2010}

% SH:
%     Mention models that only use one MW variable


%%%%%%%%%%%%%%%%%%%%%%%%%%%%%%%%%%%%%%%%%%%%%%%%%%%%%%%%%%%%%%%%%%%%%%%%%%%%%%%%%
\subsection{Identification}

Estimates of the parameters $\gamma$  and $\beta$ can be interpreted causally 
under the assumption of \textit{strict exogeneity}. 
That is, we require the unobserved shocks to rents to be uncorrelated with past 
and present values of minimum wage changes \parencite[][chapter 10]{wooldridge2010}.
Formally, we assume that the set of conditions

\begin{equation*}
	E[\Delta \varepsilon_{ict} \Delta \ln \underline{w}_{ic\tau}  
							| \alpha_i, \delta_t, \Delta \mathbf{X}_{ct}] = 0
	\ \ \ \forall \tau \in \{\underline{T}, \overline{T} \}
\end{equation*}
holds for all periods $t$. This assumption has two important implications. First of all, 
it implies no pre-trends in rents leading up to a minimum wage change (conditional on 
controls). We will test this implication more formally by including leads of the minimum 
wage. Secondly, it rules out feedback effects from current values of rents on our controls, 
i.e., MW changes are assumed not to be influenced by past values of rents. While we think 
this is a reasonable assumption---minimum wages are usually not set by considering their 
effects on the housing market---, we allow this type of feedback effects in the panel 
specifications described in the upcoming subsection. Importantly, our identifying assumption 
allows for arbitrary correlation between ZIP code effects $\alpha_i$ and minimum wages 
---e.g., our empirical strategy is robust to the fact that richer districts tend to vote 
for MW policies.


%within a ZIP code, the change of the logarithm of the minimum wage $\Delta 
%\underline{w}_{it}$ is mean independent of the change in the unobserved shock 
%$\Delta \varepsilon_{it}$ conditional on the controls.

We worry that unobserved shocks, such as local business cycles, may systematically affect 
both rents and minimum wage changes. We deploy two strategies to account for that. First 
of all, we include economic controls from the QCEW. 
These data are aggregated at the county level, and represent a second best given the 
unavailability of controls at the ZIP code level. Specifically, we use average weekly wages, 
employment and establishment count for the sectors ``Professional and business services,'' 
``Information,'' and ``Financial activities.'' We assume that these sectors are not affected 
by the minimum wage. In fact, according to \textcite[][table 5]{MinWorkersReportBLS}, in 2019 
such industries accounted for 3.5, 1, and 1.2 percent of the total number of MW workers, 
respectively.\footnote{In Section \ref{sec:app_econ_control} we show suggestive evidence 
	that they are not affected by the MW by using them as dependent variable in our models.}
We believe that these controls plausibly capture variation arising from unobserved trends 
in local markets. Our second strategy to deal with unobserved heterogeneity is to control for
ZIP code-specific linear and quadratic trends. These models, presented in the Appendix, 
should account for time-varying heterogeneity not captured by our economic controls that 
follows this pattern.

We can test the identification assumption using models that include leads and lags
of the MW variables.

\begin{equation} \label{eq:leads_lags}
	\Delta \ln y_{ict} = \delta_t
						+ \sum_{r=-s}^{s} \beta_r \Delta \ln \underline{w}_{ic,t+r}
						+ \Delta \mathbf{X}^{'}_{ct}\eta
						+ \Delta \varepsilon_{ict} \ ,
\end{equation}
where $s$ is the number of months of a symmetric window around the MW change. Our baseline
specification sets $s = 5$. We show in the appendix that our results are very similar for
$s \in \{3, 6, 9\}$. Results hold for larger values of $s$ as well; however, increasing 
$s$ implies dropping observations at the start and end of the panel, which decreases our 
sample size and makes the results less reliable.

Importantly, this model allows us to test whether $\beta_{-s} = \beta_{-s+1} = ... = \beta_{-1} 
= 0$. Under the assumption that there is no anticipatory effects in the housing market, we 
interpret the absence of pre-trends as evidence against the presence of unobserved economic 
shocks driving our results.\footnote{We can also interpret the absence of pre-trends as a test 
	for anticipatory effects if we are willing to assume that the controls embedded in 
	$\mathbf{X}^{'}_{ct}$ capture all relevant unobserved heterogeneity arising from local 
	business cycles. While we find the interpretation given in the text more palatable, the 
	data does is consistent with both.}
We think that, given the high frequency of our data and that we focus on short windows around MW 
changes, the assumption of no anticipatory effects is reasonable. We further present evidence in 
favor of this assumption by showing that MW changes do not predict the number of listings of 
houses for sale in Zillow.\footnote{Ideally, we would run this regression on the number of rental 
	units. Unfortunately, as described in Section \ref{sec:data}, this information is not available.}
Specifically, we can track the number of houses listed for sale in selected ZIP codes during the 
period 2013-2019 for our preferred house type (SFCC). We use such series to run placebo 
regressions where we estimate both the static and dynamic models using the change in (log) 
listings as dependent variable. %Significant effects of MW changes, or pronounced pre-trends will 
%indicate that policy changes actually affect the Zillow inventory composition and cast doubt on
%the identifying assumption. We fail to find that.

Appendix Table XX shows that we obtain similar results under a weaker 
\textit{sequential exogeneity} assumption \parencite{ArellanoHonore2001}.
%% SH: 
%%    Discuss estimation procedure in notes of figure


%%%%%%%%%%%%%%%%%%%%%%%%%%%%%%%%%%%%%%%%%%%%%%%%%%%%%%%%%%%%%%%%%%%%%%%%%%%%%%%%%
\subsection{Estimation concerns}

Appendix \ref{sec:did_spillovers_id} shows that under parallel trends assumptions
the model in equation \ref{eq:fd} identifies parameters $\gamma$ and $\beta$.

Inspired by this approach, we use a stacked sample in Appendix Table XX.

