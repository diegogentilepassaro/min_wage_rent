%%%%%%%%%%%%%%%%%%%%%%%%%%%%%%%%%%%%%%%%%%%%%%%%%%%%%%%%%%%%%%%%%%%%%%%%%%%%%%%%%
%%%%%                         EMPIRICAL STRATEGY                             %%%%
%%%%%%%%%%%%%%%%%%%%%%%%%%%%%%%%%%%%%%%%%%%%%%%%%%%%%%%%%%%%%%%%%%%%%%%%%%%%%%%%%

In this section, we present the empirical strategy adopted to study the effect of the 
MW on rents and we discuss the assumptions required for identification. We begin with 
a \textit{static} difference-in-differences specification that imposes no dynamics in 
the effects. To ease concerns of contemporaneous shocks systematically affecting both 
changes in rents and the MW within a ZIP code, we directly control for several 
county-level time-varying proxies of the health of the local labor market.

This static model has several shortcomings that motivate the inclusion of leads and 
lags of MW changes. The \textit{dynamic} model both allows the effect to persist for 
more than one period and permits a test of the underlying parallel-trends assumption. 
One may also worry that the dependent variable presents auto-correlation, which would 
imply bias in our estimates. To account for this possibility we present a 
panel-specification that includes the lagged dependent variable as control following 
\textcite{ArellanoBond1991} and related literature. We test the robustness of our 
results by adding ZIP code level linear and quadratic trends as alternative controls 
of local dynamics in the housing market. 

Our specifications are distinct from classic event-study models used commonly in the 
literature \parencite[discussed in, e.g.,][]{BorusyakJaravel2017, abraham2018}. Rather, 
they share features with empirical work estimating the impact of high-frequency and 
changing-intensity events \parencite[see, e.g.,][]{Fuest2018, Suarez2016}.
First, our models allow for units treated more than once. Secondly, our models allow 
for the inclusion of never-treated units, which aid in the identification of fixed 
effects and other parameters, diminishing concerns of under-identification 
\parencite{BorusyakJaravel2017}. Finally, our specifications identify the treatment 
effect of minimum wages on median rents exploiting not only the timing of a MW change 
but also its intensity.

Importantly, all regressions cluster standard errors at the state level so as to match 
the main source of variation of MW changes.

    
%%%%%%%%%%%%%%%%%%%%%%%%%%%%%%%%%%%%%%%%%%%%%%%%%%%%%%%%%%%%%%%%%%%%%%%%%%%%%%%%%
\subsection{Static Model}
Consider the following two-way fixed effects model relating rents and the minimum wage:

\begin{equation*} \label{eq:did_lev}
    \ln y_{ict} = \alpha_i + \hat{\delta}_t 
    			+ \beta \ln \underline{w}_{ict}
    			+ \mathbf{X}^{'}_{ct}\eta
    			+ \varepsilon_{it} \ .
\end{equation*}    
In this equation, $y_{ict}$ represents our main outcome variable, rents per square foot 
for the Zillow SFCC series, in ZIP code $i$ county $c$ month $t$, $\underline{w}$ is the 
statutory minimum wage, $\alpha_i$ and $\hat{\delta}_t$ are ZIP code and time period 
(month) fixed effects, respectively, and $\mathbf{X}_{ct}$ is a vector of county-level 
time-varying controls. Time runs from $\underline{T} = \text{February 2010}$ to 
$\overline{T} = \text{December 2019}$. By taking first differences on this expression 
we obtain what we label as our \textit{static} model, which is given by

\begin{equation}\label{eq:did}
	\Delta \ln y_{ict} = \delta_t
						+ \beta \Delta \ln \underline{w}_{ict}
						+ \Delta \mathbf{X}^{'}_{ct} \eta
						+ \Delta \varepsilon_{ict} \ ,
\end{equation}
where $\delta_t = \hat{\delta}_t - \hat{\delta}_{t-1}$, and $\beta$ can be interpreted 
as the elasticity of rents to the MW. We spell out the model in first differences because 
it is reasonable to expect unobserved shocks to rental prices to be persistent over 
time. Both the first differences and the level models are consistent under similar 
assumptions, but the model in first differences is more efficient if shocks are serially 
correlated. In Appendix \autoref{tab:level_auto}, we test for AR(1) auto-correlation in 
the shocks following \textcite[][chapter 10]{wooldridge2010} and we comfortably reject 
the no auto-correlation hypothesis. 

Estimates of the parameter $\beta$ can be interpreted causally under the assumption of
\textit{strict exogeneity}. That is, we require the unobserved shocks to rents to be 
uncorrelated with past and present values of minimum wage changes 
\parencite[][chapter 10]{wooldridge2010}. Formally, we assume that the set of conditions

\begin{equation*}
	E[\Delta \varepsilon_{ict} \Delta \ln \underline{w}_{ic\tau}  
							| \alpha_i, \delta_t, \Delta \mathbf{X}_{ct}] = 0
	\ \ \ \forall \tau \in \{\underline{T}, \overline{T} \}
\end{equation*}
holds for all periods $t$. This assumption has two important implications. First of all, 
it implies no pre-trends in rents leading up to a minimum wage change (conditional on 
controls). We will test this implication more formally by including leads of the minimum 
wage. Secondly, it rules out feedback effects from current values of rents on our controls, 
i.e., MW changes are assumed not to be influenced by past values of rents. While we think 
this is a reasonable assumption---minimum wages are usually not set by considering their 
effects on the housing market---, we allow this type of feedback effects in the panel 
specifications described in the upcoming subsection. Importantly, our identifying assumption 
allows for arbitrary correlation between ZIP code effects $\alpha_i$ and minimum wages 
---e.g., our empirical strategy is robust to the fact that richer districts tend to vote 
for MW policies.

%within a ZIP code, the change of the logarithm of the minimum wage $\Delta 
%\underline{w}_{it}$ is mean independent of the change in the unobserved shock 
%$\Delta \varepsilon_{it}$ conditional on the controls.

We worry that unobserved shocks, such as local business cycles, may systematically affect 
both rents and minimum wage changes. We deploy two strategies to account for that. First 
of all, we include economic controls from the QCEW. 
These data are aggregated at the county level, and represent a second best given the 
unavailability of controls at the ZIP code level. Specifically, we use average weekly wages, 
employment and establishment count for the sectors ``Professional and business services,'' 
``Information,'' and ``Financial activities.'' We assume that these sectors are not affected 
by the minimum wage. In fact, according to \textcite[][table 5]{MinWorkersReportBLS}, in 2019 
such industries accounted for 3.5, 1, and 1.2 percent of the total number of MW workers, 
respectively.\footnote{In \autoref{sec:app_econ_control} we show suggestive evidence 
	that they are not affected by the MW by using them as dependent variable in our models.}
We believe that these controls plausibly capture variation arising from unobserved trends 
in local markets. Our second strategy to deal with unobserved heterogeneity is to control for
ZIP code-specific linear and quadratic trends. These models, presented in the Appendix, 
should account for time-varying heterogeneity not captured by our economic controls that 
follows this pattern.

%%%%%%%%%%%%%%%%%%%%%%%%%%%%%%%%%%%%%%%%%%%%%%%%%%%%%%%%%%%%%%%%%%%%%%%%%%%%%%%%%
\subsection{Dynamic Models}

One concern with the static model is that, despite controlling for economic factors or ZIP 
code-specific trends, preexisting time paths of rents per square foot in anticipation of MW 
changes may be different in treated and non-treated ZIP codes due to either residual demand 
shocks or anticipatory effects in the supply of rentals. In order to assess whether pre-trends 
are present, we extend our static model with leads and lags of our minimum wage variable. 
In addition, one may believe that the effect of MW changes on rents is not a one-time discrete 
level jump but that it has persistence. In such cases the estimated coefficient $\beta$ from 
\autoref{eq:did} might only have limited relevance for policy evaluation purposes 
\parencite{callaway2019}. Extending the model with leads and lags of the MW allows us to 
explore this possibility as well. The \textit{dynamic} difference-in-differences model is

\begin{equation} \label{eq:leads_lags}
	\Delta \ln y_{ict} = \delta_t
						+ \sum_{r=-s}^{s} \beta_r \Delta \ln \underline{w}_{ic,t+r}
						+ \Delta \mathbf{X}^{'}_{ct}\eta
						+ \Delta \varepsilon_{ict} \ ,
\end{equation}
where $s$ is the number of months of a symmetric window around the MW change. Our baseline
specification sets $s = 5$. We show in the appendix that our results are very similar for
$s \in \{3, 6, 9\}$. Results hold for larger values of $s$ as well; however, increasing 
$s$ implies dropping observations at the start and end of the panel, which decreases our 
sample size and makes the results less reliable.

Importantly, this model allows us to test whether $\beta_{-s} = \beta_{-s+1} = ... = \beta_{-1} 
= 0$. Under the assumption that there is no anticipatory effects in the housing market, we 
interpret the absence of pre-trends as evidence against the presence of unobserved economic 
shocks driving our results.\footnote{We can also interpret the absence of pre-trends as a test 
	for anticipatory effects if we are willing to assume that the controls embedded in 
	$\mathbf{X}^{'}_{ct}$ capture all relevant unobserved heterogeneity arising from local 
	business cycles. While we find the interpretation given in the text more palatable, the 
	data does is consistent with both.}
We think that, given the high frequency of our data and that we focus on short windows around MW 
changes, the assumption of no anticipatory effects is reasonable. We further present evidence in 
favor of this assumption by showing that MW changes do not predict the number of listings of 
houses for sale in Zillow.\footnote{Ideally, we would run this regression on the number of rental 
	units. Unfortunately, as described in \autoref{sec:data}, this information is not available.}
Specifically, we can track the number of houses listed for sale in selected ZIP codes during the 
period 2013-2019 for our preferred house type (SFCC). We use such series to run placebo 
regressions where we estimate both the static and dynamic models using the change in (log) 
listings as dependent variable. %Significant effects of MW changes, or pronounced pre-trends will 
%indicate that policy changes actually affect the Zillow inventory composition and cast doubt on
%the identifying assumption. We fail to find that.

This model allows us to estimate the dynamics of the logarithm of the rent per square foot around 
changes in the MW. We define the cumulative elasticity of rents to the MW as $\sum_{r=0}^s \beta_r$. 
We intend to obtain reliable estimates of this statistic. However, because each coefficient is 
estimated with noise, the cumulative sum is likely to show large confidence intervals. Imposing 
the assumption of no pre-trends, we can gain efficiency by estimating a model with distributed 
lags only. We present estimates of dynamic models for both the individual coefficients and the 
cumulative sum in the results section.

% SH: Following Jesse's suggestion I comment out the below. I think that, once we run the model
%     at different levels of aggregation, we should write a brief appendix and point the reader 
%     to it for more discussion on why our results differ

%In past settings using yearly data \parencite{Tidemann2018, Yamagishi2019}, MW changes are so 
%common in a given geographic area relative to the timespan of the data that it is very hard to  
%credibly estimate the lags. Intuitively, this is the case because it is hard to distinguish which  
%variation of the rental price is due to the current MW change or to a preceding one. In our  
%estimates that concern is not justified, as given that we have month to month variation, we use 
%short windows (5 months) in which there is no overlap in MW changes within a ZIP code. 

Finally, we consider a dynamic panel specification that includes the lagged dependent variable 
as a control. Such a model has two important advantages. First and most important, it accounts 
for the possibility of structural auto-correlation in the rents variable that would bias our results. This 
is in fact a possibility, given that some listings likely remain in the Zillow data for more 
than one month. Secondly, it allows us to relax the strict exogeneity assumption 
\parencite{ArellanoHonore2001}. The less stringent \textit{sequential exogeneity} assumption 
required for identification allows for feedback effects of minimum wages on rents.\footnote{More 
	precisely, the sequential exogeneity assumption states that, conditional on controls and 
	fixed 	effects, unobserved shocks to rents are uncorrelated with \textit{past} MW changes.}
The model with can be written as

\begin{equation}\label{eq:ab_panel}
	\Delta \ln y_{ict} = \delta_t
						+ \sum_{r=0}^{s} \beta_r \Delta \ln \underline{w}_{ic,t+r}
						+ \gamma \Delta \ln y_{ic,t-1} + \Delta \mathbf{X}^{'}_{ct}\eta
						+ \Delta \varepsilon_{ict} \ .
\end{equation}
The model can also be estimated using pre-trends, which we show to be zero.

We are also interested in the parameter $\gamma$ since, per the properties of time series, the 
long run effect of minimum wages on rents is given by $(\beta_0 + \beta_1 + \dots + 
\beta_s)/(1-\gamma)$. Estimation is not that straightforward, however, because by construction 
we now have that $\Delta y_{ic,t-1}$ is correlated with the error term. To address this issue,
we follow the literature and instrument $\Delta y_{ic,t-1}$ with $\Delta y_{ic,t-2}$ 
\parencite{ArellanoHonore2001}. To obtain a more precise estimate of the long run effect, we 
shut down some lagged coefficients by imposing a short window $s$ in the model.

%%%%%%%%%%%%%%%%%%%%%%%%%%%%%%%%%%%%%%%%%%%%%%%%%%%%%%%%%%%%%%%%%%%%%%%%%%%%%%%%%
\subsection{Heterogeneity by ZIP code Characteristics}\label{sec:strategy_heterogeneity}

Consider some demographic characteristics of ZIP codes (say, the share of college graduates).
We extend the baseline panel static model defined in \autoref{eq:did} by interacting the local 
MW change variable with dummy variables that indicate whether the ZIP code belongs to each 
quartile of the distribution of such characteristic. This specification allows us to test for 
heterogeneous effects of MW changes based on ZIP code characteristics, therefore assessing 
whether our effects are driven by ZIP codes expected to have more MW workers. Formally, the 
model is

\begin{equation}\label{eq:diff_main_hetero} 
    \Delta \ln y_{ict} = \theta_t
    				+ \sum_{q = 1}^4 \beta_q \mathds{1}\{i \in q\} \Delta \ln \underline{w}_{ict}
    				+ \Delta \mathbf{X}^{'}_{ct}\eta
    				+ \Delta \varepsilon_{ict} \ ,
\end{equation}
where $q$ identifies quartiles of some ZIP code level characteristic, and $\mathds{1}\{\cdot\}$ 
is the indicator function.

To minimize any type of concern about ZIP code characteristics being endogenous to MW changes, 
we use use socio-demographic data from the 2010 Census and the 5-years 2008-2012 ACS that 
predate our panel.


%%%%%%%%%%%%%%%%%%%%%%%%%%%%%%%%%%%%%%%%%%%%%%%%%%%%%%%%%%%%%%%%%%%%%%%%%%%%%%%%%
\subsection{Experienced Minimum Wage}\label{sec:emp_strategy_expmw}

We run versions of the static and dynamic models both using the experienced MW instead 
of the statutory one and including both measures together. These models are valuable for two main 
reasons. First, as long as MW workers commute across ZIP codes, their income will be determined 
by the MW level set at their workplace location, not by the one in place at their residential 
address. As a consequence, ZIP codes with no changes in the statutory MW may nevertheless record 
changes in their residents' income, whereas ZIP codes with statutory changes may experience minimal 
variation in residents' income. By a basic supply-and-demand logic, we expect rents to increase in 
locations where residents have more income due to the MW changes. In this light, classical 
measurement error arguments will imply that, using the statutory MW only, the effect of interest 
will be downward biased \parencite{AngristPischke2009}. Our results support that view. 

Secondly, MW changes may be thought of as income transfers across ZIP codes. The experienced MW 
better captures variation relevant for those ZIP codes where, as a result of the policy, residents 
earn more money. By directly controlling for both MW measures in the baseline static model, we argue 
that the residual variation of the statutory MW will be related to places where residents see a 
negative income shock following the MW increase, since these are the ZIP codes that ``pay'' for the 
transfer to ZIP codes where low-income workers reside. As a result we would expect to see a stronger 
positive effect for the experienced MW (i.e., ZIP codes that experience a positive income shock), 
and a negative effect for the actual MW (i.e., ZIP codes that experience a negative shock).

Identification now requires the timing of within-ZIP code experienced MW changes to be orthogonal 
with respect to dynamics of rent unobservables. Given that the experienced MW is constructed as a 
weighted average of the statutory MWs of close-by zipcodes, this restriction is similar to imposing 
unobserved determinants of rents in one zipcode to be orthogonal to MW changes in other zipcodes 
where its residents work.\footnote{They are not \textit{exactly} equal because we use the natural 
	logarithm of the experienced MW in our model.}
This restriction seems unquestionable for state-wide changes, since they tend to affect near-by ZIP 
codes in the same fashion, and thus are unlikely to be correlated with specific ZIP codes' unobservables. 
For local changes the assumption could be questioned if local policy-makers set MW changes at 
the same time as rents in places where MW workers reside are rising. Considering the hard-to-predict 
commuting patterns across zipcodes and the fact that MW changes tend to be set without consideration 
of housing market dynamics in general, we think this identifying assumption is likely to hold. On the 
other hand, because the weights are fixed over time we are not worried about changes in commuting 
patterns that hamper identification. However, if those weights are inaccurately measured or change 
over time significantly following MW changes our estimations will be biased. For example, if MW workers 
move to places with low rents that are close to cities with regular MW changes the effect of 
experienced MW on rents will be understated by our model.

