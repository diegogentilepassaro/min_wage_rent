%%%%%%%%%%%%%%%%%%%%%%%%%%%%%%%%%%%%%%%%%%%%%%%%%%%%%%%%%%%%%%%%%%%%%%%%%%%%%%%%
%%%%%                          CONTEXT AND DATA                             %%%%
%%%%%%%%%%%%%%%%%%%%%%%%%%%%%%%%%%%%%%%%%%%%%%%%%%%%%%%%%%%%%%%%%%%%%%%%%%%%%%%%

We begin the section by describing the construction of a ZIP code by month panel
of MW levels in the US.
We use our panel to describe trends in MW policies in the 2010s.
Later, we discuss the relationship between income and housing consumption at the
household level.
The data suggests that rents are likely to respond to MW changes.
We also explore how housing expenditure varies across ZIP codes.
Finally, we document the construction of our analysis sample and discuss
its strengths and limitations.

%%%%%%%%%%%%%%%%%%%%%%%%%%%%%%%%%%%%%%%%%%%%%%%%%%%%%%%%%%%%%%%%%%%%%%%%%%%%%%%%
\subsection{Minimum Wage Policies in the 2010s}
\label{sec:data_mw_panel}

We collected data on federal-, state-, county-, and city-level statutory MW 
levels from \textcite{VaghulZipperer2016}.
We extended their data, available up to 2016, using data from 
\textcite{BerkeleyLaborCenter} and from official government offices for the 
years 2016--2020.%
\footnote{Some states and cities issue different MW levels for small businesses
    (usually identified by having less than 25 employees).
    In these cases, we select the general MW level as the prevalent one.
    In addition, there may be different (lower) MW levels for tipped employees.
    We do not account for them because employers are typically required to make 
    up for the difference between the tipped MW plus tips and the actual MW.}
%
% Backing up claim on tipped MW: https://www.dol.gov/general/topic/wages/wagestips
%
Most ZIP codes are contained within a jurisdiction, and for them the statutory 
MW is simply the maximum of the federal, state, and local levels.
Some ZIP codes cross jurisdictions, and so are bound by multiple statutory MW 
levels.
In these cases we assign a weighted average of the statutory MW levels in its
constituent census blocks, exploiting an original correspondence table between 
blocks and ZIP codes detailed in Online Appendix \ref{sec:blocks_to_uspszip}, 
where weights correspond to the number of housing units.
The result is a ZIP code-month panel of statutory MW levels in the US between
January 2010 and June 2020.
More details on the construction of the panel can be found in Online Appendix 
\ref{sec:assigning_mw_levels}.

Online Appendix Figure \ref{fig:mw_policies} shows the different levels of 
binding MW policies over time in our data.
Panel A focuses on state-level MW policies.
There are $\stateBindingMW$ states with MW policies in 2010--2020, all of 
which started prior to January 2010.
Panel B shows sub-state MW policies.
In total, there are $\localBindingMW$ counties and cities with some binding MW
policy in this period.
The number of new local jurisdictions instituting a MW policy increases strongly 
after 2013 and declines after 2018.
Overall, we observe strong variations in MW levels across jurisdictions.

Figure \ref{fig:map_mw_perc_changes} maps the percentage change in 
the statutory MW level from January 2010 to June 2020 in each ZIP code.
We observe a great deal of spatial heterogeneity in MW levels within the US.
Importantly, many metropolitan areas across and within state borders have 
differential MW changes, which will be central to distinguishing the effect 
of the two MW-based measures proposed in Section \ref{sec:model}.
We describe the construction of these measures later in this section.

%%%%%%%%%%%%%%%%%%%%%%%%%%%%%%%%%%%%%%%%%%%%%%%%%%%%%%%%%%%%%%%%%%%%%%%%%%%%%%%%
\subsection{Households, Income, and Housing}
\label{sec:data_income_housing}

We compare individuals and households within metropolitan areas using data 
from the 2011 and 2013 waves of the American Housing Survey \parencite{ahs2020}.
Figure \ref{fig:ahs_pr_renters} shows that low-income households are much
more likely to rent.
While only $\TopDecPrRent$ percent of households in the top income quintile 
are renters, around $\BottomDecPrRent$ percent of them are when focusing on 
the bottom one.
Online Appendix Figure \ref{fig:ahs_hhead} shows that, while low income 
individuals are less likely to be household heads, many of them are.
The average probability for the bottom three income deciles is 50 percent.
Online Appendix Figure \ref{fig:ahs_rent_sqft} shows that, among households 
that rent, rents per square foot are surprisingly constant across household 
income levels.
These figures suggest that the MW is likely to affect household income, at 
least for lower income households, and that rents per square foot can plausibly
respond to MW changes.
Online Appendix Figure \ref{fig:ahs_unit_types} shows the type of building 
households live in by household income decile.
Low-income households are more likely to live in buildings with more units,
though they are spread across all building types.

%%
%% Things that we may also say:
%%    - Only one in ten MW workers is a teenager \parencite{Manning2021}.
%%    - 40th of individuals in bottom quintile of individual income are 
%%      household heads
%%

We explore variations over space in housing expenditure.
To do so, we collected Individual Income Tax Statistics aggregated at the 
ZIP code level from the IRS \parencite{IRS},
%\footnote{For each ZIP code and year we observe the number of households, 
%    population, adjusted gross income, total wage bill, total business income, 
%    number of households that receive a wage, number of households that have 
%    business income, and the number of households with farm income.}
and Small Area Fair Market Rents (SAFMRs hereafter) data from the HUD 
\parencite{hudSAFMR}.%
\footnote{SAFMRs data are constructed by the HUD as an extension of the
    Fair Market Rents (FMRs) data using, for each year, ZIP code-level information
    from previous years' American Community Survey 
    \parencite[][, p.\ 35]{SafmrReport2018}.
    SAFMRs are an estimate of the 40th percentile of the rents distribution
    based on constant housing quality \parencite[][, p.\ 1]{SafmrReport2018}.
    The FMRs data, available at the county and year levels, have been used to study 
    the effect of the MW on rents in the US \parencite{Tidemann2018, Yamagishi2019}.}
For each ZIP code in 2018, we constructed a housing expenditure share dividing 
the average monthly wage per household from the IRS by the 2 bedroom SAFMR 
rental value from the HUD.%
\footnote{We impute a small share of missing values using a regression model 
	where the ZIP code-level covariates include data from LODES and the US 
	Census.
	See Online Appendix \ref{sec:measure_housing_expenditure} for details.}%
\textsuperscript{,}%
\footnote{This computation will be a good approximation for the housing 
	expenditure share insofar total housing expenditure and total wage income 
	are roughly proportional to their averages under the same constant of 
	proportionality.
	This computation also assumes away differences in the number of bedrooms 
	across ZIP codes.}
Online Appendix Figure \ref{fig:map_hous_exp_chicago} maps our estimates for the 
Chicago CBSA.
There is considerable variation in housing expenditure over space, with poorer
areas generally spending a higher share of their income on housing.

To get a sense of the spatial distribution of minimum wage earners we construct 
a proxy variable using the number of workers across income bins in the 5-year 
2010-2014 American Community Survey \parencite[ACS;][]{CensusACS}.
See details in Online Appendix \ref{sec:assigning_mw_levels}.
Our variable for the share of MW workers is negatively correlated with median 
household income from the ACS (corr.\ $=\corrShWkrMedInc$) and 
positively correlated with our estimate of the housing expenditure share 
(corr.\ $=\corrShWkrS$).
This latter correlation also suggests that the MW is likely to affect rents.

%%%%%%%%%%%%%%%%%%%%%%%%%%%%%%%%%%%%%%%%%%%%%%%%%%%%%%%%%%%%%%%%%%%%%%%%%%%%%%%%
\subsection{Estimation Data and Samples}

\subsubsection{Rents Data}
\label{sec:data_rents}

Zillow is the leading online real estate platform in the US, hosting more than 
170 million unique monthly users in 2019 \parencite{ZillowFacts}.
Zillow provides the median rental and sales price among units listed on the 
platform for different house types and at different geographic and time 
aggregation levels \parencite{ZillowData}.%
\footnote{As of the release of this article, the data are no longer available 
    for download.
    See \textcite{ZillowDataArchive} for a snapshot of the website as of 
    February 2020, the last month the data were available.}
We collected the ZIP code level data, available from February 2010 
to December 2019.
There is variation in the entry of a ZIP code to the data, and locations with a 
small number of listings are omitted. %
%\footnote{Two related notes:
%    (i) once a ZIP code enters the Zillow data it never drops out, and
%    (ii) the threshold used by Zillow to censor the data is not made public.}

Our main analyses use the median rental price per square foot among housing 
units listed in the category Single Family houses, Condominium and Cooperative 
units (SFCC).
This is the most populated time series, as it includes the most common US 
rental house types \parencite{Fernald2020}.
We focus on rents \textit{per square foot} to account for systematic 
differences in housing size.
Online Appendix Figure \ref{fig:trend_zillow_safmr} shows that this series 
follows a similar trend over time when compared to SAFMR.
It is important to note that these data reflect rents of newly available 
housing units which should quickly incorporate new information into prices 
\parencite{AmbroseEtAl2015}.
As a result, we expect them to react quickly to economic shocks, such as 
changes in MW policies.
On the other hand, rents among the universe of leased units should react 
more slowly, as they are only updated when the lease is renewed.
This is the pattern of results in \textcite{AgarwalEtAl2021} who uses
data from all contract rents.

The Zillow data have several limitations.
First, Zillow's market penetration dictates the sample of ZIP codes available.
Online Appendix Figure \ref{fig:map_zillow_sample} shows that the sample of ZIP 
codes with SFCC rents data coincides with areas of high population density.
Second, we only observe the median rental value.
No data on the distribution of rents, nor the number of units listed for rent, 
are available.
Finally, we observe posted rents rather than contract rents.
We did not find any information on how correlated posted rents and contract 
rents are, so we decided to ask this as a question in the online platform Quora.
Online Appendix \ref{sec:posted_rents} shows a selection of quotes from the 
replies, which overwhelmingly suggest that contract rents generally do not 
differ from posted rents.
Some answers also suggest that rents of long-tenured landlords may not
reflect current market conditions.

\subsubsection{The residence and workplace minimum wage measures}
\label{sec:data_mw_measures}

Using the panel described in Section \ref{sec:data_mw_panel} at hand, computing 
the residence MW is straightforward.
We define it as $\mw^{\res}_{it} = \ln \MW_{it}$.

To construct the workplace MW we need commuting data, which we obtain from the 
Longitudinal Employer-Household Dynamics Origin-Destination Employment Statistics 
\parencite[LODES;][]{CensusLODES} for the years 2009 through 2018.
We collected the datasets for ``All Jobs.''
The raw data are aggregated at the census block level. 
We further aggregate it to ZIP codes using the original correspondence between 
census blocks and USPS ZIP codes described in Online Appendix 
\ref{sec:blocks_to_uspszip}.
This results in residence-workplace matrices that, for each ZIP code and year, 
indicate the number of jobs of residents in every other ZIP code.

We use the 2017 residence-workplace matrix to build exposure weights.
Let $\Z(i)$ be the set of ZIP codes in which $i$'s residents work 
(including $i$).
We construct the set of weights $\{\pi_{iz}\}_{z\in\Z(i)}$ as 
$ \pi_{iz} = N_{iz}/{N_i} , $
where 
$N_{iz}$ is the number of jobs with residence in $i$ and workplace in $z$, 
and $N_i$ is the total number of jobs originating in $i$.
The workplace MW measure is defined as
\begin{equation*}\label{eq:mw_wkp_def}
    \mw^{\wkp}_{it} = \sum_{z\in\Z(i)} \pi_{iz} \ln \MW_{zt} \ .
\end{equation*}
The workplace MW has a shift-share structure.
Our strategy, which exploits differential exposure to common shocks for 
identification, is most related to the recent work in this area by 
\textcite{GoldsmithpinkhamEtAl2020}.

While our baseline uses commuting shares from 2017,
for robustness we present estimates in which the workplace MW measure
is constructed using alternative set of weights.
In particular, we use different years and alternative job categories,
such as jobs for young or low-income workers.%
\footnote{The LODES data reports origin-destination matrices for the number of 
    workers 29 years old and younger, and the number of workers earning less 
    than \$1,251 per month.
The resulting workplace MW measures with any set of weights are highly correlated 
among each other (corr.\ $>0.99$ for every pair).}
%%
%% MG: Documented in descriptive/events_count.
%%

Figure \ref{fig:map_mw_chicago_jul2019}, already discussed in the introduction,
illustrates the difference in the MW-based measures mapping their change in the 
Chicago CBSA on July 2019.
For completeness, Online Appendix Figure \ref{fig:map_rents_chicago_jul2019} 
shows the changes in our main median rents variable around the same date.


%%%%%%%%%%%%%%%%%%%%%%%%%%%%%%%%%%%%%%%%%%%%%%%%%%%%%%%%%%%%%%%%%%%%%%%%%%%%%%%%
\subsubsection{Other data sources}\label{sec:data_other}

\paragraph{Time-varying data.}

To proxy for local economic activity we collect data from the 
Quarterly Census of Employment and Wages \parencite[QCEW;][]{QCEW} 
at the county-quarter and county-month levels for several industrial divisions 
and from 2010 to 2019.%
\footnote{The QCEW covers the following industrial aggregates: 
    ``Natural resources and mining,'' ``Construction,'' ``Manufacturing,'' 
    ``Trade, transportation, and utilities,'' ``Information,'' 
    ``Financial activities'' (including insurance and real state), 
    ``Professional and business services,'' ``Education and health services,'' 
    ``Leisure and hospitality,'' ``Other services,'' ``Public Administration,''
    and ``Unclassified.''
    We observe, for each county-quarter-industry cell, the number of 
    establishments and the average weekly wage, and for each 
    county-month-industry cell, the level of employment.}
We use these data as controls for the state of the local economy in our 
regression models.

\paragraph{ZIP code characteristics.}

While our MW assignment recognizes that ZIP codes cross census geographies, 
we assign to each ZIP code a unique geography based on where the largest 
share of its houses fall.
We do this for descriptive purposes and also to use geographic indicators  
in our estimates.
Additionally, we collect ZIP code demographics from the ACS 
\parencite{CensusACS} and the 2010 US Census \parencite{CensusDecennial}.
We collect these data at the block or tract levels, and assign them to ZIP codes
using the correspondence table described in Online Appendix 
\ref{sec:blocks_to_uspszip}.

%%%%%%%%%%%%%%%%%%%%%%%%%%%%%%%%%%%%%%%%%%%%%%%%%%%%%%%%%%%%%%%%%%%%%%%%%%%%%%%%
\subsubsection{Estimation samples}\label{sec:data_final_panel}

We put together an unbalanced panel of ZIP codes available in Zillow in the 
SFCC category at the monthly date level from February 2010 to December 2019.
This panel contains $\ZIPMWeventsUnbal$ MW changes at the ZIP code level, 
which arise from $\StateMWeventsUnbal$ state and $\CityCountyMWeventsUnbal$ 
county and city changes.
Online Appendix Figure \ref{fig:mw_changes_dist_zillow} shows the distribution 
of positive MW increases among ZIP codes in the Zillow data.
To prevent our estimates from being affected by changes in sample composition,
we construct a ``baseline'' panel keeping ZIP codes with valid rents data 
starting on January 2015.
The resulting fully-balanced panel contains $\ZIPMWeventsBase$ MW changes at 
the ZIP code level.%
\footnote{To avoid losing observations in models with leads and lags we include 
    six leads and lags of the MW measures, so the dataset actually runs from 
    July 2014 to June 2020.}

Table \ref{tab:stats_zip_samples} compares the sample of ZIP codes in the Zillow
data to the population of ZIP codes along sociodemographic dimensions.
The first and second columns report data for the universe of ZIP codes and 
for the set of urban ZIP codes, respectively.
The third column shows the set of ZIP codes in the Zillow data with any 
non-missing value of rents per square foot in the SFCC category.
Finally, the fourth column shows descriptive statistics for our estimation 
sample, which we call the ``baseline'' sample.
While our baseline sample contains only 11.8 percent of urban ZIP codes, it 
covers 25.0 percent of their population and 25.8 percent of their households.
With respect to demographics, ZIP codes in the baseline sample tend to be 
more populated, richer, with a higher share of Black and Hispanic inhabitants, 
and with a higher share of renter households than both the average ZIP code 
and the average urban ZIP code.
This is so because Zillow is present in large urban regions, but 
it does not usually operate in smaller urban or rural areas.
In an attempt to capture the treatment effect for the average urban ZIP code 
we conduct an exercise where we re-weight our sample to match the average 
of a handful of characteristics of those.

Finally, Online Appendix Table \ref{tab:stats_est_panel} shows statistics 
of our baseline panel.
The distribution of the residence and workplace MW measures is, as expected,
quite similar.
We also show median rents in Zillow in the SFCC category.
The average monthly median rent is \$1,757.9 and \$1.32 per square foot, 
although these variables show a great deal of variation.
Finally, we show average weekly wage, employment, and establishment count 
for the QCEW industries we use as controls in our models.
