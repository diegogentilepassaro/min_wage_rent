%%%%%%%%%%%%%%%%%%%%%%%%%%%%%%%%%%%%%%%%%%%%%%%%%%%%%%%%%%%%%%%%%%%%%%%%%%%%%%%%
%%%%%                          CONTEXT AND DATA                             %%%%
%%%%%%%%%%%%%%%%%%%%%%%%%%%%%%%%%%%%%%%%%%%%%%%%%%%%%%%%%%%%%%%%%%%%%%%%%%%%%%%%

We begin the section by describing the construction of a ZIP code by month panel
of MW levels in the US. 
We use our panel to describe trends in MW policies in the 2010s.
Later, we discuss the relationship between income and housing consumption at the
household level.
We also explore how housing expenditure varies across ZIP codes.
Finally, we document the construction of our analysis sample.

%%%%%%%%%%%%%%%%%%%%%%%%%%%%%%%%%%%%%%%%%%%%%%%%%%%%%%%%%%%%%%%%%%%%%%%%%%%%%%%%
\subsection{Minimum Wage Policies in the 2010s}
\label{sec:data_mw_panel}

We collect data on federal-, state-, county-, and city-level statutory MW levels 
from \textcite{VaghulZipperer2016}.
We supplement their data, available up to 2016, with data from 
\textcite{BerkeleyLaborCenter} and from official sub-national government offices 
for the years 2016--2019.%
\footnote{Some states and cities issue different MW levels for small businesses
(usually identified by having less than 25 employees).
In these cases, we select the general MW level as the prevalent one.
In addition, there may be different (lower) MW levels for tipped employees.
We do not account for them because employers are typically required to make up 
for the difference between tipped MW plus tips and actual MW.}
%
% Backing up claim on tipped MW: https://www.dol.gov/general/topic/wages/wagestips
%
Most ZIP codes are contained within a jurisdiction, and for them the statutory 
MW is simply the maximum of the federal, state, and local levels.
Some ZIP codes cross jurisdictions, and so are bound by multiple statutory MW 
levels.
In these cases we assign a weighted average of the statutory MW levels in its
constituent census blocks, exploiting an original correspondence table between 
census blocks and USPS ZIP codes detailed in Appendix 
\ref{sec:blocks_to_uspszip}, where weights correspond to the number of housing
units.
More details on the construction of the ZIP code-level statutory MW panel 
can be found in Appendix \ref{sec:assigning_mw_levels}.

Appendix Figure \ref{fig:mw_policies} shows the different levels of the 
statutory MW over time in our data.
Panel (a) focuses on state level MW policies, whereas
panel (b) shows sub-state MW policies.
In total, there are $\stateBindingMW$ states and $\localBindingMW$ counties and 
cities with a MW policy at any moment in the 2010 decade.
The number of local jurisdictions instituting a MW level increases strongly
after 2015.
%% SH: guessing this fact.

Figure \ref{fig:map_mw_perc_changes} shows the percentage change 
in the statutory MW levels from January 2010 to December 2019 in each ZIP code.
We observe a great deal of spatial heterogeneity in MW levels within the US.
Importantly, there exist many many metropolitan areas across and within state 
borders that have differential MW changes, which will be central to 
distinguishing the effect of the two MW-based measures proposed in 
Section \ref{sec:model}.
We describe the construction of these measures later in this section.

%%%%%%%%%%%%%%%%%%%%%%%%%%%%%%%%%%%%%%%%%%%%%%%%%%%%%%%%%%%%%%%%%%%%%%%%%%%%%%%%
\subsection{Income and Housing}
\label{sec:data_income_housing}
%%
%% SH: Any alternatives for the name of this section?
%%

We explore the joint-distribution of income levels and housing choices using 
American Housing Survey data for 2011 and 2013 (ADD CITE FOR DATA).
Figure \ref{fig:ahs_pr_renters} shows that low-income households are much
more likely to rent.
Appendix Figure \ref{fig:ahs_rent_sqft} shows that rent per square foot is 
surprisingly constant across household income levels.
Appendix Figure \ref{fig:ahs_unit_types} shows the type of building households
live in by household income decile.

We explore variations over space in housing expenditure.
To do so, for the period 2010--2019
we collected Individual Income Tax Statistics aggregated at the ZIP code level 
from the IRS \parencite{IRS},%
\footnote{For each ZIP code and year we observe the number of households, 
population, adjusted gross income, total wage bill, total business income, 
number of households that receive a wage, number of households that have 
business income, and the number of households with farm income.}
and Small Area Fair Market Rents (SAFMRs) data for from the HUD 
\parencite{hudSAFMR}.%
\footnote{SAFMRs data is constructed by the HUD as an extension from 
Fair Market Rents (FMRs) data using, for each year, ZIP code-level information
from previous years' ACS \parencite[][, p. 35]{SafmrReport2018}..
SAFMRs data is an estimate of the 40th percentile of the rents distribution
based on constant housing quality \parencite[][, p. 1]{SafmrReport2018}..
The FMRs, available at the county and year levels, have been used to study the 
effect of the MW on rents \parencite{Tidemann2018, Yamagishi2019}.}
For each ZIP code in 2018, we constructed a housing expenditure share dividing 
the average monthly wage per household from the IRS by the 2 bedroom SAFMR 
rental value from the HUD.%
\footnote{We impute a small share of missing values using a regression model 
where the ZIP code-level covariates include data from LODES and the US Census.
See Appendix \ref{sec:measure_housing_expenditure} for details.}
Appendix Figure \ref{fig:map_hous_exp_chicago} maps our estimates for the 
metropolitan area of Chicago.
There is considerable variation in housing expenditure over space, with poorer
areas generally paying higher rents.
%%
%% SH: Should we do more to show this? Eg., we can correlate our housing expenditure
%%     share with poverty and stuff like that
%% 

%%%%%%%%%%%%%%%%%%%%%%%%%%%%%%%%%%%%%%%%%%%%%%%%%%%%%%%%%%%%%%%%%%%%%%%%%%%%%%%%
\subsection{Estimation Data and Samples}

\subsubsection{Rents Data}
\label{sec:data_rents}

One of the main challenges to estimate the effects of any policy on the rental
housing market is to obtain adequate data.
We leverage data from Zillow at the ZIP code and month levels.
The high frequency and resolution of the Zillow data is an advantage since it 
allows us to explore the effects of MW changes on rents exploiting their precise
timing and geographic scope. 

Zillow is the leading online real estate and rental platform in the US, hosting
more than 110 million homes and 170 million unique monthly users in 2019 
\parencite{ZillowFacts}.
Zillow provides, starting on February 2010, the median rental and sales price 
among homes listed on the platform for different house types and at geographic
and time aggregation levels \parencite{ZillowData}.%
\footnote{The availability of different time series changed over time, so
data used in this paper is not available for download.
See \textcite{ZillowDataArchive} for a snapshot of the website as of 
February 2020.
We downloaded the data on January 2020, a month before Zillow removed it from
its website.} 
We collect the ZIP code level monthly time series from February 2010 to 
December 2019. 
There is variation in the entry of a ZIP code to the data, and units with a small 
number of listings are omitted.%
\footnote{Two related notes:
(i) once a ZIP code enters the Zillow data it never drops out;
(ii) the threshold used by Zillow to censor the data is not made public.}
As we will explain below, we for estimation we will the subset of ZIP codes
with valid rents data as of January 2015.

We focus our primary analysis on a single housing category:
\textit{single-family} houses and \textit{condominium and cooperative} units (SFCC).
This is the series with the largest number of non-missing ZIP codes, as it 
covers the most common US rental house types \parencite{Fernald2020}.
% In fact, roughly a third of the nation's 47.2 million rental units in 2018 fit 
% the category of single-family homes \parencite{Fernald2020}.
We focus on rents \textit{per square foot} to account for systematic differences
in housing size.
In fact, as shown in Appendix Figure \ref{fig:ahs_rent_sqft}, this variable does 
not seem to vary much by income levels.
Our main outcome variable represents the median rental price per square foot in 
the SFCC category among units listed in the platform for a given ZIP code and 
month.
We show results using median rents per square foot in other rental categories 
available in the data as well.

Zillow data has several limitations.
First, Zillow's market penetration dictates the sample of ZIP codes available.
Appendix Figure \ref{fig:map_zillow_sample} shows that the sample of ZIP codes
with valid SFCC rents data typically coincides with areas of high population 
density.
Second, we only observe the median per-square-foot rental value among listings.
We do not observe actual rents paid by tenants in a given period, 
the distribution of rents among listings in the given ZIP code and month, nor
the number of units listed for rent in a given month.

% To ensure that our data captures trends in the overall US rental market in 
% urban areas, we compare Zillow's median rental price in the SFCC category with 
% Small Area Fair Market Rents (hereafter SAFMRs) series for houses with 
% different number of bedrooms.
% Appendix Figure \ref{fig:trend_zillow_safmr} shows that these series evolve
% very similarly over time.

\subsubsection{The residence and workplace minimum wage measures}
\label{sec:data_mw_measures}

In this subsection we define the minimum wage variables we use in our analysis,
which follow the intuition in Proposition \ref{prop:representation}.
With our MW panel at hand, computing the residence MW is straightforward.
We define it as
\begin{equation*}
    \mw^{\res}_{it} = \ln \MW_{it} \ .
\end{equation*}

We also construct the workplace MW, which captures the spillover effects of
statutory MW policies across locations.
To construct this measure we need to know, for each ZIP code, where workers 
residing in that location work.
We obtain this information from the Longitudinal Employer-Household 
Dynamics Origin-Destination Employment Statistics \parencite[LODES;][]{CensusLODES}
for the years 2009 through 2018.
We collected the datasets for ``All Jobs.''
The raw data is originally aggregated at the census block level. 
We further aggregate it to ZIP codes using the original correspondence between 
census blocks and USPS ZIP codes described in Appendix 
\ref{sec:blocks_to_uspszip}.
This results in ZIP code residence-workplace matrices that, for each location 
and year, indicate the number of jobs of residents in every other location.

We then use the 2017 ZIP code residence-workplace matrix to build exposure 
weights.
Let $\Z(i)$ be the set of ZIP codes in which $i$'s residents work 
(including $i$).
We construct the set of weights $\{\omega_{iz}\}_{z\in\Z(i)}$ as 
$ \omega_{iz} = N_{iz}/{N_i} , $
where 
$N_{iz}$ is the number of jobs who reside in $i$ and work in $z$, 
and $N_i$ is the total number of jobs originating in $i$.
The workplace minimum wage measure is defined as
\begin{equation*}\label{eq:mw_wkp_definition}
    \mw^{\wkp}_{it} = \sum_{z\in\Z(i)} \omega_{iz} \ln \MW_{zt} \ .
\end{equation*}
For robustness, we present estimates in which the workplace MW measure
was constructed using alternative set of weights.
We use different years and alternative job categories,
such as jobs for young or low-income workers.%
\footnote{The LODES data reports origin-destination matrices for the 
number of workers 29 years old and younger, and the number of workers 
earning less than \$1,251 per month.
The resulting workplace MW measures with any set of weights are highly correlated 
among each other ($\rho>0.99$ for every pair).}
%%
%% MG: Documented in descriptive/events_count.
%%

Figure \ref{fig:map_mw_chicago_jul2019} illustrates the difference in these 
measures by plotting the change in the residence and workplace MW measures 
in the metropolitan area of Chicago in July 2019.
%%
%% No more details needed here. We already discussed this figure in the intro
For completeness, Appendix Figure \ref{fig:map_rents_chicago_jul2019} shows
the changes in our main median rents variable around the same date.


%%%%%%%%%%%%%%%%%%%%%%%%%%%%%%%%%%%%%%%%%%%%%%%%%%%%%%%%%%%%%%%%%%%%%%%%%%%%%%%%
\subsubsection{Other data sources}\label{sec:data_other}

\paragraph{Time-varying data}

To proxy for local economic activity we collect data from the 
Quarterly Census of Employment and Wages \parencite[QCEW;][]{QCEW} 
at the county-quarter and county-month levels for several industrial divisions 
and from 2010 to 2019.%
\footnote{The QCEW covers the following industrial aggregates: 
``Natural resources and mining,'' ``Construction,'' ``Manufacturing,'' 
``Trade, transportation, and utilities,'' ``Information,'' 
``Financial activities'' (including insurance and real state), 
``Professional and business services,'' ``Education and health services,'' 
``Leisure and hospitality,'' ``Other services,'' ``Public Administration,''
and ``Unclassified.''}
We observe, for each county-quarter-industry cell, the number of establishments 
and the average weekly wage, and 
for each county-month-industry cell, the level of employment.
We use these data as controls for the state of the local economy in our 
regression models.

We complement the LODES origin-destination matrices with block-level aggregates 
on residence and workplace area characteristics for the years 2009 through 2018 
\parencite{CensusLODES}.
These data are also aggregated to the ZIP code level following Appendix 
\ref{sec:blocks_to_uspszip}.
Relative to the origin-destination matrices, these data contain counts of jobs 
broken by more detailed categories, such as NAICS industrial aggregates and 
schooling levels.
We use these data in heterogeneity analysis.

\paragraph{ZIP code characteristics}

While our MW assignment recognizes that many of ZIP codes cross census 
geographies, we assign to each ZIP code a unique geography based on where the 
largest share of its houses fall.
We do this for descriptive purposes and also to use geography indicators 
in our empirical models.

In order to describe our sample of ZIP codes we collect data from 
the 5-year 2007-2011 American Community Survey \parencite[ACS;][]{CensusACS} and
the 2010 US Census \parencite{CensusDecennial}.
We collect these data at the block or tract levels, and assign it to ZIP codes
using the correspondence table described in Appendix \ref{sec:blocks_to_uspszip}.

%%%%%%%%%%%%%%%%%%%%%%%%%%%%%%%%%%%%%%%%%%%%%%%%%%%%%%%%%%%%%%%%%%%%%%%%%%%%%%%%
\subsubsection{Estimation samples}\label{sec:data_final_panel}

We put together an unbalanced panel of ZIP codes available in Zillow in the SFCC 
category at the monthly date level from February 2010 to December 2019.
This panel contains $\ZIPMWeventsUnbal$ MW changes at the ZIP code level, 
which arise from $\StateMWeventsUnbal$ state-level changes and 
$\CityCountyMWeventsUnbal$ county- and city-level changes.
Appendix Figure \ref{fig:mw_changes_dist_zillow} shows the distribution of 
positive increases in our statutory MW variable among all ZIP codes available 
in the Zillow data.
Given that ZIP codes enter the Zillow data progressively over time affecting 
the composition of the sample,
we construct our \textit{baseline estimating panel} keeping ZIP codes that enter 
the Zillow rental data at most in January 2015.
We end up with a fully balanced panel from January 2015 to December 2019
that contains $\ZIPMWeventsBase$ MW changes at the level of the ZIP code.

Table \ref{tab:stats_zip_samples} compares the Zillow sample and our baseline 
panel to the population of ZIP codes along sociodemographic dimensions. 
The first and second columns report data for the universe of ZIP codes and 
for the set of urban ZIP codes, respectively.
The third column shows the set of ZIP codes in the Zillow data, i.e., those 
that have some non-missing value of rents per square foot in the SFCC category 
between February 2010 and December 2019.
Finally, the fourth column shows descriptive statistics for our estimation 
sample, which we refer to as the ``baseline sample.''

While our baseline sample contains only 11.8\% of all urban ZIP codes, it covers
25.0\% of their population and 25.8\% of their households.
With respect to demographic characteristics, ZIP codes in the baseline sample 
tend to be more populated, richer, with a higher share of Black and Hispanic 
inhabitants, and with a higher share of renter households than both 
the average ZIP code and the average urban ZIP code.
This is the case because Zillow is present in almost every large urban market, 
but it does not operate as often in small urban or rural areas.
In an attempt to capture the treatment effect for the average urban ZIP code 
we conduct an estimation exercise where we re-weight our sample to match the 
average of a handful of characteristics of those.

Finally, Appendix Table \ref{tab:stats_est_panel} shows some sample statistics 
of our baseline panel.
As suggested in the table, the distribution of the residence and workplace MW 
measures is similar.
However, we show in the next section that they do show independent variation
in our model.
We also show summary statistics of median rents in several housing categories.
For the SFCC category, the average monthly median rent is \$1,757.9 and \$1.32 
per square foot, although these variables show a great deal of variation.
Finally, we show average weekly wage, employment, and establishment count 
for the QCEW industries we use as controls in some models.

\paragraph{Auxiliary panels}

For some estimations we construct analogous panels where the units of 
observation are the county by month and ZIP code by year.
In the county-by-month panel we define the MW measures in an analogous fashion 
as for ZIP codes, and we use Zillow data that is already aggregated at this 
level.
We also define a county-level baseline sample keeping a fully balanced panel of 
counties with Zillow rental data as of January 2015.
In the ZIP-code-by-year panel we compute the monthly difference in the log rents 
and MW measures and compute their yearly averages.
