%%%%%%%%%%%%%%%%%%%%%%%%%%%%%%%%%%%%%%%%%%%%%%%%%%%%%%%%%%%%%%%%%%%%%%%%%%%%%%%%%
%%%%%                                RESULTS                                 %%%%
%%%%%%%%%%%%%%%%%%%%%%%%%%%%%%%%%%%%%%%%%%%%%%%%%%%%%%%%%%%%%%%%%%%%%%%%%%%%%%%%%

In this section we present our main results.
First, we show our baseline estimates and we discuss how our results change
when using alternative specifications.
Second, we conduct several robustness checks to explore the strength of 
our results.
Third, we show heterogeneity analysis based on the residence location of MW 
workers.
Finally, we summarize our results and compare them with existing literature.

%%%%%%%%%%%%%%%%%%%%%%%%%%%%%%%%%%%%%%%%%%%%%%%%%%%%%%%%%%%%%%%%%%%%%%%%%%%%%%%%
\subsection{Main Results}\label{sec:main_results}

Table \ref{tab:static} shows our estimates using the baseline sample
described in \ref{sec:data_final_panel}.
Column (1) shows the results of a first-differenced regression of the workplace
MW measure on the residence MW measure, economic controls and year-month fixed
effects.
We observe that a 10 percent increase in the residence MW induces an 
8.6 percent increase in the workplace MW.
While the measures are strongly correlated, this model shows that this 
correlation is far from perfect, suggesting that there is independent variation
to estimate the effect of both measures on rents.

Columns (2) through (4) shows estimates of equation \ref{eq:fd}, varying the
set of included MW measures.
Column (2) shows the results of a model that does not include the workplace MW.
In this case we observe that a 10 percent increase in the MW is associated with
a statistically significant 0.27 percent increase in rents.
Column (3) shows the results of a model that does not include the residence MW.
The coefficient of interest seems to increase slightly.
Column (4) estimates the model using both MW measures.
The coefficient on the residence MW has now turned negative and equals $-0.0204$, 
although it is not statistically significant at conventional levels ($t=-1.21$).
The coefficient on the workplace MW has increased in value to $0.0326$ and is 
statistically significant ($t=1.92$).
We reject the hypothesis that these coefficients are equal at the 10\% 
significance level.
Finally, the sum of the coefficients on each MW measure has a magnitude of 
$0.0342$, very similar to the one in column (3), and is highly significant 
($t=2.21$).
Our results imply that a 10-percent increase in both MW measures will increase
rents by $0.34$ percent.
However, if only the residence MW increases then rents are estimated to decline,
and if only the workplace MW goes up then the rents increase would be larger.

The first concern with these results is whether our identifying assumptions 
holds up.

To illustrate the Appendix Figure \ref{fig:map_residuals_chicago_jul2019} shows the residuals of 

\paragraph{Alternative specifications}

Discuss stacked model.

%%%%%%%%%%%%%%%%%%%%%%%%%%%%%%%%%%%%%%%%%%%%%%%%%%%%%%%%%%%%%%%%%%%%%%%%%%%%%%%%
\subsection{Robustness checks}\label{sec:robustness_results}

Robustness table.

\paragraph{Alternative rental categories}

Discuss results with other Zillow variables.

\paragraph{Sample Selection and External Validity}

Discuss unbalanced, reweighted, fully balanced, and so on.

\paragraph{Other geographies and time frames}

County results.

%%%%%%%%%%%%%%%%%%%%%%%%%%%%%%%%%%%%%%%%%%%%%%%%%%%%%%%%%%%%%%%%%%%%%%%%%%%%%%%%
\subsection{The Heterogeneous Impact of MW Changes on Rents}
\label{sec:heterogeneity_results}

Construct measure of MW workers residing in each zip code using ACS 2015?

%%%%%%%%%%%%%%%%%%%%%%%%%%%%%%%%%%%%%%%%%%%%%%%%%%%%%%%%%%%%%%%%%%%%%%%%%%%%%%%%
\subsection{Summary and Discussion}
\label{sec:results_discussion}

Faced with an increase of the statutory MW at some jurisdiction, our results
indicate that its sptail effect across rental markets will be determined by 
its incidence on each of the MW measures.


Compare to \textcite{Yamagishi2019}.
