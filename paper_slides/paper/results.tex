%%%%%%%%%%%%%%%%%%%%%%%%%%%%%%%%%%%%%%%%%%%%%%%%%%%%%%%%%%%%%%%%%%%%%%%%%%%%%%%%
%%%%%                                RESULTS                                %%%%
%%%%%%%%%%%%%%%%%%%%%%%%%%%%%%%%%%%%%%%%%%%%%%%%%%%%%%%%%%%%%%%%%%%%%%%%%%%%%%%%

In this section we present our main results.
First, we show our baseline estimates and conduct several robustness checks to 
explore the strength of our results.
Second, we present results of models that use alternative empirical strategies.
Third, we show heterogeneity analysis based on the residence location of MW 
workers and discuss other concerns that arise from the selectivity of our 
sample of ZIP codes.
Finally, we summarize our results and compare them with existing literature.

%%%%%%%%%%%%%%%%%%%%%%%%%%%%%%%%%%%%%%%%%%%%%%%%%%%%%%%%%%%%%%%%%%%%%%%%%%%%%%%%
\subsection{Main Results}
\label{sec:results_main}

Table \ref{tab:static} displays our estimates using the baseline sample 
described in Section \ref{sec:data_final_panel}.
Column (1) shows the results of a first-differenced regression of the workplace
MW measure on the residence MW measure, economic controls, and monthly date fixed
effects.
We observe that a 10 percent increase in the residence MW induces a 
$\WkpOnResCoeffBaseTen$ percent increase in the workplace MW.
While the measures are strongly correlated, this model shows that this 
correlation is far from exact, suggesting that there is independent variation
to estimate the effect of both measures on rents.

Columns (2) through (4) of Table \ref{tab:static} show estimates of equation 
\eqref{eq:fd}, varying the set of included MW measures.
Column (2) shows the results of estimating a model that does not include the 
workplace MW.
In this model, only locations with a statutory MW change are assumed to 
experience effects, similar to much of the MW literature 
\parencite[e.g.,][]{DubeEtAl2010, MeerWest2016, Yamagishi2021}.
In this case, we estimate the elasticity of median rents to the MW to be 
$\OnlyResGammaBase$ ($t=\OnlyResGammaBasetStat$).
Column (3) shows the results of a model that does not include the residence MW.
The coefficient on the MW variable increases slightly to 
$\OnlyWkpBetaBase$ ($t=\OnlyWkpBetaBasetStat$), 
consistent with the view that changes in the workplace MW are a better proxy of 
the changes in income generated by MW increases.
Column (4) estimates the model using both MW measures.
Consistent with the theoretical model in Section \ref{sec:model}, the 
coefficient on the residence MW ($\gamma$) now turns negative and equals 
$\BothGammaBase$, although it is not statistically significant 
($t=\BothGammaBasetStat$).
The coefficient on the workplace MW ($\beta$) increases to $\BothBetaBase$ and 
is statistically significant ($t=\BothBetaBasetStat$).
We reject the hypothesis that $\gamma=\beta$ at the 10\% significance level 
($p = \GammaEqBetaBasePval$).
Finally, $\gamma+\beta$ is estimated to be 
$\BothSumBase$ ($t=\BothSumBasetStat$), which is very similar in magnitude to 
the coefficients on MW variables in columns (2) and (3).
Thus, our results imply that a 10 percent increase in both MW measures will 
increase rents by $\BothSumBaseTen$ percent.
However, our results imply substantial heterogeneity across space.
If only the residence MW increases then rents are expected to decline,
and if only the workplace MW goes up then the rents increase will be larger.

A central concern with these results is whether our identifying assumptions is 
likely to hold.
Figure \ref{fig:dynamic_baseline} shows estimates of equation 
\eqref{eq:fd_leads_lags} under the baseline sample.
Panel (a) adds leads and lags of the workplace MW measure only, so that
our set of coefficients is now 
$\{\{\beta_s\}_{s=-6}^{-1},\beta,\{\beta_s\}_{s=1}^6,\gamma\}$.
We cannot reject the hypothesis that $\beta_{-6}=...=\beta_{-1}=0$ 
($p = \BetaPretrendDynBasePVal$).
Estimates of post-event coefficients $\{\beta_s\}_{s=1}^6$ are also estimated to 
be zero.
The only significant estimates are those of $\beta$ and $\gamma$, and they 
imply larger and more significant effects of both the residence and workplace MW
measures.
Our estimate of $\gamma$ is now $\BothWkpDynGammaBase$, and is statistically 
significant at the 10\% level ($t=\BothWkpDynGammaBasetStat$).
Our estimate of $\beta$ equals $\BothWkpDynBetaBase$ and is statistically 
significant at the 1\% level ($t=\BothWkpDynBetaBasetStat$).
Furthermore, in this case we can reject the hypothesis of equality of 
coefficients more precisely ($p = \GammaEqBetaBaseDynPval$).
Our estimate of $\gamma+\beta$ is now $\BothWkpDynSumBase$ and is highly 
significant ($t=\BothWkpDynSumBasetStat$), implying that a 10 percent increase 
in both measures would increase median rents by $\BothWkpDynSumBaseTen$ percent.

Panel (b) of Figure \ref{fig:dynamic_baseline} shows that a similar story 
obtains when we add leads and lags of the residence MW only.
Panel (c) of Figure \ref{fig:dynamic_baseline} adds leads and lags of both MW 
measures.
In this case, the width of confidence intervals for pre- and post-event 
coefficients increases between 2 to 4 times.
We attribute this fact to the strong correlation of the MW measures in our
panel, which implies that these coefficients are weakly identified.
However, the event-period coefficients are similar to those of 
panels (a) and (b).

Appendix Figure \ref{fig:map_residuals_chicago_jul2019} illustrates the 
identifying variation that we use by mapping the residualized workplace MW and 
residualized log rents.%
\footnote{To maximize the number of ZIP codes with valid data on this map we
use the results of the unbalanced panel discussed in Section 
\ref{sec:results_heterogeneity}.}
Panel (a) of Appendix Figure \ref{fig:map_residuals_chicago_jul2019}, to be 
contrasted with Panel (a) of Figure \ref{fig:map_mw_chicago_jul2019}, 
shows that residualized workplace MW is high outside of Cook County, the 
jurisdiction that increased the MW.
Panel (b) of Appendix Figure \ref{fig:map_residuals_chicago_jul2019} shows 
that residualized rents are very noisy.
The correlation of these residualized variables across our sample identifies
the effect of the workplace MW.

\subsubsection*{Robustness Checks}

Table \ref{tab:robustness} shows how our results change when we vary the
specification of the regression model and the commuting shares used 
to construct the workplace MW measure.
Each row of the table shows estimates analogous to those of columns (1) and (4)
of Table \ref{tab:static}.

Panel A of Table \ref{tab:robustness} groups the results when varying the 
regression model.
Row (b) shows that our results are very similar when we exclude the 
economic controls from the QCEW.
Rows (c) and (d) show that interacting our time fixed effects with indicators 
for county or CBSA yields similar results.
In all these cases our baseline estimates are contained in relevant 
confidence intervals and, in the case of CBSA $\times$ monthly date fixed 
effects, the results seem even larger.
This supports the view that our results are not caused by regional trends 
in housing markets correlated with our MW variables.
Row (e) shows that the results are different and non-significant when using 
state $\times$ monthly date fixed effects.
While our baseline estimates are within relevant confidence intervals, the 
signs of the point estimates are flipped.
We think that within-state comparisons are not appropriate because they fully 
identify coefficients from local MW changes which, in turn, are more likely to 
be passed by cities or counties that have more dynamic rental markets.
For instance, comparisons within the state of Illinois between ZIP codes in 
Cook County (the main jurisdiction with a local MW level) and the average
ZIP code in the state are likely to yield biased results, as both MW levels and
rents tend to increase at the same time of the year in Cook County.
On the other hand, within-CBSA and within-county comparisons use ZIP codes that
are likely to experience similar trends in rental markets.
In the case of Cook County, it seems plausible that ZIP codes a few miles away
in the same metropolitan area experience more similar trends in rents.
Row (f) includes ZIP code fixed effects in the first-differenced model, which
is equivalent to allowing for a ZIP code-specific linear trend in the model in 
levels.
The fact that our results are very similar implies that potential ZIP code 
level linear trends correlated with MW changes are unlikely to be the cause
of our results.

Panel B of Table \ref{tab:robustness} estimates the baseline model but 
computing the workplace MW using alternative commuting structures.
Rows (g) and (h) use commuting shares from 2014 or 2018 instead of 2017 as 
the baseline estimates.
Row (i) allows the commuting shares to vary by year, introducing additional
cross-year variation in the workplace MW measure that does not arise from 
changes in the statutory MW.
The fact that these specifications yield very similar results suggests that 
changes in commuting correlated with MW changes are unlikely to be the driver
of the results.
Rows (j) and (l) use 2017 commuting shares for workers that earn less than 
\$1,251 per month and workers that are less than 29, respectively.
If anything, the results seem to be stronger in this case, consistent
with the idea that these workers are more likely to earn close to the 
minimum wage.

\subsubsection*{Other geographies and time frames}

In this subsection, we compare our results with estimates obtained from 
alternative panels where the unit of observation is either the county by month 
or the ZIP code by year.
The reason to show these results is twofold.
First, it allows us to emphasize how important is the ZIP code and monthly 
resolution of our data for the plausibility of our identification assumption.
Namely, that the no pre-trends assumption at the ZIP code level is more 
plausible than at the county level.
Second, it allows us to compare our results with the previous literature 
estimating the effects of MW on housing rents.
Because none of the previous papers distinguish between workplace and residence
MW measures, we compare them to our short model that regresses log rents on
the workplace MW only.
The results for each dataset are summarized in Appendix Table 
\ref{tab:static_geos_times}, where Panel A repeats the results in Table 
\ref{tab:static} for convenience.

Panel B of Appendix Table \ref{tab:static_geos_times} shows our results based 
on a county-by-month panel.
Overall, the results are similar in magnitude to our baseline but are not
statistically significant.
In Appendix Figure \ref{fig:dynamic_county_month} we extend the model
that includes both MW measures adding leads and lags of the workplace MW, as
in panel (a) of Figure \ref{fig:dynamic_baseline}.
We observe considerable pre-trends in the rental prices in this model,
suggesting that estimates obtained at a larger geographical resolution may not 
use plausibly exogenous identifying variation.

Panel C of Appendix Table \ref{tab:static_geos_times} shows results estimated 
using a ZIP-code-by-year panel.
We estimate models that are yearly averages of their monthly equivalents, 
so in principle they should be valid under the same identifying assumption.
However, in practice we find that estimates are very imprecise, with standard
errors 3 to 4 times larger.
Our rental changes occur right at the month of the MW change, thus using 
yearly variation lacks the power to detect them.
Therefore, the usage of monthly data appears central to precisely estimate the 
effect of MW changes on rents.

% SH: Antes decíamos lo de abajo que creo es incorrecto, así que lo borré.
%     Usando Law of Iterated Expectations la identification assumption del
%     modelo mensual implica la del model anual.
%     Ver https://jmslab.github.io/DataRecipes/recipes/datarecipe_107.pdf
%
% On the one hand, using yearly variation, identification assumptions required to 
% interpret the results as causal are much less likely to hold because 
% unobservable determinants of MW changes over time are more likely to be 
% related with the yearly rental dynamics.


%%%%%%%%%%%%%%%%%%%%%%%%%%%%%%%%%%%%%%%%%%%%%%%%%%%%%%%%%%%%%%%%%%%%%%%%%%%%%%%%
\subsection{Alternative Strategies}
\label{sec:results_alternative_strategies}

Appendix Table \ref{tab:stacked_w6} estimates our main models using a 
``stacked'' sample, as discussed in Section \ref{sec:alt_emp_strategies}.
Our sample contains 618 ``events,'' that is, CBSA-month pairs that had some 
strict subset of ZIP codes increasing the residence MW.
These estimates interact the year-month fixed effects with event-id indicators, 
and thus compare ZIP codes within the same CBSA and time window.
This is in line with recent difference-in-differences literature that 
focuses on carefully selecting the comparison groups 
\parencite{CallawayEtAl2021, deChaisemartinEtAl2022, RothEtAl2022}.
Relative to Table \ref{tab:static} this model has fewer observations and includes 
many more fixed effects.
Nevertheless, we find very similar albeit less precise results.
We reject the hypothesis of equality of the residence and workplace MW at the
10\% significance level.
In this case, a 10 percent increase in both MW measures is estimated to 
increase rents by $\BothSumStackTen$ percent, in line with the results of the 
previous subsection.

Appendix Table \ref{tab:static_ab} shows estimates of a model that includes
the lagged difference in log rents as a covariate.
This specification relaxes the strict exogeneity assumption and allows for 
feedback effects of rent increases on the minimum wage variables.
To avoid the endogeneity problem of including this covariate the models are 
estimated using an IV strategy where we instrument the first lag of the change 
in rents with the second lag of this variable 
\parencite{ArellanoBond1991,ArellanoHonore2001}.
This estimation strategy also yields very similar but less precise results
when compared to our baseline in the previous subsection.

%%%%%%%%%%%%%%%%%%%%%%%%%%%%%%%%%%%%%%%%%%%%%%%%%%%%%%%%%%%%%%%%%%%%%%%%%%%%%%%%
\subsection{Sample Selection Concerns and Heterogeneity}
\label{sec:results_heterogeneity}

Table \ref{tab:static_sample} explores the sensitivity of our estimates to 
the sample of ZIP codes used in estimation.
Columns (1), (3), and (5) use our baseline sample of ZIP codes, 
a fully-balanced sample dropping all data prior to July 2015, and
an unbalanced sample of ZIP codes where we control for year-of-entry by
year-month fixed effects.
While the coefficient on the residence MW is very stable across specifications,
the one on the workplace MW seems to increase slightly when using the fully-balanced
sample, and to decrease slightly under the unbalanced sample.
This suggests that allowing for a change in sample composition biases
the estimates downward.
Our baseline sample achieves more precision at the cost of allowing the 
composition of ZIP codes to change before July of 2015.

We also worry that our ZIP codes might be a selected sample in ways that affect
our estimated effects.
In columns (2), (4), and (6) we estimate the same models but re-weighting 
observations to match relevant characteristics of the sample of urban 
ZIP codes.
Our weights follow \textcite{Hainmueller2012} and are designed to match the 
averages of 5 variables: 
the share of renter households, 
the shares of residents and workers that earn less than \$1,251, and
the shares of residents and workers that did not complete a high-school degree.
The re-weighting seems not to affect the estimated coefficients.

Table \ref{tab:heterogeneity} explores two channels of heterogeneity of our 
results.
Column (2) of Table \ref{tab:heterogeneity} presents estimates interacting
the MW measures with indicators that capture the likelihood of hosting MW 
workers and residents.
The results suggest that the effect of the residence MW is stronger 
by ZIP codes that are likely to have a high share of MW jobs, and that 
the effect of the workplace MW is stronger in ZIP codes with a high share 
of residents that are likely to earn a MW.
These results are consistent with our model, as one would expect ZIP codes 
where production is intensive in low-wage work to be more sensitive to the 
residence MW, 
and rents in ZIP codes that receive a large share of income from the MW
to be more sensitive to the workplace MW.
results are what one would expect given our 
model.
Column (3) of Table \ref{tab:heterogeneity} interacts both MW measures with an
indicator for the ZIP code having any public housing units.
We find that our results for both measures seem to be driven by ZIP codes that 
have public housing available.
This result does not support the view public housing diminishes the scope for 
landlords to increase rents.
However, it is possible that this indicator variable is capturing the locating 
of high presence of low-wage residents and workers where, per our previous
discussion, we expect stronger effects.

\subsection{Alternative rental categories}

Appendix Table \ref{tab:zillow_categories} shows how our results change when we 
use other rental categories available in the Zillow data.
%% SH: Before we had ' and less common rental categories'
%%     Dropped the 'less common' part because it's meaning is ambiguous. Better
%%     to say that the categories have fewer observations, as we do below
For each rental variable we use an unbalanced panel that controls for
year-month fixed effects interacted with indicators for the quarter of entry
to the data in the given rental category.
We note that the number of observations varies widely across housing categories, 
and is always much lower than for our baseline SFCC variable.

Given the reduced precision of these estimates is hard to obtain strong 
conclusions on what type of housing is reacting more strongly to MW changes.
We observe that the sum of the coefficients on our MW variables is 
statistically significant at conventional levels in the categories 
``Single Family'' (SF),  ``Condominium and Cooperative Houses'' (CC), and 
``Multifamily 5+ units.''
(The categories SF and CC are the components of our SFCC variable.)
The coefficients on each of the MW measures are typically much noisier than
baseline.
We observe inconsistent results for the category ``1 bedroom'' where the sign 
of the coefficients is flipped relative to baseline.
However, these are not statistically significant.

%%
%% SH: To be done: include other sources of rents data here, such as SAFMR
%%

%%%%%%%%%%%%%%%%%%%%%%%%%%%%%%%%%%%%%%%%%%%%%%%%%%%%%%%%%%%%%%%%%%%%%%%%%%%%%%%%
\subsection{Summary and Discussion}
\label{sec:results_discussion}

Faced with an increase of the statutory MW at some jurisdiction, our results
indicate that its spatial effect across rental markets will be determined by 
its incidence on each of the MW measures.
Consistent with the theoretical model in Section \ref{sec:model}, we find that 
increases in the MW at the residence tend to lower rents, whereas increases 
in the MW at workplace locations tend to increase rents.
Our estimates appear robust to several specification tests.
Furthermore, the magnitude of our estimates is similar to estimates of the
elasticity of restaurant prices to the MW \parencite{AllegrettoReich2018},
and the elasticity of grocery store prices to the MW 
\parencite{RenkinEtAl2020, Leung2021}.
%%
%% Backing-up claims
%%  AllegetroReich2018 estimate that "Minimum wage price elasticities averaged 
%%                 0.058 for all restaurants and ranged from 0.044 to 0.109"
%% RenkinEtAl2020 "find that a 10% minimum wage hike translates into a 0.36% 
%%                increase in the prices of grocery products"
%% Leung2021 finds that "a 10% increase in the minimum wage raises grocery 
%%                store prices by 0.6%-0.8%"
%%

We compare our estimates to those in \textcite{Yamagishi2019, AgarwalEtAl2021}.
Using Fair Market Rents data at the county by year level, 
\textcite[][, Tables 1 and 2]{Yamagishi2019} uses a long-differences 
specification and obtains null results using all counties and statistically 
significant results using densely populated counties.
In the latter case, he reports that a 10 percent increase in the MW increases
rents by $0.0365$ percent in the first year, and $0.1059$ percent four years 
later.
Our ZIP code-level estimates using only the workplace MW imply a one-time 
increase in rents of a similar magnitude as \citeauthor{Yamagishi2019}'s 
(\citeyear[][Table 2, Column 1]{Yamagishi2019}) one-year estimates.
While our results are consistent in this sense, 
\textcite[][Table 3]{Yamagishi2019} detects significant pre-trends,
questioning the validity of the longer-run results.%
\footnote{\textcite{Tidemann2018} uses the same data at the state level and 
reports the paradoxical result that MW hikes decrease monthly rents.
\textcite[][, Appendix C.1.3.]{Yamagishi2019} compares his results with 
\textcite{Tidemann2018} and concludes that for densely populated areas 
Tidemann's result turns positive and that clustering the standard errors at the
state level renders his results insignificant.}

Our results are consistent with \textcite{AgarwalEtAl2021}.
While the main goal of this paper is to estimate the effect of the MW on eviction
risk, the authors provide estimates of the effect of the MW on rents using
individual-level transactions from 2000 to 2009.
\textcite[][, Figure 4]{AgarwalEtAl2021} suggest that a 10 percent hike 
in the MW (at residence) increases rents paid by individuals by 0.5 percent.
The authors find an increasing effect over time that fully materializes after 
6 months.
This result is consistent with our estimates that show how rents of housing 
units in the rental market (which we observe in the Zillow data) jump 
discretely on the month of the MW change.
