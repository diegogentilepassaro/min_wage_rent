%%%%%%%%%%%%%%%%%%%%%%%%%%%%%%%%%%%%%%%%%%%%%%%%%%%%%%%%%%%%%%%%%%%%%%%%%%%%%%%%
%%%%%                                RESULTS                                %%%%
%%%%%%%%%%%%%%%%%%%%%%%%%%%%%%%%%%%%%%%%%%%%%%%%%%%%%%%%%%%%%%%%%%%%%%%%%%%%%%%%

In this section we present our main results.
First, we show our baseline estimates and conduct several robustness checks to 
explore the strength of our results.
Second, we present results of models that use alternative empirical strategies.
Third, we show heterogeneity analysis based on the residence location of MW 
workers and discuss other concerns that arise from the selectivity of our 
sample of ZIP codes.
Finally, we summarize our results and compare them with existing literature.

%%%%%%%%%%%%%%%%%%%%%%%%%%%%%%%%%%%%%%%%%%%%%%%%%%%%%%%%%%%%%%%%%%%%%%%%%%%%%%%%
\subsection{Main Results}
\label{sec:results_main}

Table \ref{tab:static} displays our estimates using the baseline sample 
described in Section \ref{sec:data_final_panel} under the parametric model
in equation \eqref{eq:fd}.
Column (1) shows the results of a regression of the workplace MW on the 
residence MW, economic controls, and monthly date fixed effects.
We observe that a 10 percent increase in the residence MW is associated with an
$\WkpOnResCoeffBaseTen$ percent increase in the workplace MW.
While the measures are strongly correlated, this model shows that this 
correlation is far from exact, confirming that there is independent variation
to estimate the effect of both variables on rents.

Columns (2) through (4) of Table \ref{tab:static} show estimates of equation 
\eqref{eq:fd}, varying the set of included MW measures.
Column (2) shows the results of estimating a model that does not include the 
workplace MW.
In this model, only locations with a statutory MW change are assumed to 
experience effects, similar to much of the MW literature 
\parencite[e.g.,][]{DubeEtAl2010, MeerWest2016, Yamagishi2021}.
In this case, we estimate the elasticity of median rents to the MW to be 
$\OnlyResGammaBase$ ($t=\OnlyResGammaBasetStat$).
Column (3) shows the results of a model that does not include the residence MW.
The coefficient on the MW variable increases slightly to 
$\OnlyWkpBetaBase$ ($t=\OnlyWkpBetaBasetStat$), 
supporting the view that changes in the workplace MW are a better proxy of 
the changes in income generated by MW increases.
Column (4) estimates the model using both MW measures.
Consistent with the theoretical model in Section \ref{sec:model}, the 
coefficient on the residence MW ($\gamma$) now turns negative and equals 
$\BothGammaBase$, although it is not statistically significant 
($t=\BothGammaBasetStat$).
The coefficient on the workplace MW ($\beta$) increases to $\BothBetaBase$ and 
is statistically significant ($t=\BothBetaBasetStat$).
We reject the hypothesis that $\gamma=\beta$ at the 10\% significance level 
($p = \GammaEqBetaBasePval$).
Finally, $\gamma+\beta$ is estimated to be $\BothSumBase$, which is similar 
in magnitude to the coefficient in column (3), and strongly 
significant ($t=\BothSumBasetStat$).
Thus, our results imply that a 10 percent increase in both MW measures will 
increase rents by $\BothSumBaseTen$ percent.
However, our results also imply substantial heterogeneity across space.
If only the residence MW increases then rents are expected to decline,
and if only the workplace MW goes up then the rents increase will be larger.

\subsubsection*{Identifying assumptions}

A central concern with these results is whether our identifying assumptions are 
likely to hold.
Figure \ref{fig:dynamic_baseline} shows estimates of the parametric model 
including leads and lags of either MW measure.
Panel A adds leads and lags of the workplace MW measure only, so that
the coefficients are
$\{\{\beta_s\}_{s=-6}^{-1},\beta,\{\beta_s\}_{s=1}^6,\gamma\}$.
We cannot reject the hypothesis that $\beta_{-6}=...=\beta_{-1}=0$ 
($p = \BetaPretrendDynBasePVal$).
Estimates of post-event coefficients $\{\beta_s\}_{s=1}^6$ are also estimated to 
be zero.
The only significant estimates are those of $\beta$ and $\gamma$, and they imply
similar effects relative to the model without dynamic effects.
Our estimate of $\gamma$ is $\BothWkpDynGammaBase$ 
($t=\BothWkpDynGammaBasetStat$) and of $\beta$ is $\BothWkpDynBetaBase$ 
($t=\BothWkpDynBetaBasetStat$).
We can now reject the hypothesis of equality of coefficients more precisely 
($p = \GammaEqBetaBaseDynPval$).
Our estimate of $\gamma+\beta$ is now $\BothWkpDynSumBase$.
It is highly significant ($t=\BothWkpDynSumBasetStat$) and almost identical to 
our baseline.
Panel B of Figure \ref{fig:dynamic_baseline} shows that a similar story 
obtains when we add leads and lags of the residence MW only.%
\footnote{Including leads and lags of both MW measures results in confidence 
intervals for pre- and post-event coefficients between two and four times wider.
As discussed in Section \ref{sec:empirical_strategy}, we exclude these estimates 
because the pre-trends test based on this model is statistically weak.}
These results are clear evidence in favor of the parallel trends assumption.

Appendix Figure \ref{fig:non_parametric} plots the relationship between 
log rents and each of the MW measures for ZIP codes in CBSAs and months 
in which at least one statutory MW changed.
Panel A shows the raw data, which shows a positive correlation between log rents
and both MW measures.
Panel B shows the same relationships after residualizing each variable on 
ZIP code fixed effects and indicators for different values of the other MW 
measure.
We observe a positive slope for the workplace MW, and a negative one for
the residence MW.
This provides evidence in favor of the assumption of no selection on gains, and
also of the linear functional form assumed in equation \eqref{eq:func_form}. 
Furthermore, the slopes in these figures are in line with our baseline estimates
of $\gamma$ and $\beta$.
Appendix Figure \ref{fig:map_residuals_chicago_jul2019} illustrates the 
identifying variation we use by mapping the residualized workplace MW and 
residualized log rents.%
\footnote{To maximize the number of ZIP codes with valid data on this map we
use the results of the unbalanced panel discussed in Section 
\ref{sec:results_heterogeneity}.}
Panel A of Appendix Figure \ref{fig:map_residuals_chicago_jul2019}, to be 
contrasted with the left panel of Figure \ref{fig:map_mw_chicago_jul2019}, 
shows that the residualized change in the workplace MW is high outside of Cook 
County, where the statutory MW increased.
For completeness, Panel B of Appendix Figure 
\ref{fig:map_residuals_chicago_jul2019} shows residualized rents.

\subsubsection*{Robustness Checks}

Table \ref{tab:robustness} shows how our results change when we vary the
specification of the regression model and the commuting shares used 
to construct the workplace MW measure.
Each row of the table shows estimates analogous to those of columns (1) and (4)
of Table \ref{tab:static}.

Panel A of Table \ref{tab:robustness} groups the results when varying the 
regression model.
Row (b) shows that our results are very similar when we exclude the 
economic controls from the QCEW.
Rows (c) and (d) show that interacting our time fixed effects with indicators 
for county or CBSA yields similar results.
In all these cases our baseline estimates are contained in relevant 
confidence intervals and, in the case of CBSA $\times$ monthly date fixed 
effects, the results seem even larger.
This supports the view that our results are not caused by regional trends 
in housing markets correlated with our MW variables.
Row (e) shows that the results are different and non-significant when using 
state by monthly date fixed effects.
While our baseline estimates are within relevant confidence intervals, the 
signs of the point estimates are flipped.
We think that within-state comparisons are not appropriate because they fully 
identify coefficients off of local MW changes which, in turn, are more likely to 
be passed by cities or counties that have more dynamic rental markets.
For instance, comparisons within the state of Illinois between ZIP codes in 
Cook County (the main jurisdiction with a local MW level) and the average
ZIP code in the state are likely to yield biased results, as both MW levels and
rents tend to increase at the same time of the year in Cook County.
On the other hand, within-CBSA and within-county comparisons use ZIP codes that
are likely to experience similar trends in rental markets.
Row (f) includes ZIP code fixed effects in the first-differenced model, which
is equivalent to allowing for a ZIP code-specific linear trend in the model in 
levels.
The fact that our results are very similar implies that potential ZIP code 
level linear trends correlated with MW changes are unlikely to be the cause
of our results.

Panel B of Table \ref{tab:robustness} estimates the baseline model but 
computing the workplace MW using alternative commuting structures.
Rows (g) and (h) use commuting shares from 2014 or 2018 instead of 2017 as 
the baseline estimates.
Row (i) allows the commuting shares to vary by year, introducing additional
cross-year variation in the workplace MW measure that does not arise from 
changes in the statutory MW.
The fact that these specifications yield very similar results suggests that 
changes in commuting correlated with MW changes are unlikely to be the driver
of the results.
Rows (j) and (k) use 2017 commuting shares for workers that earn less than 
\$1,251 per month and workers that are less than 29, respectively.
If anything, the results seem to be stronger and more significant in this case, 
consistent with the idea that these workers are more likely to earn close to the 
minimum wage.

\subsubsection*{Other geographies and time frames}

In this subsection, we compare our results with estimates obtained from 
alternative panels where the unit of observation is either the county by month 
or the ZIP code by year.
The reason to show these results is twofold.
First, it allows us to emphasize the importance of the ZIP-code-by-month 
resolution of our data for the plausibility of our identification assumptions.
Second, it allows us to compare our results with the previous literature 
estimating the effects of MW on housing rents.
Because none of the previous papers distinguish between workplace and residence
MW levels, we compare them to our short model that excludes the residence MW.
The results for each dataset are summarized in Appendix Table 
\ref{tab:static_geos_times}, where Panel A repeats the results in Table 
\ref{tab:static} for convenience.

Panel B of Appendix Table \ref{tab:static_geos_times} shows our results based 
on a county-by-month panel.
Overall, the results are similar in magnitude to our baseline but are not
statistically significant.
In Appendix Figure \ref{fig:dynamic_county_month} we extend the model
that includes both MW measures adding leads and lags of the workplace MW, as
in Panel A of Figure \ref{fig:dynamic_baseline}.
We observe considerable pre-trends in the rental prices in this model,
suggesting that estimates obtained at a larger geographical resolution may not 
use plausibly exogenous identifying variation.

Panel C of Appendix Table \ref{tab:static_geos_times} shows results estimated 
using a ZIP-code-by-year panel.
We estimate models that are yearly averages of their monthly equivalents, 
so in principle they should be valid under the same identifying assumption.
However, in practice we find that estimates are very imprecise, with standard
errors 3 to 4 times larger.
Our rental changes occur right at the month of the MW change, thus using 
yearly variation lacks the power to detect them.
The usage of monthly data appears central to precisely estimate the 
effect of MW changes on rents.

%%%%%%%%%%%%%%%%%%%%%%%%%%%%%%%%%%%%%%%%%%%%%%%%%%%%%%%%%%%%%%%%%%%%%%%%%%%%%%%%
\subsection{Alternative Strategies}
\label{sec:results_alternative_strategies}

Appendix Table \ref{tab:stacked_w6} estimates our main models using a 
``stacked'' sample, as discussed in Section \ref{sec:alt_emp_strategies}.
Our sample contains 618 ``events,'' that is, CBSA-month pairs that had some 
strict subset of ZIP codes increasing the residence MW.
These estimates interact the year-month fixed effects with event ID indicators, 
and thus compare ZIP codes within the event and time window.
This is in line with recent difference-in-differences literature that 
focuses on carefully selecting the comparison groups 
\parencite{CallawayEtAl2021, deChaisemartinEtAl2022, RothEtAl2022}.
We find that our key MW-based measures have little predictive power on their own,
but the model including both measures yields similar patterns as our baseline.
If anything, results seem stronger in this case.
Both MW measures are strongly significant in this case.
A 10 percent increase in both MW measures is estimated to increase rents 
by $\BothSumStackTen$ percent.
Appendix Figure \ref{fig:dynamic_stacked} shows the results of a similar model 
that leads and lags of the workplace MW.
Estimates of leads and lags are statistically non-distinguishable from zero.
However, they are noisier than in our baseline.

Appendix Table \ref{tab:arellano_bond} shows estimates of a model that includes
the lagged difference in log rents as a covariate.
This specification relaxes the strict exogeneity assumption and allows for 
feedback effects of rent increases on the minimum wage variables.
To avoid the endogeneity problem of including this covariate the models are 
estimated using an IV strategy where we instrument the first lag of the change 
in rents with the second lag of this variable 
\parencite{ArellanoBond1991,ArellanoHonore2001}.
Columns (1) and (2) show estimates of models in levels, both of which imply
confidence intervals for the coefficients that include our preferred estimates.
Columns (3) and (4) show preferred models in first differences, where results
are very similar. 

%%%%%%%%%%%%%%%%%%%%%%%%%%%%%%%%%%%%%%%%%%%%%%%%%%%%%%%%%%%%%%%%%%%%%%%%%%%%%%%%
\subsection{Sample Selection Concerns and Heterogeneity}
\label{sec:results_heterogeneity}

Table \ref{tab:static_sample} explores the sensitivity of our estimates to 
the sample of ZIP codes used in estimation.
Columns (1) and (3) use our baseline sample and
an unbalanced sample of ZIP codes where we control for quarter-year-of-entry by
year-month fixed effects, respectively.
While the coefficient on the residence MW is very stable across specifications,
the one on the workplace MW seems to decrease when using the unbalanced
sample.
This suggests that allowing for a change in sample composition biases the 
estimates downwards.

We also worry that our ZIP codes might be a selected sample in ways that affect
our estimated effects.
In columns (2) and (4) we estimate the same models but re-weighting 
observations to match relevant pre-treatment characteristics of the sample of 
urban ZIP codes.
Our weights follow \textcite{Hainmueller2012} and are designed to match the 
averages of 3 variables: 
the share of renter occupied households according to the 2010 US Census, and
the shares of residents and workers that earn less than \$1,251, according to
2014 LODES.
Effects appear to be stronger for the baseline sample in column (2), although
we cannot reject they are equal to the estimates in column (1).

Table \ref{tab:heterogeneity} explores three channels of heterogeneity of our 
results.
Column (1) reproduces our baseline results.
Column (2) of Table \ref{tab:heterogeneity} presents estimates interacting
the MW measures with an estimated share of MW workers residing in each ZIP code.
At the mean share of MW workers, our estimates indicate that the coefficient on 
the residence MW is $\TildeGammaZero$ (SE$=\TildeGammaZeroSE$) and 
on the workplace MW is $\TildeBetaZero$ (SE$=\TildeBetaZeroSE$).
For a ZIP code that is one standard deviation above the average share of MW 
workers, the effect of the residence MW is of $\TildeGammaZeroPlusGammaOne$ 
(SE$=\TildeGammaZeroPlusGammaOneSE$) and, for the workplace MW, the effect is
$\TildeBetaZeroPlusBetaOne$ (SE$=\TildeBetaZeroPlusBetaOneSE$).
Hosting more MW workers in a ZIP code implies that income is likely to be more 
sensitive to the MW and so, consistent with our model, the effect of the MW
on rents is larger.

Column (3) of Table \ref{tab:heterogeneity} interacts both MW measures with the
standardized median household income from the ACS.
We find analogous patterns to Column (2), as a higher median income is 
correlated with a lower share of MW workers.
Column (4) interacts the MW measures with the standardized share of public 
housing units.
We find that the effects for ZIP codes that have more public housing are larger,
although the coefficient on the interaction is not statistically significant. 
This result suggests that public housing does not necessarily diminish the scope 
for landlords to increase rents.
However, it is possible that this variable is capturing the high presence of 
low-wage residents and workers where, per our previous discussion, we would 
expect stronger effects.

\subsection{Alternative rental categories}

Appendix Table \ref{tab:zillow_categories} shows how our results change when we 
use other rental categories available in the Zillow data.
%% SH: Before we had ' and less common rental categories'
%%     Dropped the 'less common' part because it's meaning is ambiguous. Better
%%     to say that the categories have fewer observations, as we do below
For each rental variable we use an unbalanced panel that controls for
year-month fixed effects interacted with indicators for the quarter of entry
to the data in the given rental category.
We note that the number of observations varies widely across housing categories, 
and is always much lower than for our baseline SFCC variable.

Given the reduced precision of these estimates is hard to obtain strong 
conclusions on what type of housing is reacting more strongly to MW changes.
We observe that the sum of the coefficients on our MW variables is 
statistically significant at conventional levels in the categories 
``Single Family'' (SF),  ``Condominium and Cooperative Houses'' (CC), and 
``Multifamily 5+ units.''
(The categories SF and CC are the components of our SFCC variable.)
Appendix Figure \ref{fig:ahs_unit_types} shows that low-wage households are 
likely to reside in these type of housing units.
However, the coefficients on each of the MW measures are typically much noisier 
than baseline.
We observe inconsistent results for the category ``1 bedroom'' where the sign 
of the coefficients is flipped relative to baseline, although these estimates 
are not statistically significant.

%%%%%%%%%%%%%%%%%%%%%%%%%%%%%%%%%%%%%%%%%%%%%%%%%%%%%%%%%%%%%%%%%%%%%%%%%%%%%%%%
\subsection{Summary and Discussion}
\label{sec:results_discussion}

Faced with an increase of the statutory MW at some jurisdiction, our results
indicate that its spatial effects across rental markets will be determined by 
its incidence on each of the MW measures.
Consistent with the theoretical model in Section \ref{sec:model}, we find that 
increases in the MW at the residence tend to lower rents, whereas increases 
in the MW at workplace locations tend to increase rents.
Our estimates appear robust to several specification tests.
Furthermore, the magnitude of our estimates is similar to estimates of the
elasticity of restaurant prices to the MW \parencite{AllegrettoReich2018},
and the elasticity of grocery store prices to the MW 
\parencite{RenkinEtAl2020, Leung2021}.
%%
%% Backing-up claims
%%  AllegetroReich2018 estimate that "Minimum wage price elasticities averaged 
%%                0.058 for all restaurants and ranged from 0.044 to 0.109"
%%  RenkinEtAl2020 "find that a 10% minimum wage hike translates into a 0.36% 
%%                increase in the prices of grocery products"
%%  Leung2021 finds that "a 10% increase in the minimum wage raises grocery 
%%                store prices by 0.6%-0.8%"
%%

We compare our estimates to those in \textcite{Yamagishi2019, AgarwalEtAl2021}.
Using Fair Market Rents data at the county by year level, 
\textcite[][, Tables 1 and 2]{Yamagishi2019} uses a long-differences 
specification and obtains null results using all counties and statistically 
significant results using densely populated counties.
In the latter case, he reports that a 10 percent increase in the MW increases
rents by $0.0365$ percent in the first year, and $0.1059$ percent four years 
later.
Our ZIP code-level estimates using only the workplace MW imply a one-time 
increase in rents of a similar magnitude as \citeauthor{Yamagishi2019}'s 
(\citeyear[][Table 2, Column 1]{Yamagishi2019}) one-year estimates.
While our results are consistent in this sense, 
\textcite[][Table 3]{Yamagishi2019} detects significant pre-trends,
questioning the validity of the longer-run results.%
\footnote{\textcite{Tidemann2018} uses the same data at the state level and 
reports the paradoxical result that MW hikes decrease monthly rents.
\textcite[][, Appendix C.1.3.]{Yamagishi2019} compares his results with 
\textcite{Tidemann2018} and concludes that for densely populated areas 
Tidemann's result turns positive and that clustering the standard errors at the
state level renders his results insignificant.}

Our results are consistent with \textcite{AgarwalEtAl2021}.
While the main goal of this paper is to estimate the effect of the MW on eviction
risk, the authors provide estimates of the effect of the MW on rents using
individual-level transactions from 2000 to 2009.
\textcite[][, Figure 4]{AgarwalEtAl2021} suggest that a 10 percent hike 
in the MW (at residence) increases rents paid by individuals by 0.5 percent.
The authors find an increasing effect over time that fully materializes after 
6 months.
This result is consistent with our estimates that show how rents of housing 
units in the rental market (which we observe in the Zillow data) jump 
discretely on the month of the MW change.
