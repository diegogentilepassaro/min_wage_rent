%%%%%%%%%%%%%%%%%%%%%%%%%%%%%%%%%%%%%%%%%%%%%%%%%%%%%%%%%%%%%%%%%%%%%%%%%%%%%%%%%
%%%%%                                RESULTS                                 %%%%
%%%%%%%%%%%%%%%%%%%%%%%%%%%%%%%%%%%%%%%%%%%%%%%%%%%%%%%%%%%%%%%%%%%%%%%%%%%%%%%%%

In this section we present our main results.
First, we show our baseline estimates and we discuss how our results change
when using alternative specifications.
Second, we conduct several robustness checks to explore the strength of 
our results.
Third, we show heterogeneity analysis based on the residence location of MW 
workers.
Finally, we summarize our results and compare them with existing literature.

%%%%%%%%%%%%%%%%%%%%%%%%%%%%%%%%%%%%%%%%%%%%%%%%%%%%%%%%%%%%%%%%%%%%%%%%%%%%%%%%
\subsection{Main Results}\label{sec:main_results}

Table \ref{tab:static} shows our estimates using the baseline sample
described in \ref{sec:data_final_panel}.
Column (1) shows the results of a first-differenced regression of the workplace
MW measure on the residence MW measure, economic controls and year-month fixed
effects.
We observe that a 10 percent increase in the residence MW induces an 
8.6 percent increase in the workplace MW.
While the measures are strongly correlated, this model shows that this 
correlation is far from exact, suggesting that there is independent variation
to estimate the effect of both measures on rents.

Columns (2) through (4) shows estimates of equation \eqref{eq:fd}, varying the
set of included MW measures.
Column (2) shows the results of a model that does not include the workplace MW. 
This model has been extensively used in other contexts (see for example \textcite{MeerWest2016}), and it assumes no spatial spillovers of the extra MW 
income so that that only locations with a statutory MW change are predicted to 
experience effects on rents. 
In this case we observe that a 10 percent increase in the MW is associated with
a statistically significant 0.27 percent increase in the average rents.
Column (3) shows the results of a model that does not include the residence MW.
The coefficient of interest seems to increase slightly, consistent with the idea 
that changes in the workplace MW are a better measure of changes in disposable 
income across space than the statutory one.
Column (4) estimates the model using both MW measures, and it both allows for 
spatial spillovers, through the workplace MW measure, and it controls for local 
changes in prices of other goods, through the statutory MW measure.
As suggested by our model in \ref{sec:model}, the coefficient on the residence 
MW has now turned negative and equals $-0.0204$, although it is not 
statistically significant at conventional levels ($t=-1.21$).
The coefficient on the workplace MW has increased in value to $0.0545$ and is 
statistically significant ($t=1.93$).
We reject the hypothesis that these coefficients are equal at the 10\% 
significance level.
Finally, the sum of the coefficients on each MW measure has a magnitude of 
$0.0342$, very similar to the one in column (3), and is highly significant 
($t=2.21$).
Our results imply that a 10-percent increase in both MW measures will increase
rents by $0.34$ percent.
However, if only the residence MW increases then rents are estimated to decline,
and if only the workplace MW goes up then the rents increase would be larger.

Table \ref{tab:static_sample} shows that we obtain similar results when
using an unbalanced sample (column 2) or a fully-balanced sample (column 3).
We obtain qualitatively similar but statistically insignificant results when
we reweight observations to match moments of the distribution of urban ZIP 
codes.
However, all estimates are slightly less precise than our baseline results in 
Table \ref{tab:static}.

A main concern with these results is whether our identifying assumptions is 
likely to hold.
Figure \ref{fig:dynamic_baseline} shows estimates of equation 
\eqref{eq:fd_leads_lags} under the baseline sample.
Panel (a) adds leads and lags of the workplace MW measure only, and shows
precisely estimated zeros in the pre-event period.
We cannot reject the hypothesis that pre-event coefficients are jointly equal
to zero ($p = 0.856$).
Post-event coefficients are also estimated to be zero.
The only significant coefficients are those of the event period and they imply 
larger and more significant effects of both residence and workplace MW.
The coefficient on the residence equals $-0.0306$, and it is 
statistically significant at the 10\% level ($t=-1.82$).
The coefficient on the workplace MW equals $0.0683$ and is 
statistically significant at the 1\% level ($t=2.52$).
The sum of the coefficients on each MW measure also increase and is now of
$0.0377$ and highly significant ($t=2.45$), implying that a 10-percent increase 
in both MW measures will increase rents by $0.38$ percent.
In this case we can even reject the hypothesis of equality of coefficients more
precisely ($p = 0.025$).

Panel (b) shows a similar story when we add leads and lags of the residence MW
only.
Panel (c) adds leads and lags of both MW measures.
In this case, precision decreases sharply given the strong correlation of 
MW variables cross sectionally and over time. However, we observe similar 
coefficients in the event period.

To illustrate the identifying variation that we use 
Appendix Figure \ref{fig:map_residuals_chicago_jul2019} shows a map of 
residualized workplace MW and residualized log rents.%
\footnote{To maximize the number of ZIP codes with valid data on this map we
use the unbalanced panel (column 2 of Table \ref{tab:static_sample}).}
Panel (a) of Appendix Figure \ref{fig:map_residuals_chicago_jul2019}, to be 
contrasted with Panel (a) of Figure \ref{fig:map_mw_chicago_jul2019}, 
shows that residualized workplace MW is high outside of Cook County, the 
jurisdiction that increased the MW.
Panel (b) of Appendix Figure \ref{fig:map_residuals_chicago_jul2019} shows that 
residualized rents are very noisy.
The correlation of these residualized variables across our sample identifies
the effect of the workplace MW.

\subsection{Alternative Strategies}

Appendix Table \ref{tab:stacked_w6} estimates our main models using a ``stacked''
sample, as discussed in Section \ref{sec:alt_emp_strategies}.
Our sample contains 618 ``events,'' that is, CBSA-month pairs that had some 
strict subset of ZIP codes increasing the residence MW.
These estimates interact the year-month fixed effects with event-id indicators, 
and thus compare ZIP codes within the same event only in line with the recent
difference-in-differences literature \parencite{CallawaySantAnna2021, CallawayEtAl2021}.
Relative to Table \ref{tab:static} this model has less observations and includes 
many more fixed effects. Nevertheless, we find very similar albeit less precise results.

Appendix Table \ref{tab:static_ab} shows estimates of a model that includes
the lagged difference in log rents as a covariate.
This specification relaxes the strict exogeneity assumption and allows for 
feedback effects of rent increases on the minimum wage variables.
To avoid the endogeneity problem of including this covariate the models are 
estimated using an IV strategy following \textcite{ArellanoBond1991,ArellanoHonore2001}.
Relative to our baseline, this estimation strategy yields very similar results.

\subsection{Robustness Checks}

Table \ref{tab:robustness} shows how our results change when we vary the
specification of the regression model (Panel A) and the commuting shares used 
to construct the workplace MW measure (Panel B).
Each row of the table shows estimates analogous to those of columns (1) and (4)
of Table \ref{tab:static}.

Panel A row (b) of Table \ref{tab:robustness} shows that our results are very similar 
when we we exclude the economic controls from the QCEW. 
The first fact confirms that our economic controls are not driving the results.
Panel A rows (c) to (e) show that including County $\times$ year-month, 
CBSA $\times$ year-month, and Place $\times$ year-month fixed effects
the effects are similar and our baseline estimates are within the confidence 
intervals. This supports the view that our results are not caused by regional 
trends in housing markets correlated with our MW variables.
Panel A row (f) includes ZIP code fixed effects in the first-differenced model,
thus allowing for a ZIP code specific linear trend results are similar. This 
accounts for potential ZIP code level heterogeneity that could be biasing our results.

\subsection{Alternative Rental Categories}

Appendix Table \ref{tab:zillow_categories} shows how our results change when we 
use alternative and less common rental categories in Zillow.
The sample in each row is constructed starting from the baseline sample used in 
Table \ref{tab:static} and keeping only ZIP codes that have valid data in the 
given rental category.
We note that the number of observations varies widely and is always much lower
that in our baseline.

Given the reduced precision of these estimates is hard to obtain strong conclusions 
of what type of housing is reacting the most.
Results in the categories ``Condominium and Cooperative Houses'' and ``3 bedroom''
resemble the baseline estimates the most.
We observe inconsistent results for ``Studios,'' where the sign of the coefficients
is flipped relative to baseline.
However, these are not statistically significant.

\subsubsection{Other geographies and time frames} \label{sec:oth_geo_time}

In this subsection, we compare our results with estimates on alternative panels
that use County-month, ZIP code-year, and Count-year variation. The reason to show 
these results is twofold. First, it allows us to emphasize how important 
is the ZIP code and monthly resolution of our data for the plausibility of our 
identification assumption. Namely, that pretrends ahead of a MW change is much 
less likely at the ZIP code level than at the county level. Second, it allows us 
to benchmark our results with the previous literature estimating the effects of 
MW on housing rents. \textcite{Yamagishi2019} and \textcite{Tidemann2018} use 
county-year variation, and we discuss that why identification is likely to fail.

Appendix Table \ref{tab:cty_vs_zip_mth_vs_yr } Panel B shows our results based on a County 
month panel in which we assigned MW measures by considering as statutory MW the maximum MW within
the County and where we used Zillow data readily available at the County-month. The results in 
row (iii) are similar in magnitude to our baseline results but they are not significant and 
much less precisely estimated. In Appendix Figure \ref{fig:dynamic_county_month}, we display 
for the county-month the analogous to Figure \ref{fig:dynamic_baseline} panel (a), to show that 
at the county month there are considerable pretrends in the rental prices. This illustrates 
the importance of having data with high frequency and high geographic resolution. 

Appendix Table \ref{tab:cty_vs_zip_mth_vs_yr } Panels C and D, show results estimated
using a panel at the ZIP code-year and County-year respectively. In both cases, estimates
are very imprecise and the magnitude of the coefficients is far from the baseline results. 
We interpret this facts as evidence that using yearly variation of the rental prices is not 
well suited for estimating the effects of the MW changes. On the one hand, using yearly variation, 
identification assumptions required to interpret the results as causal are much less likely 
to hold because unobservable determinants of MW changes over time are more likely to be 
related with the yearly rental dynamics. Nevertheless, the exact month when MW changes 
happen is heavily influenced by different legislative calendars of the state and local 
representatives, and therefore observing rents to increase when MW increases is much 
more likely to be causal when using small geographic units and controlling for local 
state of the economy factors. On the other hand, without data around the exact timing of the 
MW changes pretrends test are less informative.

Using Fair Market Rents data at the County-year level, \textcite{Yamagishi2019} employs a 
specification close to the one shown in Appendix Table \ref{tab:cty_vs_zip_mth_vs_yr } 
panel (D) and row (i) and finds that the elasticity of housing rents to residence MW changes 
is of $0.0150$, although his results are not significant. He reports that this elasticity is 
significant at the 10\% level and of about $0.0365$ when focusing in densely populated areas.
\textcite{Tidemann2018} uses the same data and a level model to reports the paradoxical 
result that MW hikes decrease monthly rents in statistically significant way. \textcite{Yamagishi2019} 
Appendix C.1.3. compares his results with \textcite{Tidemann2018} and concludes that for densely 
populated areas Tidemann's result turns positive and that clustering the standard errors at the
state level renders his results insignificant.


%%%%%%%%%%%%%%%%%%%%%%%%%%%%%%%%%%%%%%%%%%%%%%%%%%%%%%%%%%%%%%%%%%%%%%%%%%%%%%%%
\subsection{The Heterogeneous Impact of MW Changes on Rents}
\label{sec:heterogeneity_results}

Construct measure of MW workers residing in each zip code using ACS 2015?

%%%%%%%%%%%%%%%%%%%%%%%%%%%%%%%%%%%%%%%%%%%%%%%%%%%%%%%%%%%%%%%%%%%%%%%%%%%%%%%%
\subsection{Summary and Discussion}
\label{sec:results_discussion}

Faced with an increase of the statutory MW at some jurisdiction, our results
indicate that its sptail effect across rental markets will be determined by 
its incidence on each of the MW measures.


Compare to \textcite{Yamagishi2019}.
