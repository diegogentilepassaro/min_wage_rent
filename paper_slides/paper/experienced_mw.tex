%%%%%%%%%%%%%%%%%%%%%%%%%%%%%%%%%%%%%%%%%%%%%%%%%%%%%%%%%%%%%%%%%%%%%%%%%%%%%%%%%
%%%%%                           EXPERIENCED MW                               %%%%
%%%%%%%%%%%%%%%%%%%%%%%%%%%%%%%%%%%%%%%%%%%%%%%%%%%%%%%%%%%%%%%%%%%%%%%%%%%%%%%%%

Thus far we have used the statutory minimum wage as our variable of interest (i.e., 
the maximum across federal, state and local levels). While this variable correctly 
captures the underlying relevant increase in income across low-wage workers for large
geographical areas, such as states, it is likely to be less precise at the local level, 
such as the zipcode. As discussed throughout the paper, a central feature of MW 
policies is the geographical scope inherently associated with them. Hence, accounting 
for workplace and residence locations of low-wage workers can provide valuable insight 
into the question of how MW changes affect the housing market. 

The underlying assumptions behind this view are that rents are determined by the 
interaction of supply and demand of housing for rent, and that demand for rentals is
positively associated with income. As a result, we would expect MW policies to increase 
rents in neighborhoods where low-wage workers live, since those experience a positive
income shock due to the policy, and, to the extent that they work on zipcodes with few 
low-wage workers, to decrease rents in their workplace, as those zipcodes experience
a negative income shock. As a matter of fact, the heterogeneity analysis in 
\autoref{sec:heter} suggests that the effect is stronger in zipcodes with a high
concentration of low-wage workers. We explore this possibility more deeply in this 
section.

We begin by discussing the experienced MW, introduced in \autoref{sec:data}, and 
comparing it with the statutory MW. We use this measure in two ways. First, we use
it as dependent variable in our main models, and find that the estimated elasticity 
increases. Second, ... COMPLETE

%%%%%%%%%%%%%%%%%%%%%%%%%%%%%%%%%%%%%%%%%%%%%%%%%%%%%%%%%%%%%%%%%%%%%%%%%%%%%%%%%
\subsection{A New Minimum Wage Measure}

As shown in the heterogeneity analysis of \autoref{sec:heter}, the effect MW policies 
appears to be stronger for zipcodes with a higher concentration of MW residents. This 
provides an empirical justification for refining the main explanatory variable so to 
better account for the fact that residence and workplace of workers tends to diverge, 
and that MW policies in the latter determine low-wage workers' income. As explained in 
\autoref{sec:mw_construction}, we compute the experienced MW for a given zipcode $i$ as 
the weighted average of the statutory MW in zipcodes where residents of $i$ work, where 
weights correspond to the share of the workforce in $i$ that works in each zipcode in 
the LODES data.

\autoref{fig:expmw_san_diego} illustrates the difference between our MW measures by plotting 
the percent change in statutory and experienced MW variables following the California 
increase of January 2019. As of December 2018 the MW in San Diego city was \$11.50, 
whereas the state's MW binding outside the city was \$11.\footnote{For employers larger 
	than 26 employees. For those below 26 employees the level was \$10.50. As explained in 
	\autoref{sec:data}, our variable uses the statutory MW variable takes the MW for
	large employers in this case. [WE EXPLAIN THIS? CHECK]}
The increase in the state level MW to \$12 in January 2020 appears as a discontinuity in the 
city border. However, when we account for the fact that MW workers commute, we observe a 
gradient in the intensity of the policy.

\begin{figure}
	\caption{The California MW Increase of January 2019 in San Diego}
	\label{fig:expmw_san_diego}
	\centering
	\begin{subfigure}[b]{0.65\textwidth}
		\caption{Statutory MW change}
		\includegraphics[width = \textwidth]
		{../../analysis/descriptive_maps/output/San_Diego_mw_msa.png}
	\end{subfigure}\\
	\begin{subfigure}[b]{0.65\textwidth}
		\caption{Experienced MW change}
		\includegraphics[width = \textwidth]
		{../../analysis/descriptive_maps/output/San_Diego_expmw_msa.png}
	\end{subfigure}
	\begin{minipage}{0.95\textwidth} \footnotesize
		\vspace{2mm} 
		\textit{Notes}: The figure maps the percent increase in our minimum wage and 
		experienced minimum wage measures following the state increase in California
		on January 2019. The map colors only those zipcodes for which we have 
		non-missing rents data from Zillow.
	\end{minipage}
\end{figure}

The new measure we obtain is highly correlated with the original statutory MW. The 
correlation in the levels of the variables is 0.985, whereas the correlation between the 
difference of natural logarithms is 0.971. %% Document in repo
However, these variables are not identical, and they show an important degree of independent 
variation. To see this, first note that our experienced MW results in a larger number of 
treated zipcode-month observations in our baseline panel, increasing the number of events 
from 5,302 to 8,942. %% Document in repo
Hence, there are 3,640 zipcode-month cells that went from being zero to a positive value. 
The variables not only differ in the number of non-zero changes, but also in their intensity. 
The first column of \autoref{tab:expmw_main} reports the results of running our main static 
model but using as dependent variable the change in the experienced MW $\Delta \ln 
\underline{w}_{itc}^{\text{exp}}$. After conditioning on zipcode and time period fixed effects
and our preferred set of economic controls, we observe that, on average, a 1 percent increase 
in the statutory MW translates into a 0.89 percent increase in the experienced MW. This 
regression suggests that the measures have more independent variation than suggested by 
raw correlations alone.

Now, given that the measures show a reasonable degree of independent variation, where is that
variation coming from? State-wide MW increases are unlikely to induce that variation, since 
they tend to affect near-by zipcodes equally. Local MW policies (either at the county or city
levels) are the obvious candidates, since they impact differentially zipcodes in close 
proximity. 

%%%%%%%%%%%%%%%%%%%%%%%%%%%%%%%%%%%%%%%%%%%%%%%%%%%%%%%%%%%%%%%%%%%%%%%%%%%%%%%%%
\subsection{Estimation Results}

We re-estimate our baseline models using the experienced MW as treatment variable and present 
the results in \autoref{tab:expmw_main}. Column 2 shows that both the static and cumulative 
effect on rents slightly increase: a 10 percent increase in the experienced MW leads to a simultaneous
$0.31$ percent rent increase, rising to $0.66$ within 6-months. The higher estimates suggest how 
indeed measurement error is likely to introduce downward bias in baseline results.
%FINISH COMMENT ON RESULTS

%We re-estimate The estimated effect of MW on rents slightly increases in the two statistically significant periods,  
%$t=0$ and $t=1$, but the new results largely confirm the magnitude and dynamics uncovered by the baseline model. 
%A 10 percent increase in experienced MW leads to a $0.31$ percent contemporaneous increase in median rents, as well 
%as a $0.145$ increase in the following month. The absence of statistically significant pre-trend is confirmed.    


\begin{table}[h!]\centering
	\caption{The Impact of Experienced Minimum Wage on Rents}
	\label{tab:expmw_main}
	{
\def\sym#1{\ifmmode^{#1}\else\(^{#1}\)\fi}
\begin{tabular}{l*{4}{c}}
\hline\hline
          &\multicolumn{1}{c}{$\Delta \ln \underline{w}_{itc}^{\text{exp}}$}&\multicolumn{3}{c}{$\Delta \ln y_{itc}$}                \\\cmidrule(lr){2-2}\cmidrule(lr){3-5}
          &\multicolumn{1}{c}{(1)}         &\multicolumn{1}{c}{(2)}         &\multicolumn{1}{c}{(3)}         &\multicolumn{1}{c}{(4)}         \\
\hline
$\Delta \ln \underline{w}_{itc}$&   0.8977\sym{***}&   0.0258\sym{**} &                  &  -0.0272         \\
          & (0.0315)         & (0.0124)         &                  & (0.0199)         \\
[1em]
$\Delta \ln \underline{w}_{itc}^{\text{exp}}$&                  &                  &   0.0308\sym{**} &   0.0590\sym{**} \\
          &                  &                  & (0.0132)         & (0.0283)         \\
\hline
\vspace{-2mm}&                  &                  &                  &                  \\
R-squared &    0.943         &    0.022         &    0.022         &    0.022         \\
Observations&  107,814         &  107,814         &  107,814         &  107,814         \\
\hline\hline
\end{tabular}
}

	\begin{minipage}{0.95\textwidth}\footnotesize
		\vspace{3mm}	
		\textit{Notes:} The table shows versions of the static model using different dependent and 
		MW variables. Column 1 shows a regression of the change in the natural logarithm of the 
		experienced MW on the change in the natural logarithm of the statutory MW. Column 2, 3, and 4
		use the change natural logarithm of median rents per square foot in the SFCC category as 
		dependent variable. Column 2 uses the change in the natural logarithm of the statutory MW, 
		column 3 uses the change in the natural logarithm of the experienced MW, and column 4 uses both.
		All models control for time period fixed effects and, additionally, for our preferred set of 
		economic controls from the QCEW. 
		Standard errors clustered at the state level are reported in parenthesis. Significance codes: 
		*** $p < 0.01$, ** $p < 0.05$, * $p < 0.1$.
	\end{minipage}
\end{table}



