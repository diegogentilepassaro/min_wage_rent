%%%%%%%%%%%%%%%%%%%%%%%%%%%%%%%%%%%%%%%%%%%%%%%%%%%%%%%%%%%%%%%%%%%%%%%%%%%%%%%%%
%%%%%                             CONCLUSION                                 %%%%
%%%%%%%%%%%%%%%%%%%%%%%%%%%%%%%%%%%%%%%%%%%%%%%%%%%%%%%%%%%%%%%%%%%%%%%%%%%%%%%%%

We explore whether minimum wage changes affect housing rental prices, and 
whether MW shocks propagate spatially through commuting.
To answer this question we develop a theoretical approach that accounts for
the fact that MW workers typically reside and work in different locations.
Our model suggests that MW changes at workplaces will tend to increase
rents, and highlights the importance of accounting for the MW at the residence
location when estimating the effect of the workplace MW on rents.

We collect data on rents, statutory MW levels, and commuting flows, and estimate 
the effect of the residence and workplace MW on rents.
We find evidence supporting the main conclusions of our model: the workplace MW
increases rents, and thus MW policies spill over spatially through commuting.
Our conclusions are robust to a variety of robustness checks, and suggest
stronger effects in locations that are residence to more MW workers.
Our two-parameter model is able to capture rich heterogeneity in the effect 
of the MW on rents depending on the prevailing commuting structure.

To explore the incidence of the MW on landlords, we explore two counterfactual 
MW policies.
Our results suggest that landlords pocket a non-negligible portion of the newly 
generated wage income, and that this share varies spatially.
Because low-wage households are more affected by MW policies and more likely to 
be renters,
%%
%% SH: Verify this fact (compare average earnings of homeowners to average earnings 
%%                       of renters in census)
%%
the omission of the housing market channel would lead to an overstatement of the 
equalizing effects of the MW on disposable income.

Our analysis takes a partial equilibrium perspective, exploring the incidence 
of small increases in the MW within metropolitan areas.
However, one would expect general equilibrium adjustments to large changes 
in MW levels, such as worker mobility and changes in housing supply.
Exploring these issues in the context of a spatial model with worker mobility
that distinguishes between renters and homeowners appears as a fruitful 
avenue for future work.
