%%%%%%%%%%%%%%%%%%%%%%%%%%%%%%%%%%%%%%%%%%%%%%%%%%%%%%%%%%%%%%%%%%%%%%%%%%%%%%%%%
%%%%%                             CONCLUSION                                 %%%%
%%%%%%%%%%%%%%%%%%%%%%%%%%%%%%%%%%%%%%%%%%%%%%%%%%%%%%%%%%%%%%%%%%%%%%%%%%%%%%%%%

In this paper, we ask whether minimum wage changes affect housing rental prices.
To answer this question we develop a theoretical approach that accounts for
the fact that MW workers typically reside and work in different locations.
We show in a partial-equilibrium model that one should expect different effects
on rental prices in a given location depending on whether MW changes arise from 
the residence or the workplace location of its residents.
Our model suggests two summary statistics that should capture the effect of 
the MW on rents in a particular location, which we called the residence MW and 
the workplace MW.

We collect data on rents, minimum wage levels, and commuting patterns, and
estimate the effect of residence MW and workplace MW on rents as suggested by
the theoretical model.
The monthly frequency and high geographic resolution of our data allows us to 
analyze state-, county-, and city-level changes in the MW to identify the causal 
impact of raising the MW on the local rental housing market.
We find evidence supporting the main conclusions of our model: the workplace and 
residence MW have opposing effects on rents, and thus MW changes appear to 
spill over spatially through commuting.
Our two-parameter model is able to capture rich heterogeneity in the effect 
of the MW on rents depending on the prevailing commuting structure and 
gradient of statutory MW levels in metropolitan areas.
We use our results to explore the consequences of a counterfactual increase 
in the federal MW to \$9.
Our results suggest that homeowners pocket a non-negligible portion of the newly 
generated wage income by the new MW policy.
Because landlords tend to earn more income than low-wage workers,
%%
%% SH: Verify this fact (compare average earnings of homeowners to average earnings 
%%                       of renters in census)
%%
the omission of this channel would lead to an overstatement of the equalizing 
effects of the MW on disposable income.

Our paper has some important limitations.
First, we focus on price effects on rents.
However, in a longer time horizon residents and workers will likely change 
their residence and workplace locations within and across cities as a result 
of MW policies.
Studying the effects of these policies on the structure of a city and of the 
system of cities requires rich migration data and a different empirical approach
that endogenizes this channel.
Second, a full account of the welfare effects of the MW remains an open question.
We move a few steps into that direction by showing that landlords capture a 
share of the additional wage income that MW policies transfer to workers, and 
by pointing out that taking into account the spatial distribution of residents 
and workers is crucial to understanding where to expect effects on income and 
rents.
Exploring these issues in the context of a rich spatial model 
appears as a fruitful avenue for future work.
