%%%%%%%%%%%%%%%%%%%%%%%%%%%%%%%%%%%%%%%%%%%%%%%%%%%%%%%%%%%%%%%%%%%%%%%%%%%%%%%%%
%%%%%                             CONCLUSION                                 %%%%
%%%%%%%%%%%%%%%%%%%%%%%%%%%%%%%%%%%%%%%%%%%%%%%%%%%%%%%%%%%%%%%%%%%%%%%%%%%%%%%%%

In this paper, we ask whether minimum wage changes affect housing rental prices.
To answer this question we develop a theoretical approach that accounts for
the fact that MW workers typically reside and work in different locations.
We show in a partial-equilibrium model that one should expect different effects
on rental prices in a given location depending on whether they arise from the 
residence or the workplace location of its residents.

We collect data on rents, minimum wage levels and commuting patterns and 
estimate the effect of residence and workplace MW on rents.
The high frequency and resolution of our data allows us to analyze state-, county-, 
and city-level changes in the MW to identify the causal impact of raising the MW 
on the local rental housing market.
We find evidence supporting the main conclusions of our model: the effect of the 
MW is larger when it arises from the workplace.
Minimum wage changes appear to spill over spatially through commuting.

Our results suggest that homeowners pocket some portion of the increase in 
income of some workers generated by the MW.
The omission of this channel leads to an overstatement of the equalizing effects 
of the MW on disposable income.
