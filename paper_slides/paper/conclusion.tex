%%%%%%%%%%%%%%%%%%%%%%%%%%%%%%%%%%%%%%%%%%%%%%%%%%%%%%%%%%%%%%%%%%%%%%%%%%%%%%%%%
%%%%%                             CONCLUSION                                 %%%%
%%%%%%%%%%%%%%%%%%%%%%%%%%%%%%%%%%%%%%%%%%%%%%%%%%%%%%%%%%%%%%%%%%%%%%%%%%%%%%%%%

In this paper, we ask whether minimum wage changes affect housing rental prices. To answer this 
question we use rental listings from Zillow and MW changes collected from 
\textcite{VaghulZipperer2016} and \textcite{BerkeleyLaborCenter}, to construct a panel 
at the ZIP code-month level. The high frequency and resolution of our data allows us to analyze state, 
county, and city-level changes in the MW to identify the causal impact of raising the MW on the 
local rental housing market. 

To do that, we leverage on a panel difference-in-differences approach that 
exploits the staggered implementation and the intensity of hundreds of MW increases 
across thousands of ZIP codes. Our results indicate that minimum wage increases have a
positive impact on rents, which we find to be robust to different specifications. Our models
suggest that a 10 percent increase in MW causes rents to increase approximately by 
0.26 percent in the same month, and 0.5 to 0.6 percent in the long run. We go beyond the average 
MW effect and we look at the heterogeneity of effects across ZIP codes. 
We show that rents disproportionately increase in ZIP codes where: (i) it is more likely to find 
MW workers as residents; (ii) there is a lower share of college graduates;  (iii) a larger share 
of younger residents (15-24 yers old);  and (iv) a larger share of African-American residents. 

We then leverage LODES data to create a measure of experienced MW that accounts for the difference 
between residence and workplace location of MW earners. We are able in this way to better track the 
changes in MW that residents of different ZIP codes experience, which allows us to (i) estimate an 
arguably more relevant treatment effect when using the experienced MW instead of the statutory one,
and (ii) identify suggestive evidence of income transfers across ZIP codes to locations where MW
workers leave.

To assess the magnitude of the effect of MW on housing rents, we perform benchmarking exercises to 
recover the income-to-rent pass-through associated with the causal estimates. We use both QCEW 
county-quarter wage data, and results from \textcite{CegnizEtAl2019} to first recover the average 
wage elasticity to the MW used to obtain compute the pass-through. All exercises consistently show 
that a share approximately between 19 and 28 percent of the additional income generated by MW 
policies end up captured by landlords via rents.

Our results highlights that place-based policies aimed at the labor market can also have 
significant impacts on other related markets. In particular, MW provisions are usually thought as a 
way to guarantee economic means to low income workers. However, they may also be benefiting 
landlords in ways that are unintended. In this sense, studying how place-based policies affect the 
housing market becomes an important step to better understand income inequality across U.S. 
neighborhoods.
