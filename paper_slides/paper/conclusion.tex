%%%%%%%%%%%%%%%%%%%%%%%%%%%%%%%%%%%%%%%%%%%%%%%%%%%%%%%%%%%%%%%%%%%%%%%%%%%%%%%%%
%%%%%                             CONCLUSION                                 %%%%
%%%%%%%%%%%%%%%%%%%%%%%%%%%%%%%%%%%%%%%%%%%%%%%%%%%%%%%%%%%%%%%%%%%%%%%%%%%%%%%%%

We explore whether minimum wage changes affect housing rental prices.
To answer this question we develop a theoretical approach that accounts for
the fact that MW workers typically reside and work in different locations.
Our model suggests two summary statistics that should capture the effect of 
statutory MW changes on rents in a particular location, which we call the 
residence MW and the workplace MW.

We collect data on rents, statutory MW levels, and commuting flows, and estimate 
the effect of the residence and workplace MW on rents.
We find evidence supporting the main conclusions of our model: the workplace and 
residence MW have opposing effects on rents, and thus MW changes appear to 
spill over spatially through commuting.
Our two-parameter model is able to capture rich heterogeneity in the effect 
of the MW on rents depending on the prevailing commuting structure.

To gauge the distributional consequences of the effect of the MW on 
housing markets we explore two counterfactual MW policies.
We compute the incidence of these policies on landlords.
Our results suggest that landlords pocket a non-negligible portion of the newly 
generated wage income.
Because low-wage households are more affected by MW policies and more likely to 
be renters,
%%
%% SH: Verify this fact (compare average earnings of homeowners to average earnings 
%%                       of renters in census)
%%
the omission of the housing market channel would lead to an overstatement of the 
equalizing effects of the MW on disposable income.

Our analysis takes a partial equilibrium perspective, exploring the incidence 
of small increases in the MW within metropolitan areas.
However, one would expect general equilibrium adjustments to large changes 
in MW levels, such as worker mobility and changes in housing supply.
Exploring these issues in the context of a spatial model with worker mobility 
appears as a fruitful avenue for future work.
