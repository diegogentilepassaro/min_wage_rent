%%%%%%%%%%%%%%%%%%%%%%%%%%%%%%%%%%%%%%%%%%%%%%%%%%%%%%%%%%%%%%%%%%%%%%%%%%%%%%%%%
%%%%%                             CONCLUSION                                 %%%%
%%%%%%%%%%%%%%%%%%%%%%%%%%%%%%%%%%%%%%%%%%%%%%%%%%%%%%%%%%%%%%%%%%%%%%%%%%%%%%%%%

In this paper, we ask whether minimum wage changes affect housing rental prices.
To answer this question we develop a theoretical approach that accounts for
the fact that MW workers typically reside and work in different locations.
We show in a partial-equilibrium model that one should expect different effects
on rental prices in a given location depending on whether MW changes arise from the 
residence or the workplace location of its residents.

We collect data on rents, minimum wage levels and commuting patterns, and we
estimate the effect of residence and workplace MW on rents.
The monthly frequency and high geographic resolution of our data allows us to 
analyze state-, county-, and city-level changes in the MW to identify the causal 
impact of raising the MW on the local rental housing market.
We find evidence supporting the main conclusions of our model: the workplace and 
residence MW have different effects on rents, and therefore, minimum wage changes 
appear to spill over spatially through commuting. Our baseline results imply that 
a 10\% increase in the workplace MW measure increases rents about XXX percent. 
On the other hand, increasing the residence MW measure by 10\% implies that 
rents will go down around XXXX percent. This results imply that effects on rents are 
very heterogeneous depending on the prevailing commuting structure and gradient of 
statutory minimum wages in metropolitan areas. To report an elasticity that is somehow 
analogous to the ones  that have been often reported in the classic MW literature, 
we report that on average increasing 10\% the statutory MW will increase rents (through 
changes in both MW measures) by about XXXX percent. \footnote{This estimate is the sum of 
both the residence and workplace coefficients in our baseline model.}

Finally, we perform a counterfactual exercise of raising the federal MW to \$9. Our 
results suggest that homeowners pocket some portion of the newly generated wage income 
by the MW change. The omission of this channel leads to an overstatement of the equalizing 
effects of the MW on disposable income.

Our paper has some important limitations, that opens the door to future research. First,
we focus on the short term effects on rents. However, residents and workers can change their 
location within and across cities as a result of MW policies. The way that these policies shape 
the nature of a city and the national system of cities requires rich data on migration and an 
analysis focusing on a longer horizon.
Second, a full account of the welfare effects and distribution following changes in MW remains 
an open question. We move a few steps into that direction by showing that landlords are 
capturing a share of the additional wage income that MW policies transfer to workers, and 
by pointing out that taking into account the spatial distribution of residents and workers 
is crucial to understand where to expect effects on income and rents.  
