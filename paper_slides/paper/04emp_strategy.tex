%%%%%%%%%%%%%%%%%%%%%%%%%%%%%%%%%%%%%%%%%%%%%%%%%%%%%%%%%%%%%%%%%%%%%%%%%%%%%%%%%
%%%%%                         EMPIRICAL STRATEGY                             %%%%
%%%%%%%%%%%%%%%%%%%%%%%%%%%%%%%%%%%%%%%%%%%%%%%%%%%%%%%%%%%%%%%%%%%%%%%%%%%%%%%%%

In this section, we present the empirical strategy adopted to study the effect of MW on rents, and 
we discuss the assumptions needed for identification. We begin with a panel DiD model and we build 
on that following \textcite{MeerWest2016}. This allows us to estimate the full dynamics of rents 
around MW changes under various identifying assumptions. Our dynamic specifications are distinct 
from the usual DiD and event-study ones \parencite{BorusyakJaravel2017, abraham2018} for two main 
reasons: first, our models allow for the use of variation coming from more than one MW change per 
geographic unit and from geographic units that never experience a MW change. This is desirable 
because we both avoid under-identification issues with the two-way fixed effects and because we 
use never-treated zipcodes as control units. Secondly, our specifications not only exploit the 
timing of a MW change for identification but also its intensity.
    
% Whereas this shows a statistically significant results, one may worry about pre-trends that 
% confound the effect. To account for this, in the dynamic model we add leads and lags of this 
% variable, which allows us to formally test the hypothesis of pre-trends.


%%%%%%%%%%%%%%%%%%%%%%%%%%%%%%%%%%%%%%%%%%%%%%%%%%%%%%%%%%%%%%%%%%%%%%%%%%%%%%%%%
\subsection{Baseline Specifications}
Consider the following panel difference-in-differences model relating rents and the minimum wage:

\begin{equation}\label{eq:did_lev}
    y_{it} = \alpha_i + \alpha_t + \gamma_i t + \beta \underline{w}_{it} + \epsilon_{it}
\end{equation}
    
where $y_{it}$ is the log rent per square foot for the Zillow SFCC series, $\underline{w}$ is the 
log of the minimum wage, $\alpha_i$ is a zipcode fixed effect, $\alpha_t$ is a time fixed effect, 
and $\gamma_i$ is a zipcode-specific linear trend.\footnote{We add a zipcode-specific linear trend 
	to allow for heterogeneity in the time path of zipcodes \parencite{angrist2008mostly}. In the 
	next section we additionally present results from models without zipcode-specific linear trends 
	as well as with zipcode-specific quadratic trends.} 
We then re-write \autoref{eq:did_lev} in first differences:
    
\begin{equation}\label{eq:did}
        \Delta y_{it} = \theta_t + \gamma_i + \beta \Delta \underline{w}_{it} + \Delta \epsilon_{it}
\end{equation}

We reference this model as \textit{static DiD}. We spell out the model in first differences because 
we believe that the unobserved shocks to rental prices are likely to be persistent over time. Both 
the first differences and the level models are consistent under similar assumption but the first 
difference model is more efficient if the shocks are serially correlated \parencite{wooldridge2010}.

Identification comes from assuming that within a zipcode the change in the level of the logarithm 
of the minimum wage is mean independent of the change in the unobserved shock $\Delta \epsilon_{it}$ 
conditional on the time fixed effects and the zipcode-specific linear trend. This implies that if 
the true effect is a one-time level change, then $\beta$ has a causal interpretation and it can be 
seen as the elasticity of the rent per square foot to the MW.
    
One potential concern with the static DiD model, is that, despite controlling for a zipcode-specific 
linear trend, preexisting time-paths of rents per square foot might be different in zipcodes that 
had a MW change relative to zipcodes that did not experienced a change. To assess if that is the 
case, one can extend the model to include leads of $\Delta \underline{w}_{it}$. In addition, one 
may be believe that the effect of MW changes on rents is not a one time discrete level jump but that 
it also affects the growth rate of rental prices. In such cases the estimated coefficient $\beta$ 
from \autoref{eq:did} might only have limited relevance in evaluating the policy of interest 
\parencite{callaway2019difference}. To allow for dynamics in the effects, we extend the model to 
also include lags of $\Delta \underline{w}_{it}$. The \textit{dynamic} model is

\begin{equation}\label{eq:leads_lags}
    \Delta y_{it} = \theta_t + \gamma_i + \sum_{r=-s}^{s}\beta_r \Delta \underline{w}_{i(t-r)} 
    				+ \Delta \epsilon_{it} \ ,
\end{equation}
where $s$ is the number of months of a symmetric window around the MW change. Note that this 
dynamic DiD model still allows for treatment and control groups to have different averages, even though 
it now requires a more stringent identification: 
\[E \left[ \Delta \epsilon_{it} \Delta \underline{w}_{it-r} \big| \theta_{t}, \gamma_{i} \right] = 0
	\ \ \forall r\in\{-s, ..., -1, 0, 1, ..., s\} \ . \] 
In this context, a violation of the identification assumption would require a change in MW to be 
systematically correlated with unobserved shocks to treated zipcode relative to untreated ones. 
Importantly, this model allows us to test whether $\beta_{-s} = \beta_{-s+1} = ... = \beta_{-1} = 0$, 
the well known pre-trends test, to establish whether there are significant rent responses preceding 
a change in MW. Under the assumption of no pre-trends, we can gain efficiency through estimating a 
model only with distributed lags as follows:

\begin{equation}\label{eq:lags}
        \Delta y_{it} = \theta_t + \gamma_i 
        		+ \sum_{r=0}^{s}\beta_r \Delta \underline{w}_{i(t-r)} 
        		+ \Delta \epsilon_{it} \ .
\end{equation}

This model allows us to estimate the dynamics of the logarithm of the rent per square foot around 
changes in the MW and we can recover the elasticity of rents to MW by summing $\beta_0$ to 
$\beta_{s}$. We present results from this model in the results section. In past settings using 
yearly data \parencite{Tidemann2018, Yamagishi2019}, MW changes are so common in a given 
geographic area relative to the timespan of the data that it is very hard to credibly estimate the 
lags. Intuitively, this is the case because it is hard to distinguish which variation of the rental 
price is due to the current MW change or to a preceding one. In our estimates that concern is not 
justified, as given that we have month to month variation, we use short windows (5 months) in which 
there is no overlap in MW changes within a zipcode. The absence of pre-trends does not exhaust the 
potential threats to identification. Effects could still be driven by contemporaneous shocks 
systematically affecting both changes in rents and MW within a zipcode. To ease those concerns, we 
directly control for several county-level time-varying proxies of the health of the local labor and 
housing markets.\footnote{This amounts to adding a vector $\Delta X_{ct}$ on the right-hand side of 
	our models, where $c$ indexes counties, and we map zipcodes to a single county as explained in 
	\autoref{sec:data}.}

As mentioned in \autoref{sec:data}, part of the variation in the median rental price comes from 
unobserved changes in the Zillow inventory for a given zipcode through time. This may pose a threat 
to identification in the case which changes to MW directly affect the composition of rentals posted 
on the platform in a given zipcode-month period. Such concerns are partly mitigated by directly 
controlling for county-level time-varying housing market conditions, but we additionally investigate 
the issue by leveraging on the richer set of information Zillow provides on houses listed for sales. Specifically, we can track the number of houses listed \textit{for sale} in the selected zipcodes 
during the period 2013-2019 for our preferred house type (SFCC). We use such series to run a placebo 
regression where we estimate \autoref{eq:did} and \autoref{eq:leads_lags} using the (log) change in 
listings as outcome variable.  Significant effects of MW changes, or pronounced pre-trends will 
indicate that policy changes actually affect the Zillow inventory composition and cast doubt on the 
identifying assumption.

Finally, in our appendix, we consider a dynamic panel specification to allow for full dynamics on 
the rental prices. The model then becomes

\begin{equation}\label{eq:ab_panel}
        \Delta y_{it} = \Delta y_{i(t-1)} + \theta_t + \gamma_i 
        		+ \sum_{r=0}^{s}\beta_r \Delta \underline{w}_{i(t-r)} 
        		+ \Delta \epsilon_{it} \ .
\end{equation}

However, by construction we now have that $\Delta y_{i(t-1)}$ is necessarily correlated with 
$\Delta \epsilon_{it}$. To address that, we take two separate approaches. First, we follow 
\textcite{ArellanoBond1991} and, as it is customary in the literature, we instrument $\Delta 
y_{i(t-1)}$ with $\Delta y_{i(t-2)}$. Second, we follow \textcite{MeerWest2016} and instrument 
$\Delta y_{i(t-1)}$ with an off-window lag of the change in the logarithm of the MW. In particular, 
as most of our models have a window $s=5$, we use as an instrument $\Delta \underline{w}_{i(t-6)}$. 
Intuitively, if there is an effect of MW changes to rents past MW changes should predict future 
rents and past MW changes should not be correlated with contemporaneous unobserved determinants 
of rents once we take into account the dynamic effect of MW on rents. 


%%%%%%%%%%%%%%%%%%%%%%%%%%%%%%%%%%%%%%%%%%%%%%%%%%%%%%%%%%%%%%%%%%%%%%%%%%%%%%%%%
\subsection{Heterogeneity by Zipcode Characteristics }

In order to allow for heterogeneous effects based on zipcode characteristics, and to make sure 
that our effects are driven by the zipcodes that are expected to have more MW earners, we extend the 
baseline panel difference-in-differences model defined in \autoref{eq:did} by interacting the local 
MW change with zipcode level characteristics. To minimize the possibility of any characteristic 
being endogenous to MW changes, we use use socio-demographic data that predate our panel. We take 
them from the 2010 Census and the 5-years 2008-2012 ACS. Then, the model we take to the data becomes

\begin{equation}\label{eq:diff_main_hetero} 
    \Delta y_{it} = \theta_t + \gamma_i 
    		+ \sum_{q = 1}^4 \beta_q \mathds{1}\{i \in q\} \Delta \underline{w}_{it} 
    		+ \Delta \epsilon_{it} \ ,
\end{equation}
where $q$ identifies quartiles of some zipcode level characteristic, and $\mathds{1}\{ \cdot \}$ is 
the indicator function. We report results for these models in \autoref{sec:heter}.