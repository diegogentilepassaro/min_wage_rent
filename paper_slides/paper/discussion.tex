%%%%%%%%%%%%%%%%%%%%%%%%%%%%%%%%%%%%%%%%%%%%%%%%%%%%%%%%%%%%%%%%%%%%%%%%%%%%%%%%%
%%%%%                             DISCUSSION                                 %%%%
%%%%%%%%%%%%%%%%%%%%%%%%%%%%%%%%%%%%%%%%%%%%%%%%%%%%%%%%%%%%%%%%%%%%%%%%%%%%%%%%%

Our models consistently estimate a positive impact of MW changes on rents. We now 
assess whether the magnitude of the estimated effects are in line with the literature. 
We quantify the share of MW increases captured by landlords by computing the 
pass-through of the policy to rents. Interestingly, the rent increases generated are 
borne both by MW residents and residents not directly affected by changes in the MW.
This implies that policymakers---when ignoring rent responses to the MW---will both 
overestimate welfare increases for lower-income workers and ignore the real allocation 
of implicit costs. 

Our analysis assumes no disemployment effects. In fact, recent literature has highlighted 
the lack of overall employment responses to MW 
increases \parencite{CegnizEtAl2019}.\footnote{More generally, evidence on the impact on 
	overall employment is scant \parencite{dube2019impacts}. Most of the debate has focused 
	on teen employment \parencite{card1992using, allegretto2017credible} or specific sectors 
	\parencite{katz1992effect, card2000minimum, DubeEtAl2010}} 
The short-term focus of the analysis further minimizes the possibility of observing a 
significant change in employment. All exercises return remarkably similar pass-through 
figures: the average percentage change in median rents causally ascribed to changes in the 
MW makes up approximately $19$ to $28$ percent of the average percentage change in the total 
wage bill. Most notably, the pass-through appears to be close (although somewhat smaller) 
than the share of income spent on rent by renters \parencite{fernald2020americas}. 

Later, we use our pass-through estimates as input in a brief discussion of the policy 
implications of our results.

%%%%%%%%%%%%%%%%%%%%%%%%%%%%%%%%%%%%%%%%%%%%%%%%%%%%%%%%%%%%%%%%%%%%%%%%%%%%%%%%%
\subsection{Assessing the Magnitude of the Effects}\label{sec:discussion_benchmarking}

To benchmark our results, we perform a back-of-the-envelope calculation to obtain an 
approximate estimate of MW policies' pass-through rate from income to rents. Specifically, 
we compare the average causal effect we find with an estimate of the average percentage 
change in the wage bill for our sample. We obtain the latter figure in two ways. First, we 
use our quarterly QCEW wages at the county level combined with our MW measures 
to estimate the elasticity of earnings to MW empirically. As exemplified by 
\textcite{gentzkow2015newspapers}, we combine the two sources of data by taking MW averages 
at the county-quarter level.\footnote{We can estimate the linear regression at the 
	county-quarter level by exploiting the fact that the average of a linear function is a 
	linear function of the average. For more details, see \autoref{sec:app_econ_control}.} 
Secondly, we import estimates from \textcite{CegnizEtAl2019}. We compute the pass-through for
both the statutory and experienced MW measures.


\paragraph{QCEW Regression} In lack of more refined data on wages, we exploit the QCEW 
county-quarter panel for average weekly wages measured for private-sector occupations coupled 
with our MW measures from our ZIP code and monthly panels\footnote{According to the QCEW overview page 
	\textit{``wages data represent 
	the total compensation paid during the calendar quarter, regardless of when the services were 
	performed. Under most state laws or regulations, wages include bonuses, stock options, severance 
	pay, the cash value of meals and lodging, tips and other gratuities. In some states, wages also 
	include employer contributions to certain deferred compensation plans, such as 401(k) plans.}" For more information, see \href{https://www.bls.gov/help/def/en.htm\#all}{https://www.bls.gov/help/def/en.htm\#all}}. We set up a first-difference linear 
regression model at the county-quarter level. Our main explanatory variable is the change in 
the log of the MW obtained by averaging the original data both geographically (using the number 
of housing units as weights) and temporally. Crucially, we then rely on the fact that we can 
compute the average monthly change within a quarter given a variable values at the beginning 
and at the end of that quarter. To reduce the amount of bias, and to obtain a comparable 
estimate with the previous exercise, we focus on a static model. Additionally, we only 
select counties spanning the underlying set of ZIP codes in the final rent panel. We estimate 
the following model:

\begin{equation}
	\overline{\Delta \ln w}_{cq} = \alpha_{q} + \lambda_{c} 
							+ \rho \ \overline{\Delta \ln \underline{w}}_{cq}
							+ \nu_{cq} ,
\end{equation}
where $w_{cq}$ represents average weekly wages in county $c$ and quarter $q$. In the 
first-difference model, $\lambda_{c}$ controls for county-level linear trends. The parameter of 
interest is $\rho$, which captures the elasticity of wages with respect to MW changes.

We obtain statically significant impact of 0.138 (s.e\. 0.038). We report the point estimate, 
along with that of our baseline effect of MW on rents, in column 2 of Panel A in 
\autoref{tab:passthrough}. The implied pass-through is 18.9 percent. Roughly a fifth of the 
additional income generated by the policy ends up captured by landlords via higher rents. As 
previously discussed in \autoref{sec:experienced_mw}, the use of changes in the statutory 
MW likely introduces measurement error due to the fact that MW workers do not necessarily 
live in affected ZIP codes. We therefore re-estimate the model using the experienced MW 
variable constructed in \autoref{sec:experienced_mw}, and report point estimates in column 
1 of Panel B in \autoref{tab:passthrough}. The implied pass-through is almost unchanged given 
that both the effect on rents and the estimated increase in the wage bill are slightly larger.

\paragraph{Comparison to literature} Our approach to estimating the elasticity of average wages
to the MW may be incorrect, and thus our pass-through numbers biased. After all, our empirical
approach was not crafted to answer this question. For this reason, in \autoref{tab:passthrough}, 
column 2 we use results from \textcite{CegnizEtAl2019} to compute the elasticity of average wages 
to thr MW. The authors use an event-study design to estimate the impact of MW increases on the 
entire frequency distribution of wages. From that we are able to back out the effect of MW 
changes on average wages. The authors estimate a 6.8 percent increase in wages following a MW
event, and compute sample-averaged MW increase of 10.1 percent \parencite[][Table I]
{CegnizEtAl2019}. We couple such numbers with our estimated sample average share of MW workers 
across zipcodes to obtain a 0.115 percent increase in the total wage bill (panel A). This 
corresponds to a pass-through ratio of 22.7 percent, a remarkably similar number to those 
obtained using QCEW-based estimates. Although \textcite{CegnizEtAl2019} focus on the statutory MW, 
we couple their results with estimates obtained using the experienced MW in Panel B. Using the 
experienced MW variable changes the share of MW workers that we identify, and therefore it 
slightly affects the average wage elasticity, which is now 0.112. We obtain in this case a 
27.7 percent pass-through, more in line with the overall average share of earnings spent on 
rents in the U.S. \parencite{fernald2020americas}.   

\begin{table}[!h] 
	\centering
	\caption{Pass-Through of MW Changes to Rents}
	\label{tab:passthrough}  
	{
\def\sym#1{\ifmmode^{#1}\else\(^{#1}\)\fi}
\begin{tabular}{l*{4}{c}}
\hline\hline
            &\multicolumn{1}{c}{(1)}&\multicolumn{1}{c}{(2)}&\multicolumn{1}{c}{(3)}&\multicolumn{1}{c}{(4)}\\
            &\multicolumn{1}{c}{\shortstack{r}}&\multicolumn{1}{c}{\shortstack{QCEW \\ regression}}&\multicolumn{1}{c}{\shortstack{QCEW Regression + \\ Experienced MW}}&\multicolumn{1}{c}{\shortstack{Dube et al. (2019) + \\ Experienced MW}}\\
\hline
Effect on Rents&   0.123&   0.026&   0.031&   0.031\\
Effect on Wages&   0.214&   0.045&   0.058&   0.112\\
Pass-Through&   0.575&   0.577&   0.530&   0.277\\
\hline\hline
\end{tabular}
}

	\begin{minipage}{0.95\textwidth} \footnotesize
		\vspace{2mm}
		\textit{Notes:} The table reports back-of-the-envelope computation of MW 
		policy pass-through to rents. This is obtained by taking the ratio of the 
		estimated rent elasticity to MW and the average wage elasticity to MW. 
		In Panel A we report the pass-through obtained using the statutory MW as 
		main dependent variable. In column (1) we combine the rent elasticity with 
		the average wage elasticity to MW obtained through QCEW-based regression
		analysis, as explained in the text. In column 2 we use estimates
		from \textcite{CegnizEtAl2019} to recover the average wage elasticity to MW. 
		Panel B replicates the exercises conducted in Panel A while replacing the 
		statutory MW with the experience MW. 
	\end{minipage}
\end{table}

%%%%%%%%%%%%%%%%%%%%%%%%%%%%%%%%%%%%%%%%%%%%%%%%%%%%%%%%%%%%%%%%%%%%%%%%%%%%%%%%%
\subsection{Policy Implications}\label{sec:policy}

Our analysis highlighted how MW changes causally increase rents at a local level: a 10 
percent increase in MW generates a a contemporaneous effect of 0.25 percent in ZIP code 
level median rents, and a cumulative 0.6 percent increase over the course of a semester. 
Thus, the static effect can be interpreted as a lower bound for the effect of MW changes on rents. 
In addition, this effect disproportionately impacts those neighborhoods characterized by 
socio-demographic features associated with economic insecurity: younger population, lower 
education, higher share of African-American population and, more generally, a higher share 
of residents earning the MW. We also showed that accounting for workplace and 
residence differences of MW workers implies a larger effect on the housing market.

First and foremost, such results suggest how MW policies affect the housing market and 
spill over geographically. This stresses the importance for policymakers to account for 
the link between income-supporting interventions aimed at relieving the conditions of 
populations at risk, and the broader impact that such policies have over housing 
instability for those very same households. \textcite{fernald2020americas} reports that 
the share of high-income renters (above \$75,000 per year) have contributed for more than 
three-quarters of rental demand growth in the 2010-2018 period, making up to 45 percent of 
the total renter population. This, coupled with a very inelastic housing demand and very low 
vacancy rates, have created greater pressure on lower-income households who experience the
effects of higher rental prices. 
In such a framework, our analysis presents evidence suggesting that the effect of MW policies 
on rents reduces the efficacy of policies designed to curb local income inequality. By 
identifying the differential effect on rents in workplace versus residence location, our
heterogeneity results underscore the presence of income redistribution across renters 
living (and consuming) in ZIP codes with different shares of MW residents. Answering these 
questions in a definitive way would require a welfare analysis beyond the scope of this paper. 
Nevertheless, we conduct back-of-the-envelope calculations suggesting that between 20 and 
30 percent of the income generated by MW policies ends up absorbed in higher rents.

\textcite{allegretto2018local} state how price increases in the restaurant industry represents 
the primary mechanism trough which MW are (fully) absorbed. If we focus on restaurants only, 
the burden is passed from owners to customers. Conditional on the consumption habits of higher 
and lower income households (and assuming a negligible impact on employment), the latter group 
appears to rip the majority of the benefits entailed by MW policies. However, as those workers
take the extra income to their residence location, we observe a significant pass-through to 
rents there. Our results suggest that an informed policy debate cannot overlook that additional 
pass-through for two main reasons. First, the evidence provided suggest how rents increase more 
in neighborhoods with younger, less-educated and more African-American population. The extent 
to which these groups actually benefit from the additional income provided is of paramount 
importance, given that in 2019 households earning on average less than \$15,000 had less than 
\$410 left each month after paying rent and utilities \parencite{fernald2020americas}. Secondly, 
our results indicate that rising rent prices impact not only treated workers, but they also
transmit to other workers through the housing market. Hence, to further understand the impact 
on local inequality and urban sorting, it would be important for future research to uncover the 
total welfare impact of MW policies. 

 