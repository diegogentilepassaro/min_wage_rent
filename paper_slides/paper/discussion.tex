%%%%%%%%%%%%%%%%%%%%%%%%%%%%%%%%%%%%%%%%%%%%%%%%%%%%%%%%%%%%%%%%%%%%%%%%%%%%%%%%%
%%%%%                             DISCUSSION                                 %%%%
%%%%%%%%%%%%%%%%%%%%%%%%%%%%%%%%%%%%%%%%%%%%%%%%%%%%%%%%%%%%%%%%%%%%%%%%%%%%%%%%%
Our results suggest a small-yet-robust impact of MW on rents. From a policy perspective, 
the presence of spillovers in the housing market makes it important to benchmark such results 
and to assess the welfare consequences of MW polices. 
We quantify the total share of additional MW income that is captured by landlord 
by computing the policy pass-through to rents. Interestingly, the rent increases generated is 
borne both by MW residents, but also by residents not directly affected by changes in the MW. 
This implies that policy makers -- when ignoring rent responses to MW - will both overestimate 
welfare increases for lower-income workers as well as ignore the real allocation of implicit costs. 

Due to the short-term focus of our analysis, we abstract from employment effects. Additionally, 
the recent literature has indeed highlighted the lack of overall employment responses 
\parencite{CegnizEtAl2019}.\footnote{More generally, evidence on the impact on overall 
	employment is scant \parencite{dube2019impacts}, while most of the debate has focused on 
	teen employment \parencite{card1992using, allegretto2017credible} or specific sectors 
	\parencite{katz1992effect, card2000minimum, dube2010minimum}} All exercises return 
remarkably similar pass-through figures: the average percentage change in median rents causally 
ascribed to changes in the MW makes up approximately $19$ to $28$ percent of the average 
percentage change in the total wage bill. Most notably, the pass-through appears to be close 
to the share of income spent on rent by renters \parencite{fernald2020americas}.
%%%%%%%%%%%%%%%%%%%%%%%%%%%%%%%%%%%%%%%%%%%%%%%%%%%%%%%%%%%%%%%%%%%%%%%%%%%%%%%%%
\subsection{Benchmarking}\label{sec:discussion_benchmarking}
To benchmark our results, we perform a back of the envelope calculation allowing us to get a ballpark estimate 
of the pass-through rate of MW policies from income to rents. Specifically, we compare the average 
causal effect we find with an estimate of the average percentage change in the wage 
bill for our sample. We obtain the latter figure in two ways: 
%first, we compute that using an intuitive formula based on ACS data. These computations are 
%clearly based on several simplifying assumptions and must therefore be taken as an approximation 
%of the true magnitude of the pass-through. 
First, we use county-quarter QCEW data on wages 
combined with zipcode-month MW data to empirically estimate the elasticity of earnings to MW. 
As exemplified by \textcite{gentzkow2015newspapers}, we combine the two sources of data by 
taking MW averages at the county-quarter level.\footnote{We can estimate the linear 
	regression at the county-quarter level by exploiting the fact that the average of a linear 
	function is a linear function of the average. For more details, see Jesse Shapiro's notes at \href{https://jmslab.github.io/DataRecipes/recipes/datarecipe_107.pdf}{\texttt{https://jmslab.github.io/DataRecipes/recipes/datarecipe\_107.pdf}}.} 
Secondly, we import estimates from \textcite{CegnizEtAl2019}. We compute the pass-through for
both statutory and experienced MW. 



%Lastly, we provide back-of-the-envelope calculations to help assessing the welfare implications of 
%having landlords capturing part of the additional income generated by MW changes. We first perform three
%different exercises that allow us to quantify the average impact of MW changes on wages. Equipped with these 
%measures, we are then able to obtain a ballpark estimates of the implied pass-through by comparing them to 
%the our main results. We initially compute the average impact of MW on wages with a simple formula based on 
%ACS data, and we get a 57.5 percent pass-through. Then, we alternatively use county-quarter QCEW data to run a 
%DiD regression of wage changes on the average MW change for this level of geographical and time aggregation.\footnote{see 
%	section XX for more details on the aggregation methodology and identification assumptions.} 
%We use to such estimates to obtain a 57.7 percent pass-through. This figure decreases to 53
%percent when replacing the statutory MW with the LODES-based experienced MW measure. While these computations are
%likely to overestimate the real magnitude of the pass-through as they rely on several simplifying assumptions, they 
%nevertheless return strikingly similar numbers. To reduce the margin of error,  we finally use estimates of the impact of MW 
%on wages taken from the literature \parencite{CegnizEtAl2019}. We obtain an implied pass-through of 27.7 percent. 

%\paragraph{Formula-based Approach}
%First, we use 2013-2018 ACS data to approximate the zipcode-level share of minimum wage 
%workers, $l_{i}$. We use ACS data later than 2010 reflecting the fact that 62 percent of the 
%MW changes in our sample come from the 2015-2019 period. A limitation of using ACS 
%aggregated data is that we cannot leverage on the number of hours worked to assess who is a 
%MW worker and who is not. We therefore adopt an different approach by estimating the hypothetical 
%monthly wage for workers earning the statutory minimum for a given zipcode. To do that, 
%we further assume that the distribution of full- and part-time workers is uniform and orthogonal 
%to the MW status. While this most likely leads to overestimate the actual number of MW workers, 
%it allows us to use the overall distribution of full- and part-time workers to 
%approximate that of MW ones, not provided in the ACS. 
%
%We initially compute the full- and part-time \textit{annual} MW, $amw_{ij}$, $j\in \{f, p\}$, 
%for each zipcode $i$ as follows: 
%
%\begin{align}
%	amw_{if} &= MW_{i, 13/18} * 40 * 4.35 * 12 \\
%	amw_{ip} &= MW_{i, 13/18} * 20 * 4.35 * 12
%\end{align}
%
%Here $MW_{i, 13/18}$ represents the average 2013-2018 hourly minimum wage for 
%zipcode $i$. Equipped with these thresholds, we then use the binned distribution of individual income for 
%full- and part-time workers to obtain, for each zipcode $i$, a proxy for the number of MW workers 
%($l_{if}$ and $l_{ip}$, respectively).\footnote{Whenever MW estimates fall within a bin, we include 
%	that only when more than half of the range lies below the threshold.} 
%
%Once we established the sample average number of full- and part-time MW workers, we use 
%these estimates to obtain the average percentage change in total monthly wage bill $\overline{\Delta mmw}$ 
%associated with the average change in MW in our sample, $\overline{\Delta MW}$. To do that, we first 
%compute the change in the wage bill associated with rising wages for full- and part-time MW workers. 
%These in turn are obtained as the product of the change in monthly income caused by $\overline{\Delta MW}$ 
%and the estimated average number of MW workers $\overline{l_{if}}$ and $\overline{l_{ip}}$: 
%
%\begin{align}
%	\overline{\Delta W}_{f} = \overline{\Delta mmw}_{f} * \overline{l_{f}} = (\overline{\Delta MW} * 40 * 4.25) * \overline{l_{f}} \\ 
%	\nonumber \\
%	\overline{\Delta W}_{p} = \overline{\Delta mmw}_{p} * \overline{l_{p}} = (\overline{\Delta MW} * 20 * 4.25) * \overline{l_{p}}
%\end{align}
%
%
%We finally obtain the percentage change in the monthly total wage  bill, $\overline{\% \Delta W}$, 
%by using the formula: 
%
%\begin{align}\label{eq:benchmark_Dwage_pct}
%	\overline{\% \Delta W} = \frac{\overline{\Delta W}_{f} + \overline{\Delta W}_{p}}{\overline{I}}
%\end{align}
%
%We use the average aggregate income reported in the ACS as our denominator.\footnote{In 
%	order to compute the \textit{monthly} aggregate income for each zipcode $i$, we divide 
%	the ACS annual estimates reported by 12.}
%
%Equipped with such number, we then compare it to the average percentage change in rent that 
%can be causally imputed to MW changes, obtained multiplying the estimated elasticity $\hat{\beta}$ 
%by the average MW change in percentage terms: 
%
%\begin{align}
%	r = \frac{\overline{\% \Delta m}}{\overline{\% \Delta W}} = \frac{\hat{\beta}*\overline{\% \Delta MW}}{\overline{\% \Delta W}} 
%\end{align}
%
%This ratio allows us to then asses the magnitude of our estimated effect: how much of the 
%increase in income ends up causally increasing rents? We report our estimates for $r$ in 
%\autoref{tab:passthrough}, column 1. We can see how our sampled MW changes lead to a $0.123$ 
%percent increase in rents and a $0.214$ percent increase in the total wage bill. This corresponds 
%to an implied pass-trough of $57.5$ percent.\footnote{We can apply the same reasoning to compute 
%	the whole sample distribution for $r$. Results for the average implied pass-through are very similar (0.612).}
%As specified before, this is just an approximation based on several simplifying assumptions, but it 
%nevertheless shows how the estimated effect of MW wages on rents is within a plausible range. 

\paragraph{QCEW Regression} To validate our results, we provide a 
regression-based estimation of the elasticity of wages to MW. In lack of finer data on wages, 
we exploit the QCEW county-quarter panel for average weekly wages measured for all occupations 
in the private sector coupled with zipcode-month level data on MW. We set up a first-difference 
linear regression model at the county-quarter level, with our main explanatory variable being the 
changes in (log) MW obtained by averaging the original data both geographically (using the 
number of housing units as weights) and temporally. Our main explanatory variable will now capture 
for each county-quarter, the average change across zipcodes and across the 3 underlying months 
in wages. Crucially, we then rely on the fact that the average change in rent in a quarter can be calculated 
given rents at the beginning and at the end of that quarter. To reduce the amount of bias, and to obtain 
a comparable estimate with the previous exercise, we focus on a static DID model. Additionally, we only 
select counties spanning the underlying set of zipcodes in the final rent panel. We estimate the following model: 

\begin{align}
	\overline{\Delta w}_{cq} = \rho \overline{\Delta \underline{w}}_{cq} + \alpha_{q} + \lambda_{c} + \nu_{cq}
\end{align}

In the first-difference model, $\lambda_{c}$ controls for county-level linear trends. 
The parameter of interest is $\rho$ which captures the elasticity of wages with respect to 
MW, $\overline{\% \Delta W}$. 

We obtain statically significant impact of $0.138$ (s.e. $0.038$). 
We report the point estimate, along with that of our baseline effect of MW on rents, in 
Table \ref{tab:passthrough}, Panel A column 2. The implied pass-through is $18.9$ percent: 
roughly a fifth of the additional income generated by the policy ends up captured by landlords 
via higher rents. As previously mentioned in \autoref{sec:experienced_mw}, the use of changes in the statutory 
MW likely introduces measurement error due to the fact that MW workers do not necessarily 
live in affected zipcodes. In the present context, this would result in measurement error in 
the variable of interest $\overline{\Delta \underline{w}}_{cq}$. We therefore re-estimate the model 
using the experienced MW variable constructed in \autoref{sec:experienced_mw}, and report the point 
estimates in Table \ref{tab:passthrough}, Panel B column 1. The implied pass-through does not
change. \\

\paragraph{Comparison to literature.} While previous analysis return consistently stable 
pass-through shares, they inherently are rough approximations based on multiple simplifying 
assumptions and lightweight modeling. The estimated elasticity of income to MW, for example, 
might be biased due to unobserved confounders. For this reason, in Table \ref{tab:passthrough}, 
column 2 we use results from \textcite{CegnizEtAl2019} to compute the elasticity of 
average wage to MW. The authors use a DID design to estimate the impact of 
MW increases on the entire frequency distribution of wages. From that they are able to back out the effect of 
MW on average wages, 6.8 percent with a sample-averaged MW increase of 10.1 percent 
\parencite[Table I]{CegnizEtAl2019}. We couple such numbers with our estimated sample share 
of MW workers to obtain an $0.115$ percent increase in the total wage bill (panel A). This corresponds to 
a pass-through ration of $22.7$ percent, a remarkably similar number to those obtained using 
QCEW data. We finally couple their results with estimates originated from the experienced MW that
indicate a $27.7$ percent pass-through, more in line with the 
overall average share of earnings spent on rent in the U.S. \parencite{fernald2020americas}.   

\begin{table}[!h] 
	\centering
	\caption{Pass-Through of MW Changes to Rents}
	\label{tab:passthrough}  
	{
\def\sym#1{\ifmmode^{#1}\else\(^{#1}\)\fi}
\begin{tabular}{l*{4}{c}}
\hline\hline
            &\multicolumn{1}{c}{(1)}&\multicolumn{1}{c}{(2)}&\multicolumn{1}{c}{(3)}&\multicolumn{1}{c}{(4)}\\
            &\multicolumn{1}{c}{\shortstack{r}}&\multicolumn{1}{c}{\shortstack{QCEW \\ regression}}&\multicolumn{1}{c}{\shortstack{QCEW Regression + \\ Experienced MW}}&\multicolumn{1}{c}{\shortstack{Dube et al. (2019) + \\ Experienced MW}}\\
\hline
Effect on Rents&   0.123&   0.026&   0.031&   0.031\\
Effect on Wages&   0.214&   0.045&   0.058&   0.112\\
Pass-Through&   0.575&   0.577&   0.530&   0.277\\
\hline\hline
\end{tabular}
}

	\begin{minipage}{\textwidth} \footnotesize
		\vspace{2mm}
		\textit{Notes:} The table reports back-of-the-envelope computation of MW 
		policy pass-through to rents. This is obtained by taking the ration of the 
		estimated rent elasticity to MW, and the average wage elasticity to MW. 
		In Panel A we report the pass-through obtained using the statutory MW as 
		main dependent variable. In column (1) we combine the rent elasticity with 
		the average wage elasticity to MW obtained through QCEW-based regression
		analysis (\autoref{sec:discussion_benchmarking}). In column 2 we use estimates
		from \textcite{CegnizEtAl2019} to recover the average wage elasticity to MW. 
		Panel B replicate the exercises conducted in Panel A while replacing the statutory
		MW with the experience MW. 
	\end{minipage}
\end{table}

%%%%%%%%%%%%%%%%%%%%%%%%%%%%%%%%%%%%%%%%%%%%%%%%%%%%%%%%%%%%%%%%%%%%%%%%%%%%%%%%%
%\subsection{The magnitude of the estimates}\label{sec:benchmark}
% ARE WE KEEPING THIS ?  HOW ROBUSTLY CAN WE JUSTIFY THE PARAMETERS CHOICE?
%In this subsection we use the model introduced in \autoref{sec:model} combined with some auxiliary 
%assumptions to assess whether the magnitude of our estimates is plausible.
%
%Assume that functions that characterize supply and demand of rental units are constant elasticity, 
%so that $\underline{\gamma}$ and $\underline{\beta}$ are the elasticity of MW households demand to 
%rents and income, $\overline{\gamma}$ and $\overline{\beta}$ are analogous parameters for non-MW 
%households, and $k$ is the elasticity of housing supply to rents.\footnote{More precisely, we 
%	assume that $\underline{H}(r, \underline{w}) = A e^{\underline{\gamma} \ln r + \underline{\beta} 
%		\ln\underline{w}}$, $\overline{H}(r, w) = B e^{\overline{\gamma} \ln r + \overline{\beta} \ln 
%		w}$, and $H(r) = C e^{k \ln r}$. $A, B, C > 0$ are constants.}. 
%As a result, it can be shown that \autoref{eq:model-elasticity} takes the form
%
%\begin{equation}\label{eq:benchmarking-elasticity}
%\rho = \frac{\underline{\beta} \ \underline{s}}
%{k - \underline{\gamma} \ \underline{s} 
%	- \overline{\gamma} \overline{s}}
%\end{equation}
%where $\underline{s} = \frac{\underline{H}}{H}$ is share of housing occupied by MW households 
%and $\overline{s} = \frac{\overline{H}}{H}$ is the share of housing occupied by non-MW households. 
%Note that $\overline{s} = 1 - \underline{s}$.
%
%The above expression is intuitive, in the sense that factors which increase housing demand make 
%the elasticity higher, whereas factors that increase supply lower it. For instance, a higher 
%$\underline{\beta}$ --elasticity of housing demand to income-- implies a higher $\rho$, whereas 
%a higher $k$ --elasticity of housing demand to rents-- implies a lower $\rho$.
%
%Suppose that the share of MW is $\underline{s} = 0.3$, so that $\overline{s}=0.7$. Assume that 
%demand elasticities and $(\underline{\gamma}, \overline{\gamma}, \underline{\beta}) = (- 0.7, 
%- 0.5, 0.1)$, implying that MW households are more sensible to increases in rents, and that 
%demand for housing is price-inelastic and a normal good. Finally, let $k = 0.1$, similar to 
%estimate of \textcite[][Table 5]{Diamond2016}. Substituting these values in 
%\eqref{eq:benchmarking-elasticity} results in an elasticity of 0.45. This value turns out to be 
%very close to the cumulative sum of our $t$ and $t-1$ coefficients.


%%%%%%%%%%%%%%%%%%%%%%%%%%%%%%%%%%%%%%%%%%%%%%%%%%%%%%%%%%%%%%%%%%%%%%%%%%%%%%%%%
\subsection{Policy Implications}\label{sec:policy}

Our analysis highlighted how MW changes causally increase rents at a local level: a 10 percent
increase in MW generates a simultaneous 0.25 percent spike in zipcode-level median rents, and 
a cumulative 0.6 percent increase over the course of a semester. In addition, this effect 
disproportionately impacts those neighborhoods characterized by socio-demographic
features associated with economic insecurity: younger population, lower education, 
higher share of African-American population and, more generally, a higher share of residents 
earning MW. 

First and foremost, such results suggest how MW policies generate spillovers in the housing market. 
This stresses the importance for policymakers to account for the link between 
income-support interventions aimed at relieving the conditions of populations-at-risk, 
and the broader impact that such policies have over housing instability for those very same households.
\textcite{fernald2020americas} reports how the share of high-income renters ($>\$75,000$/year) have contributed for more
than three-quarters of rental demand growth in the 2010-2018 period, making up to 45 percent 
if the total renter population. This, coupled with a very inelastic housing demand and very low 
vacancy rates, have created greater pressure on lower-income households. In such framework, our analysis 
therefore present evidence suggesting that the very same policies designed to curb local income inequality
might actually contribute to exacerbating it. Answering this questions in a definitive way 
would require a welfare analysis beyond the scope of this paper. We nevertheless show how back-of-the
-envelope calculations suggests that between 20 and 30 percent of the income generated by MW policies 
ends up absorbed in higher rents.

\textcite{allegretto2018local} state how price increases in the restaurant industry represents the primary mechanism
trough which MW are (fully) absorbed.  If we focus on restaurants only, the burden is passed from owners to customers. 
Conditional on the consumption habits of higher and lower income households (and assuming a negligible impact on 
employment), the latter group appear to rip the majority of the benefits entailed by MW policies. Our results suggest that 
an informed policy debate cannot overlook the additional pass-through operating via-rents for two main reasons. 
First, the evidence provided suggest how rents increase more in neighborhoods with higher level of unemployment and 
lower level of education. The extent to which these groups can actually benefit from the additional income provided 
is of paramount importance, given that In 2019 households earning on average less than $\$15,000$ 
had less than $\$410$ left each month after paying rent and utilities \parencite{fernald2020americas}. Secondly, our 
results indicate how rising rent prices impact not only treated workers, but all zipcode renters. To understand  the further 
impact on local income inequality and urban sorting becomes hence of primary importance for future policy research 
whose goal is to uncover the welfare impact of MW legislation. 




 