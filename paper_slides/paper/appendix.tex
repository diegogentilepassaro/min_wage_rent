%%%%%%%%%%%%%%%%%%%%%%%%%%%%%%%%%%%%%%%%%%%%%%%%%%%%%%%%%%%%%%%%%%%%%%%%%%%%%%%%
%%%%%                              APPENDIX                                 %%%%
%%%%%%%%%%%%%%%%%%%%%%%%%%%%%%%%%%%%%%%%%%%%%%%%%%%%%%%%%%%%%%%%%%%%%%%%%%%%%%%%

\section{A dynamic supply and demand model}\label{sec:dyn_theory_model}

The geography is represented by a set of ZIP codes $\Z$.
There is an exogenously given distribution of people with differing 
residence $i$ and 
workplace $z$ locations across these ZIP codes which,
as in the main body of the paper, we denote by $\{L_{iz}\}_{i,z\in\Z\times\Z}$.

Let $H_{it}$ be the stock of square feet rented in period $t$, where $t$ is 
characterized by a month $m$ and year $y$.
This stock is composed of contracts starting at different calendar months. 
We assume that all contracts last for one year.
We impose that $H_{it} \leq S_i$ for all $t$, 
where $S_i$ denotes the total number of available square feet in $i$.

We further decompose $H_{it}$ as follows.
Let $h_{izt} = h_{iz}\left(R_{it}, \MW_{it}, \MW_{zt}\right)$ be the 
per-capita demand of housing of group $(i,z)$ in period $t$,
which depends on the prevailing MW at the time of contract sign-up.
We assume that this demand function is decreasing in residence MW and decreasing
in workplace MW, just as in Section \ref{sec:model}.
For simplicity, we omitted the mediation channels of prices and income.
Let $\lambda_{it}$ denote the share of $i$'s residents who started their contracts 
in period $t$.%
\footnote{We assume that these shares do not vary by workplace.}
Then, we can write the stock of contracted square feet during period $t$ as
$$
H_{it} = \sum_{\tau = t - 11}^{t} \lambda_{i\tau} \sum_{z\in\Z} L_{iz} 
h_{iz\tau} (r_{i\tau}, \MW_{i\tau}, \MW_{z\tau})
$$
where 
$r_{i\tau}$ represents rents \textit{per square foot} in period $\tau$.
It is convenient to define the stock of contracted square feet excluding the ones 
that were signed 12 months ago. We denote them by
$$
\tilde H_{it} = \sum_{\tau = t - 10}^{t} \lambda_{i\tau} \sum_{z\in\Z} L_{iz} 
h_{iz\tau} (r_{i\tau}, \MW_{i\tau}, \MW_{z\tau}) .
$$

We assume that all square feet are homogeneous, and so they have the same price 
in the market.

\subsection*{Within-period equilibrium}

Within this simple model, we assume the following timing: 
(1) At the beginning of period $t$, a share $\lambda_{it}$ of people's contracts 
expire (the ones that started on $t-12$);
(2) The square feet from expiring contracts are added to the pool of available 
rental space for new renters;
(3) Renters in $t$ and a flow supply of rental space in $t$ determine equilibrium 
rents $R_{it}$.
We now develop each of these steps more formally.

As of the start of every period $t$, $\lambda_{i,t-12} \sum_z L_{iz} h_{iz,t-12}$ 
square feet become available for rent from each group of workers $(i,z)$.
The square feet available to rent in period $t$ (vacant) are then
$$
\lambda_{i,t-12} \sum_z L_{iz} h_{iz,t-12} + (S_i - H_{i,t-1}) = S_i - \tilde H_{i,t-1}.
$$
Note that $(S_i - H_{i,t-1})$ are the non-rented square feet as of $t-1$.

We denote by $V_{it}(R_{it}, \lambda_t)$ the supply of housing, increasing in 
$R_{it}$.
A feasibility constraint is that 
\begin{equation}\label{eq:feasibility}
    V_{it}(R_{it}, \lambda_t) \leq S_i - \tilde H_{i,t-1} .
\end{equation}

The flow demand for new rentals in $t$ by those whose contract expired is given by
$$
\lambda_{it} \sum_z L_{iz} h_{izt} \left(R_{it}, \MW_{it}, \MW_{zt} \right) .
$$
This demand arises because a share of the ZIP code's contracts expired. 
Those people go to the market and may desire to rent more square feet given changes 
in their income.

The market in period $t$ cleans if
\begin{equation}\label{eq:equilibrium_dynamic}
    \lambda_t \sum_z L_{iz} h_{iz} \left(R_{it}, \MW_{it}, \MW_{zt} \right) = 
    V_{it}(R_{it}, \lambda_t) .
\end{equation}
Given minimum wages in $t$, $\{\MW_{it}\}_{i\in\Z}$,
the share of workers looking to rent in period $t$, $\lambda_t$, 
and a number of vacancies that satisfies \eqref{eq:feasibility}, 
equation \eqref{eq:equilibrium_dynamic} determines equilibrium rents in period $t$.
Because the properties of housing demand and housing supply are the same as in the
model in Section \ref{sec:model},
the equilibrium condition \eqref{eq:equilibrium_dynamic} implies an analogue of 
Propositions \ref{prop:comparative_statics} and \ref{prop:representation}.


\clearpage
%%%%%%%%%%%%%%%%%%%%%%%%%%%%%%%%%%%%%%%%%%%%%%%%%%%%%%%%%%%%%%%%%%%%%%%%%%%%%%%%
\section{Matching census blocks to USPS ZIP codes}
\label{sec:blocks_to_uspszip}

One challenge of this project is that LODES data on commuting patterns are 
aggregated at the census-block level of the 2010 US Census.
However, Zillow data are aggregated at the level of \textit{USPS ZIP codes},
and census blocks and USPS ZIP codes are not nested.
In this appendix we describe the steps we took to construct a correspondence
table between these geographies.

First, we collected the GIS map of 11,053,116 census blocks from \textcite{cbTiger}
and compute their centroids.
Second, we assigned each block to a unique USPS ZIP code using the GIS map 
from \textcite{ESRI} based on assigning to each block the USPS ZIP code 
that contains its centroid.
If the centroid falls outside the block, we pick a point inside it at random.
We assign 11,013,203 using the spatial match (99.64 percent of the total).%
\footnote{545,566 of ZIP codes assigned via spatial match use 
a point of the census block picked at random (4.94 percent of the total).}
Third, for the blocks that remain unassigned we use the 
tract-to-USPS-ZIP-code correspondence from \textcite{hudCrosswalks}.
Specifically, for each tract we keep the USPS ZIP code where the largest number 
of houses of the tract fall, and we assign it to each block using the tract 
identifier.
We assign 22,819 blocks using this approach (0.21 percent).
There remain 17,094 unassigned blocks (0.15 percent), which we drop from the 
analysis.
This creates a unique mapping from census blocks to USPS ZIP codes.

In the end, there are 11,036,022 census blocks which are assigned to 31,754 
USPS ZIP codes, implying an average of 347.55 census blocks per ZIP code.
Thus, even though there may be blocks that go beyond one ZIP code, 
we expect the error introduced by this process to be very small.

%% DGP: Reminder to document these numbers!

\clearpage
%%%%%%%%%%%%%%%%%%%%%%%%%%%%%%%%%%%%%%%%%%%%%%%%%%%%%%%%%%%%%%%%%%%%%%%%%%%%%%%%
\section{Assigning minimum wage levels to ZIP codes}
\label{sec:assigning_mw_levels}

Our main rents data is aggregated at the level of the USPS ZIP code.
To match this geographical level, we assign statutory MW levels to ZIP codes.
ZIP codes usually cross jurisdictions, and as a result parts of them
are subject to different statutory MW levels.
Trying to overcome this problem, we assign averages of the relevant statutory MW 
levels to each ZIP code.

We proceed as follows.
First, we collect a census crosswalk constructed by \parencite{CensusLODES} that 
contains, for each block, identifiers for block group, tract, county, CBSA 
(i.e., core-based statistical area), place (i.e., census-designated place), and 
state.
Second, we assign the MW level of each jurisdiction to the relevant block.
We use the state code for state MW policies, and we match local MW policies 
based on the names of the county and the place.
We define the statutory MW at each census block as the maximum of the federal,
state, county, and place levels.
Then, based on the original correspondence table described in Appendix 
\ref{sec:blocks_to_uspszip}, we assign a ZIP code to each block.
Finally, we define \textit{the statutory MW} at ZIP code $i$ and month $t$, 
$\MW_{it}$, as the weighted average of the statutory MW levels in its
constituent blocks, where the weights are given by the number of housing 
units.%
\footnote{ZIP codes between 00001 and 00199 correspond to federal territories.
Thus, we assign as statutory MW the federal level.}
For ZIP codes that have no housing units in them, such as those corresponding to 
universities or airports, we use a simple average instead.


\clearpage
%%%%%%%%%%%%%%%%%%%%%%%%%%%%%%%%%%%%%%%%%%%%%%%%%%%%%%%%%%%%%%%%%%%%%%%%%%%%%%%%
\section{Measuring housing expenditure at the ZIP code level}
\label{sec:measure_housing_expenditure}

For our counterfactual exercises we require several pieces of information.
First, to estimate the overall incidence of a MW policy we need the levels 
of total wages and total housing expenditure in each location.
Second, to estimate the ZIP code-specific incidence, we require a housing 
expenditure share that varies by ZIP code.
We construct these measures for 2018 using data from the \textcite{IRS} and
the \textcite{hudSAFMR}.

To construct these data we start by collecting the following variables.
We approximate the levels of total wages and housing expenditure using per 
household variables.
From the IRS we obtain annual wage per household, which we 
divide by 12 to obtain a monthly measure.
From the HUD, we use the 2-bedroom SAFMR series as our monthly housing 
expenditure variable.%
\footnote{Average rents in a location would be better approximated as a
weighted average of rents for houses with different number of bedrooms,
weighted by the share of households that rent each type of housing.
However, these data are not publicly available.}
We define the ZIP code-specific housing share as the ratio of these two 
variables.

The computed variables have several missing values across the board, and 
small percentage of missing values within within urban CBSAs 
(as defined in Table \ref{tab:stats_zip_samples}).
We impute missing values independently for each variable using an OLS
regression based on sociodemographic characteristics of each ZIP code 
(including data from the US Census and LODES) and CBSA by county fixed effects.
To limit the influence of outliers, we winsorize the results at the 0.5th and 
99.5th percentiles. 
The percentage of urban ZIP codes with non-imputed housing expenditure shares 
is $93.23$.

\clearpage
%%%%%%%%%%%%%%%%%%%%%%%%%%%%%%%%%%%%%%%%%%%%%%%%%%%%%%%%%%%%%%%%%%%%%%%%%%%%%%%%
\section{Identification of first-differenced model with spillovers}
\label{sec:did_spillovers_id}

Consider the causal model for rents given by
$$r_{it}=f_{it}(\{\mw_{zt}\}_{z\in\Z})$$
where $\mw_{zt}$ represents the ``dose'' of treatment received by unit $i$ from
ZIP code $z$ in period $t$.
We say that a ZIP code $i$ is treated ``directly'' at time $t$ if $\mw_{it}>0$.
For this appendix, we think of $\Z$ as the ZIP codes in a closed metropolitan 
area. 
%% SH: To match the stacked model
Following Section \ref{sec:empirical_strategy} we assume the following functional
form for the causal model in first differences:
\begin{equation}\label{eq:causal_model}
    \Delta r_{it} = \gamma \Delta \mw_{it} 
                  + \beta \sum_{z\in\Z} \pi_{iz} \Delta \mw_{zt}
                  + \delta_t + \Delta \varepsilon_{it}
\end{equation}
where $\delta_t$ is a time effect,
$\Delta \varepsilon_{it}$ stands for other factors that determine the evolution
of rents in ZIP code $i$, and
other objects are defined as in the paper.
We show that the parameters $\beta$ and $\gamma$ can be recovered from data on 
rent changes and minimum wage changes under suitable parallel-trends 
assumptions.

\begin{prop}[Identification]\label{prop:diS_id}
    Consider a policy such that, 
    for some (directly treated) ZIP codes  $z\in\Z_0\subset\Z$ for non-empty $\Z_0$,
        $\Delta\mw_{zt}=\Delta\mw>0$ if $t=\bar{t}$ and
        $\Delta\mw_{zt}=0$           if $t\neq\bar{t}$,
    and for (not directly treated) ZIP codes $z\notin\Z_0$ ,
        $\Delta\mw_{zt}=0$ for all $t$.
    Assume
    (1) there exist at least two not directly treated ZIP codes with differential 
    exposure to the policy, 
    (2) parallel trends among two non-empty subgroups of not directly treated 
    ZIP codes (specified below),
    (3) parallel trends across directly treated ZIP codes and not directly 
    treated ZIP codes.
    Then, the parameters $\beta$ and $\gamma$ of the model 
    \eqref{eq:causal_model} are identified.
\end{prop}

\begin{proof}
    The expected evolution of rents in ZIP codes not treated directly is
    \begin{equation}\label{eq:expected_rents_nottreated}
        E[\Delta r_{it} | i\notin\Z_0] = 
        \begin{cases}
            \delta_t + E[\Delta \varepsilon_{it} | z\notin\Z_0] 
                                       & \text{ if } t\neq\bar{t} \\
            \beta\sum_{z\in\Z_0}\pi_{iz}\mw_{zt} + \delta_t 
                     + E[\Delta \varepsilon_{it} | z\notin\Z_0] 
                                       & \text{ if } t=\bar{t} ,
        \end{cases}
    \end{equation}
    Now, rank these ZIP codes according to their exposure to the policy 
    $\Pi_i = \sum_{z\in Z_0} \pi_{iz}$.
    Consider a partition of ZIP codes in $\Z\setminus\Z_0$ in 
    two non-empty subsets such that 
    ZIP codes with $\Pi_i > \bar\Pi$ belong to a ``high exposure'' group $\Z_h$, 
    and the rest to a ``low exposure'' group $\Z_l$,
    where $\bar\Pi\in\left(\min{\Pi_i}, \max{\Pi_i}\right)$.
    Using \eqref{eq:expected_rents_nottreated} we can compute, for $t\neq \bar{t}$,
    \begin{equation*}
        E[\Delta r_{it} | i\in\Z_h] - E[\Delta r_{it} | i\in\Z_l] = 
        E[\Delta \varepsilon_{it} | i\in\Z_h] - E[\Delta \varepsilon_{it} | i\in\Z_l] ,
    \end{equation*}
    and for $t = \bar{t}$,
    \begin{equation*}
        \begin{split}
            E[\Delta r_{it} | i\in\Z_h] - E[\Delta r_{it} | i\in\Z_l] 
            & = \beta \left(E\left[\sum_{z\in\Z_0}\pi_{iz} \Big| i\in\Z_h\right]
                           - E\left[\sum_{z\in\Z_0}\pi_{iz} \Big| i\in\Z_l\right]\right) \Delta \mw  \\
            & + E[\Delta \varepsilon_{it} | i\in\Z_h] - E[\Delta \varepsilon_{it} | i\in\Z_l] .
        \end{split}
    \end{equation*}
    Under assumption (2), namely that 
    $E[\Delta \varepsilon_{it} | i\in\Z_h] 
     - E[\Delta \varepsilon_{it} | i\in\Z_l] = 0$,
    we can re-arrange the previous equation to obtain
    \begin{equation}\label{eq:id_beta}
        \beta = \frac{E[\Delta r_{it} | i\in\Z_h] - E[\Delta r_{it} | i\in\Z_l]}
                     {\left(E[\sum_{z\in\Z_0}\pi_{iz} | i\in\Z_h] - E[\sum_{z\in\Z_0}\pi_{iz} | i\in\Z_l]\right) \Delta \mw}.
    \end{equation}
    Assumption (1) guarantees that 
    $\left(E[\sum_{z\in\Z_0}\pi_{iz} | i\in\Z_h] 
          - E[\sum_{z\in\Z_0}\pi_{iz} | i\in\Z_l]\right) \neq 0$,  
    thus $\beta$ is identified.

    For those ZIP codes that are treated directly, the expected evolution of 
    rents is
    \begin{equation*}\label{eq:expected_rents_treated}
        E[\Delta r_{it} | i\in\Z_0] = 
        \begin{cases}
            \delta_t + E[\Delta \varepsilon_{it} | z\notin\Z_0] 
                                       & \text{ if } t\neq\bar{t} \\
            \gamma\Delta\mw + \beta\sum_{z\in\Z_0}\pi_{iz}\Delta\mw + \delta_t 
                     + E[\Delta \varepsilon_{it} | z\notin\Z_0] 
                                       & \text{ if } t=\bar{t} .
        \end{cases}
    \end{equation*}
    Differencing with respect to \eqref{eq:expected_rents_nottreated} obtains, 
    for $t=\bar{t}$,
    \begin{equation*}
        \begin{split}
            E[\Delta r_{it} | i\in\Z_0] - E[\Delta r_{it} | i\notin\Z_0] 
              & = \gamma \Delta \mw  \\
              & + \beta \left(E\left[\sum_{z\in\Z_0}\pi_{iz} \Big| i\in\Z_0\right]
                           - E\left[\sum_{z\in\Z_0}\pi_{iz} \Big| i\notin\Z_0\right]\right) \Delta \mw  \\
              & + E[\Delta \varepsilon_{it} | i\in\Z_0] - E[\Delta \varepsilon_{it} | i\notin\Z_0] .
        \end{split}
    \end{equation*}
    Assumption (3) means that
    $E[\Delta \varepsilon_{it} | i\in\Z_0] 
      - E[\Delta \varepsilon_{it} | i\notin\Z_0] = 0$.
    Substituting in the previous equation yields
    \begin{equation}\label{eq:id_gamma}
        \gamma = \frac{E[\Delta r_{it} | i\in\Z_0] - E[\Delta r_{it} | i\notin\Z_0]}{\Delta\mw}
            - \beta \left(E\left[\sum_{z\in\Z_0}\pi_{iz} \Big| i\in\Z_0   \right] 
                        - E\left[\sum_{z\in\Z_0}\pi_{iz} \Big| i\notin\Z_0\right]\right) .
    \end{equation}
    Since $\beta$ is known per equation \eqref{eq:id_beta}, $\gamma$ is identified
    as well.
\end{proof}


\clearpage
%%%%%%%%%%%%%%%%%%%%%%%%%%%%%%%%%%%%%%%%%%%%%%%%%%%%%%%%%%%%%%%%%%%%%%%%%%%%%%%%
\section{Estimation of the effect of the MW on ZIP code-level wages}
\label{sec:mw_on_income}

Our main data source to estimate the effect of the MW on wage income are
the yearly ZIP code aggregates from \textcite{IRS}.
Thus, we construct a ZIP code by month panel where we collect, for each ZIP code
and year between 2010 and 2018, IRS income aggregates, the yearly average of our 
MW measures, and the yearly average of our QCEW variables. 
See Section \ref{sec:data} for details on the construction of these data.

We estimate versions of
\begin{equation*}\label{eq:wage_level_model}
    y_{it} = \gamma_i + \psi_t + \varepsilon \overline{\MW_{it}^{\wkp}} + 
                 \overline{\mathbf{X}^{'}_{it}}\eta + \nu_{it} ,
\end{equation*}
where 
$y_{it} = \log Y_{it}$ is the log of total wages at ZIP code $i$ in year $t$,
$\gamma_i$ and $\psi_t$ are ZIP code and year fixed effects,
$\overline{\MW_{it}^{\wkp}}$ is the yearly average of the workplace MW,
$\overline{\mathbf{X}^{'}_{it}}$ represents the yearly average of economic 
controls, and
$\nu_{it}$ is an error term.
Sometimes we interact the year fixed effects with indicators for different
geographies.
We use a level model because it is not feasible to take monthly first differences 
with yearly data.
A yearly model estimated with monthly averaged variables is identified under the 
same assumptions as the corresponding monthly model.
%%
%% Note that here we can think of $y_{it}$ as the monthly average of yearly 
%%    wages and everything works out
%%

Appendix Table \ref{tab:static_wages} shows estimates of $\varepsilon$ for 
different specifications of the model given in \eqref{eq:wage_level_model}.
Columns (1) through (4) estimate the effect of the workplace MW on log total
wages under different specifications.
The estimates of $\varepsilon$ fluctuate between 0.1083 and 0.1488.
Our preferred specification in column (3) suggests that a 10 percent increase
in the workplace MW generates a 1.08 percent increase in total wages.
These estimates are consistent with those in \textcite{CegnizEtAl2019}.
\textcite[][Table I]{CegnizEtAl2019} estimates that a MW event increases wages
by 6.8 percent, and in their data the average MW event represents an increase in 
the statutory MW of 10.1 percent.
For illustration, assume that 15 percent of workers in a location earn the
minimum wage.
Then, \citeauthor{CegnizEtAl2019}'s (\citeyear{CegnizEtAl2019}) estimates imply 
that a 10 percent increase in the MW will increase total wages by 
$(6.8/10.1)\times 10\times 0.15 \approx 1.01$ percent.

Column (5) of Appendix Table \ref{tab:static_wages} shows, as a falsification
test, estimates of the same model as in column (3) but using the log of total
dividends as dependent variable.
We obtain a positive but much lower effect in this case, suggesting that 
dividends do not respond to the MW as wages do.
