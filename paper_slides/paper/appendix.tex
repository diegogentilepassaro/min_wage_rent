%%%%%%%%%%%%%%%%%%%%%%%%%%%%%%%%%%%%%%%%%%%%%%%%%%%%%%%%%%%%%%%%%%%%%%%%%%%%%%%%
%%%%%                              APPENDIX                                 %%%%
%%%%%%%%%%%%%%%%%%%%%%%%%%%%%%%%%%%%%%%%%%%%%%%%%%%%%%%%%%%%%%%%%%%%%%%%%%%%%%%%

\section{A dynamic supply and demand model}\label{sec:dyn_theory_model}

The geography is represented by a set of ZIP codes $\Z$.
There is an exogenously given distribution of people with differing 
residence $i$ and 
workplace $z$ locations across these ZIP codes which,
as in the main body of the paper, we denote by $\{L_{iz}\}_{i,z\in\Z\times\Z}$.

Let $H_{it}$ be the stock of square feet rented in period $t$, where $t$ is 
characterized by a month $m$ and year $y$.
This stock is composed of contracts starting at different calendar months. 
We assume that all contracts last for one year.
We impose that that $H_{it} \leq D_i$ for all $t$, 
where $D_i$ denotes the total number of available square feet in $i$.

We further decompose $H_{it}$ as follows.
Let $h_{izt} = h_{iz}\left(R_{it}, \MW_{it}, \MW_{zt}\right)$ be the 
per-person demand of housing of group $(i,z)$ in period $t$,
which depends on the prevailing MW at the time of contract sign-up.
We assume that this demand function is decreasing in residence MW and decreasing
in workplace MW, just as in Section \ref{sec:model}.
For simplicity we omitted the mediation channels of prices and income.
Let $\lambda_{it}$ denote the share of $i$'s residents who started their contracts 
in period $t$.%
\footnote{We assume that these shares do not vary by workplace.}
Then, we can write the stock of contracted square feet during period $t$ as
$$
H_{it} = \sum_{\tau = t - 11}^{t} \lambda_{i\tau} \sum_{z\in\Z} L_{iz} 
h_{iz\tau} (r_{i\tau}, \MW_{i\tau}, \MW_{z\tau})
$$
where 
$r_{i\tau}$ represents rents \textit{per square foot} in period $\tau$.
It is convenient to define the stock of contracted square feet excluding the ones 
that were signed 12 months ago. We denote them by
$$
\tilde H_{it} = \sum_{\tau = t - 10}^{t} \lambda_{i\tau} \sum_{z\in\Z} L_{iz} 
h_{iz\tau} (r_{i\tau}, \MW_{i\tau}, \MW_{z\tau}) .
$$

We assume that all square feet are homogeneous and so they have the same price in 
the market.

\subsection*{Within-period equilibrium}

Within this simple model, we assume the following timing: 
(1) At the beginning of period $t$, a share $\lambda_{it}$ of people's contracts 
expire (the ones that started on $t-12$);
(2) The square feet from expiring contracts are added to the pool of available 
rental space for new renters;
(3) Renters in $t$ and a flow supply of rental space in $t$ determine equilibrium 
rents $R_{it}$.
Let's go by these steps more formally.

As of the start of every period $t$, $\lambda_{i,t-12} \sum_z L_{iz} h_{iz,t-12}$ 
square feet become available for rent from each group of workers $(i,z)$.
The square feet available to rent in period $t$ (vacant) are then
$$
\lambda_{i,t-12} \sum_z L_{iz} h_{iz,t-12} + (D_i - H_{i,t-1}) = D_i - \tilde H_{i,t-1}.
$$
Note that $(D_i - H_{i,t-1})$ are the non-rented square feet as of $t-1$.

We denote by $V_{it}(R_{it}, \lambda_t)$ the supply of housing, increasing in 
$R_{it}$.
A feasilibity constraint is that 
\begin{equation}\label{eq:feasibility}
    V_{it}(R_{it}, \lambda_t) \leq D_i - \tilde H_{i,t-1} .
\end{equation}

The flow demand for new rentals in $t$ by those whose contract expired is given by
$$
\lambda_{it} \sum_z L_{iz} h_{izt} \left(R_{it}, \MW_{it}, \MW_{zt} \right) .
$$
This demand arises because a share of the ZIP code's contracts expired. 
Those people go to the market and may desire to rent more square feet given changes 
in their income.

The market in period $t$ cleans if
\begin{equation}\label{eq:equilibrium_dynamic}
    \lambda_t \sum_z L_{iz} h_{iz} \left(R_{it}, \MW_{it}, \MW_{zt} \right) = 
    V_{it}(R_{it}, \lambda_t) .
\end{equation}
Given minimum wages in $t$, $\{\MW_{it}\}_{i\in\Z}$,
the share of workers looking to rent in period $t$, $\lambda_t$, 
and a number of vacancies that satisfies \eqref{eq:feasibility}, 
equation \eqref{eq:equilibrium_dynamic} determines equilibrium rents in period $t$.
Because the properties of housing demand and housing supply are the same as in the
model in Section \ref{sec:model},
the equilibrium condition \eqref{eq:equilibrium_dynamic} implies an analogue of 
Propositions \ref{prop:comparative_statics} and \ref{prop:representation}.

\clearpage
%%%%%%%%%%%%%%%%%%%%%%%%%%%%%%%%%%%%%%%%%%%%%%%%%%%%%%%%%%%%%%%%%%%%%%%%%%%%%%%%
\section{Identification of first-differenced model with spillovers}\label{sec:did_spillovers_id}

Consider the causal model for rents given by
$$r_{it}=f_{it}(\{\mw_{zt}\}_{z\in\Z})$$
where $\mw_{zt}$ represents the ``dose'' of treatment received by unit $i$ from
ZIP code $z$ in period $t$.
We say that a ZIP code $i$ is treated ``directly'' at time $t$ if $\mw_{it}>0$.
For this appendix, we think of $\Z$ as the ZIP codes in a closed metropolitan 
area. 
%% SH: To match the stacked model
Following Section \ref{sec:empirical_strategy} we assume the following functional
form for the causal model in first differences:
\begin{equation}\label{eq:causal_model}
    \Delta r_{it} = \gamma \Delta \mw_{it} 
                  + \beta \sum_{z\in\Z} \pi_{iz} \Delta \mw_{zt}
                  + \delta_t + \Delta \varepsilon_{it}
\end{equation}
where $\delta_t$ is a time effect,
$\Delta \varepsilon_{it}$ stands for other factors that determine the evolution
of rents in ZIP code $i$, and
other objects are defined as in the paper.
We show that the parameters $\beta$ and $\gamma$ can be recovered from data on 
rents and minimum wage changes under suitable parallel-trends assumptions.

\begin{prop}[Identification]\label{prop:did_id}
    Consider a policy such that, 
    for some (directly treated) ZIP codes  $z\in\Z_0\subset\Z$ for non-empty $\Z_0$,
        $\Delta\mw_{zt}=\Delta\mw>0$ if $t=\bar{t}$ and
        $\Delta\mw_{zt}=0$           if $t\neq\bar{t}$,
    and for (not directly treated) ZIP codes $z\in\Z\setminus\Z_0$ ,
        $\Delta\mw_{zt}=0$ for all $t$.
    Assume
    (i) $\exists z', z''\in \Z\setminus\Z_0 | \sum_{z\in Z_0} \pi_{z'z} \neq  \sum_{z\in Z_0} \pi_{z''z}$, 
    (ii) parallel trends across not directly treated ZIP codes,
    (iii) parallel trends across ZIP codes treated and not directly treated ZIP codes.
    Then, the parameters $\beta$ and $\gamma$ in the causal model \ref{eq:causal_model}
    are identified.
\end{prop}

\begin{proof}
    The expected evolution of rents in ZIP codes not treated directly is
    \begin{equation}\label{eq:expected_rents_nottreated}
        E[\Delta r_{it} | i\notin\Z_0] = 
        \begin{cases}
            \delta_t + E[\Delta \varepsilon_{it} | z\notin\Z_0] 
                                       & \text{ if } t\neq\bar{t} \\
            \beta\sum_{z\in\Z_0}\pi_{iz}\mw_{zt} + \delta_t 
                     + E[\Delta \varepsilon_{it} | z\notin\Z_0] 
                                       & \text{ if } t=\bar{t} ,
        \end{cases}
    \end{equation}
    Now, rank the not directly treated ZIP codes according to the value of 
    $\Pi_i = \sum_{z\in Z_0} \pi_{iz}$. Consider a partition of ZIP codes in 
    $\Z\setminus\Z_0$ in two non-empty subsets such that all ZIP codes with $\Pi_i > \Pi_{z'}$ 
    belong to a group, that we call  $Z_l$, and the rest to another one called $Z_h$.
    Using \ref{eq:expected_rents_nottreated} we can compute, for $t\neq \bar{t}$,
    \begin{equation*}
        E[\Delta r_{it} | i\in\Z_h] - E[\Delta r_{it} | i\in\Z_l] = 
        E[\Delta \varepsilon_{it} | i\in\Z_h] - E[\Delta \varepsilon_{it} | i\in\Z_l] ,
    \end{equation*}
    and for $t = \bar{t}$,
    \begin{equation*}
        \begin{split}
            E[\Delta r_{it} | i\in\Z_h] - E[\Delta r_{it} | i\in\Z_l] 
            & = \beta \left(E\left[\sum_{z\in\Z_0}\pi_{iz} \Big| i\in\Z_h\right]
                           - E\left[\sum_{z\in\Z_0}\pi_{iz} \Big| i\in\Z_l\right]\right) \Delta \mw  \\
            & + E[\Delta \varepsilon_{it} | i\in\Z_h] - E[\Delta \varepsilon_{it} | i\in\Z_l] .
        \end{split}
    \end{equation*}
    %DGP: Not sure I understand the math here. $\Delta \underbar{w}$ is indexed by ${z\bar{t}}$ so how can we pull it out of the sum?
    Under assumption (ii), namely that $E[\Delta \varepsilon_{it} | i\in\Z_h] 
    - E[\Delta \varepsilon_{it} | i\in\Z_l]=0$
    we can re-arrange the previous equation to obtain
    \begin{equation}\label{eq:id_beta}
        \beta = \frac{E[\Delta r_{it} | i\in\Z_h] - E[\Delta r_{it} | i\in\Z_l]}
                     {\left(E[\sum_{z\in\Z_0}\pi_{iz} | i\in\Z_h] - E[\sum_{z\in\Z_0}\pi_{iz} | i\in\Z_l]\right) \Delta \mw}.
    \end{equation}
    Assumption (i) guarantees that $\left(E[\sum_{z\in\Z_0}\pi_{iz} | i\in\Z_h] 
    - E[\sum_{z\in\Z_0}\pi_{iz} | i\in\Z_l]\right) \neq 0$,  
    thus $\beta$ is identified.

    For those ZIP codes that are treated directly, the expected evolution of 
    rents is:
    \begin{equation*}\label{eq:expected_rents_treated}
        E[\Delta r_{it} | i\in\Z_0] = 
        \begin{cases}
            \delta_t + E[\Delta \varepsilon_it | z\notin\Z_0] 
                                       & \text{ if } t\neq\bar{t} \\
            \gamma\Delta\mw + \beta\sum_{z\in\Z_0}\pi_{iz}\Delta\mw + \delta_t 
                     + E[\Delta \varepsilon_it | z\notin\Z_0] 
                                       & \text{ if } t=\bar{t} .
        \end{cases}
    \end{equation*}
    Differencing with respect to \ref{eq:expected_rents_nottreated} obtains, 
    for $t=\bar{t}$,
    \begin{equation*}
        \begin{split}
            E[\Delta r_{it} | i\in\Z_0] - E[\Delta r_{it} | i\notin\Z_0] 
              & = \gamma \Delta \mw  \\
              & + \beta \left(E\left[\sum_{z\in\Z_0}\pi_{iz} \Big| i\in\Z_0\right]
                           - E\left[\sum_{z\in\Z_0}\pi_{iz} \Big| i\notin\Z_0\right]\right) \Delta \mw  \\
              & + E[\Delta \varepsilon_{it} | i\in\Z_0] - E[\Delta \varepsilon_{it} | i\notin\Z_0] .
        \end{split}
    \end{equation*}
    Assumption (iii) means that
    $E[\Delta \varepsilon_{it} | i\in\Z_0] - E[\Delta \varepsilon_{it} | i\notin\Z_0]=0$.
    Substituting in the previous equation yields
    \begin{equation}\label{eq:id_gamma}
        \gamma = \frac{E[\Delta r_{it} | i\in\Z_0] - E[\Delta r_{it} | i\notin\Z_0]}{\Delta\mw}
            - \beta \left(E\left[\sum_{z\in\Z_0}\pi_{iz} \Big| i\in\Z_0   \right] 
                        - E\left[\sum_{z\in\Z_0}\pi_{iz} \Big| i\notin\Z_0\right]\right) .
    \end{equation}
    Since $\beta$ is known per equation \ref{eq:id_beta}, $\gamma$ is identified
    as well.
\end{proof}


\clearpage
%%%%%%%%%%%%%%%%%%%%%%%%%%%%%%%%%%%%%%%%%%%%%%%%%%%%%%%%%%%%%%%%%%%%%%%%%%%%%%%%
\section{The Rationale for Selecting Economic Controls}\label{sec:app_econ_control}

Throughout the paper we use wages, employment, and establishment-count from the QCEW to control 
for the local business cycle. However, those variables may themselves be impacted directly by 
changes in the MW. As a result, incorporating those controls raises the concern of a ``bad 
control'' problem \parencite{AngristPischke2009}. In the scenario that the MW affects the 
controls used in the regression, this approach will not unveil the average treatment effect of 
the MW on rents even if the policy is known to be \textit{randomly assigned}. The reason is 
that conditioning on these variables opens a potential alternative channel through which the MW 
affects rents, introducing bias in our estimates.

To avoid the ``bad controls" problem, while at the same time include control variables that 
proxy for the local economic conditions, we select QCEW county-level time-series for the sectors 
that we think are unlikely to be affected by MW legislation: "Professional and Business Services", 
"Information", and "Finance". Our interpretation of the effects of the MW on rents as causal 
relies on the assumption that these controls are not influenced by the MW. In this appendix we 
provide evidence in favor of this assumption.

We start by noting that these sectors employ a rather small portion of MW labor. According to 
\textcite[][table 5]{MinWorkersReportBLS}, in 2019 such industries accounted for 3.5, 1, and 
1.2 percent of the total number of MW workers, respectively. These low percentages make direct
impacts unlikely. However, these sectors may be influenced by MW legislation indirectly. We 
test this possibility by using a version of the dynamic model to analyze whether MW affects 
either wages, employment, or establishment-count in these sectors. The presence of significant 
pre-treatment trends would suggest that these sectors react to MW, and cast doubt on the 
identification strategy discussed in the paper.

For each industry, we observe employment at the county and month levels, and average weekly 
wages and establishment-count at the county and quarter levels. As a result, estimation of the 
dynamic model defined at the zipcode and month level is not straightforward. To be able to 
estimate our model, we aggregate MW zipcode-month information at the county-month level by 
taking, for each period, the weighted average of MW levels in each zipcode associated with a 
given county, using the number of housing units as weights. We call this our MW variable 
$\underline{w}_{ct}$, where $(c, t)$ is a county and monthly date cell.\footnote{For counties 
	without city level ordinances this procedure would simply reflect the state- or 
	county-level 	MW. For places with a MW at the city level our MW variable corresponds to a 
	weighted average between places affected by the city MW and places not affected by it.}

Given that we observe employment with monthly frequency, we are able to estimate the following 
model:

\begin{equation} \label{eq:dynamic_econ_cont_month}
\Delta \ln y_{ct} = \delta_{t} 
+ \sum_{r=-s}^{s} \beta_r \Delta \ln \underline{w}_{c,t+r} 
+ \Delta \nu_{ct} ,
\end{equation}
where $y_{ct}$ is the outcome variable (employment) in county $c$ and month $t$, and $s$ defines 
the number of leads and lags in the model.

Because the QCEW provides average weekly wages and establishment-count at the county-quarter level, 
we estimate the model for these variables in quarterly averages of monthly observations. We compute 
quarterly average of QCEW measures as $\overline{\Delta y}_{cq} = \frac{1}{3} \Delta y_{cq}$. As for 
MW data, we define $\underline{w}_{cq}$ as the third month in each quarter $q$. We are then able to 
compute the quarterly average for MW changes, $\overline{\Delta \ln \underline{w}}_{cq} = \frac{1}{3} 
\Delta \ln \underline{w}_{cq}$. The estimating model then becomes

\begin{equation} \label{eq:dynamic_econ_cont_quarter}
\overline{\Delta \ln y}_{cq} = \overline{\delta}_q 
+ \sum_{r=\tau}^{\tau} \rho_r \overline{\Delta \ln \underline{w}}_{c,q+r}
+ \overline{\Delta \nu}_{ct} ,
\end{equation}
where $\tau$ defines the quarterly window. We interpret the $\rho_r$ coefficients in 
\autoref{eq:dynamic_econ_cont_quarter} as averages of the monthly coefficients $\beta_r$ in 
\autoref{eq:dynamic_econ_cont_month}. The reason is that the average change over a quarter is a 
linear combination of monthly changes.\footnote{To see this, let's define $q_1, q_2, q_3$ as the 
	first, second, and third months in a quarter $q$, respectively. Then, for any variable $x$,
	\begin{equation*}
	\begin{split}
	\frac{1}{3}\Delta x_{cq} 
	& = \frac{1}{3} \left( x_{cq} - x_{c,q-1} \right) 
	= \frac{1}{3} \left( x_{cq_3} - x_{c,q_3-1} \right) \\
	& = \frac{1}{3} \left( x_{cq_3} - x_{cq_2} 
	+ x_{cq_2} - x_{cq_1} + x_{cq_1} - x_{c,q_3-1} \right) \\ 
	& = \frac{1}{3} \left( \Delta x_{cq_3} + \Delta x_{cq_2}  + \Delta x_{cq_1} \right) .
	\end{split}
	\end{equation*}}

\autoref{fig:controls_models} shows the results of our estimation for the three industries 
selected as controls. Even though some of the coefficients are significant, we interpret the 
results as suggestive of a noisy zero effect. The most worrisome sector is ``Professional 
and Business Services'', which shows a significant same quarter coefficient of average weekly
wages, and a slight negative pre-trend in employment. While we decided to keep this sector as
control in our main models, we stress that our results are virtually identical when dropping it.

\clearpage
%%%%%%%%%%%%%%%%%%%%%%%%%%%%%%%%%%%%%%%%%%%%%%%%%%%%%%%%%%%%%%%%%%%%%%%%%%%%%%%%
\section{Appendix Tables}



\clearpage
%%%%%%%%%%%%%%%%%%%%%%%%%%%%%%%%%%%%%%%%%%%%%%%%%%%%%%%%%%%%%%%%%%%%%%%%%%%%%%%%
\section{Appendix Figures}


