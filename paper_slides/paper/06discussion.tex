%%%%%%%%%%%%%%%%%%%%%%%%%%%%%%%%%%%%%%%%%%%%%%%%%%%%%%%%%%%%%%%%%%%%%%%%%%%%%%%%%
%%%%%                             DISCUSSION                                 %%%%
%%%%%%%%%%%%%%%%%%%%%%%%%%%%%%%%%%%%%%%%%%%%%%%%%%%%%%%%%%%%%%%%%%%%%%%%%%%%%%%%%

In this discussion we discuss the implications of our results.

%%%%%%%%%%%%%%%%%%%%%%%%%%%%%%%%%%%%%%%%%%%%%%%%%%%%%%%%%%%%%%%%%%%%%%%%%%%%%%%%%
\subsection{The magnitude of the estimates}\label{sec:benchmark}

In this subsection we use the model introduced in \autoref{sec:model} combined with some auxiliary 
assumptions to assess whether the magnitude of our estimates is plausible.

Assume that functions that characterize supply and demand of rental units are constant elasticity, 
so that $\underline{\gamma}$ and $\underline{\beta}$ are the elasticity of MW households demand to 
rents and income, $\overline{\gamma}$ and $\overline{\beta}$ are analogous parameters for non-MW 
households, and $k$ is the elasticity of housing supply to rents.\footnote{More precisely, we 
	assume that $\underline{H}(r, \underline{w}) = A e^{\underline{\gamma} \ln r + \underline{\beta} 
		\ln\underline{w}}$, $\overline{H}(r, w) = B e^{\overline{\gamma} \ln r + \overline{\beta} \ln 
		w}$, and $H(r) = C e^{k \ln r}$. $A, B, C > 0$ are constants.}. 
As a result, it can be shown that \autoref{eq:model-elasticity} takes the form

\begin{equation}\label{eq:benchmarking-elasticity}
\rho = \frac{\underline{\beta} \ \underline{s}}
{k - \underline{\gamma} \ \underline{s} 
	- \overline{\gamma} \overline{s}}
\end{equation}
where $\underline{s} = \frac{\underline{H}}{H}$ is share of housing occupied by MW households 
and $\overline{s} = \frac{\overline{H}}{H}$ is the share of housing occupied by non-MW households. 
Note that $\overline{s} = 1 - \underline{s}$.

The above expression is intuitive, in the sense that factors which increase housing demand make 
the elasticity higher, whereas factors that increase supply lower it. For instance, a higher 
$\underline{\beta}$ --elasticity of housing demand to income-- implies a higher $\rho$, whereas 
a higher $k$ --elasticity of housing demand to rents-- implies a lower $\rho$.

Suppose that the share of MW is $\underline{s} = 0.3$, so that $\overline{s}=0.7$. Assume that 
demand elasticities and $(\underline{\gamma}, \overline{\gamma}, \underline{\beta}) = (- 0.7, 
- 0.5, 0.1)$, implying that MW households are more sensible to increases in rents, and that 
demand for housing is price-inelastic and a normal good. Finally, let $k = 0.1$, similar to 
estimate of \textcite[][Table 5]{Diamond2016}. Substituting these values in 
\eqref{eq:benchmarking-elasticity} results in an elasticity of 0.45. This value turns out to be 
very close to the cumulative sum of our $t$ and $t-1$ coefficients.


%%%%%%%%%%%%%%%%%%%%%%%%%%%%%%%%%%%%%%%%%%%%%%%%%%%%%%%%%%%%%%%%%%%%%%%%%%%%%%%%%
\subsection{Policy implications}\label{sec:policy}

