
In this section we layout a simple demand and supply model of local rental markets.
We use the model to illustrate why we expect a different impact of MW changes 
on rents at workplace and residence locations.
Because we study the consequences of MW changes in the very-short run, our model is 
static and we assume an exogenous distribution of workers across residence and 
workplace locations.
We discuss the addition of the time dimension in Appendix \ref{sec:dyn_theory_model}.
We think of a spatial model with worker mobility across ZIP codes as an avenue 
for future work.

\subsection{Setup}

We consider the rental market of some ZIP code $i$ embedded in a larger geography 
composed of a finite number of ZIP codes $\Z$.
Workers with residence $i$ work in a ZIP code $z\in\Z(i)$, where 
$\Z(i)\subseteq\Z$.
More precisely, we let $L_{iz}$ denote the measure of $i$'s residents who work in 
$z$; and 
$L_i = \sum_{z \in \Z(i)} L_{iz}$ and $L_z = \sum_{i \in \Z(i)} L_{iz}$ the number
of residents in $i$ and workers in $z$, respectively.
We assume that the distribution of residence-workplace pairs is fixed.%
\footnote{To simplify we assume that all of $i$' residents work, so that the number
of residents equals the number of workers.}
This assumption is consistent with the effect of the MW having a small effect
on employment, as in \textcite{CegnizEtAl2019}, and on migration, as in
\textcite{PerezPerez2021}.

Each ZIP code has a binding minimum wage, which we denote by $\{\MW_z\}_{z\in\Z(i)}$.

\subsubsection*{Housing demand}

Each group $(i,z)$ consume
square feet of living space $H_{iz}$, 
a non-tradable good produced in their residence $C_{iz}^{NT}$, and
a tradable good $C_{iz}^T$.
A representative $(i,z)$ worker chooses between these alternatives by maximizing
a quasi-concave utility function 
$u_{iz} = u \left(H_{iz}, C^{\text{NT}}_{iz}, C^{\text{T}}_{iz}\right)$
subject to a budget constraint
$$R_i H_{iz} + P_i(\MW_i) C^{\text{NT}}_{iz} + C^{\text{T}}_{iz} \leq Y_{iz}(\MW_z),$$
where
$R_i$ gives the rental price of housing per square feet,
$P_i(\MW_i)$ gives the price of local consumption,
the price of tradable consumption is normalized to one, and 
$Y_{iz}(\MW_z)$ is an income function.
We summarize the effect of MWs below.

\begin{assu}[Effect of Minimum Wages]\label{assu:mws}
    We assume that
    (i) the prices of non-tradable goods are increasing in $i$'s MW, 
    $\frac{d P_i}{d \MW_i} > 0$, and
    (ii) income is weakly increasing in $z$'s MW, 
    $\frac{d Y_{iz}}{d \MW_z} \geq 0$, with strict inequality 
    for at least one $z\in\Z(i)$.
\end{assu}

We think that the structure of the problem and
Assumption \ref{assu:mws} are consistent with the literature.
First, recent evidence by \textcite{MiyauchiEtAl2021} shows that individuals 
tend to consume close to home.
As a result, we expect them to be sensitive to prices of local consumption in 
their same neighborhood, justifying the inclusion of $C^{\text{NT}}_{iz}$ in the 
utility function.%
\footnote{An extension of the model would allow workers to consume in any ZIP code
in the metropolitan area.
While theoretically straightforward, this extension would require data on consumption
trips, which we lack.
We think of our model as an approximation.}
Second, MWs hikes have been shown to increase prices of local consumption 
\parencite[e.g.,][]{AllegrettoReich2018, Leung2021},
and also to increase wage income even for wages above the MW 
level \parencite[e.g.,][]{CegnizEtAl2019,Dube2019Income}.%
\footnote{An extension would allow separate wage income and business income in 
the budget constraint.
If firm owners tend to live where they work, and MW increases damage profits
\parencite[as found by][, among others]{DracaMachinVanreenen2011},
then business income would depend negatively on the MW level.}
%% SH: It would be cool to cite some paper on the relationship between consumption
%%     prices and rents.

For convenience we define the per capita housing demand function as 
$h_{iz} \equiv \frac{H_{iz}}{L_{iz}}$.
The solution to the worker's problem for each $z$ then yields a set of 
continuously differentiable per capita housing demand functions 
$\{h_{iz} (R_i, P_i, Y_z)\}_{z\in\Z(i)}$.
First, we assume that housing is a normal good, so that housing demand is 
increasing in income $Y_z$.
Standard arguments then imply that this function is decreasing in its own price 
$R_i$.
%% Note that we need to mention the income effect first to rule out Giffen goods
%%
Finally, we assume that housing demand is decreasing in local prices $P_i$.
A sufficient (albeit not necessary) condition is that housing and non-tradable
consumption are complements.%
\footnote{To formalize the required condition, let $h_{iz}$ and $c_{iz}$ denote 
Marshallian demands resulting from the choice problem, and $\tilde h_{iz}$ 
denote the Hicksian housing demand.
The Slutsky equation implies that
$$\frac{\partial h_{iz}}{\partial P_i} 
   = \frac{\partial \tilde h_{iz}}{\partial P_i} 
   - \frac{\partial h_{iz}}{\partial Y_{iz}} c_{iz}.$$
To obtain $\frac{\partial h_{iz}}{\partial P_i} < 0$, we require that 
$\frac{\partial \tilde h_{iz}}{\partial P_i} 
< \frac{\partial h_{iz}}{\partial Y_{iz}} c_{iz}$, i.e., the income effect of an 
increase in non-tradable prices is larger than the corresponding substitution 
effect.}
The fact that families are willing to pay higher rents for better amenities,
as found by \textcite{CoutureEtAl2019}, is consistent with this assumption.

Note that, given our assumptions, 
an increase in a group's $(i,z)$ workplace MW will tend to increase housing 
demand in $i$, 
and an increase in residence MW will have a negative effect---conditional on 
its effect via the workplace MW of the group $(i,i)$---.

\subsubsection*{Housing supply}

We assume that absentee landlords supply square feet in $i$ according to the 
function $S_i(R_i)$,
and we assume that this function is weakly increasing in $R_i$.
Note that this formulation allows for an upper limit on the number of houses at 
which point the supply becomes perfectly inelastic.

\subsection{Equilibrium and Comparative Statics}

Total demand of housing in ZIP code $i$ is given by the sum of the demands of 
each group.
Thus, we can write the equilibrium condition in this market as
\begin{equation}\label{eq:equilibrium}
	\sum_{z\in\Z(i)} L_{iz} h_{iz} \left(R_i, P_i(\MW_i), Y_z(\MW_z)\right) = S_i(R_i) .
\end{equation}
Given that the per-capita housing demand functions are continuous and 
decreasing in rents,
under a suitable regularity condition there is a unique equilibrium in this market.%
\footnote{To see this, assume that 
$S_i(0) - \sum_{z\in\Z(i)} L_{iz} h_{iz} (0, P_i, Y_z) < 0$
and apply the intermediate value theorem.
Intuitively, at low rental prices demand has to exceed supply.}
Equilibrium rents are a function of the entire set of minimum wages, formally, 
$R^*_i = f(\{\MW_i\}_{i\in\Z(i)})$.

We are interested in two questions.
First, what is the effect of a change in the vector of MWs 
$(\{d \ln \MW_i\}_{i\in\Z(i)})'$ on equilibrium rents?
Second, under what conditions can we reduce the dimensionality of the rents 
function and represent the effects of MW changes on equilibrium rents in a 
simpler way?
We start with the first question.

\begin{prop}[Comparative Statics]\label{prop:comparative_statics}
    Consider residence ZIP code $i$ and a change in MW policy at a larger
    jurisdiction such that for $z\in\Z_0 \subseteq \Z(i)$ binding MWs increase, 
    and for $z'\in\Z(i)\setminus \Z_0$ binding MWs do not change,
    where $\Z_0$ is non-empty.
    Under the assumptions of unchanging $\{L_{iz}\}_{z\in\Z(i)}$ 
    and Assumption \ref{assu:mws},
    we have that
    \begin{enumerate}
        \item[(i)]
        for $z'\in\Z_0\setminus\{i\}$ for which $\frac{d Y_{z'}}{d \MW_{z'}}>0$, 
        the policy has a positive partial effect on rents, 
        $\frac{d\ln R_i}{d\ln\MW_{z'}} > 0$;
        \item[(ii)]
        the partial effect of the MW increase $i$ on rents is ambiguous, 
        $\frac{d\ln R_i}{d\ln\MW_i} \lessgtr 0$; and
        \item[(iii)]
        as a result, the overall effect on rents is ambiguous if $i\in\Z_0$ 
        and weakly positive if $i\notin\Z_0$.
    \end{enumerate}
\end{prop}

\begin{proof}
    Fully differentiate the market clearing condition with respect to $\ln R_i$ 
    and $\ln \MW_i$ for all $i\in\Z(i)$.
    Dividing by \eqref{eq:equilibrium} and each of the variables appropriately, 
    one can show that
    \begin{equation}\label{eq:diff_equilibrium}
        \Big(\eta_i - \sum_z \pi_{iz} \xi^R_{iz} \Big) d \ln R_i
        = 
        \sum_z \pi_{iz} \left(\xi^P_{iz} \epsilon_{i}^P d \ln \MW_i 
                            + \xi^Y_{iz} \epsilon_{z}^Y d \ln \MW_z \right) ,
    \end{equation}
    where
    $\pi_{iz} = \frac{L_{iz}}{L_i}$ represents the share of $i$'s residents 
    working in $z$;
    $\xi_{iz}^x = \frac{d h_{iz}}{d x_i} \frac{x_i}{\sum_z \pi_{iz} h_{iz}}$ for
    $x\in\{R,P,Y\}$ is the elasticity of the per-capita housing demand evaluated 
    at the average per-capita demand of ZIP code $i$;
    $\epsilon_{i}^P = \frac{d P_i}{d \MW_i} \frac{\MW_i}{P_i}$ and 
    $\epsilon_{z}^Y = \frac{d Y_z}{d \MW_z} \frac{\MW_z}{Y_z}$ are
    elasticities of prices and income to minimum wages; and
    $\eta_i = \frac{d S_i}{d R_i} \frac{R_i}{S_i}$ is the elasticity 
    of housing supply in ZIP code $i$.

    For each $z\in\Z_0\setminus\{i\}$ the partial effect on rents of the policy
    is given by 
    $$\left(\eta_i - \sum_z \pi_{iz} \xi^R_{iz}\right)^{-1} 
      \pi_{iz}\xi^Y_{iz}\epsilon_{z}^Y d\ln\MW_z.$$
    Because $\eta_i>0$ and $\xi^R_{iz} < 0$  $\forall z\in\Z(i)$ with strict, 
    the first factor is positive.
    Thus, because we assumed $\epsilon_{z}^Y>0$ this effect is positive, as 
    desired.

    For ZIP code $i$ the partial effect is given by
    $$\left(\eta_i - \sum_z \pi_{iz} \xi^R_{iz}\right)^{-1} 
      \left(\epsilon_{i}^P \sum \pi_{iz}\xi^P_{iz} 
            + \pi_{ii}\xi^y_{ii}\epsilon_i^y \right)d\ln\MW_i.$$
    Because $\epsilon_{i}^P>0$, $\xi^P_{iz}<0$ $\forall z\in\Z(i)$,
    and $\epsilon_{i}^Y>0$, 
    then the sign of this partial effect is ambiguous.
    The third statement of the Proposition follows directly.
\end{proof}

Proposition \ref{prop:comparative_statics} (i) shows that,
if at least some low-wage worker (for whom $\frac{d Y_z}{d \MW_z}>0$)
commutes to a ZIP code where the MW increased,
then the MW hike will tend to increase rents.
Proposition \ref{prop:comparative_statics} (ii) establishes that a decreasing
effect on rents may follow if the minimum wage also increases in ZIP code $i$.
As a result, the sign of the overall effect of the policy is not determined a 
priori.

The following proposition establishes conditions under which the dimensionality
of equation \eqref{eq:diff_equilibrium} can be reduced.

\begin{prop}[Representation]\label{prop:representation}
    Assume that for all ZIP code $z\in\Z(i)$ we have
    (i) homogeneous elasticity of per-capita housing demand to incomes $Y_z$,
    $\xi^Y_{iz}=\xi^Y_{i}$, and
    (ii) homogeneous elasticity of income to minimum wages $\MW_z$,
    $\epsilon_z^y=\epsilon^y$.
    Then, we can write the change in log rents as a function of the change in 
    two MW-based measures: ZIP code $i$'s \textbf{workplace MW}, defined
    as $\sum_{z\in\Z(i)} \pi_{iz} \ln \MW_z$, and 
    ZIP code $i$'s \textbf{residence MW}, defined as $\ln\MW_i$.
    Furthermore, the workplace MW has a positive effect on rents, whereas the
    residence MW has a negative effect.
\end{prop}
\begin{proof}
    Under the stated assumptions we can manipulate \eqref{eq:diff_equilibrium} 
    to write
    \begin{equation} \label{eq:theory_represenation}
        d r_i = \beta_i  d \mw^{\text{exp}}_i
              + \gamma_i d \mw^{\text{res}}_i
    \end{equation}
    where
    $r_i=\ln R_i$ reprensents the log of rents,
    $\mw^{\wkp}_i = \sum_{z\in\Z(i)} \pi_{iz} \ln \MW_z$ and
    $\mw^{\res}_i = \ln \MW_i$ are defined as $i$'s 
    \textit{workplace} and \textit{residence} MW levels; and
    $\beta_i = \frac{\sum_{z\in\Z(i)}\pi_{iz}\xi_{iz}^{p}\epsilon_{i}^{p}}
                    {\eta_{i} - \sum_z \pi_{iz} \xi_{iz}} 
             >0$ and
    $\gamma_i = \frac{\sum_z \xi_{i}^{y}\epsilon^{y}}
                     {\eta_{i} - \sum_z \pi_{iz} \xi_{iz}} 
              < 0$
    are parameters.
\end{proof}

Proposition \ref{prop:representation} shows that, under an homogeneity assumption,%
\footnote{We acknoweldge that this simpliflying assumption will not hold exactly
in practice.
For our empirical estimates, we need the weaker assumption that heterogeneity in 
the effect of workplace MWs is not correlated to shocks in the housing market.}
the change in rents following a small changes in the profile of MWs can be expressed 
as a function of two MW-based measures: 
one summarizing the effect of MW changes in workplaces,
and another one summarizing the effect of the MW in the same ZIP code $i$.
This motivates our empirical strategy, where we regress log rents on the empirical
counterparts of these measures.

