
In this section we lay down a motivating demand and supply model of the rental market. 
We intend to illustrate why one may expect a different impact of workplace and residence
MW changes.
We also show how, in the model under certain assumptions, changes in log rents can 
be expressed as a function of the changes in two MW-based measures.

The model is purposefully stylized.
Because we study the consequences of MW changes in the very-short run, our model is 
static. 
We discuss the addition of the time dimension in Appendix \ref{sec:dyn_theory_model}.
We also assume an exogenous distribution of workers across residence and workplace 
locations.
We think of the specification of a spatial model with worker mobility as an avenue 
for future work.

We emphasize that the model is designed to highlight a possible mechanism through 
which one may expect residence and workplace MWs to have different impacts. 
However, our empirical results do not hinge on any of the assumptions made in this 
section.

\subsection{Set-up}

We consider the rental market of some ZIP code $i$ embedded in a metropolitan area, 
which is characterized by a set of ZIP codes $\Z$.
Workers with residence $i\in\Z$ work in some other ZIP code $z\in\Z$. 
More precisely, we let $L_{iz}$ denote the measure $i$'s residents who work in $z$;
$L_i = \sum_{z \in \Z} L_{iz}$ and $L_z = \sum_{i \in \Z} L_{iz}$ 
the number of residents in $i$ and workers in $z$, respectively;
and $\mathcal{L}=\sum_{z \in \Z}\sum_{i \in \Z}L_{iz}$ the total measure of workers. 
We assume that the distribution of residence-workplace pairs is fixed.%
\footnote{To simplify we assume that all of $i$' residents work, so that the number
of residents equals the number of workers.}
Each ZIP code is characterized by a binding minimum wage, which we denote by 
$\{\MW_z\}_{z\in\Z}$.

\paragraph{Housing demand}

In this simple static model all workers have to rent a house in a common market, 
where the rental rate is $r_i$. 
We assume that group $(i,z)$'s demand of square feet per person is given by $h_{iz} (r_i, 
\MW_i, \MW_z)$, where the second argument corresponds to the \textit{residence} MW, and 
the third to the \textit{workplace} MW. 
We characterize the properties of this set of functions below.

\begin{assu}[Housing demand]\label{assu:housing_function}
	For all residence-workplace pairs, the housing demand function $h_{iz} (r_i, 	
	\MW_i, \MW_z)$ is 
	(i) continuously differentiable in its three arguments;
	(ii) decreasing in rental prices $r_i$;
	(iii) decreasing in residence MW, $\MW_i$;
	(iv) increasing in workplace MW, $\MW_z$.
\end{assu}

Points (i) and (ii) simply say that $h_{iz}$ is a ``smooth'' demand function.
Point (iv) follows from the fact that housing is a normal good.
Given that, under negligible disemployment effects, workplace MW increases income 
(at least of the workers who make close to the MW), it should increase housing 
demand.
Point (iii) is less obvious, and we motivate it as follows.
Residence MW, while increasing the incomes of people working in the same place, 
also increases the cost of production of non-tradable goods.
The higher cost of non-tradables will translate into a lower demand of housing 
if the substitution effect of an change in local prices on local demand of housing
is smaller than the corresponding income effect.
A sufficient condition is that housing and local consumption are complements.%
\footnote{We can formalize this discussion with a simple worker choice problem.
Say a representative $(i,z)$ worker chooses between housing demand $h_{iz}$,
non-tradable consumption $c^{\text{NT}}_{iz}$, and tradable consumption $c^{\text{T}}_{iz}$,
by maximizing
$$u_{iz} = u \left(h_{iz}, c^{\text{NT}}_{iz}, c^{\text{T}}_{iz}\right)$$
subject to 
$r_i h_{iz} + p_i(\MW_i) c^{\text{NT}}_{iz} + c^{\text{T}}_{iz} \leq y_{iz}(\MW_z),$
where $p_i(\MW_i)$ gives the price of local consumption, which is increasing in residence MW;
the price of tradable consumption is normalized to one; and 
$y_{iz}(\MW_z)$ is an income function that depends positively on the workplace MW.
Let $h_{iz}^*$ and $c_{iz}^*$ denote marshallian demands, and 
$\tilde h_{iz}^*$ denote the hicksian housing demand.
The slutsky equation implies that
$$\frac{\partial h_{iz}^*}{\partial p_i} 
   = \frac{\partial \tilde h_{iz}^*}{\partial p_i} 
   - \frac{\partial h_{iz}^*}{\partial y_{iz}} c_{iz}^*.$$
We have that $\frac{\partial h_{iz}^*}{\partial p_i} < 0$ if and only if 
$\frac{\partial \tilde h_{iz}^*}{\partial p_i} 
< \frac{\partial h_{iz}^*}{\partial y_{iz}} c_{iz}^*$.}
Yet another possibility is that the owners of firms tend to live in the same ZIP 
code where the firm is located and suffer losses in profits after the MW increase 
(thus, for them, residence MW decreases income even further).

We think that the interpretation underlying point (iii) is plausible for several 
reasons.
First, recent evidence by \textcite{MiyauchiEtAl2021} shows that individuals tend 
to consume close to home.
As a result, we expect them to be sensible to prices of local consumption in their 
same neighborhood.
Second, MWs have been shown to increase prices of local consumption 
\parencite[e.g.,][]{AllegrettoReich2018, LeungForthcoming}.
These empirical facts suggest that residence MW changes might (conditional on workplace)
negatively affect incomes and thus demand for housing.

\paragraph{Housing supply}

We assume a simple supply side. Denote by $D_i(r_i)$ the supply of square feet in 
$i$, which is increasing in $r_i$.
Note that this formulation allows for an upper limit on the number of houses at 
which point the supply becomes perfectly inelastic.

\subsection{Equilibrium and Comparative Statics}

Total demand of housing in ZIP code $i$ is given by the sum of the demands of each group. 
Thus, we can write the equilibrium condition in this market as
\begin{equation}\label{eq:equilibrium}
	\sum_{z\in\Z} L_{iz} h_{iz} (r_i, \MW_i, \MW_z) = D_i(r_i) .
\end{equation}
We organize the main results in a couple of propositions.

\begin{prop}[Equilibrium]\label{prop:equilibrium}
    Assume that $h_{iz}(\cdot)$ is continuous and decreasing in $r_i$, $D_i(\cdot)$ 
    is     continuous and increasing in $r_i$, and $D_i(0) - \sum_{z\in\Z} L_{iz} 
    h_{iz} (0, \MW_i, \MW_z) < 0$. Then, a unique equilibrium level of rents exists 
    as a function of MWs:
    $$r_i^* =  f\left(\{\MW_i\}_{i\in\Z}\right)$$
\end{prop}
\begin{proof}
    From the equilibrium condition define $g(r_i) = D_i(r_i) - \sum_{z\in\Z} L_{iz} 
    h_{iz} (r_i, \MW_i, \MW_z)$. Per the intermediate value theorem, there exists 
    a value such that $g(r_i^*) = 0$. Furthermore, by monotonicity of $g(\cdot)$ 
    such value is unique.
\end{proof}

Note that equilibrium rents are a function of the entire vector of minimum wages. 
We are interested in two questions. What is the effect of a change in the vector of 
MWs $(\{d \ln \MW_i\}_{i\in\Z})'$ on equilibrium rents?
Under what conditions can one reduce the dimensionality of the rents function?
The remaining propositions answer those questions.

\begin{prop}[Comparative Statics]\label{prop:comparative_statics}
    Under the assumptions of (i) exogenous distribution of workers across workplace 
    and residence, (ii) housing demand equation satisfying Assumption 
    \ref{assu:housing_function}, and (iii) continuously differentiable and increase 
    housing supply, we have that workplace-MW hikes increase rents, and residence-MW 
    hikes, conditional on workplace-MW hikes, decrease rents.
\end{prop}

\begin{proof}
    Fully differentiate the market clearing condition with respect to $\ln r_i$ and 
    $\ln \MW_i$ for all $i\in\Z$ and re-arrange terms to get
    \begin{equation}\label{eq:diff_equilibrium}
        \Big(\eta_i - \sum_z \pi_{iz} \xi_{iz} \Big) d \ln r_i
        = 
        \sum_z \pi_{iz} \left(\epsilon_{iz}^i d \ln \MW_i 
                            + \epsilon_{iz}^z d \ln \MW_z \right)
    \end{equation}
    where 
    $\pi_{iz} = \frac{L_{iz}}{L_i}$ represents the share of workers from $i$ working 
    in	$z$;
    $\xi_{iz} = \frac{d h_{iz}}{d r_i} \frac{r_i}{\sum_i \pi_{iz} h_{iz}}$ is the 
    elasticity of housing demand at the average per-capita demand of ZIP code $i$;
    $\epsilon_{iz}^i = \frac{d h_{iz}}{d \MW_i} \frac{\MW_i}{\sum_i \pi_{iz} h_{iz}}$ and 
    $\epsilon_{iz}^z = \frac{d h_{iz}}{d \MW_z} \frac{\MW_z}{\sum_i \pi_{iz} h_{iz}}$ 
    are the elasticities of housing demand to workplace and residence MWs also at 
    the average per-capita demand of ZIP code $i$; and
    $\eta_i = \frac{1}{L_i} \frac{d D_i}{d r_i} \frac{r_i}{D_i}$ is the	elasticity 
    of housing supply in ZIP code $i$.

    Because $\xi_{iz} < 0$, $\epsilon_{iz}^i < 0$, and $\epsilon_{iz}^z > 0$ for 
    all $z\in Z$, it is apparent from \eqref{eq:diff_equilibrium} that an increase 
    in workplace MW unambiguously increases rents, whereas a residence MW increase 
    will have an unconditional muted effect
    and a negative effect conditional on the experienced MW.%
    \footnote{The sign of the overall partial effect depends on the sign of 
    $\sum_z \pi_{iz} \epsilon_{iz}^i + \pi_{ii} \epsilon_{ii}^z$.} 
\end{proof}

Proposition \ref{prop:comparative_statics} shows that, under conditions on the 
direction of the effect of MW changes and regularity conditions on the demand 
function, we can uniquivocably establish the influence of the MW on rents. 
Interestingly, increases in MW changes in some subset of zipcodes $Z\in\Z\setminus\{i\}$ 
will affect $i$ if some of $i$'s residents work in some $z\in\Z$.
In this simple model of supply and demand we find spatial spillovers.

\begin{prop}[Representation]\label{prop:representation}
    Under the assumption of constant elasticity of housing demand to workplace minimum 
    wages, we can write the change in log rents as a function of the change in two 
    MW-based measures: the \textbf{experienced log MW} and the \textbf{statutory 
    log MW}.
\end{prop}

\begin{proof}
    Under the assumption that $\epsilon_{iz}^z = \epsilon_i^z$ for all $z\in\Z$ we can 
    manipulate \eqref{eq:diff_equilibrium} to write
    $$
    d \ln r_i = \beta_i \sum_i \pi_{iz} d\ln \MW_z + \gamma_i d \ln \MW_i
    $$
    where $\beta_i = \frac{\epsilon_{i}^z}{\eta_{i} - \sum_z \pi_{iz} \xi_{iz}} 
    >0$ and $\gamma_i = \frac{\sum_z \pi_{iz} \epsilon_{iz}^i}{\eta_{i} 
            - \sum_z \pi_{iz} \xi_{iz}} < 0$.
\end{proof}

Proposition \ref{prop:representation} shows that, as an approximation to small changes in 
the profile of MWs, the change in rents in the ZIP code as a function of two MW-based 
measures.
This motivates our empirical strategy, where we further impose that $\beta_i = \beta$ and 
$\gamma_i=\gamma$ for all $i\in\Z$.%
\footnote{The assumption of constant effects is sufficient, although not necessary. A 
weaker assumption is that, in the context of the empirical model, the heterogeneity in 
these parameters is uncorrelated to rents.}

\subsection{Extensions and discussion}

The supply and demand framework is static.
In Appendix \ref{sec:dyn_theory_model} we extend the framework to allow for 
month-to-month dynamics, and show that the similar conclusions apply.
% SH: Something interesting to say about dynamics?

The analysis arrives at the effect of MW changes on rents through a market clearing
assumption.
An alternative story is that rental markets are characterized by a bargaining process 
between landlords and tenants.
Appendix \ref{sec:bargaining_model} shows a simple Nash bargaining model in which
the tenant's outside option depends positively on workplace MW and negatively on
residence MW.
Comparative statics analogous to those in Proposition \ref{prop:comparative_statics} 
obtain in this case.
