
In this section we layout a simple demand and supply model of local rental markets.
We use the model to illustrate why we expect a different impact of MW changes 
on rents at workplace and residence locations.
Because we study the consequences of MW changes in the very-short run, our model is 
static.
We discuss the addition of the time dimension in Appendix \ref{sec:dyn_theory_model}.
We also assume an exogenous distribution of workers across residence and workplace 
locations.
We think of a spatial model with worker mobility across ZIP codes as an avenue 
for future work.

We emphasize that the model is designed to highlight a possible mechanism through 
which one may expect residence and workplace MWs to have a different impact on the 
housing market.
Our empirical results do not hinge on any of the assumptions made in this section;
however, they reject a model in which workplace and residence locations have the 
same effect.

\subsection{Setup}

We consider the rental market of some ZIP code $i$ embedded in a larger geography 
composed of a finite number of ZIP codes $\Z$.
Workers with residence $i$ work in some other ZIP code $z\in\Z(i)$, where 
$\Z(i)\subseteq\Z$.
More precisely, we let $L_{iz}$ denote the measure of $i$'s residents who work in 
$z$; and 
$L_i = \sum_{z \in \Z(i)} L_{iz}$ and $L_z = \sum_{i \in \Z(i)} L_{iz}$ the number
of residents in $i$ and workers in $z$, respectively.
We assume that the distribution of residence-workplace pairs is fixed.%
\footnote{To simplify we assume that all of $i$' residents work, so that the number
of residents equals the number of workers.}
Each ZIP code has a binding minimum wage, which we denote by $\{\MW_z\}_{z\in\Z(i)}$.

\subsubsection*{Housing demand}

Each group $(i,z)$ consume
square feet of living space $h_{iz}$, 
a non-tradable good produced in their residence $c_{iz}^{NT}$, and
a tradable good $c_{iz}^T$.
A representative $(i,z)$ worker chooses between these alternatives by maximizing
a quasi-concave utility function 
$u_{iz} = u \left(h_{iz}, c^{\text{NT}}_{iz}, c^{\text{T}}_{iz}\right)$
subject to a budget constraint
$$R_i h_{iz} + P_i(\MW_i) c^{\text{NT}}_{iz} + c^{\text{T}}_{iz} \leq y_{iz}(\MW_z),$$
where
$R_i$ gives the rental price of housing per square feet,
$P_i(\MW_i)$ gives the price of local consumption,
the price of tradable consumption is normalized to one, and 
$y_{iz}(\MW_z)$ is an income function.
We summarize the effect of MWs below.

\begin{assu}[Effect of Minimum Wages]\label{assu:mws}
    We assume that
    (i) the prices of non-tradable goods are increasing in $i$'s MW, 
    $\frac{d P_i}{d \MW_i} > 0$, and
    (ii) income is weakly increasing in $z$'s MW, and stricly increasing for at 
    least one $z\in\Z(i)$.
\end{assu}

We think that the structure of the problem and
Assumption \ref{assu:mws} are supported by the literature.
First, recent evidence by \textcite{MiyauchiEtAl2021} shows that individuals tend 
to consume close to home.
As a result, we expect them to be sensible to prices of local consumption in their 
same neighborhood, justifying the inclusion of $c^{\text{NT}}_{iz}$ in the utility
function.%
\footnote{An extension of the model would allow workers to consume in any ZIP code
in the metropolitan area.
While theoretically straightforward, this extension would require data on consumption
trips, which we lack.
We think of our model as an approximation.}
Second, MWs hikes have been shown to increase prices of local consumption 
\parencite[e.g.,][]{AllegrettoReich2018, Leung2021},
and also to increase wage income even beyond the new MW 
level \parencite[e.g.,][]{CegnizEtAl2019},.%
\footnote{An extension would allow separate wage income and business income in 
the budget constraint.
If firm owners tend to live where they work, and MW increases lower profits,
then business income would depend negatively on the local MW.}
%% SH: It would be cool to cite some paper on the relationship between consumption
%%     prices and rents.

The solution to the worker's problem for each $z$ yields a set of continuously 
differentiable housing demand functions $\{h_{iz} (R_i, P_i, Y_z)\}_{z\in\Z(i)}$.
Standard arguments imply that this function is decreasing in its own price $R_i$.
We assume conditions that ensure that housing demand is decreasing in local 
prices $P_i$.%
\footnote{To formalize the required condition, let $h_{iz}$ and $c_{iz}$ denote 
Marshallian demands resulting from the choice problem, and $\tilde h_{iz}$ 
denote the Hicksian housing demand.
The Slutsky equation implies that
$$\frac{\partial h_{iz}}{\partial P_i} 
   = \frac{\partial \tilde h_{iz}}{\partial P_i} 
   - \frac{\partial h_{iz}}{\partial y_{iz}} c_{iz}.$$
To obtain $\frac{\partial h_{iz}}{\partial P_i} < 0$, we require that 
$\frac{\partial \tilde h_{iz}}{\partial P_i} 
< \frac{\partial h_{iz}}{\partial y_{iz}} c_{iz}$, i.e., the income effect of an 
increase in non-tradable prices is larger than the corresponding substitution 
effect.}
Finally, we assume that housing is a normal good, so that housing demand is 
increasing in income $Y_z$.

Note that, given our assumptions, an increase in a group's $(i,z)$ workplace MW 
will tend to increase housing demand in $i$, and an increase in residence MW
will have a negative effect (conditional on its effect via the workplace MW
of the group $i,i$).


\subsubsection*{Housing supply}

We assume that the supply of square feet in $i$ is given by a function $D_i(R_i)$,
and we assume that this function is weakly increasing in $R_i$.
Note that this formulation allows for an upper limit on the number of houses at 
which point the supply becomes perfectly inelastic.

\subsection{Equilibrium and Comparative Statics}

Total demand of housing in ZIP code $i$ is given by the sum of the demands of 
each group.
Thus, we can write the equilibrium condition in this market as
\begin{equation}\label{eq:equilibrium}
	\sum_{z\in\Z(i)} L_{iz} h_{iz} \left(R_i, P_i(\MW_i), Y_z(\MW_z)\right) = D_i(R_i) .
\end{equation}
Given that housing demand functions are continuous and decreasing in rents, 
under a suitable regularity condition there is a unique equilibrium in this market.%
\footnote{To see this, assume that 
$D_i(0) - \sum_{z\in\Z(i)} L_{iz} h_{iz} (0, P_i, Y_z) < 0$
and apply the intermediate value theorem.}
Equilibrium rents are a function of the entire set of minimum wages, formally, 
$r^*_i = f(\{\MW_i\}_{i\in\Z(i)})$.

We are interested in two questions.
What is the effect of a change in the vector of MWs $(\{d \ln \MW_i\}_{i\in\Z(i)})'$
on equilibrium rents?
Under what conditions can we reduce the dimensionality of the rents function and 
represent the effects of MW changes on equilibrium rents in a simpler way?
We start with the first question.

\begin{prop}[Comparative Statics]\label{prop:comparative_statics}
    Consider a change in MW policy such that 
    for $z\in\Z_0 \subseteq \Z(i)$ binding MWs increase, 
    and for $z\in\Z(i)\setminus \Z_0$ binding MWs do not change,
    where $\Z_0$ is non-empty.
    Under the assumptions of unchanging $\{L_iz\}_{z\in\Z(i)}$ 
    and Assumption \ref{assu:mws},
    we have that
    (i) for $z\in\Z_0\setminus\{i\}$ for which $\frac{d Y_z}{d \MW_z}>0$, 
    the policy has a positive partial effect on rents;
    (ii) for ZIP code $i$ the partial effect of the policy is ambiguous; and
    (iii) as a result, the overall effect on rents is ambiguous if $i\in\Z_0$ 
    and weakly positive in $i\notin\Z_0$.
\end{prop}

\begin{proof}
    Fully differentiate the market clearing condition with respect to $\ln R_i$ 
    and $\ln \MW_i$ for all $i\in\Z(i)$.
    Dividing by \ref{eq:equilibrium} and each of the variables appropriately, 
    one can show that
    \begin{equation}\label{eq:diff_equilibrium}
        \Big(\eta_i - \sum_z \pi_{iz} \xi^r_{iz} \Big) d \ln R_i
        = 
        \sum_z \pi_{iz} \left(\xi^p_{iz} \epsilon_{i}^p d \ln \MW_i 
                            + \xi^y_{iz} \epsilon_{z}^y d \ln \MW_z \right) ,
    \end{equation}
    where
    $\pi_{iz} = \frac{L_{iz}}{L_i}$ represents the share of $i$'s residents 
    working in $z$;
    $\xi_{iz}^x = \frac{d h_{iz}}{d x_i} \frac{x_i}{\sum_z \pi_{iz} h_{iz}}$ for
    $x\in\{R,P,Y\}$ is the elasticity of housing demand at the average demand of
    ZIP code $i$;
    $\epsilon_{i}^p = \frac{d P_i}{d \MW_i} \frac{\MW_i}{P_i}$ and 
    $\epsilon_{z}^y = \frac{d Y_z}{d \MW_z} \frac{\MW_z}{Y_z}$ are
    elasticities of prices and income to minimum wages; and
    $\eta_i = \frac{d D_i}{d R_i} \frac{R_i}{D_i}$ is the elasticity 
    of housing supply in ZIP code $i$.

    For each $z\in\Z_0\setminus\{i\}$ the partial effect on rents of the policy
    is given by 
    $$\left(\eta_i - \sum_z \pi_{iz} \xi^r_{iz}\right)^{-1} 
      \pi_{iz}\xi^y_{iz}\epsilon_{z}^y d\ln\MW_z.$$
    Because $\eta_i>0$ and $\xi^r_{iz}\leq 0$  $\forall z\in\Z(i)$, the first 
    factor is positive.
    Thus, if $\epsilon_{z}^y>0$ then this effect is positive, as desired.

    For ZIP code $i$ the partial effect is given by
    $$\left(\eta_i - \sum_z \pi_{iz} \xi^r_{iz}\right)^{-1} 
      \left(\epsilon_{i}^p \sum \pi_{iz}\xi^p_{iz} 
            + \pi_{ii}\xi^y_{ii}\epsilon_i^y \right)d\ln\MW_i.$$
    Because $\epsilon_{i}^p>0$, $\xi^p_{iz}<0$ $\forall z\in\Z(i)$,
    and $\epsilon_{i}^y\geq0$, 
    then the sign of this partial effect is ambiguous.
    The statement under (iii) follows directly.
\end{proof}

Proposition \ref{prop:comparative_statics} (i) shows that, 
if at least some low-wage worker (for whom $\frac{d Y_z}{d \MW_z}>0$) 
commutes to a ZIP code where the MW increased, 
then the MW hike will tend to increase rents.
Proposition \ref{prop:comparative_statics} (ii) establishes that a decreasing
effect on rents may follow if ZIP code $i$ also increases rents.
As a result, the overall effect of the policy is not determined a priori.

The following proposition establishes conditions under which the dimensionality
of equation \ref{eq:diff_equilibrium} can be reduced.

\begin{prop}[Representation]\label{prop:representation}
    Under the assumptions of 
    homogeneous elasticity of housing demand to incomes $Y_z$ and
    homogeneous elasticity of income to minimum wages $\MW_z$,
    we can write the change in log rents as a function of the change in two 
    MW-based measures: ZIP code's $i$ \textbf{workplace MW} and 
    \textbf{residence MW}.
    Furthermore, the workplace MW has a positive effect on rents, whereas the
    residence MW has a negative effect.
\end{prop}
\begin{proof}
    We assume that, for all $z\in\Z(i)$, we have 
    $\xi^y_{iz}=\xi^y_{i}$ and 
    $\epsilon_z^y=\epsilon^y$.
    Then, we can manipulate \eqref{eq:diff_equilibrium} to write
    \begin{equation} \label{eq:theory_represenation}
        d r_i = \beta_i  d \mw^{\text{exp}}_i
              + \gamma_i d \mw^{\text{res}}_i
    \end{equation}
    where
    $r_i=\ln R_i$ reprensents the log of rents,
    $\mw^{\wkp}_i = \sum_{z\in\Z(i)} \pi_{iz} d \ln \MW_z$ and
    $\mw^{\res}_i = d \ln \MW_i$ are defined as in $i$'s 
    \textit{workplace} and \textit{residence} MW levels; and
    $\beta_i = \frac{\sum_{z\in\Z(i)}\pi_{iz}\xi_{iz}^{p}\epsilon_{i}^{p}}
                    {\eta_{i} - \sum_z \pi_{iz} \xi_{iz}} 
             >0$ and
    $\gamma_i = \frac{\sum_z \xi_{i}^{y}\epsilon^{y}}
                     {\eta_{i} - \sum_z \pi_{iz} \xi_{iz}} 
              < 0$
    are parameters.
\end{proof}

Proposition \ref{prop:representation} shows that, under an homogeneity assumption,%
\footnote{We acknoweldge that this simpliflying assumption will not hold exactly
in practice.
For our empirical estimates, we need the weaker assumption that heterogeneity in 
the effect of workplace MWs is not correlated to shocks in the housing market.}
the change in rents following a small changes in the profile of MWs can be expressed 
as a function of two MW-based measures: 
one summarizing the effect of MW changes in workplaces,
and another one summarizing the effect of the MW in the same ZIP code $i$.
This motivates our empirical strategy, where we regress log rents on the empirical
counterparts of these measures.

