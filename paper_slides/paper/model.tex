
In this section we lay down a motivating demand and supply model of the rental market. 
We intend to illustrate why one may expect a different impact of workplace and residence
MW changes.
We also show how, under certain assumptions, changes in log rents can be expressed as a 
function of the changes in two MW-based measures.

The model is purposefully stylized.
Because we study the consequences of MW changes in the very-short run, our model is 
static. We discuss the addition of the time dimension in Appendix XX.
The model also assumes several properties of demand and supply equations. We discuss
an example of a micro-foundation of them in Appendix XX.
Finally, our model features an exogenous distribution of people across residence and 
workplace locations. This assumption is, again, motivated by the short-run nature of our 
empirical question.
We think of the specification of a spatial model to study the longer-run and welfare 
consequences of MW changes as an avenue for future work.

We emphasize that the model is designed to highlight a possible mechanism through which 
one may expect residence and workplace MWs to have different impacts. 
However, our empirical results do not hinge on any of the assumptions made in this 
section.

\subsection{Set-up}

We consider the rental market of some ZIP code $i$ embedded in a geography, which is 
characterized by a set of ZIP codes $\Z$.
Workers with residence $i$ work in some other ZIP code $z\in\Z$. More precisely, we let 
$L_{iz}$ denote the measure $i$'s residents who work in $z$;
$L_i = \sum_{z \in \Z} L_{iz}$ and $L_z = \sum_{i \in \Z} L_{iz}$ 
the number of residents in $i$ and workers in $z$, respectively;
and $\mathcal{L}=\sum_{z \in \Z}\sum_{i \in \Z}L_{iz}$ the total number of workers. 
We assume that the distribution of residence-workplace pairs is fixed.%
\footnote{To simplify we assume that all of $i$' residents work.}

There is a distribution of minimum wages $\{\MW_z\}_{z\in\Z}$, which will affect 
differently to each group $(i,z)$ depending on whether the MW is in their residence or 
workplace. 
We want to explore what are the consequences of some change in the MWs in static 
equilibrium. We explore the consequences of adding a time dimension in Appendix XX.

\subsection{Demand and Supply of Rentals}

In this simple static model all workers have to rent a house in a common market, where 
the rental rate is $r_i$. 
We assume that group $(i,z)$'s demand of square feet per person is given by $h_{iz} (r_i, 
\MW_i, \MW_z)$, where the second argument corresponds to the \textit{residence} MW, and 
the third to the \textit{workplace} MW. 
We characterize the properties of this set of functions below.

\begin{assu}[Properties of housing demand]\label{assu:housing_function}
	For all residence-workplace pairs, the housing demand function $h_{iz} (r_i, 	
	\MW_i, \MW_z)$ is 
	(i) continuously differentiable in its three arguments;
	(ii) decreasing in rental prices $r_i$;
	(iii) decreasing in residence MW, $\MW_i$;
	(iv) increasing in workplace MW, $\MW_z$.
\end{assu}

Explain intuition of assumption here. Cite papers about the effect of MW on prices. 
Point to appendix for micro-foundation.

The supply, on the other hand, is more standard. We assume that $D_i(r_i)$ gives the 
supply of square feet in $i$, which is increasing in $r_i$. Note that this formulation 
allows for an upper limit on the number of houses at which point the supply becomes 
perfectly inelastic.

\subsection{Equilibrium and Comparative Statics}

Total demand of housing in ZIP code $i$ is given by the sum of the demands of each group. 
Thus, we can write the equilibrium condition in this market as

\begin{equation}\label{eq:equilibrium}
	\sum_{z\in\Z} L_{iz} h_{iz} (r_i, \MW_i, \MW_z) = D_i(r_i)
\end{equation}

We organize the main results in a couple of propositions.

\begin{prop}[Equilibrium]\label{prop:equilibrium}
	Assume that $h_{iz}(\cdot)$ is continuous and decreasing in $r_i$, $D_i(\cdot)$ is 
	continuous and increasing in $r_i$, and $D_i(0) - \sum_{z\in\Z} L_{iz} h_{iz} (0, 
	\MW_i, \MW_z) < 0$. Then, a unique equilibrium level of rents exists as a function of 
	MWs:
	$$\ln r_i^* =  f\left(\{\MW_i\}_{i\in\Z}\right)$$
\end{prop}
\begin{proof}
	From the equilibrium condition define $g(r_i) = D_i(r_i) - \sum_{z\in\Z} L_{iz} 
	h_{iz} (r_i, \MW_i, \MW_z)$. Per the intermediate value theorem, there exists a value 
	such that $g(r_i) = 0$. Furthermore, by monotonicity of $g(r_i)$ such value is unique.
\end{proof}

Note that equilibrium rents are a function of the entire vector of minimum wages. 
We are interested in two questions. What is the effect of a change in the vector of 
MWs $(\{d \ln \MW_i\}_{i\in\Z})'$ on equilibrium rents?
Under what conditions can one reduce the dimensionality of the rents function?
The remaining propositions answer those questions.

\begin{prop}[Comparative Statics]\label{prop:comparative_statics}
	Under the assumptions of (i) exogenous distribution of workers across workplace 
	and residence, (ii) housing demand equation satisfying Assumption 
	\ref{assu:housing_function}, and (iii) continuously differentiable and increase 
	housing supply, we have that workplace-MW hikes increase rents, and residence-MW 
	hikes, conditional on workplace-MW hikes, decrease rents.
\end{prop}

\begin{proof}
	Fully differentiate the market clearing condition with respect to $\ln r_i$ and 
	$\ln \MW_i$ for all $i\in\Z$ and re-arrange terms to get
	\begin{equation}\label{eq:diff_equilibrium}
		\sum_i \pi_{iz} \xi_{iz} d \ln r_i
		+ \sum_i \pi_{iz} \epsilon_{iz}^i d \ln \MW_i 
		+ \sum_i \pi_{iz} \epsilon_{iz}^z d \ln \MW_z
		= \eta_i d \ln r_i
	\end{equation}	
	where 
	$\pi_{ni} = \frac{L_{ni}}{L_n}$ represent the share of workers from $i$ working in 
	$z$;
	$\epsilon_{iz}^i = \frac{d h_{iz}}{d \MW_i} \frac{\MW_i}{\sum_i \pi_{iz} h_{iz}}$ and 
	$\epsilon_{iz}^z = \frac{d h_{iz}}{d \MW_z} \frac{\MW_z}{\sum_i \pi_{iz} h_{iz}}$ 
	are the elasticities of housing demand to workplace and residence MWs evaluated at 
	the average per-capita housing demand of ZIP code $i$; and
	$\eta_i = \frac{1}{L_i} \frac{d D_i}{d r_i} \frac{r_i}{\sum_i \pi_{iz} h_{iz}}$ is 
	the per-person elasticity of housing supply in ZIP code $i$.
	
	Because $\epsilon_{iz}^i < 0$ and $\epsilon_{iz}^z > 0$ $\forall z\in Z$, it is 
	apparent from \eqref{eq:diff_equilibrium} that an increase in workplace
	MW unambiguously increases rents, whereas a	residence MW increase will have an 
	unconditional muted effect,%
	\footnote{The sign of the partial effect depends on the sign of 
		$\sum_i \pi_{iz} \epsilon_{iz}^i + \pi_{ii} \epsilon_{ii}^z$.} 
	and a negative effect conditional on the experienced MW.
\end{proof}

Proposition \ref{prop:comparative_statics} shows that, under sign and regularity 
conditions on the direction of the effect of MW changes, we can establish their influence 
on rents. 
Interestingly, increases in MW changes in some subset of zipcodes $Z\in\Z\setminus\{i\}$ 
will affect it if some of $i$'s residents work in $Z$. 
In this simple model of supply and demand we find spatial spillovers.

\begin{prop}[Representation]\label{prop:representation}
	Under the assumption of constant elasticity of housing demand to workplace minimum 
	wages, we can write the change in log rents as a function of the change in two 
	MW-based measures: the \textbf{statutory log MW} and the \textbf{experienced log MW}.
\end{prop}

\begin{proof}
	Under the assumption that $\epsilon_{iz}^z = \epsilon_i^z$ for all $z\in\Z$ we can 
	manipulate \eqref{eq:diff_equilibrium} to write
	$$
	d \ln r_i = \beta_i \sum_i \pi_{iz} d\ln \MW_z + \gamma_i d \ln \MW_i
	$$
	where $\beta_i = \frac{\epsilon_{i}^z}{\eta_{i} - \sum_z \pi_{iz} \epsilon_{iz}^i} 
	>0$ and $\gamma_i = \frac{\sum_z \pi_{iz} \epsilon_{iz}^i}{\eta_{i} - \sum_z \pi_{iz} 
	\epsilon_{iz}^i} < 0$.
\end{proof}

Proposition \ref{prop:representation} shows that, as an approximation to small changes in 
the profile of MWs, we can approximate the change in rents in the ZIP code as a function 
of two MW-based measures.
This motivates our empirical strategy, where we further impose that $\beta_i = \beta$ and 
$\gamma_i=\gamma$ for all $i\in\Z$.%
\footnote{The assumption of constant effects is sufficient, although not necessary. A 
weaker assumption is that, in the context of the empirical model, the heterogeneity in 
these parameters is uncorrelated to rents.}

\subsection{Extensions}

We entertain two extensions of the basic framework.

In Appendix \ref{sec:model_microfoundation} we show how a housing demand equation as in 
Assumption \ref{assu:housing_function} can be derived from a maximization problem at the 
level of the individual. DISCUSS.

In Appendix \ref{sec:dyn_theory_model} we extend the framework to allow for dynamics.

