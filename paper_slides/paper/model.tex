
In this section we layout a simple demand and supply model of local rental markets.
We use the model to illustrate why we expect a different impact of MW changes 
on rents at workplace and residence locations.
Because we study the consequences of MW changes in the very-short run, our model is 
static.
We discuss the addition of the time dimension in Appendix \ref{sec:dyn_theory_model}.
We also assume an exogenous distribution of workers across residence and workplace 
locations.
We think of a spatial model with worker mobility across ZIP codes as an avenue 
for future work.

We emphasize that the model is designed to highlight a possible mechanism through 
which one may expect residence and workplace MWs to have a different impact on the 
housing market.
Our empirical results do not hinge on any of the assumptions made in this section;
however, they reject a model in which workplace and residence locations have the 
same effect.

\subsection{Setup}

We consider the rental market of some ZIP code $i$ embedded in a larger geography 
composed of a finite number of ZIP codes $\Z$.
Workers with residence $i$ work in some other ZIP code $z\in\Z(i)$, where 
$\Z(i)\subseteq\Z$.
More precisely, we let $L_{iz}$ denote the measure of $i$'s residents who work in 
$z$; and 
$L_i = \sum_{z \in \Z(i)} L_{iz}$ and $L_z = \sum_{i \in \Z(i)} L_{iz}$ the number
of residents in $i$ and workers in $z$, respectively.
We assume that the distribution of residence-workplace pairs is fixed.%
\footnote{To simplify we assume that all of $i$' residents work, so that the number
of residents equals the number of workers.}
Each ZIP code has a binding minimum wage, which we denote by $\{\MW_z\}_{z\in\Z(i)}$.

\subsubsection*{Housing demand}

Each group $(i,z)$ consume
square feet of living space $h_{iz}$, 
a non-tradable good produced in their residence $c_{iz}^{NT}$, and
a tradable good $c_{iz}^T$.
A representative $(i,z)$ worker chooses between these alternatives by maximizing
a continuously differentiable utility function 
$u_{iz} = u \left(h_{iz}, c^{\text{NT}}_{iz}, c^{\text{T}}_{iz}\right)$
subject to a budget constraint
$$r_i h_{iz} + p_i(\MW_i) c^{\text{NT}}_{iz} + c^{\text{T}}_{iz} \leq y_{iz}(\MW_z),$$
where
$r_i$ gives the rental price of housing per square feet,
$p_i(\MW_i)$ gives the price of local consumption,
the price of tradable consumption is normalized to one, and 
$y_{iz}(\MW_z)$ is an income function.
We assume that 
$\frac{d p_i}{d \MW_i} \geq 0$, so that local consumption prices are weakly 
increasing in the local MW, and
$\frac{d y_{iz}}{d \MW_z} \geq 0$, so that income is weakly increasing in the 
workplace MW.%
\footnote{An extension would allow separate wage income and business income in 
the budget constraint.
If firm owners tend to live where they work, and MW increases lower profits,
then business income would depend negatively on the local MW.}

The solution to this problem for each $z$ delivers a set of continuously 
differentiable housing demand functions $\{h_{iz} (r_i, p_i, y_z)\}_{z\in\Z(i)}$.
Standard arguments imply that this function is decreasing in its own price $r_i$.
We assume that housing demand is decreasing in local prices $p_i$.%
\footnote{To formalize this argument, let $h_{iz}$ and $c_{iz}$ denote 
Marshallian demands resulting from the choice problem, and $\tilde h_{iz}$ 
denote the Hicksian housing demand.
The Slutsky equation implies that
$$\frac{\partial h_{iz}}{\partial p_i} 
   = \frac{\partial \tilde h_{iz}}{\partial p_i} 
   - \frac{\partial h_{iz}}{\partial y_{iz}} c_{iz}.$$
To obtain $\frac{\partial h_{iz}}{\partial p_i} < 0$, we require that 
$\frac{\partial \tilde h_{iz}}{\partial p_i} 
< \frac{\partial h_{iz}}{\partial y_{iz}} c_{iz}$, i.e., the income effect is 
larger than the substitution effect.}
Finally, we assume that housing is a normal good, so that housing demand is 
increasing in income $y_z$.

Note that, given our assumptions, an increase in a group's $(i,z)$ workplace MW 
will tend to increase housing demand in $i$, and an increase in residence MW
will have a negative effect (conditional on its effect via the workplace MW
of group $(i,i)$).
The assumptions that deliver this implication are supported by the literature.
First, recent evidence by \textcite{MiyauchiEtAl2021} shows that individuals tend 
to consume close to home.
As a result, we expect them to be sensible to prices of local consumption in their 
same neighborhood.%
\footnote{An extension of the model would allow workers to consume in any ZIP code
in the metropolitan area.
While theoretically straightforward, this extension would require data on consumption
trips, which we lack.
We think of our model as an approximation.}
Second, MWs hikes have been shown to increase wage income even beyond the new MW 
level \parencite[e.g.,][]{CegnizEtAl2019} 
and also to increase prices of local consumption 
\parencite[e.g.,][]{AllegrettoReich2018, Leung2021}.
%% SH: It would be cool to cite some paper on the relationship between consumption
%%     prices and rents.

\subsubsection*{Housing supply}

We assume a simple supply side. Denote by $D_i(r_i)$ the supply of square feet in 
$i$, which is increasing in $r_i$.
Note that this formulation allows for an upper limit on the number of houses at 
which point the supply becomes perfectly inelastic.

\subsection{Equilibrium and Comparative Statics}

Total demand of housing in ZIP code $i$ is given by the sum of the demands of each group. 
Thus, we can write the equilibrium condition in this market as
\begin{equation}\label{eq:equilibrium}
	\sum_{z\in\Z(i)} L_{iz} h_{iz} (r_i, \MW_i, \MW_z) = D_i(r_i) .
\end{equation}
Given that housing demand functions are continuous and decreasing in rents, 
under a suitable regularity condition there is a unique equilibrium in this market.%
\footnote{Assume $D_i(0) - \sum_{z\in\Z(i)} L_{iz} h_{iz} (0, \MW_i, \MW_z) < 0$
and apply the intermediate value theorem.}
We denote equilibrium rents as $r^*_i = f(\{\MW_i\}_{i\in\Z(i)})$.

%%% SH: Formal proposition for Equilibrium below
%%%     I think it's not necessary to include it since this is kind of trivial to show
% \begin{prop}[Equilibrium]\label{prop:equilibrium}
%     Assume that $h_{iz}(\cdot)$ is continuous and decreasing in $r_i$, $D_i(\cdot)$ 
%     is     continuous and increasing in $r_i$, and $D_i(0) - \sum_{z\in\Z} L_{iz} 
%     h_{iz} (0, \MW_i, \MW_z) < 0$. Then, a unique equilibrium level of rents exists 
%     as a function of MWs:
%     $$r_i^* =  f\left(\{\MW_i\}_{i\in\Z}\right)$$
% \end{prop}
% \begin{proof}
%     From the equilibrium condition define $g(r_i) = D_i(r_i) - \sum_{z\in\Z} L_{iz} 
%     h_{iz} (r_i, \MW_i, \MW_z)$. Per the intermediate value theorem, there exists 
%     a value such that $g(r_i^*) = 0$. Furthermore, by monotonicity of $g(\cdot)$ 
%     such value is unique.
% \end{proof}

Note that equilibrium rents are a function of the entire vector of minimum wages. 
We are interested in two questions.
What is the effect of a change in the vector of MWs $(\{d \ln \MW_i\}_{i\in\Z(i)})'$
on equilibrium rents?
Under what conditions can we reduce the dimensionality of the rents function and 
represent the effects of MW changes on equilibrium rents in a simpler way?

\begin{prop}[Comparative Statics]\label{prop:comparative_statics}
    Under the assumptions of
    (i) exogenously given distribution of workers across workplace and residence
    pairs,
    (ii) housing demand equation satisfying Assumption \ref{assu:housing_function}, 
    and 
    (iii) continuously differentiable and increasing housing supply, we have that
    workplace-MW hikes increase rents, and residence-MW hikes, holding constant 
    workplace-MW hikes, decrease rents.
\end{prop}

\begin{proof}
    Fully differentiate the market clearing condition with respect to $\ln r_i$ and 
    $\ln \MW_i$ for all $i\in\Z(i)$ and re-arrange terms to get
    \begin{equation}\label{eq:diff_equilibrium}
        \Big(\eta_i - \sum_z \pi_{iz} \xi_{iz} \Big) d \ln r_i
        = 
        \sum_z \pi_{iz} \left(\epsilon_{iz}^i d \ln \MW_i 
                            + \epsilon_{iz}^z d \ln \MW_z \right) ,
    \end{equation}
    where 
    $\pi_{iz} = \frac{L_{iz}}{L_i}$ represents the share of $i$'s residents working 
    in $z$;
    $\xi_{iz} = \frac{d h_{iz}}{d r_i} \frac{r_i}{\sum_z \pi_{iz} h_{iz}}$ is the 
    elasticity of housing demand at the average per-capita demand of ZIP code $i$;
    $\epsilon_{iz}^i = \frac{d h_{iz}}{d \MW_i} \frac{\MW_i}{\sum_z \pi_{iz} h_{iz}}$ and 
    $\epsilon_{iz}^z = \frac{d h_{iz}}{d \MW_z} \frac{\MW_z}{\sum_z \pi_{iz} h_{iz}}$ 
    are the elasticities of housing demand to workplace and residence MWs also at 
    the average per-capita demand of ZIP code $i$; and
    $\eta_i = \frac{1}{L_i} \frac{d D_i}{d r_i} \frac{r_i}{D_i}$ is the elasticity 
    of housing supply in ZIP code $i$.

    Because $\xi_{iz} < 0$ and, for at least some workplace, $\epsilon_{iz}^i < 0$ 
    and $\epsilon_{iz}^z > 0$, it is apparent from \eqref{eq:diff_equilibrium} that 
    an increase in workplace MW unambiguously increases rents, whereas the effect 
    of an increase in residence MW on rents is generally ambiguous (as long as some 
    residents of $i$ also work in $i$) as it is composed of a direct negative effect 
    and an indirect positive effect through changing the experienced MW.%
    \footnote{The sign of the overall partial effect depends on the sign of 
    $\pi_{ii} \epsilon_{ii}^z + \sum_z \pi_{iz} \epsilon_{iz}^i$.}
    Holding constant workplace MWs, the effect of the residence MW is negative.
\end{proof}

Proposition \ref{prop:comparative_statics} shows that, under conditions on the 
direction of the effect of MW changes and regularity conditions on the demand 
function, we can unequivocally establish the influence of the MW on rents. 
Crucially, increases in MW changes in a set of ZIP codes other than $i$ will affect 
rents at $i$ if some of $i$'s residents work in some of those ZIP codes.

\begin{prop}[Representation]\label{prop:representation}
    Under the assumption of constant elasticity of housing demand (across workplace locations)
    to workplace minimum wages,
    we can write the change in log rents as a function of the change in two 
    MW-based measures: the \textbf{experienced log MW} and the \textbf{statutory 
    log MW}.
\end{prop}
\begin{proof}
    Under the assumption that $\epsilon_{iz}^z = \epsilon_i^z$ for all $z\in\Z(i)$ 
    we can manipulate \eqref{eq:diff_equilibrium} to write
    \begin{equation} \label{eq:theory_represenation}
        d \ln r_i = \beta_i \sum_i \pi_{iz} d\ln \MW_z + \gamma_i d \ln \MW_i
    \end{equation}
    where $\beta_i = \frac{\epsilon_{i}^z}{\eta_{i} - \sum_z \pi_{iz} \xi_{iz}} 
    >0$ and $\gamma_i = \frac{\sum_z \pi_{iz} \epsilon_{iz}^i}{\eta_{i} 
            - \sum_z \pi_{iz} \xi_{iz}} < 0$.
\end{proof}

Proposition \ref{prop:representation} shows that, under an homogeneity assumption,%
\footnote{We acknoweldge that this simpliflying assumption will not hold exactly
in practice.
For our empirical estimates, we need the weaker assumption that heterogeneity in 
the effect of workplace MWs is not correlated to shocks in the housing market.}
the change in rents following a small changes in the profile of MWs can be expressed 
as a function of two MW-based measures: one summarizing the effect of workplace MW,
and another one summarizing the effect of residence MW.
This motivates our empirical strategy, where we regress log rents on the empirical
counterparts of the measures.

