
In this section we layout a simple demand and supply model of local rental markets.
We use the model to illustrate why we expect a different impact of MW changes 
on rents at workplace and residence locations.
We show how, under reasonable assumptions, the short-term effects of MW changes 
in log rents can be expressed as a function of the changes in two MW-based measures
that take into account residence and workplace locations.
The model is informative in itself but it will also guide our empirical strategy.

The model is purposefully stylized.
Because we study the consequences of MW changes in the very-short run, our model is 
static.
We discuss the addition of the time dimension in Appendix \ref{sec:dyn_theory_model}.
We also assume an exogenous distribution of workers across residence and workplace 
locations.
We think of a spatial model with worker mobility across ZIP codes as an avenue 
for future work.

We emphasize that the model is designed to highlight a possible mechanism through 
which one may expect residence and workplace MWs to have a different impact on the 
housing market.
Our empirical results do not hinge on any of the assumptions made in this section;
however, they reject a model in which workplace and residence locations have the 
same effect.

\subsection{Setup}

We consider the rental market of some ZIP code $i$ embedded in a larger geography 
$\Z$ composed of a finite number of ZIP codes.
Workers with residence $i$ work in some other ZIP code $z\in\Z(i)$, where 
$\Z(i)\subseteq\Z$.
More precisely, we let $L_{iz}$ denote the measure of $i$'s residents who work in 
$z$; and 
$L_i = \sum_{z \in \Z(i)} L_{iz}$ and $L_z = \sum_{i \in \Z(i)} L_{iz}$ the number
of residents in $i$ and workers in $z$, respectively.
We assume that the distribution of residence-workplace pairs is fixed.%
\footnote{To simplify we assume that all of $i$' residents work, so that the number
of residents equals the number of workers.}
Each ZIP code has a binding minimum wage, which we denote by $\{\MW_z\}_{z\in\Z(i)}$.

\paragraph{Housing demand}

In this simple static model all workers have to rent a house in a common market, 
where the rental rate is $r_i$.
We assume that group $(i,z)$'s demand of square feet per person is given by $h_{iz}
(r_i, \MW_i, \MW_z)$, where the second argument corresponds to the \textit{residence} 
MW, and the third to the \textit{workplace} MW.
We characterize the properties of this set of functions below.

\begin{assu}[Housing demand]\label{assu:housing_function}
    For all residence-workplace pairs, the housing demand function $h_{iz} (r_i, 	
    \MW_i, \MW_z)$ is:
    (i) continuously differentiable in its three arguments;
    (ii) decreasing in rental prices $r_i$;
    (iii) non-increasing in residence minimum wage $\MW_i$;
    (iv) non-decreasing in workplace minimum wage $\MW_z$.
    Furthermore, for at least one $z\in\Z(i)$, the inequalities in points (iii)
    and (iv) are strict.
\end{assu}

Points (i) and (ii) simply say that $h_{iz}$ is a ``smooth'' demand function.
Point (iv) follows from the fact that housing is a normal good.
Given that, under negligible employment effects, workplace MW increases income, 
it should also increase housing demand for workers with earnings close to the MW,
unless that there is no MW workers in a ZIP code.
Point (iii) is a bit more subtle.
Residence MW, while increasing the income of people working and residing in the 
same ZIP code, will also increase the cost of production of local non-tradable 
goods (assuming that workers are an input in non-tradable production).
The higher cost of non-tradables will translate into a lower demand of housing 
if the substitution effect of a change in the price of non-tradables on local demand 
of housing is smaller than the corresponding income effect.
A sufficient condition for that is that housing and local consumption are complements.%
\footnote{We can formalize this discussion with a simple choice problem.
Say a representative $(i,z)$ worker chooses between housing demand $h_{iz}$,
non-tradable consumption $c^{\text{NT}}_{iz}$, and tradable consumption $c^{\text{T}}_{iz}$,
by maximizing
$$u_{iz} = u \left(h_{iz}, c^{\text{NT}}_{iz}, c^{\text{T}}_{iz}\right)$$
subject to 
$r_i h_{iz} + p_i(\MW_i) c^{\text{NT}}_{iz} + c^{\text{T}}_{iz} \leq y_{iz}(\MW_z),$
where $p_i(\MW_i)$ gives the price of local consumption, which is increasing in residence MW;
the price of tradable consumption is normalized to one; and 
$y_{iz}(\MW_z)$ is an income function that depends positively on the workplace MW.
Let $h_{iz}^*$ and $c_{iz}^*$ denote Marshallian demands, and 
$\tilde h_{iz}^*$ denote the Hicksian housing demand.
By assumption, the price of the MW will increase prices of non-tradable consumption.
Thus, consider the effect of an increase in $p_i$ on housing demand.
The Slutsky equation implies that
$$\frac{\partial h_{iz}^*}{\partial p_i} 
   = \frac{\partial \tilde h_{iz}^*}{\partial p_i} 
   - \frac{\partial h_{iz}^*}{\partial y_{iz}} c_{iz}^*.$$
We have that $\frac{\partial h_{iz}^*}{\partial p_i} < 0$ if and only if 
$\frac{\partial \tilde h_{iz}^*}{\partial p_i} 
< \frac{\partial h_{iz}^*}{\partial y_{iz}} c_{iz}^*$.}
Another possibility is to introduce firms that produce non-tradable local goods, and that 
use MW workers as an input. Under perfect competition, after a MW increase, the firms will 
charge a higher price to hit the zero profit condition and not go out of business. Now the 
residents that don't work in that ZIP code will pay a higher price for their local good 
and they will have less disposable income for housing.
%%% DGP: I changed the alternative explanation because I thought it is simpler to connect with basic
%%%   general equilibrium micro theory. However, the explanation about firm owners living close to firms 
%%%   that they own also makes sense in a world with firms having some market power. Should we also write it?
%%% SH: I think that, with this addition, we are giving the same explanation twice.
%%%   Note that prices are already increasing the MW, where the mechanism you describe
%%%   is implicit.
%%%   I think it'd be better to briefly mention another explanation, like the one about
%%%   owner's losing profits.
%%%   Let me know if you agree and we can go back!
%%%   DGP: I agree with your point about giving the same explanation twice. However, I
%%%   don't think that the alternative explanation works without some market power structure as
%%%   in perfect competition profits would be 0 always. 

We think that the interpretation underlying point (iii) is plausible for several 
reasons.
First, recent evidence by \textcite{MiyauchiEtAl2021} shows that individuals tend 
to consume close to home.
As a result, we expect them to be sensible to prices of local consumption in their 
same neighborhood.
Second, MWs have been shown to increase prices of local consumption 
\parencite[e.g.,][]{AllegrettoReich2018, LeungForthcoming}.
These empirical facts suggest that residence MW changes might (conditional on workplace)
negatively affect incomes and thus demand for housing.

\paragraph{Housing supply}

We assume a simple supply side. Denote by $D_i(r_i)$ the supply of square feet in 
$i$, which is increasing in $r_i$.
Note that this formulation allows for an upper limit on the number of houses at 
which point the supply becomes perfectly inelastic.

\subsection{Equilibrium and Comparative Statics}

Total demand of housing in ZIP code $i$ is given by the sum of the demands of each group. 
Thus, we can write the equilibrium condition in this market as
\begin{equation}\label{eq:equilibrium}
	\sum_{z\in\Z(i)} L_{iz} h_{iz} (r_i, \MW_i, \MW_z) = D_i(r_i) .
\end{equation}
Given that housing demand functions are continuous and decreasing in rents, 
under a suitable regularity condition there is a unique equilibrium in this market.%
\footnote{Assume $D_i(0) - \sum_{z\in\Z(i)} L_{iz} h_{iz} (0, \MW_i, \MW_z) < 0$
and apply the intermediate value theorem.}
We denote equilibrium rents as $r^*_i = f(\{\MW_i\}_{i\in\Z(i)})$.

%%% SH: Formal proposition for Equilibrium below
%%%     I think it's not necessary to include it since this is kind of trivial to show
% \begin{prop}[Equilibrium]\label{prop:equilibrium}
%     Assume that $h_{iz}(\cdot)$ is continuous and decreasing in $r_i$, $D_i(\cdot)$ 
%     is     continuous and increasing in $r_i$, and $D_i(0) - \sum_{z\in\Z} L_{iz} 
%     h_{iz} (0, \MW_i, \MW_z) < 0$. Then, a unique equilibrium level of rents exists 
%     as a function of MWs:
%     $$r_i^* =  f\left(\{\MW_i\}_{i\in\Z}\right)$$
% \end{prop}
% \begin{proof}
%     From the equilibrium condition define $g(r_i) = D_i(r_i) - \sum_{z\in\Z} L_{iz} 
%     h_{iz} (r_i, \MW_i, \MW_z)$. Per the intermediate value theorem, there exists 
%     a value such that $g(r_i^*) = 0$. Furthermore, by monotonicity of $g(\cdot)$ 
%     such value is unique.
% \end{proof}

Note that equilibrium rents are a function of the entire vector of minimum wages. 
We are interested in two questions.
What is the effect of a change in the vector of MWs $(\{d \ln \MW_i\}_{i\in\Z(i)})'$
on equilibrium rents?
Under what conditions can we reduce the dimensionality of the rents function and 
represent the effects of MW changes on equilibrium rents in a simpler way?

\begin{prop}[Comparative Statics]\label{prop:comparative_statics}
    Under the assumptions of
    (i) exogenously given distribution of workers across workplace and residence
    pairs,
    (ii) housing demand equation satisfying Assumption \ref{assu:housing_function}, 
    and 
    (iii) continuously differentiable and increasing housing supply, we have that
    workplace-MW hikes increase rents, and residence-MW hikes, holding constant 
    workplace-MW hikes, decrease rents.
\end{prop}

\begin{proof}
    Fully differentiate the market clearing condition with respect to $\ln r_i$ and 
    $\ln \MW_i$ for all $i\in\Z(i)$ and re-arrange terms to get
    \begin{equation}\label{eq:diff_equilibrium}
        \Big(\eta_i - \sum_z \pi_{iz} \xi_{iz} \Big) d \ln r_i
        = 
        \sum_z \pi_{iz} \left(\epsilon_{iz}^i d \ln \MW_i 
                            + \epsilon_{iz}^z d \ln \MW_z \right) ,
    \end{equation}
    where 
    $\pi_{iz} = \frac{L_{iz}}{L_i}$ represents the share of $i$'s residents working 
    in $z$;
    $\xi_{iz} = \frac{d h_{iz}}{d r_i} \frac{r_i}{\sum_z \pi_{iz} h_{iz}}$ is the 
    elasticity of housing demand at the average per-capita demand of ZIP code $i$;
    $\epsilon_{iz}^i = \frac{d h_{iz}}{d \MW_i} \frac{\MW_i}{\sum_z \pi_{iz} h_{iz}}$ and 
    $\epsilon_{iz}^z = \frac{d h_{iz}}{d \MW_z} \frac{\MW_z}{\sum_z \pi_{iz} h_{iz}}$ 
    are the elasticities of housing demand to workplace and residence MWs also at 
    the average per-capita demand of ZIP code $i$; and
    $\eta_i = \frac{1}{L_i} \frac{d D_i}{d r_i} \frac{r_i}{D_i}$ is the elasticity 
    of housing supply in ZIP code $i$.

    Because $\xi_{iz} < 0$ and, for at least some workplace, $\epsilon_{iz}^i < 0$ 
    and $\epsilon_{iz}^z > 0$, it is apparent from \eqref{eq:diff_equilibrium} that 
    an increase in workplace MW unambiguously increases rents, whereas the effect 
    of an increase in residence MW on rents is generally ambiguous (as long as some 
    residents of $i$ also work in $i$) as it is composed of a direct negative effect 
    and an indirect positive effect through changing the experienced MW.%
    \footnote{The sign of the overall partial effect depends on the sign of 
    $\pi_{ii} \epsilon_{ii}^z + \sum_z \pi_{iz} \epsilon_{iz}^i$.}
    Holding constant workplace MWs, the effect of the residence MW is negative.
\end{proof}

Proposition \ref{prop:comparative_statics} shows that, under conditions on the 
direction of the effect of MW changes and regularity conditions on the demand 
function, we can unequivocally establish the influence of the MW on rents. 
Crucially, increases in MW changes in a set of ZIP codes other than $i$ will affect 
rents at $i$ if some of $i$'s residents work in some of those ZIP codes.

\begin{prop}[Representation]\label{prop:representation}
    Under the assumption of constant elasticity of housing demand (across workplace locations)
    to workplace minimum wages,
    we can write the change in log rents as a function of the change in two 
    MW-based measures: the \textbf{experienced log MW} and the \textbf{statutory 
    log MW}.
\end{prop}
\begin{proof}
    Under the assumption that $\epsilon_{iz}^z = \epsilon_i^z$ for all $z\in\Z(i)$ 
    we can manipulate \eqref{eq:diff_equilibrium} to write
    \begin{equation} \label{eq:theory_represenation}
        d \ln r_i = \beta_i \sum_i \pi_{iz} d\ln \MW_z + \gamma_i d \ln \MW_i
    \end{equation}
    where $\beta_i = \frac{\epsilon_{i}^z}{\eta_{i} - \sum_z \pi_{iz} \xi_{iz}} 
    >0$ and $\gamma_i = \frac{\sum_z \pi_{iz} \epsilon_{iz}^i}{\eta_{i} 
            - \sum_z \pi_{iz} \xi_{iz}} < 0$.
\end{proof}

Proposition \ref{prop:representation} shows that, under an homogeneity assumption,%
\footnote{We acknoweldge that this simpliflying assumption will not hold exactly
in practice.
For our empirical estimates, we need the weaker assumption that heterogeneity in 
the effect of workplace MWs is not correlated to shocks in the housing market.}
the change in rents following a small changes in the profile of MWs can be expressed 
as a function of two MW-based measures: one summarizing the effect of workplace MW,
and another one summarizing the effect of residence MW.
This motivates our empirical strategy, where we regress log rents on the empirical
counterparts of the measures.

