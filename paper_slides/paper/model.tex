
In this section we layout a simple demand and supply model of local rental markets.
We use the model to illustrate why we expect a different impact of MW changes 
on rents at workplace and residence locations. We show how, under reasonable 
assumptions, the short-term effects of MW changes in log rents can be expressed 
as a function of the changes in two MW-based measures that take into account 
residence and workplace locations. The model is informative in itself but it will
also guide empirical strategy.

The model is purposefully stylized.
Because we study the consequences of MW changes in the very-short run, our model is 
static. 
We discuss the addition of the time dimension in Appendix \ref{sec:dyn_theory_model}.
We also assume an exogenous distribution of workers across residence and workplace 
locations.
We think of a spatial model with worker mobility across ZIP codes as an avenue 
for future work.

We emphasize that the model is designed to highlight a possible mechanism through 
which one may expect residence and workplace MWs to have a different impact on the 
housing market. Our empirical results do not hinge on any of the assumptions made 
in this section, however, they reject a model in which workplace and residence locations
have the same effect.

\subsection{Setup}

We consider the rental market of some ZIP code $i$ embedded in a metropolitan area, 
which is characterized by a set of ZIP codes $\Z$.
Workers with residence $i\in\Z$ work in some other ZIP code $z\in\Z$. 
More precisely, we let $L_{iz}$ denote the number of $i$'s residents who work in $z$;
$L_i = \sum_{z \in \Z} L_{iz}$ and $L_z = \sum_{i \in \Z} L_{iz}$ 
the number of residents in $i$ and workers in $z$, respectively.
%%%% DGP: I deleted the total number of workers because it is not clear whether
%%%% it is the total number workers that live in a metropolitan area or the number of 
%%%% workers that work in a metropolitan area. In addition, I don't think that we need that definition.
We assume that the distribution of residence-workplace pairs is fixed.%
\footnote{To simplify we assume that all of $i$' residents work, so that the number
of residents equals the number of workers.}
%%%% DGP: Do we actually need this assumption? For example, in the empirical application, we allow
%%%% for residents of a metropolitan area to work on a different metropolitan area. I guess that
%%%% we can either sum workers from any other zipcode (regardless of its met area) in a given 
%%%% residential zipcode i, we can add an "outside met area zipcode per met area", or we can just
%%%% leave it as is and clarify that in the empirical application we allow for those cases. 
Each ZIP code has a binding minimum wage, which we denote by 
$\{\MW_z\}_{z\in\Z}$.
%%%% DGP: I wouldn't say characterize because zipcodes also have,for example, a rental price.

\paragraph{Housing demand}

In this simple static model all workers have to rent a house in a common market, 
where the rental rate is $r_i$. 
We assume that group $(i,z)$'s demand of square feet per person is given by $h_{iz} (r_i, 
\MW_i, \MW_z)$, where the second argument corresponds to the \textit{residence} MW, and 
the third to the \textit{workplace} MW. 
We characterize the properties of this set of functions below.

\begin{assu}[Housing demand]\label{assu:housing_function}
	For all residence-workplace pairs, the housing demand function $h_{iz} (r_i, 	
	\MW_i, \MW_z)$ is:
	 
	(i) continuously differentiable in its three arguments;
	
	(ii) decreasing in rental prices $r_i$;
	
	(iii) increasing in workplace MW, $\MW_z$;
	
	(iv) decreasing in residence MW, $\MW_i$;
\end{assu}
%%%% DGP: I wonder if we should be saying non-decreasing instead of increasin and non-increasing 
%%%% instead of decreasing. I think it would be easier to digest. Think of the following: a workplace
%%% MW increase cannot possibly decrease the cost of production of a local good. At most it will be zero
%%%% if labor is not an input or if it is perfectly substituteable.

Points (i) and (ii) simply say that $h_{iz}$ is a ``smooth'' demand function.
Point (iii) follows from the fact that housing is a normal good.
Given that, under negligible employment effects, workplace MW increases income 
(at least for the workers who make close to the MW), it should also increase housing 
demand.
Residence MW, while increasing the income of people working in that location, 
it will also increases the cost of production of non-tradable goods. %
\footnote{Unless that labor is not an input for producing it, or if it is perfectly substituteable.}
%%% DGP: We should introduce a non-tradeable local good, at least by saying that it is non-tradeable across zipcodes.
%%% Whether non-treadeables across zipcodes exist is an empirical question that we don't know research about
%%%% but we should be ale to somehow justify it through an example. A good example is a McDonalds: they will 
%%% necesarilly pay the higher MW if they are willing to keep serving food there, and it seems that is a small cost to pay
%%% relative to stop operations.
The higher cost of non-tradables will translate into a lower demand of housing 
if the substitution effect of a change in local prices on local demand of housing
is smaller than the corresponding income effect. A sufficient condition for that is that 
housing and local consumption are complements.%
\footnote{We can formalize this discussion with a simple MW  worker choice problem.
Say a representative MW $(i,z)$ worker chooses between housing demand $h_{iz}$,
non-tradable consumption $c^{\text{NT}}_{iz}$, and tradable consumption $c^{\text{T}}_{iz}$,
by maximizing
$$u_{iz} = u \left(h_{iz}, c^{\text{NT}}_{iz}, c^{\text{T}}_{iz}\right)$$
subject to 
$r_i h_{iz} + p_i(\MW_i) c^{\text{NT}}_{iz} + c^{\text{T}}_{iz} \leq y_{iz}(\MW_z),$
where $p_i(\MW_i)$ gives the price of local consumption, which is increasing in residence MW;
the price of tradable consumption is normalized to one; and 
$y_{iz}(\MW_z)$ is an income function that depends positively on the workplace MW.
Let $h_{iz}^*$ and $c_{iz}^*$ denote Marshallian demands, and 
$\tilde h_{iz}^*$ denote the Hicksian housing demand.
The slutsky equation implies that
$$\frac{\partial h_{iz}^*}{\partial p_i} 
   = \frac{\partial \tilde h_{iz}^*}{\partial p_i} 
   - \frac{\partial h_{iz}^*}{\partial y_{iz}} c_{iz}^*.$$
We have that $\frac{\partial h_{iz}^*}{\partial p_i} < 0$ if and only if 
$\frac{\partial \tilde h_{iz}^*}{\partial p_i} 
< \frac{\partial h_{iz}^*}{\partial y_{iz}} c_{iz}^*$.} %%%% DGP: Very nice!!!!
Another possibility is to introduce firms that produce non-tradable local goods, and that 
use MW workers as an input. Under perfect competition, after a MW increase, the firms will 
charge a higher price to hit the zero profit condition and not go out of business. Now the 
residents that don't work in that ZIP code will pay a higher price for their local good 
and they will have less disposable income for housing.
%%% DGP: I changed the alternative explanation because I thought it is simpler to connect with basic
%%% general equilibrium micro theory. However, the explanation about firm owners living close to firms 
%%% that they own also makes sense in a world with firms having some market power. Should we also write it?

We think that the interpretation underlying point (iii) is plausible for several 
reasons.
First, recent evidence by \textcite{MiyauchiEtAl2021} shows that individuals tend 
to consume close to home.
As a result, we expect them to be sensible to prices of local consumption in their 
same neighborhood.
Second, MWs have been shown to increase prices of local consumption 
\parencite[e.g.,][]{AllegrettoReich2018, LeungForthcoming}.
%%% DGP: Research idea. It is an empirical fact that prices are higher in places where rents are high, 
%%% can we figure out whether hosuing and local good are are complements or not? Can we also use MW changes for identification?
These empirical facts suggest that residence MW changes might (conditional on workplace)
negatively affect incomes and thus demand for housing.

\paragraph{Housing supply}

We assume a simple supply side. Denote by $D_i(r_i)$ the supply of square feet in 
$i$, which is increasing in $r_i$.
Note that this formulation allows for an upper limit on the number of houses at 
which point the supply becomes perfectly inelastic.

\subsection{Equilibrium and Comparative Statics}

Total demand of housing in ZIP code $i$ is given by the sum of the demands of each group. 
Thus, we can write the equilibrium condition in this market as
\begin{equation}\label{eq:equilibrium}
	\sum_{z\in\Z} L_{iz} h_{iz} (r_i, \MW_i, \MW_z) = D_i(r_i) .
\end{equation}
Given that housing demand functions are continuous and decreasing in rents, 
under a suitable regularity condition there is a unique equilibrium in this market.%
\footnote{Assume $D_i(0) - \sum_{z\in\Z} L_{iz} h_{iz} (0, \MW_i, \MW_z) < 0$
and apply the intermediate value theorem.}
We denote equilibrium rents as $r^*_i = f(\{\MW_i\}_{i\in\Z})$.

%% SH: Formal proposition for Equilibrium below
%%     I think it's not necessary since this is kind of trivial to show
% \begin{prop}[Equilibrium]\label{prop:equilibrium}
%     Assume that $h_{iz}(\cdot)$ is continuous and decreasing in $r_i$, $D_i(\cdot)$ 
%     is     continuous and increasing in $r_i$, and $D_i(0) - \sum_{z\in\Z} L_{iz} 
%     h_{iz} (0, \MW_i, \MW_z) < 0$. Then, a unique equilibrium level of rents exists 
%     as a function of MWs:
%     $$r_i^* =  f\left(\{\MW_i\}_{i\in\Z}\right)$$
% \end{prop}
% \begin{proof}
%     From the equilibrium condition define $g(r_i) = D_i(r_i) - \sum_{z\in\Z} L_{iz} 
%     h_{iz} (r_i, \MW_i, \MW_z)$. Per the intermediate value theorem, there exists 
%     a value such that $g(r_i^*) = 0$. Furthermore, by monotonicity of $g(\cdot)$ 
%     such value is unique.
% \end{proof}

Note that equilibrium rents are a function of the entire vector of minimum wages. 
We are interested in two questions. What is the effect of a change in the vector of 
MWs $(\{d \ln \MW_i\}_{i\in\Z})'$ on equilibrium rents? Under what conditions can we 
represent the effects of MW changes on equilibrium rents in a simple way?

\begin{prop}[Comparative Statics]\label{prop:comparative_statics}
    Under the assumptions of (i) exogenous distribution of workers across workplace 
    and residence, (ii) housing demand equation satisfying Assumption 
    \ref{assu:housing_function}, and (iii) continuously differentiable and increasing
    housing supply, we have that workplace-MW hikes increase rents, and residence-MW 
    hikes, holding constant workplace-MW hikes, decrease rents.
\end{prop}

\begin{proof}
    Fully differentiate the market clearing condition with respect to $\ln r_i$ and 
    $\ln \MW_i$ for all $i\in\Z$ and re-arrange terms to get
    \begin{equation}\label{eq:diff_equilibrium}
        \Big(\eta_i - \sum_z \pi_{iz} \xi_{iz} \Big) d \ln r_i
        = 
        \sum_z \pi_{iz} \left(\epsilon_{iz}^i d \ln \MW_i 
                            + \epsilon_{iz}^z d \ln \MW_z \right)
    \end{equation}
    where 
    $\pi_{iz} = \frac{L_{iz}}{L_i}$ represents the share of workers from $i$ working 
    in	$z$;
    $\xi_{iz} = \frac{d h_{iz}}{d r_i} \frac{r_i}{\sum_i \pi_{iz} h_{iz}}$ is the 
    elasticity of housing demand at the average per-capita demand of ZIP code $i$;
    $\epsilon_{iz}^i = \frac{d h_{iz}}{d \MW_i} \frac{\MW_i}{\sum_i \pi_{iz} h_{iz}}$ and 
    $\epsilon_{iz}^z = \frac{d h_{iz}}{d \MW_z} \frac{\MW_z}{\sum_i \pi_{iz} h_{iz}}$ 
    are the elasticities of housing demand to workplace and residence MWs also at 
    the average per-capita demand of ZIP code $i$; and
    $\eta_i = \frac{1}{L_i} \frac{d D_i}{d r_i} \frac{r_i}{D_i}$ is the elasticity 
    of housing supply in ZIP code $i$.
    %%% DGP: I think that in $\xi_{iz}$, $\epsilon_{iz}^i$, $\epsilon_{iz}^z$ the sum in the denominator of 
    %%% the fractions should be across $z$'s, so that the average is across all possible resident types within $i$.
    %%% When you say per-capita you mean per $i$'s resident? 
    %%% I don't understand why $\xi_{iz}$, $\epsilon_{iz}^i$, $\epsilon_{iz}^z$ are defined per capita?
    %%% $\eta_i$ is the per $i$'s resident supply elasticity? 
    

    Because $\xi_{iz} < 0$, $\epsilon_{iz}^i < 0$, and $\epsilon_{iz}^z > 0$ for 
    all $z\in Z$, it is apparent from \eqref{eq:diff_equilibrium} that an increase 
    in workplace MW unambiguously increases rents, whereas the effect of an increase in 
    residence MW on rents is ambiguous (as long as some residents of $i$ also work in $i$) 
    as it is composed of a direct negative effect and an indirect positive effect 
    through changing the experienced MW.%
    \footnote{The sign of the overall partial effect depends on the sign of 
    $\sum_z \pi_{iz} \epsilon_{iz}^i + \pi_{ii} \epsilon_{ii}^z$.} 
\end{proof}

Proposition \ref{prop:comparative_statics} shows that, under conditions on the 
direction of the effect of MW changes and regularity conditions on the demand 
function, we can unequivocally establish the influence of the MW on rents. 
Interestingly, increases in MW changes in a set of ZIP codes other than $i$
will affect rents at $i$ if some of $i$'s residents work in some of those ZIP codes.

\begin{prop}[Representation]\label{prop:representation}
    Under the assumption of constant elasticity of housing demand (across workplace locations)
    to workplace minimum wages, % 
      \footnote{There are many reasons to believe that this assumption is an over simplification. 
      For example, in a model where agents have heterogeneous costs for commuting, it would be 
      reasonable that the elasticity of housing demand to workplace MW is higher in ZIP codes that 
      are close to $i$ relative to ZIP codes that are far. However, if wages are a close to a sufficient 
      statistic of the utility that they derive out of working in a location then this assumption is plausible.}
      %%%% DGP: Not sure at all that the example is right, but just out an example reasoning to make the 
      %%%% point that we should discuss some example or some reason why this is not an impossible assumption. 
    we can write the change in log rents as a function of the change in two 
    MW-based measures: the \textbf{experienced log MW} and the \textbf{statutory 
    log MW}.
\end{prop}
\begin{proof}
    Under the assumption that $\epsilon_{iz}^z = \epsilon_i^z$ for all $z\in\Z$ we can 
    manipulate \eqref{eq:diff_equilibrium} to write
    \begin{equation} \label{eq:theory_represenation}
        d \ln r_i = \beta_i \sum_i \pi_{iz} d\ln \MW_z + \gamma_i d \ln \MW_i
    \end{equation}
    where $\beta_i = \frac{\epsilon_{i}^z}{\eta_{i} - \sum_z \pi_{iz} \xi_{iz}} 
    >0$ and $\gamma_i = \frac{\sum_z \pi_{iz} \epsilon_{iz}^i}{\eta_{i} 
            - \sum_z \pi_{iz} \xi_{iz}} < 0$.
\end{proof}

Proposition \ref{prop:representation} shows that, as an approximation to small changes 
in the profile of MWs, we can write the change in rents in the ZIP code as a function 
of two MW-based measures.
This motivates our empirical strategy, where we further impose that $\beta_i = \beta$ and 
$\gamma_i=\gamma$ for all $i\in\Z$.%
\footnote{This constant effects assumption is sufficient, although not necessary, 
to map our empirical model to \eqref{eq:theory_represenation}. 
A weaker assumption is that, in the context of the empirical model, the heterogeneity 
across residence locations in these parameters is uncorrelated to rents.}
%%%% DGP: Not 100% sure that I understand this footnote. Let's discuss it!
