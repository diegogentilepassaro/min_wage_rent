
In this section we lay out a simple demand and supply model of local rental markets.
We use the model to illustrate why we expect a different impact of MW changes 
on rents at workplace and residence locations.
Our model is static and assumes a fixed distribution of workers across 
residence and workplace locations.
The addition of a time dimension is discussed in Online Appendix 
\ref{sec:dyn_theory_model}.

\subsection{Setup}

We consider the rental market of some ZIP code $i$ embedded in a larger geography 
composed of a finite number of ZIP codes $\Z$.
Workers with residence $i$ work in a ZIP code $z\in\Z(i)$, where 
$\Z(i)\subseteq\Z$.
More precisely, we let $L_{iz}$ denote the number of $i$'s residents who work 
in $z$; and 
$L_i = \sum_{z \in \Z(i)} L_{iz}$ and $L_z = \sum_{i \in \Z(i)} L_{iz}$ the 
number of residents in $i$ and workers in $z$, respectively.%
\footnote{To simplify, we assume that all of $i$' residents work, so that the 
number of residents equals the number of workers.}
We assume that the vector $\{L_{iz}\}_{z\in\Z(i)}$ is fixed.
This assumption is intended as an approximation to our empirical setting where 
we look at the effects of MW changes at a monthly frequency.%
\footnote{\textcite{AllenEtAl2020} study the within-city transmission of 
expenditure shocks by tourists within Barcelona over a period of two years.
The authors maintain an analogous assumption of constant shares of income that
each location in the city earns from every other location.}
This assumption is consistent with estimates of small effects of the MW on 
employment, as in \textcite{CegnizEtAl2019, DustmannEtAl2022}, and 
on migration, as in \textcite{PerezPerez2021}, 
in a time frame of several years.

Each ZIP code has a binding minimum wage.
The set of binding MWs relevant for $i$ is $\{\MW_z\}_{z\in\Z(i)}$.

\subsubsection*{Housing demand}

Each group $(i,z)$ consumes
square feet of living space $H_{iz}$, 
a non-tradable good produced in their residence $C_{iz}^{NT}$, and
a tradable good $C_{iz}^T$.
A representative $(i,z)$ worker chooses between these alternatives by maximizing
a quasi-concave utility function 
$u_{iz} = u \left(H_{iz}, C^{\text{NT}}_{iz}, C^{\text{T}}_{iz}\right)$
subject to a budget constraint
$$R_i H_{iz} + P_i(\MW_i) C^{\text{NT}}_{iz} + C^{\text{T}}_{iz} \leq Y_{iz}(\MW_z),$$
where
$R_i$ gives the rental price of housing per square feet,
$P_i(\MW_i)$ gives the price of local consumption,
the price of tradable consumption is normalized to one, and 
$Y_{iz}(\MW_z)$ is an income function.
We summarize the effect of MW levels on these functions below.

\begin{assu}[Effect of Minimum Wages]\label{assu:mws}
    We assume that
    (i) the price of non-tradable goods is increasing in $i$'s MW, 
    $\frac{d P_i}{d \MW_i} > 0$, and
    (ii) incomes are weakly increasing in $z$'s MW, 
    $\frac{d Y_{iz}}{d \MW_z} \geq 0$, with strict inequality 
    for at least one $z\in\Z(i)$.
\end{assu}

The structure of the problem and Assumption \ref{assu:mws} are in line with 
the literature.
First, evidence by \textcite{MiyauchiEtAl2021} shows that individuals tend to 
consume close to home.
As a result, we expect them to be sensitive to prices of local consumption in 
their same neighborhood, justifying the assumption that workers consume 
non-tradables in the same ZIP code.%
\footnote{An extension of the model would allow workers to consume in any ZIP 
code.
While theoretically straightforward, this extension would require data on 
consumption trips, which we lack.
We think of our model as an approximation.}
Second, MW hikes have been shown to increase prices of local consumption 
\parencite[e.g.,][]{AllegrettoReich2018, Leung2021},
and also to increase wage income even for wages above the MW level 
\parencite[e.g.,][]{CegnizEtAl2019,Dube2019Income}.%
\footnote{An extension would allow separate wage income and business income in 
the budget constraint.
If firm owners tend to live where they work, and MW increases damage profits
\parencite[as found by, e.g.,][]{DracaMachinVanreenen2011, HarasztosiLidner2019},
then business income would depend negatively on the MW level.}
%% SH: It would be cool to cite some paper on the relationship between consumption
%%     prices and rents.

For convenience, we define the per-capita housing demand function as 
$h_{iz} \equiv \frac{H_{iz}}{L_{iz}}$.
The solution to the worker's problem for each $z$ then yields a set of 
continuously differentiable per-capita housing demand functions 
$\{h_{iz} (R_i, P_i, Y_z)\}_{z\in\Z(i)}$.
We summarize the properties of these functions below.

\begin{assu}[Housing demand]\label{assu:housing_demand}
    Consider the set of functions $\{h_{iz} (R_i, P_i, Y_z)\}_{z\in\Z(i)}$.
    We assume that
    (i) housing is a normal good, 
    $\frac{d h_{iz}}{d Y_z} > 0$ for all $z\in\Z(i)$,
    and
    (ii) housing demand is decreasing in prices of non-tradable consumption, 
    $\frac{d h_{iz}}{d P_i} < 0$.
\end{assu}

Using the first assumption, standard arguments imply that 
$\frac{d h_{iz}}{d R_i} < 0$.
%% 
%% Specifically, normal goods rule out the possibility of rents being a Giffen good
%%
For the second assumption to hold, 
a sufficient (albeit not necessary) condition is that housing and non-tradable
consumption are complements.%
\footnote{To formalize the required condition, let $h_{iz}$ and $c_{iz}$ denote 
per-capita Marshallian demands resulting from the choice problem, and 
$\tilde h_{iz}$ denote the corresponding Hicksian housing demand.
The Slutsky equation implies that
$$\frac{\partial h_{iz}}{\partial P_i}
   = \frac{\partial \tilde h_{iz}}{\partial P_i}
   - \frac{\partial h_{iz}}{\partial Y_{iz}} c_{iz}.$$
To obtain $\frac{\partial h_{iz}}{\partial P_i} < 0$, we require that
$\frac{\partial \tilde h_{iz}}{\partial P_i}
< \frac{\partial h_{iz}}{\partial Y_{iz}} c_{iz}$, i.e., the income effect of an
increase in non-tradable prices is larger than the corresponding substitution
effect.}
While direct empirical evidence on this particular channel is lacking,
we view the evidence of workers sorting towards locations with high housing 
costs and high-quality and more expensive amenities as consistent with it 
\parencite[e.g.,][]{CoutureEtAl2019}.

Note that, given our assumptions,
an increase in a group $(i,z)$'s workplace MW will tend to increase housing 
demand in $i$,
and an increase in residence MW will have a negative effect---conditional on 
its effect via the workplace MW of the group $(i,i)$.

\subsubsection*{Housing supply}

We assume that absentee landlords supply square feet in $i$ according to the 
function $S_i(R_i)$,
and we assume that this function is weakly increasing in $R_i$, 
$\frac{d S_i(R_i)}{d R_i} \ge 0$.
Note that this formulation allows for an upper limit on the number of houses at 
which point the supply becomes perfectly inelastic.

\subsection{Equilibrium and Comparative Statics}

Total demand of housing in ZIP code $i$ is given by the sum of the demands of 
each group.
Thus, we can write the equilibrium condition in this market as
\begin{equation}\label{eq:equilibrium}
	\sum_{z\in\Z(i)} L_{iz} h_{iz} \left(R_i, P_i(\MW_i), Y_z(\MW_z)\right) = S_i(R_i) .
\end{equation}
Given that the per-capita housing demand functions are continuous and 
decreasing in rents,
under a suitable regularity condition there is a unique equilibrium in this 
market.%
\footnote{To see this, assume that 
$S_i(0) - \sum_{z\in\Z(i)} L_{iz} h_{iz} (0, P_i, Y_z) < 0$
and apply the intermediate value theorem.
Intuitively, we require that at low rental prices demand exceeds supply.}
Equilibrium rents are a function of the entire set of minimum wages, formally, 
$R^*_i = f(\{\MW_i\}_{i\in\Z(i)})$.

We are interested in two questions.
First, what is the effect of a change in the vector of MWs 
$(\{d \ln \MW_i\}_{i\in\Z(i)})'$ on equilibrium rents?
Second, under what conditions can we reduce the dimensionality of the rents 
function and represent the effects of MW changes on equilibrium rents in a 
simpler way?
We start with the first question.

\begin{prop}[Comparative Statics]\label{prop:comparative_statics}
    Consider residence ZIP code $i$ and a change in MW policy at a larger
    jurisdiction such that for $z\in\Z_0 \subset \Z(i)$ binding MWs increase 
    and for $z\in\Z(i)\setminus \Z_0$ binding MWs do not change,
    where $\Z_0$ is non-empty.
    Under the assumptions of unchanging $\{L_{iz}\}_{z\in\Z(i)}$ 
    and Assumptions \ref{assu:mws} and \ref{assu:housing_demand},
    we have that
    \begin{enumerate}
        \item[(a)]
        for some $z'\in\Z_0\setminus\{i\}$ for which $\frac{d Y_{z'}}{d \MW_{z'}}>0$, 
        the policy has a positive partial effect on rents, 
        $\frac{d\ln R_i}{d\ln\MW_{z'}} > 0$;
        \item[(b)]
        the partial effect of the MW increase in $i$ on rents is ambiguous, 
        $\frac{d\ln R_i}{d\ln\MW_i} \lessgtr 0$; and
        \item[(c)]
        as a result, the overall effect on rents is ambiguous if $i\in\Z_0$ 
        and positive if $i\notin\Z_0$.
    \end{enumerate}
\end{prop}

\begin{proof}
    Fully differentiate the market clearing condition with respect to $\ln R_i$ 
    and $\ln \MW_z$ for all $z\in\Z(i)$.
    Using \eqref{eq:equilibrium} and appropriate algebraic manipulations, 
    one can show that
    \begin{equation}\label{eq:diff_equilibrium}
        \Big(\eta_i - \sum_z \pi_{iz} \xi^R_{iz} \Big) d \ln R_i
        = 
        \sum_z \pi_{iz} \left(\xi^P_{iz} \epsilon_{i}^P d \ln \MW_i 
                            + \xi^Y_{iz} \epsilon_{iz}^Y d \ln \MW_z \right) ,
    \end{equation}
    where
    $\pi_{iz} = \frac{L_{iz}}{L_i}$ represents the share of $i$'s residents 
    working in $z$,
    $\xi_{iz}^x = \frac{d h_{iz}}{d x_i} \frac{x_i}{\sum_z \pi_{iz} h_{iz}}$ for
    $x\in\{R,P\}$ is the elasticity of the per-capita housing demand with respect
    to $x$ evaluated at the average per-capita demand of ZIP code $i$,
    $\xi_{iz}^Y = \frac{d h_{iz}}{d Y_z} \frac{Y_z}{\sum_z \pi_{iz} h_{iz}}$ 
    represents the analogous elasticity with respect to income $Y$ from each 
    workplace $z$,
    $\epsilon_{i}^P = \frac{d P_i}{d \MW_i} \frac{\MW_i}{P_i}$ and 
    $\epsilon_{iz}^Y = \frac{d Y_z}{d \MW_z} \frac{\MW_z}{Y_z}$ are
    elasticities of prices and income to the MW, and
    $\eta_i = \frac{d S_i}{d R_i} \frac{R_i}{S_i}$ is the elasticity 
    of housing supply in ZIP code $i$.

    For any $z'\in\Z_0\setminus\{i\}$ the partial effect on rents of the policy
    is given by
    $$
    \frac{d\ln R_i}{d\ln\MW_{z'}} 
      = \left(\eta_i - \sum_z \pi_{iz} \xi^R_{iz}\right)^{-1} 
              \pi_{iz'}\xi^Y_{iz}\epsilon_{iz'}^Y.
    $$
    Because $\eta_i\geq0$ and $\xi^R_{iz} < 0$ for all $z\in\Z(i)$, 
    the first factor is positive.
    From Assumptions \ref{assu:mws} and \ref{assu:housing_demand},
    $\epsilon_{iz}^Y\geq0$ and $\xi^Y_{iz}>0$.
    Therefore, the effect is positive if for $z'$ we have 
    $\frac{d Y_{z'}}{d \MW_{z'}}>0$ (or $\epsilon_{iz'}^Y>0$), 
    and the effect is zero otherwise.

    For ZIP code $i$ the partial effect is given by
    $$
    \frac{d\ln R_i}{d\ln\MW_{i}} 
      = \left(\eta_i - \sum_z \pi_{iz} \xi^R_{iz}\right)^{-1} 
        \left(\epsilon_{i}^P \sum_z \pi_{iz}\xi^P_{iz} 
             + \pi_{ii}\xi^Y_{ii}\epsilon_{ii}^Y \right) .
    $$
    By Assumption \ref{assu:mws} we have that $\epsilon_{i}^P>0$ and that 
    $\epsilon_{ii}^Y\geq0$.
    By Assumption \ref{assu:housing_demand} we have that $\xi^Y_{ii}>0$ and that, 
    for all $z\in\Z(i)$, $\xi^P_{iz}<0$.
    Then, the second parenthesis has an ambiguous sign.
    The third statement of the Proposition follows directly.
\end{proof}

The first part of Proposition \ref{prop:comparative_statics} shows that,
if at least some low-wage worker commutes to a ZIP code $z'$ where the MW 
increased  (so that $\frac{d Y_{z'}}{d \MW_{z'}}>0$),
then the MW hike will tend to increase rents.
The second part of Proposition \ref{prop:comparative_statics} establishes that 
a decreasing effect on rents may follow if the minimum wage also increases in 
ZIP code $i$.
As a result, the sign of the overall effect of the policy in $i$ is not 
determined a priori.

As apparent from the proof of Proposition \ref{prop:comparative_statics}, 
the effect of the MW on rents at workplaces depends on the elasticities of 
per-capita housing demand to incomes
$\xi^Y_{iz} = \frac{d h_{iz}}{d Y_z} \frac{Y_z}{\sum_z \pi_{iz} h_{iz}}$ and
on the elasticities of income to minimum wages
$\epsilon_{iz}^Y = \frac{d Y_z}{d \MW_z} \frac{\MW_z}{Y_z}$.
These $(i,z)$-specific terms weigh the change in MW levels at workplace locations,
and their sum over $z$ impacts the change in rents.
The following proposition establishes conditions under which we can reduce the 
dimensionality of the rent gradient to two MW-based measures.

\begin{prop}[Representation]\label{prop:representation}
    Assume that for all ZIP codes $z\in\Z(i)$ we have
    (a) homogeneous elasticity of per-capita housing demand to incomes,
    $\xi^Y_{iz}=\xi^Y_{i}$, and
    (b) homogeneous elasticity of income to minimum wages,
    $\epsilon_{iz}^Y=\epsilon_i^Y$.
    Then, we can write
    $$
    d r_{i} = \beta_i d \mw_{i}^{\wkp} + \gamma_i d \mw^{\text{res}}_i
    $$
    where 
    $r_{i} = \ln R_i$,
    $\mw_{i}^{\wkp} = \sum_{z\in\Z(i)} \pi_{iz} \ln \MW_z$ 
    is ZIP code $i$'s \textbf{workplace MW}, 
    $\mw^{\res}_i = \ln \MW_i$ 
    is ZIP code $i$'s \textbf{residence MW}, and 
    $\beta_i > 0$ and $\gamma_i < 0$ are parameters.
\end{prop}
\begin{proof}
    Under the stated assumptions we can manipulate \eqref{eq:diff_equilibrium} 
    to write
    \begin{equation} \label{eq:theory_representation}
        d r_i = \beta_i  d \mw^{\wkp}_i
              + \gamma_i d \mw^{\res}_i
    \end{equation}
    where
    $\beta_i = \frac{\xi_{i}^{Y}\epsilon_i^{Y}}
                     {\eta_{i} - \sum_z \pi_{iz} \xi_{iz}^R} 
              > 0$ and
    $\gamma_i = \frac{\sum_{z\in\Z(i)}\pi_{iz}\xi_{iz}^{P}\epsilon_{i}^{P}}
                    {\eta_{i} - \sum_z \pi_{iz} \xi_{iz}^R} 
              < 0$
    are parameters, which signs can be verified using
    Assumptions \ref{assu:mws} and \ref{assu:housing_demand}.
\end{proof}

Proposition \ref{prop:representation} shows that, under a homogeneity assumption
on the elasticities of per-capita housing demand to income and 
of income to the MW,%
\footnote{The assumptions stated in Proposition \ref{prop:representation} are 
actually stronger than needed.
It is enough to have that the product $\xi^Y_{iz} \epsilon_{iz}^Y$ does not vary 
by $z$.}
the change in rents following a small change in the profile of MWs can be 
expressed as a function of two MW-based measures:
one summarizing the effect of MW changes in workplaces $z\in\Z(i)$,
and another one summarizing the effect of the MW in the same ZIP code $i$.
This motivates our empirical strategy, where we regress log rents on the 
empirical counterparts of these measures.

How likely are these assumptions to hold?
The assumption that the elasticity of income to the MW is constant will fail if 
the income of some $(i,z)$ groups is more sensitive to the MW than others.
This would be the case if, for example, the share of low-wage workers within 
each flow $L_{iz}$ varies strongly by workplace $z$.
% \footnote{For instance, say that 
% $$ Y_z = \underline{L}_{iz} \MW_z + \left(L_{iz} - \underline{L}_{iz}\right) \overline{W}_z $$
% where $\underline{L}_{iz}$ is the number of MW workers in the flow $L_{iz}$, 
% and $\overline{W}_z$ is the average wage earned by non-MW workers.
% In this case we have
% $\epsilon^Y_z = (\MW_z\underline{L}_{iz}/Y_z) ,$
% so if $\underline{L}_{iz}$ is constant the assumption will hold.}
The assumption that the elasticity of housing demand to income is constant 
will hold trivially for all preferences with
$h_{iz} = g\left(R_i, P_i\right) Y_i$ for some $g\left(\cdot\right)$, such as 
those embedded in Cobb-Douglas or Constant Elasticity of Substitution utility 
functions.
However, one would expect the elasticity of $(i,z)$ groups with many low-wage 
workers to be larger, suggesting that this type of preferences may not be 
appropriate.

We thus see that Proposition \ref{prop:representation} requires some strong 
homogeneity assumptions that will likely not hold in practice.
However, we expect our empirical model based on Proposition 
\ref{prop:representation} to offer a decent approximation to study the spillover
effects of MW policies on the housing market.
In fact, unless the heterogeneity in $\{\xi_{iz}^Y\epsilon_{iz}^Y\}_{z\in\Z(i)}$ 
has a strongly asymmetric distribution across workplace locations, we expect to 
correctly capture the average contribution of the workplace MW on rents.
In other words, the value of $\beta_i  d \mw^{\wkp}_i$ will be close to value of 
the elasticity-weighted changes in workplace MW levels that, according to the 
model, determine rents.%
\footnote{More precisely, say that 
$\xi^Y_{iz}\epsilon_{iz}^Y = \overline{\xi\epsilon}_i + \nu_{iz}$ where 
$\nu_{iz}$ has a mean of zero.
In that case, a similar logic than the one in the proof of 
Proposition \ref{prop:representation} will result in  the following expression 
for rents changes:
$$
    d r_i = \gamma_i d \mw^{\res}_i
          + \frac{\overline{\xi\epsilon}_i}
                 {\eta_{i} - \sum_z \pi_{iz} \xi_{iz}^R} \sum_z d\ln \MW_z
          + \frac{1}
                 {\eta_{i} - \sum_z \pi_{iz} \xi_{iz}^R} \sum_z \nu_{iz} d\ln \MW_z .
$$
The second term on the right-hand side is equivalent to $\beta_i \mw_{i}^{\wkp}$
in Proposition \ref{prop:representation}.
The third term reflects the heterogeneity.
If $\nu_{iz}$ has a symmetric distribution, and $d\ln \MW_z$ is the same across 
workplaces (because it originates from a single jurisdiction), then this third 
term will equal zero.}
%%
%% SH: Need to document in claims note
%%
Moreover, in our empirical exercises we allow for heterogeneity in elasticities
based on observable characteristics of workers, 
such as the share of MW workers residing in each location.
This exercise allows ZIP codes with more MW workers to have potentially larger 
(although still constant across workplace locations) elasticites.%
\footnote{This exercise can be mapped to the model by assuming that 
$\xi^Y_{iz} \epsilon_{iz}^Y$ is a linear function of the share of MW workers
in $i$.}
Reassuringly, we find evidence that locations with more MW workers are indeed 
more affected by the MW measures.
