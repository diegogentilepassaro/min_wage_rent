%%%%%%%%%%%%%%%%%%%%%%%%%%%%%%%%%%%%%%%%%%%%%%%%%%%%%%%%%%%%%%%%%%%%%%%%%%%%%%%%%
%%%%%                             DATA SAMPLE                                %%%%
%%%%%%%%%%%%%%%%%%%%%%%%%%%%%%%%%%%%%%%%%%%%%%%%%%%%%%%%%%%%%%%%%%%%%%%%%%%%%%%%%

We put together a panel dataset at the US postal service zipcode and monthly date level from 
January 2010 to December 2019 which contains data on minimum wages, rents and several other
variables we use in the analysis. More precisely, our data comes from five distinct sources.

First of all, our data contains MW changes at the federal, state, county, and city level.\footnote{
	Note that federal level MW changes still could induce meaningful variation as it is binding in 
	some zipcodes and not in others, so that identification does not come only from time series 
	variation. However, the last federal MW increase was in 2009 so changes used in our estimates 
	come from state, county, and city level.} 
The main source of data for these changes is \textcite{VaghulZipperer2016} up to mid-2016. We 
extend these data for the years 2017, 2018, and 2019 mainly from \textcite{BerkeleyLaborCenter}. 
For each zipcode we assume that the prevailing MW at a given month is the maximum between the 
required by the federal, state, county, and city levels.\footnote{We abstract from the fact that 
	occasionally minimum wages vary by industry.} %%% SH: ALSO COMMENT ON MW CHANGING BY SIZE
We only use MW changes that are binding, meaning that they actually impact that maximum. Our 
baseline estimating panel collects 5,301 MW changes at the zipcode-month level, which arise 
from 166 state-level changes and 229 county- and city-level changes.

Second, we use rent and house value data from properties listed in Zillow for our sample period 
\parencite{zillow}. Zillow is the leader online real estate and rental platform in the U.S., 
hosting more than 110 million homes and 170 million unique monthly users in 2019.\footnote{
	Source: \href{https://www.zillowgroup.com/facts-figures/}
	{\texttt{https://www.zillowgroup.com/facts-figures/}}. Accessed on October 23rd, 2020.}
Zillow provides the median rental and listing price (both total and per square foot) among homes 
listed on the platform in a given period. Time series are provided for different 
house types and at different geographic and date aggregation level.\footnote{
	\href{https://www.zillow.com/research/data/}{https://www.zillow.com/research/data/} 
	provides more information on the data shared by Zillow. The availability of different time 
	series changed over time, so not all series used for the analysis might be still available 
	to download.}
Given that we are interested in the behavior of the housing market in the short period following a 
MW change, we focus on USPS zipcode level monthly time series. Clearly, even within a single 
zipcode, there could be great heterogeneity in terms of house sizes and types, making it more 
difficult to assess the impact of local intervention. In an effort to minimize price variation 
coming from houses' characteristics, such as the number of bedrooms, we focus our main analysis 
on the single family, condominium and cooperative homes (SFCC) series. This is by far the series 
with the largest number of non-missing zipcodes, as it covers the most common U.S. rental house 
types. In fact, roughly a third of the nation's 47.2 million rental units in 2018 fit the category 
single-family homes, with the other 43 percent made up from buildings with 5 or more units 
\parencite{JCHS2020}. Finally, we select for our analysis \textit{per square foot} variables: 
this allows us to reduce confounding variation based on supply-side factors such as land 
availability. A limitation in the use of Zillow data comes from the fact that we cannot observe the 
underlying number of houses listed for rent in a given month. Changes in the Zillow inventory 
therefore introduce additional variation in the reported median rental price.\footnote{We do
	observe the number of houses listed for sale, which we use as a proxy for this variable.}

Third, we add socio-demographic information to each zipcode in our sample using the 2010 Census 
and the 5-years 2008-2012 ACS. The data is originally obtained at the Census tract level and mapped 
into USPS zipcodes using HUD crosswalks (\citeyear{hud}).
%Crosswalks are obtained from \url{https://www.huduser.gov/portal/datasets/usps_crosswalk.html}} 
We assign to each zipcode the following characteristics: population, number of housing units, 
median income, black population, number of unemployed, and number of college students. We use 
this information to classify zipcodes into, for example, high or low median income to then perform 
heterogeneity analysis. In addition, given that zipcodes can cross county borders, we use the 
census data and geographic codes to map each zipcode to a county by assigning it to the one 
with the highest share of houses from that zipcode. We also map each zipcode to a metropolitan 
statistical area or a rural town analogously. We use this information to assign the prevailing 
MW to each zipcode.

\autoref{tab:desc_stats} compares descriptive statistics for our data and for representative 
US aggregates from the 2010 Census and the 5 years 2008 ACS. Columns 1 and 2 report data for the 
whole universe of US zipcodes and for the top 100 US metropolitan areas, respectively. In column 3 
we show the complete set of zipcodes in the Zillow data. Finally, column 4 restricts the sample by 
keeping the set of zipcodes that have valid SFCC rental data as of July 2015. We call this
our \textit{baseline sample}, since it is the one we use for most of our analysis. Focusing on our 
preferred series, Zillow provides information on rents for 4,604 unique zipcodes accounting for 
11.8 percent of the US zipcodes and 46.7 percent of the 2015 US population. The average median 
household annual income for those zipcodes is \$64,289, 22.5 percent higher than the same figure 
for the average US zipcode, but it is slightly lower than the figure for the average zipcode 
in the top 100 metropolitan areas. Zipcodes in the baseline sample are more populous and slightly 
higher income than the average US zipcode. Zillow is a real estate company and as such it is 
present in more dynamic rental markets. Those markets have a higher share of urban population, 
a higher share of college students, and a higher share of house for rent that the average US 
zipcode. In an attempt to capture the average treatment effect we conduct an estimation 
re-weighting our sample to match characteristics of the US sample of zipcodes.
%% SH: Review description of table

\begin{table}[h!]
	\caption{Descriptive statistics of different sets of zipcodes}
	\centering
	\label{tab:desc_stats}    
	
% Table created by stargazer v.5.2.2 by Marek Hlavac, Harvard University. E-mail: hlavac at fas.harvard.edu
% Date and time: Sun, Nov 08, 2020 - 3:45:47 PM
\begin{tabular}{@{\extracolsep{5pt}} ccccc} 
\\[-1.8ex]\hline 
\hline \\[-1.8ex] 
 & U.S. & Top 100 CBSA & Full Panel & Est. Panel \\ 
\hline \\[-1.8ex] 
Population (millions) (2010) & $311.18$ & $189.71$ & $110.17$ & $50.62$ \\ 
Population as share of U.S. & $1$ & $0.61$ & $0.35$ & $0.16$ \\ 
Housing Units (millions) (2010) & $132.83$ & $78.74$ & $46.72$ & $21.32$ \\ 
Housing Units as share of U.S. & $1$ & $0.59$ & $0.35$ & $0.16$ \\ 
Urban Share (2010) & $0.46$ & $0.75$ & $0.96$ & $0.97$ \\ 
College Share (2010) & $0.46$ & $0.75$ & $0.96$ & $0.97$ \\ 
African-American Share (2010) & $0.46$ & $0.75$ & $0.96$ & $0.97$ \\ 
Hispanic Share (2010) & $0.10$ & $0.14$ & $0.17$ & $0.19$ \\ 
Elder Share (2010) & $0.15$ & $0.13$ & $0.12$ & $0.11$ \\ 
Poor Share (2010) & $0.46$ & $0.75$ & $0.96$ & $0.97$ \\ 
Unemployed Share (2010) & $0.09$ & $0.09$ & $0.09$ & $0.09$ \\ 
Mean HH income (2010) & $52,492.56$ & $62,773.64$ & $65,475.16$ & $66,919.72$ \\ 
Rent House Share (2010) & $0.29$ & $0.35$ & $0.38$ & $0.38$ \\ 
Work in same county share (2010) & $0.70$ & $0.68$ & $0.76$ & $0.76$ \\ 
Unique zipcodes & $38,893$ & $14,583$ & $3,315$ & $1,305$ \\ 
Share of state events & $$ & $$ & $0.03$ & $0.03$ \\ 
Share of county events & $$ & $$ & $0.001$ & $0.001$ \\ 
Share of  localevents & $$ & $$ & $0.003$ & $0.0005$ \\ 
Mean SFCC psqft rent & $$ & $$ & $1.30$ & $1.27$ \\ 
Unique zipcodes SFCC psqft rent & $$ & $$ & $3,316$ & $1,143$ \\ 
\hline \\[-1.8ex] 
\end{tabular} 

	\begin{minipage}{0.95\textwidth} \footnotesize
		\vspace{3mm} 
		\textit{Notes}: The table shows characteristics of four sets of US postal service zipcodes.
		Column 1 reports demographic statistics for the universe of USPS zipcode we were able to 
		map. Column 2 shows the same statistics for for the top 100 Core-Based Statistical Areas 
		(CBSA). Column 3 shows the characteristics of the set of zipcodes available in the Zillow 
		data. Finally, column 4 shows the restricted balanced sample we use as baseline in our 
		empirical analysis. All demographic information comes from the 2010 Census and the 5-years 
		2008-2012 ACS.
	\end{minipage}
\end{table}

To ensure that our data correctly captures the price evolution of the US rental market, we compare 
Zillow's median rental price with 5 Small Area Fair Market Rents (SAFMRs) series for houses with 
different number of bedrooms (0, 1, 2, 3, and 4 or more) coming from the US Department of Housing 
and Urban Development (\citeyear{hud}). SAFMRs are calculated for zipcodes within metropolitan areas 
at a yearly level, and generally equal the 40th percentile of the rent distribution for that 
zipcode.\footnote{For more information on how SAFMRs are calculated, see page 41641 of the 
	\href{https://www.huduser.gov/portal/datasets/fmr/fmr2018/FY2018-FMR-Preamble.pdf}
	{Federal Register/Vol. 82, No. 169}} %% SH: Add this cite to biblio.bib
The yearly time series correlation between Zillow SFCC and all of the SAMFRs series is consistently 
above 90 percent. Single family houses, as well as condos and cooperative houses, are fairly loose 
categories and are therefore expected to vary in terms of the number of bedrooms they might have. 
For this reason, in \autoref{fig:trend_zillow_safmrwgt} we compare the Zillow SFCC series with a 
weighted combination of the different SAMFRs series.\footnote{\label{foot:zillow_time_series}
	To compute the weighted SAMFR series we proceed as follows. First, we compute the national yearly 
	average for both the Zillow SFCC and the 5 	SAFMR series. Then, for each of the latter we compute 
	the U.S. share of single family, condo, and cooperative houses with that number of bedrooms using 
	the \textit{American Housing Survey} (AHS). To ensure comparability, we only use the estimated 
	count for rental houses in this step. 	(Additionally, AHS data is available only for years 2011, 
	2013, 2015, 2017, and 2019. We therefore fill missing years with previous year's share.) Finally, 
	we weight SAFMR series using the aforementioned shares.} 
The Zillow rent data is always higher in levels. Part of this difference is intuitively related to the 
fact that Zillow reports median rent prices while SAFMRs are based on the 40th percentile of the rent 
distribution. The two series however show similar trends, confirming that Zillow rental series indeed 
captures the dynamics of the U.S. rental prices.

Fourth, to proxy for local economic activity we collect data from the Quarterly Census of Employment 
and Wages (QCEW) at the county-quarter and county-month level for each industry and level of 
government.\footnote{The QCEW covers the following industries: goods-producing; natural resources and 
	mining; construction; manufacturing; service-providing; trade, transportation and utilities; 
	information; financial activities; professional and business services; education and health 
	services; leisure and hospitality. The QCEW additionally provides employment data for federal, 
	state, and local government.} 
For each county-quarter-industry cell we observe the number of establishments and the average weekly 
wage. For each county-month-industry cell we additionally observe the number of employed people. We 
merge this data onto our zipcode-month panel based on county and quarterly date.

We add data from the \textit{Building Permit Survey} (BPS) at the county-month level to account for 
time-varying shocks in the housing market. The BPS provides building permit statistics on new 
privately-owned residential construction disaggregated by house type. Lacking information on condos 
and cooperative houses, we only add the number of new units and the permits valuation for single 
family houses to each zipcode-month observation based on the county and month they belong.

Finally, we use data from the 2017 Longitudinal Employer-Household Dynamics Origin-Destination 
Employment Statistics (LODES) to proxy for MW workers' residence and workplace location. The LODES 
data sets provide block-level information on jobs and are organized in 3 groups: residence area 
characteristics (RAC), with information about characteristics of jobs for various types of workers 
(e.g. number of jobs in different sectors, number of job for workers under 30 years old, etc.); 
workplace area characteristics (WAC) that provide the same information as RAC files but aggregated 
with respect to workplace location; and a origin-destination matrix mapping jobs from residence to 
workplace locations. We use RAC and WAC datasets to ``locate" workers likely to be MW by looking at 
the state-level distribution of such type of workers: we build, for each zipcode in the sample, the 
share (out of the state total) of workers under 30 years old earning less than $\$1251$ that either 
\textit{live} or \textit{work} there. 

As we mentioned above, our baseline estimating sample is constructed by fixing the composition of 
zipcodes to those that had valid SFCC rent data as of July 2015. \autoref{tab:estimating_panel_stats}
shows some statistics of those zipcodes. % DESCRIBE

\begin{table}[h!]
	\caption{Descriptive statistics and comparison with representative zipcodes}
	\centering
	\label{tab:estimating_panel_stats}    
	
% Table created by stargazer v.5.2.2 by Marek Hlavac, Harvard University. E-mail: hlavac at fas.harvard.edu
% Date and time: Mon, Nov 30, 2020 - 11:59:17 AM
\begin{tabular}{@{\extracolsep{5pt}}lccccc} 
\\[-1.8ex]\hline 
\hline \\[-1.8ex] 
Statistic & \multicolumn{1}{c}{N} & \multicolumn{1}{c}{Mean} & \multicolumn{1}{c}{St. Dev.} & \multicolumn{1}{c}{Min} & \multicolumn{1}{c}{Max} \\ 
\hline \\[-1.8ex] 
Statutory MW & 156,600 & 8.08 & 1.21 & 7 & 16 \\ 
Experienced MW & 156,600 & 8.06 & 1.20 & 6.29 & 14.98 \\ 
Median rent psqft. SFCC & 113,375 & 1.27 & 0.83 & 0.47 & 7.25 \\ 
Median rent SFCC & 125,644 & 1,651.10 & 702.99 & 595.00 & 6,595.00 \\ 
Avg. wage Fin. activities & 152,334 & 1,561.78 & 965.27 & 0.00 & 9,557.00 \\ 
Employment Fin. activities & 152,334 & 59,554.22 & 75,796.09 & 0.00 & 397,839.00 \\ 
Estab. count Fin. activities & 152,334 & 5,103.83 & 5,200.06 & 31.00 & 30,405.00 \\ 
\hline \\[-1.8ex] 
\end{tabular} 

	\begin{minipage}{0.95\textwidth} \footnotesize
		\vspace{3mm} 
		\textit{Notes}: The table shows characteristics of four sets of US postal service zipcodes.
		Column 1 reports demographic statistics for the universe of USPS zipcode we were able to 
		map. Column 2 shows the same statistics for for the top 100 Core-Based Statistical Areas 
		(CBSA). Column 3 shows the characteristics of the set of zipcodes available in the Zillow 
		data. Finally, column 4 shows the restricted balanced sample we use as baseline in our 
		empirical analysis. All demographic information comes from the 2010 Census and the 5-years 
		2008-2012 ACS.
	\end{minipage}
\end{table}
