%%%%%%%%%%%%%%%%%%%%%%%%%%%%%%%%%%%%%%%%%%%%%%%%%%%%%%%%%%%%%%%%%%%%%%%%%%%%%%%%
%%%%%                             DATA SAMPLE                               %%%%
%%%%%%%%%%%%%%%%%%%%%%%%%%%%%%%%%%%%%%%%%%%%%%%%%%%%%%%%%%%%%%%%%%%%%%%%%%%%%%%%

In this section, we explain where we obtained our data and the steps we take to 
put them together in a USPS ZIP code by month panel.
First, we describe the sources and construction of our residence and workplace 
MW measures.
We describe in detail the trends, timing, and geographic patterns of the 
statutory MW changes that give raise to rich variation in our MW variables.
Second, we describe the Zillow data on rents.
We explore how the sample of ZIP codes available in Zillow compares to the US 
sample of ZIP codes.
Third, we detail sources of other data.
Finally, we guide the reader through the construction of the baseline sample we
use in estimation.

%%%%%%%%%%%%%%%%%%%%%%%%%%%%%%%%%%%%%%%%%%%%%%%%%%%%%%%%%%%%%%%%%%%%%%%%%%%%%%%%
\subsection{Rents Data from Zillow}

One of the main challenges to estimate the effects of any policy on the rental
housing market is to obtain adequate data.
Recent papers have used the Small Area Fair Market Rents (SAFMRs) series from 
\textcite{hudSAFMR}, available at the USPS ZIP code and year level 
\parencite{Tidemann2018, Yamagishi2019}.
We, on the other hand, leverage data from Zillow at the ZIP code and month level.
The higher frequency of the Zillow data is an advantage since it allows us to 
explore the effects of MW changes on rents exploiting their precise timing.

Zillow is the leader online real estate and rental platform in the US, hosting
more than 110 million homes and 170 million unique monthly users in 2019 
\parencite{ZillowFacts}.
Zillow provides the median rental and sale price among homes listed on the 
platform for different house types and at geographic and time aggregation 
levels \parencite{ZillowData}.%
\footnote{The availability of different time series changed over time, so not 
all series used for the analysis might be still available to download.
See \textcite{ZillowData} for more details on the data shared by Zillow, and 
\textcite{ZillowDataArchive} for a snapshot of the website as of February 2020.} 
We collect the USPS ZIP code level monthly time series.
The timespan of the data varies at the ZIP code level, and units with a small 
number of listings are omitted.%
\footnote{Two related notes:
(i) once a ZIP code enters our panel, it remains until the final month of our 
data (December 2019);
(ii) the threshold used by Zillow to censor the data is not made public.}
As explained below, we construct a balanced panel to address the changing 
composition of the sample.

We focus our primary analysis on a single housing category:
\textit{single-family} houses, \textit{condominium, and cooperative} units (SFCC).
This is the series with the largest number of non-missing ZIP codes, as it 
covers the most common U.S. rental house types.
In fact, roughly a third of the nation's 47.2 million rental units in 2018 fit 
the category of single-family homes \parencite{Fernald2020}.

To account for systematic differences in house size across locations we focus 
on \textit{per square foot} rents.
Our main outcome variable represents the median rental price per square foot in 
the SFCC category among units listed in the platform for a given ZIP code and 
month.
Appendix Table XXX shows that our results are similar when we use other rental 
variables.
%
% SH: To do, heterogeneity table
%

Zillow data has several limitations.
First, we do not observe the underlying number of units listed for rent in a 
given month.
We do observe the number of houses listed \textit{for sale}, which we use as a
proxy for the number of rentals in robustness analysis.
A second limitation is that Zillow's market penetration dictates the sample of 
ZIP codes available.
Appendix Figure \ref{fig:map_zillow_sample} shows that the sample of ZIP codes
we observe in Zillow typically coincide with high population-density areas.
We conduct several exercises to account for this issue.

To ensure that our data correctly captures the price evolution of the US rental 
market, we compare 
Zillow's median rental price in the SFCC category with 
three SAFMRs series for houses with a different number of bedrooms (2, 3, and 
4 or more).
SAFMRs are calculated for ZIP codes within metropolitan areas at a yearly level, 
and generally correspond to the 40th percentile of the rent distribution for 
that ZIP code.%
\footnote{For more information on how SAFMRs are calculated, see 
\textcite[][page 41641]{hudPreamble}.}
Appendix Figure \ref{fig:trend_zillow_safmr} shows that these series evolve
very similarly over time.

%%%%%%%%%%%%%%%%%%%%%%%%%%%%%%%%%%%%%%%%%%%%%%%%%%%%%%%%%%%%%%%%%%%%%%%%%%%%%%%%
\subsection{Minimum Wage}\label{sec:mw_construction}

\subsubsection*{The Statutory Minimum Wage}

We collect data on federal-, state-, county-, and city-level statutory MW levels 
from \textcite{VaghulZipperer2016}.
We supplement their data, available up to 2016, with data from 
\textcite{BerkeleyLaborCenter} for the years 2016--2019.
We assign MW levels to USPS ZIP codes by taking the following steps.
First, we assign a single census-based ZIP Code Tabulation Area (ZCTA) to each 
USPS ZIP code using a correspondence from \textcite{UDSMapper}.
Then, based on the ZCTAs we assing to each USPS ZIP code all possible 
place-county combinations using data from the \textcite{MissouriCDC}, and 
we keep only the place-county combination with the highest number of housing 
units in the 2010 census.
Then, based on place and county codes we assign a unique state and local MW 
level to each USPS ZIP code.
We define \textit{the statutory MW} at ZIP code $i$ and month $t$, $\MW_{it}$, 
as the maximum between the federal, state, county, and city levels.%
\footnote{Some states and cities issue different MW levels for small businesses
(usually identified by having less than 25 employees).
In these cases, we select the general MW level as the prevalent one.
In addition, there may be different (lower) MW levels for tipped employees.
We do not account for them because employers are typically required to make up 
for the difference between tipped MW plus tips and actual MW.}
% Backing up claim on tipped MW: https://www.dol.gov/general/topic/wages/wagestips

As a result, we only use MW changes that are binding, meaning that they actually 
modify that maximum.

When restricting 
to the sample of ZIP codes available in Zillow, and 
to our sample period, our data reports 
18,689 statutory MW changes at the ZIP code-month level.
These, in turn, arise from 
151 state-level and 
182 county and city-level changes.
%
% SH: Adding a note to remind us to document these numbers
%
Figure \ref{fig:mw_changes_dist} shows the distribution of positive increases in
our statutory MW variable among all ZIP codes available in the Zillow data.%
\footnote{There are a few cases of decrease in the MW arising from judicial 
decisions overthrowing local MW ordinances.
For expository reasons, they are not shown in the figure.
However, they are used in estimations throughout the paper.}

Panel (a) shows the distribution of the intensity of our residence MW changes. 
The average percent change among Zillow ZIP codes is 5.5\%.
%% From unbalanced panel in derived_large
%%
%% SH: We should also document this in `/descriptive/stats/'
%%
Our estimation strategy exploits the intensity of MW changes.
On the other hand, panel (b) shows the timing of those changes between 2010 and 
2019.
Most changes occur in either January or July, and the majority of them take 
place later in the panel, where our rents data is more abundant.
Statutory MW changes have also been concentrated geographically.
Appendix Figure \ref{fig:map_mw_perc_changes} shows the percentage change 
in the statutory MW levels from January 2010 to December 2019.
There exist many areas across and within state borders that have differential 
MW changes,
which will be central to distinguish the effect of the two MW-based measures
proposed in Section \ref{sec:model}.
We describe these measures in the next subsection. 

\subsubsection*{The Residence and Workplace Minimum Wage Measures}

In this subsection we define the minimum wage variables we use in our analysis.
Both follow the defintions in Proposition \ref{prop:representation}.
With our MW panel at hand, computing the residence MW is starightforward.
We define it as
\begin{equation*}
    \mw^{\res}_{it} = \ln \MW_{it} \ .
\end{equation*}

We also construct the workplace MW, which captures the spillover effects of
statutory MW policies across locations.
To construct this measure we need to know, for each ZIP code, where workers 
residing in that location work.
We obtain this information from the Longitudinal Employer-Household 
Dynamics Origin-Destination Employment Statistics \parencite[LODES;][]{LODES}
for the years 2009 through 2018.
The data come at the block-group level.
We aggregate it to construct a ZIP code origin-destination matrix where we 
observe, for each location, the number of residents working in every other 
location.

We then use this ZIP code residence-workplace matrix to build exposure weights.
Let $\Z(i)$ be the set of ZIP codes in which $i$'s residents work 
(including $i$).
We construct the set of weights $\{\omega_{iz}\}_{z\in\Z(i)}$ as 
$$
\omega_{iz} = \frac{N_{iz}}{N_i} ,
$$
where 
$N_{iz}$ is the number of workers who reside in $i$ and work in $z$, 
and $N_i$ is the total working-age population of $i$.
Appendix table XXX shows how our results change when we use young workers
and low-income workers as weights.%
%%
%% SH: To-do, robustness table 
%%
\footnote{The LODES data additionally reports origin-destination matrices for 
the number of workers 29 years old and younger, and the number of workers 
earning less than \$1,251 per month.
The resulting workplace MW measures with any set of weights are highly correlated 
among each other ($\rho>0.99$ for every pair).}
%%% 
%%% SH: We should document this claim in /descriptive/sumstats/
%%%
We define the workplace minimum wage measure as
\begin{equation}
    \mw^{\wkp}_{it} = \sum_{z\in\Z(i)} \omega_{iz} \ln \MW_{zt} \ .
\end{equation}

Figure \ref{fig:map_mw_chicago_jul2019} illustrates the difference in these 
measures by plotting the change in the residence and workplace MW 
in the metropolian area of Chicago on July 2019.
On that month, both Cook County and the city of Chicago increased the statutory 
MW from \$12 to \$13.
We observe how the increase affects ZIP codes far beyond the limits of the 
county, suggesting that rents may be affected there as well.
For completeness, Appendix Figure \ref{fig:map_rents_chicago_jul2019} shows
the changes in our main rents variable around the same date.

%%%%%%%%%%%%%%%%%%%%%%%%%%%%%%%%%%%%%%%%%%%%%%%%%%%%%%%%%%%%%%%%%%%%%%%%%%%%%%%%
\subsection{Other Data Sources}\label{sec:data_other}

To proxy for local economic activity we collect data from the 
Quarterly Census of Employment and Wages \parencite[QCEW;][]{QCEW} 
at the county-quarter and county-month levels 
for every main industrial division and from 2010 to 2019.%
\footnote{The QCEW covers the following industrial aggregates: 
``Agriculture, Forestry, and Fishing'', ``Mining'', ``Construction'', ``Manufacturing'', 
``Transportation and Public Utilities'', ``Wholesale Trade'', ``Retail Trade'',
``Financial activities'' (including insurance and real state), ``Services'', and 
``Public Administration''.}
For each county-quarter-industry, we observe the number of establishments and the 
average weekly wage. For each county-month-industry cell, we additionally observe the number 
of employed people. We use this data for descriptive purposes and as controls for the 
state of the local economy in our regression models.

We collect Individual Income Tax Statistics aggregated at the USPS ZIP code 
level for the period 2010--2019 \parencite{IRS}.
For each ZIP code year we observe the number of households, population, adjusted 
gross income, total wage bill, total business income, number of households that 
receive a wage, number of households that have business income, and the number 
of households with farm income.
We use this data in our counterfactual exercises.

Finally, in order to describe our sample of ZIP codes we use data from the 
the 5-year 2007-2011 American Community Survey \parencite[ACS;][]{ACS} and from 
Small Area Fair Market Rents \parencite[SAFMR;][]{hudSAFMR}.

% DGP: We need to add a paragraph here about identifying MW workers in each ZIP code. 
% This would probably use LODES but it couls use something else. 
%%
%% SH: I agree. We may even want to give this LODES its own section
%%     Leaving the note as reminder

%%%%%%%%%%%%%%%%%%%%%%%%%%%%%%%%%%%%%%%%%%%%%%%%%%%%%%%%%%%%%%%%%%%%%%%%%%%%%%%%
\subsection{Estimation Samples}\label{sec:data_final_panel}

We put together an unbalanced panel dataset at the ZIP code and monthly date 
levels from February 2010 to December 2019.
Given that ZIP codes enter the Zillow data progressively over time affecting 
the composition of the sample,
%%
%% SH: We could add here the figure that tracks the numbers of ZIP codes in the data.
%%
we construct our baseline estimating panel by keeping in the sample those ZIP 
codes with valid rents data as of July 2015.
This partially balanced panel contains 5,302 MW increases, which arise from 
124 state-level changes and 99 county- and city-level changes.
4,224 of those changes take place after ZIP codes already entered the panel, 
and thus are used in estimation.
%%
%% DGP: reminder to document these claims.
%%
We also construct a fully balanced panel since July 2015 that delivers similar 
though noisier results.
%%
%% SH: make sure that this is true
%%

Table \ref{tab:stats_zip_samples} further compares the Zillow sample and our
baseline panel to the population of ZIP codes along several demographic 
dimensions. 
The first and second columns report data for the whole universe of US ZIP codes 
and for the set of urban ZIP codes, respectively.
The third column shows the set of ZIP codes in the Zillow data, i.e., those 
that have some value of rents per square foot in the SFCC category.
We refer to this as the Zillow sample.
Finally, the fourth column shows descriptive statistics for our baseline 
estimating sample described above.

Our baseline ZIP codes tend to be more populated, with a higher share of black 
and hispanic inhabitants, richer, and with a higher share of renter households 
than both the average US ZIP code and the average urban US ZIP code.
This is the case because Zillow is present in almost every large urban market, 
but it does not operate as often in small urban or rural areas.
Therefore, our results can be interpreted as relevant for large urban areas.
However, even in this sample our ZIP codes are richer than the average 
(i.e., arguably less influenced by MW changes), so we expect to subestimate 
the true parameters.
In an attempt to capture the treatment effect for the average US urban ZIP code 
we conduct an estimation excersie where we re-weight our sample to match the 
average of a handful of sociodemographic characteristics of those.

Finally, Appendix Table \ref{tab:stats_est_panel} shows some sample statistics 
of our baseline estimating panel.
As suggested in the table, the distribution of the residence and workplace MW 
measures is similar.
However, we show in the next section that they do show independent variation
in our model.
We also show summary statistics of median rents in the SFCC category.
The average of monthly median rents is \$1,665 in absolute values and \$1.23 
per square foot, although these variables show a great deal of variation.
Finally, we show average weekly wage, employment and establishment count 
for the QCEW industries we use as controls in some models.
