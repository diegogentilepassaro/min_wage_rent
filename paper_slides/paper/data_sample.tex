%%%%%%%%%%%%%%%%%%%%%%%%%%%%%%%%%%%%%%%%%%%%%%%%%%%%%%%%%%%%%%%%%%%%%%%%%%%%%%%%
%%%%%                             DATA SAMPLE                               %%%%
%%%%%%%%%%%%%%%%%%%%%%%%%%%%%%%%%%%%%%%%%%%%%%%%%%%%%%%%%%%%%%%%%%%%%%%%%%%%%%%%

In this section, we explain where we obtained our data and the steps we take to 
put them together in a USPS ZIP code by month panel.
First, we describe the sources and construction of our residence and workplace 
MW measures.
We describe in detail the trends, timing, and geographic patterns of the 
statutory MW changes that give raise to rich variation in our MW variables.
Second, we describe the Zillow data on rents.
We explore how the sample of ZIP codes available in Zillow compares to the US 
sample of ZIP codes.
Third, we detail sources of other data.
Finally, we guide the reader through the construction of the baseline sample we
use in estimation.

%%%%%%%%%%%%%%%%%%%%%%%%%%%%%%%%%%%%%%%%%%%%%%%%%%%%%%%%%%%%%%%%%%%%%%%%%%%%%%%%
\subsection{Rents Data from Zillow}

One of the main challenges to estimate the effects of any policy on the rental
housing market is to obtain adequate data.
Recent papers have used the Small Area Fair Market Rents (SAFMRs) series from 
\textcite{hudSAFMR}, available at the USPS ZIP code and year level 
\parencite{Tidemann2018, Yamagishi2019}.
We, on the other hand, leverage data from Zillow at the ZIP code and month 
levels.
The higher frequency of the Zillow data is an advantage since it allows us to 
explore the effects of MW changes on rents exploiting their precise timing.

Zillow is the leader online real estate and rental platform in the US, hosting
more than 110 million homes and 170 million unique monthly users in 2019 
\parencite{ZillowFacts}.
Zillow provides the median rental and sale price among homes listed on the 
platform for different house types and at geographic and time aggregation 
levels \parencite{ZillowData}.%
\footnote{The availability of different time series changed over time, so not 
all series used for the analysis might be still available to download.
See \textcite{ZillowData} for more details on the data shared by Zillow, and 
\textcite{ZillowDataArchive} for a snapshot of the website as of February 2020.} 
We collect the USPS ZIP code level monthly time series.
The timespan of the data varies at the ZIP code level, and units with a small 
number of listings are omitted.%
\footnote{Two related notes:
(i) once a ZIP code enters our panel, it remains until the final month of our 
data (December 2019);
(ii) the threshold used by Zillow to censor the data is not made public.}
As explained below, we construct a balanced panel to address the changing 
composition of the sample.

We focus our primary analysis on a single housing category:
\textit{single-family} houses, \textit{condominium, and cooperative} units (SFCC).
This is the series with the largest number of non-missing ZIP codes, as it 
covers the most common US rental house types.
In fact, roughly a third of the nation's 47.2 million rental units in 2018 fit 
the category of single-family homes \parencite{Fernald2020}.
To account for systematic differences in house size across locations we focus 
on \textit{per square foot} rents.
Our main outcome variable represents the median rental price per square foot in 
the SFCC category among units listed in the platform for a given ZIP code and 
month.
However, we show results using median rents per square foot in other rental 
categories available in the data as well.

Zillow data has several limitations.
First, we do not observe the underlying number of units listed for rent in a 
given month.
We do observe the number of houses listed \textit{for sale}, which we use as a
proxy for the number of rentals in robustness analysis.
A second limitation is that Zillow's market penetration dictates the sample of 
ZIP codes available.
Appendix Figure \ref{fig:map_zillow_sample} shows that the sample of ZIP codes
we observe in Zillow typically coincides with high population-density areas.

To ensure that our data correctly captures the price evolution of the US rental 
market, we compare 
Zillow's median rental price in the SFCC category with 
three SAFMRs series for houses with a different number of bedrooms (2, 3, and 
4 or more).
SAFMRs are calculated for ZIP codes within metropolitan areas at a yearly level, 
and generally correspond to the 40th percentile of the distribution of rents.
Appendix Figure \ref{fig:trend_zillow_safmr} shows that these series evolve
very similarly over time.

%%%%%%%%%%%%%%%%%%%%%%%%%%%%%%%%%%%%%%%%%%%%%%%%%%%%%%%%%%%%%%%%%%%%%%%%%%%%%%%%
\subsection{Minimum Wage}\label{sec:mw_construction}

\subsubsection{The Statutory Minimum Wage}

We collect data on federal-, state-, county-, and city-level statutory MW levels 
from \textcite{VaghulZipperer2016}.
We supplement their data, available up to 2016, with data from 
\textcite{BerkeleyLaborCenter} for the years 2016--2019.%
\footnote{Some states and cities issue different MW levels for small businesses
(usually identified by having less than 25 employees).
In these cases, we select the general MW level as the prevalent one.
In addition, there may be different (lower) MW levels for tipped employees.
We do not account for them because employers are typically required to make up 
for the difference between tipped MW plus tips and actual MW.}
% Backing up claim on tipped MW: https://www.dol.gov/general/topic/wages/wagestips
We assign MW levels to USPS ZIP codes by taking the following steps.
First, we collect a crosswalk constructed by \parencite{CensusLODES} that contains,
for each census block, identifiers for block group, tract, county, CBSA 
(i.e., core-based statistical area), place (i.e., census designated place), and state.
Second, we assign the MW level of each jurisdiction to the relevant census block.
We use the state code for state MW policies, and we match local MW policies 
based on the names of the county and the place.
We define the statutory MW at each census block as the maximum of the federal,
state, county and place levels.
Then, based on an original correspondence table described in Appendix 
\ref{sec:blocks_to_uspszip}, we assign a USPS ZIP code to each census
block.
Finally, we define \textit{the statutory MW} at ZIP code $i$ and month $t$, $\MW_{it}$, 
as the weighted average of the statutory MW levels in its constituent blocks, 
where the weights are given by the number of housing units.

When restricting 
to the sample of ZIP codes available in Zillow, and 
to our sample period, our data reports 
$\ZIPMWeventsUnbal$ statutory MW changes at the ZIP code-month level.
%%
%% SH: Add thousands separator to this number
%%
These, in turn, arise from 
$\StateMWeventsUnbal$ state-level and 
$\CityCountyMWeventsUnbal$ county and city-level changes.
Figure \ref{fig:mw_changes_dist} shows the distribution of positive increases in
our statutory MW variable among all ZIP codes available in the Zillow data.%
\footnote{There are a few cases of decreases in the MW arising from judicial 
decisions overthrowing local MW ordinances.
For expository reasons, they are not shown in the figure.
However, they are used in estimations throughout the paper.}

Panel (a) shows the distribution of the intensity of the MW changes. 
The average percent change among Zillow ZIP codes is $\AvgPctChange$.
Our estimation strategy exploits the intensity of MW changes.
On the other hand, panel (b) shows the timing of those changes between 2010 and 
2019.
Most changes occur in either January or July, and the majority of them take 
place later in the panel, where our rents data is more abundant.
Statutory MW changes have also been concentrated geographically.
Appendix Figure \ref{fig:map_mw_perc_changes} shows the percentage change 
in the statutory MW levels from January 2010 to December 2019.
There exist many areas across and within state borders that have differential 
MW changes,
which will be central to distinguish the effect of the two MW-based measures
proposed in Section \ref{sec:model}.
We describe these measures in the next subsection. 

\subsubsection{The Residence and Workplace Minimum Wage Measures}

In this subsection we define the minimum wage variables we use in our analysis,
which follow Proposition \ref{prop:representation}.
With our MW panel at hand, computing the residence MW is straightforward.
We define it as
\begin{equation*}
    \mw^{\res}_{it} = \ln \MW_{it} \ .
\end{equation*}

We also construct the workplace MW, which captures the spillover effects of
statutory MW policies across locations.
To construct this measure we need to know, for each ZIP code, where workers 
residing in that location work.
We obtain this information from the Longitudinal Employer-Household 
Dynamics Origin-Destination Employment Statistics \parencite[LODES;][]{CensusLODES}
for the years 2009 through 2018.
We collected the datasets for ``All Jobs.''
The data is aggregated at the census block level, and is aggregated to ZIP codes 
using the original correspondence between census blocks and USPS ZIP codes 
described in Appendix \ref{sec:blocks_to_uspszip}.
This results in ZIP code residence-workplace matrices that, for each location 
and year, tell the number of jobs of residents in every other location.

We then use the 2017 ZIP code residence-workplace matrix to build exposure 
weights.
Let $\Z(i)$ be the set of ZIP codes in which $i$'s residents work 
(including $i$).
We construct the set of weights $\{\omega_{iz}\}_{z\in\Z(i)}$ as 
$$
\omega_{iz} = \frac{N_{iz}}{N_i} ,
$$
where 
$N_{iz}$ is the number of workers who reside in $i$ and work in $z$, 
and $N_i$ is the total working population of $i$.
Appendix table \ref{tab:robustness} shows how our results change when we 
use young workers and low-income workers to construct the weights.%
\footnote{The LODES data additionally reports origin-destination matrices for 
the number of workers 29 years old and younger, and the number of workers 
earning less than \$1,251 per month.
The resulting workplace MW measures with any set of weights are highly correlated 
among each other ($\rho>0.99$ for every pair).}
%%
%% MG: Documented in descriptive/events_count.
%%
We define the workplace minimum wage measure as
\begin{equation}
    \mw^{\wkp}_{it} = \sum_{z\in\Z(i)} \omega_{iz} \ln \MW_{zt} \ .
\end{equation}

Figure \ref{fig:map_mw_chicago_jul2019} illustrates the difference in these 
measures by plotting the change in the residence and workplace MW 
in the metropolian area of Chicago on July 2019.
On that month, both Cook County and the city of Chicago increased the statutory 
MW from \$12 to \$13.
We observe how the increase affects ZIP codes far beyond the limits of the 
county, suggesting that rents may be affected there as well.
For completeness, Appendix Figure \ref{fig:map_rents_chicago_jul2019} shows
the changes in our main rents variable around the same date.


%%%%%%%%%%%%%%%%%%%%%%%%%%%%%%%%%%%%%%%%%%%%%%%%%%%%%%%%%%%%%%%%%%%%%%%%%%%%%%%%
\subsection{Other Data Sources}\label{sec:data_other}

\subsubsection{Time-varying data}
\label{sec:data_other_timevarying}

We complement the origin-destination LODES matrices with block level aggregates 
on residence and workplace area characteristics from LODES 
\parencite{CensusLODES} for the years 2009 through 2018.
We aggregate these data to the USPS ZIP code level using the correspondence
table discussed in Appendix \ref{sec:blocks_to_uspszip}.
While in principle these data can be constructed from aggregating the 
origin-destination matrices, in practice they contain counts of workers broken
by more detailed categories, such as NAICS industrial aggregates and 
schooling levels.

To proxy for local economic activity we collect data from the 
Quarterly Census of Employment and Wages \parencite[QCEW;][]{QCEW} 
at the county-quarter and county-month levels for several industrial divisions 
and from 2010 to 2019.%
\footnote{The QCEW covers the following industrial aggregates: 
``Natural resources and mining,'' ``Construction,'' ``Manufacturing,'' 
``Trade, transportation, and utilities,'' ``Information,'' 
``Financial activities'' (including insurance and real state), 
``Professional and business services,'' ``Education and health services,'' 
``Leisure and hospitality,'' ``Other services,'' ``Public Administration,''
and ``Unclassified.''}
For each county-quarter-industry, we observe the number of establishments and 
the average weekly wage.
For each county-month-industry cell, we additionally observe the number of 
employed people.
We use this data for descriptive purposes and as controls for the state of 
the local economy in our regression models.

We collect Individual Income Tax Statistics aggregated at the USPS ZIP code 
level for the period 2010--2019 \parencite{IRS}.
For each ZIP code year we observe the number of households, population, adjusted 
gross income, total wage bill, total business income, number of households that 
receive a wage, number of households that have business income, and the number 
of households with farm income.
We use this data in our counterfactual exercises.

\subsubsection{ZIP code characteristics}
\label{sec:data_other_cross}

Our sample of ZIP codes consists of those that are matched to some census block 
in Appendix \ref{sec:blocks_to_uspszip}.
While our MW assignment recognizes that many of these ZIP codes cross census 
geographies, we asign to each ZIP code a unique geography based on where the 
largest share of its houses fall.
We do this for descriptive purposes and also to use geography indicator 
variables in our empirical models.

In order to describe our sample of ZIP codes we collect data from the the 
2010 US Census \parencite{CensusDecennial}, 
the 5-year 2007-2011 American Community Survey \parencite[ACS;][]{CensusACS}, and 
from Small Area Fair Market Rents \parencite[SAFMR;][]{hudSAFMR}.
We collect most of these data at the block level, and aggregate it to ZIP codes
using the correspondence in Appendix \ref{sec:blocks_to_uspszip}.
Finally, we collect data on the number of workers in several income bins at
the tract level.
For each census tract, we observe the number of workers that in the last 12 
months earned current dollar wages within certain bins.%
\footnote{The bin categories are: 
less than \$10,000, between \$10,000 and \$14,999, between \$15,000 and \$24,999, 
between \$25,000 and \$34,999, between \$35,000 and \$49,999, between \$50,000 and \$64,999,
between \$65,000 and \$74,999, and more than \$75,000.}

\subsubsection{Locating Minimum Wage Earners}
\label{sec:locating_mw_workers}

An important piece of information for our analysis requires knowledge of the
residence and workplace locations of workers likely to earn the minimum wage.
To compute this variable one would need to know the distribution of income
by residence and workplace in each ZIP code.
Unfortunately, such figures are not readily available in public data.%
\footnote{These data cannot be constructed from usual sources.
Both for the Current Population Survey (CPS) and the 5\% Census samples available 
in IPUMS the smaller geographical unit is the PUMA (IPUMS CITE PENDING).}
%%
%% SH: COMPLETE CITATION
%%
Thus, to get a sense of the spatial distribution of minimum wage earners we 
proceed in several ways.

To estimate of the residence location of minimum wage workers we proceed as 
follows.
Using our assignment of hourly statutory MW in January 2011 we compute 
the total yearly wage of a full-time worker earning the MW in each census tract,
which we denote $\underline{YW}$.%
\footnote{We use the definition of full-time workers from \textcite{IRSfulltime}.
Specifically, we assume that a full-time employee works for 130 hours per week
for 12 months.}
Then, we compute in which wage bin $\underline{YW}$ falls.
We estimate the number of MW earners in a tract as the total number of workers 
in all bins below that one plus a fraction of the total number of workers in 
that bin given by 
$\left(\underline{YW} - b_\ell\right)/\left(b_h - b_\ell\right),$
where $b_h$ and $b_\ell$ represent the upper and lower limits of 
the bin.
We impute the tract estimates to ZIP codes proportionally to the share of 
houses in each tract that fall in every ZIP code the tract overlaps with.%
\footnote{More precisely, we compute a tract to ZIP code correspondence from
the LODES correspondence between block and tract, available in 
\parencite{CensusLODES}, and the geographical match between blocks and ZIP codes
in Appendix \ref{sec:blocks_to_uspszip}.
We compute for each tract the share of houses that fall in each ZIP code, and we
assume that the share in the tract-ZIP code combination equals the share of
houses times the estimated number of minimum wage workers in the tract.}
Finally, we compute the share of minimum wage workers who reside in each ZIP 
code dividing our estimate of the number of minimum wage workers to the total
number of workers in the data. 

It is not possible to use the same approach to estimate the workplace location
of workers.
Thus, in this case we solely rely on workplace characteristics data from LODES.
We exploit use the intuition that ZIP codes where, for example, many workers 
belong to the industry ``Accommodation and Food Services'' (NAICS sector 72),
are likely to be the workplace location of many MW earners.

%%%%%%%%%%%%%%%%%%%%%%%%%%%%%%%%%%%%%%%%%%%%%%%%%%%%%%%%%%%%%%%%%%%%%%%%%%%%%%%%
\subsection{Estimation Samples}\label{sec:data_final_panel}

We put together an unbalanced panel dataset at the ZIP code and monthly date 
levels from February 2010 to December 2019.
This panel contains $\ZIPMWeventsUnbal$ MW changes, which arise from 
$\StateMWeventsUnbal$ state-level changes and $\CityCountyMWeventsUnbal$ 
county- and city-level changes.
Given that ZIP codes enter the Zillow data progressively over time affecting 
the composition of the sample,
%%
%% SH: We could add here the figure that tracks the numbers of ZIP codes in the data.
%%
we construct our baseline estimating panel by keeping in the sample those ZIP 
codes with valid rents data as of July 2015.
This partially balanced panel contains $\ZIPMWeventsBase$ MW changes
at the level of the ZIP code.
We also construct a fully balanced panel by dropping dates before July 2015 
from our baseline panel, which contains $\ZIPMWeventsFullbal$ MW changes at the 
ZIP code level.

Table \ref{tab:stats_zip_samples} compares the Zillow sample and our baseline 
panel to the population of ZIP codes along several demographic dimensions. 
The first and second columns report data for the whole universe of US ZIP codes 
and for the set of urban ZIP codes, respectively.
The third column shows the set of ZIP codes in the Zillow data, i.e., those 
that have some non-missing value of rents per square foot in the SFCC category.
Finally, the fourth column shows descriptive statistics for our baseline 
estimating sample described above.
Throughout the paper we refer to this set of ZIP codes as the baseline sample.

Our baseline sample contains ZIP codes tend to be more populated, richer, 
with a higher share of Black and Hispanic inhabitants, and with a higher share 
of renter households than both 
the average US ZIP code and the average urban US ZIP code.
This is the case because Zillow is present in almost every large urban market, 
but it does not operate as often in small urban or rural areas.
Therefore, our results can be interpreted as relevant for large urban areas.
However, even in this sample our ZIP codes are richer than the average.
In an attempt to capture the treatment effect for the average US urban ZIP code 
we conduct an estimation exercise where we re-weight our sample to match the 
average of a handful of sociodemographic characteristics of those.

Finally, Appendix Table \ref{tab:stats_est_panel} shows some sample statistics 
of our baseline estimating panel.
As suggested in the table, the distribution of the residence and workplace MW 
measures is similar.
However, we show in the next section that they do show independent variation
in our model.
We also show summary statistics of median rents in the SFCC category.
The average of monthly median rents is \$1,665 in absolute values and \$1.23 
per square foot, although these variables show a great deal of variation.
Finally, we show average weekly wage, employment and establishment count 
for the QCEW industries we use as controls in some models.

For some estimations we construct analogous panels where the units of 
observation is the county by month and ZIP code by year.
In the county by month panel we define the statutory MW in an analogous fashion 
as for ZIP codes, and we use Zillow data that is already aggregated at this 
level.
We also define a baseline sample keeping counties with Zillow data as of July
2015.
In the ZIP code by year panel we compute the monthly difference in the log rents 
and MW measures and compute their yearly averages.
