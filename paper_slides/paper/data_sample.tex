%%%%%%%%%%%%%%%%%%%%%%%%%%%%%%%%%%%%%%%%%%%%%%%%%%%%%%%%%%%%%%%%%%%%%%%%%%%%%%%%
%%%%%                             DATA SAMPLE                               %%%%
%%%%%%%%%%%%%%%%%%%%%%%%%%%%%%%%%%%%%%%%%%%%%%%%%%%%%%%%%%%%%%%%%%%%%%%%%%%%%%%%

In this section, we explain where we obtained our data and the steps we take to 
put them together in a ZIP-code-by-month panel.
First, we describe the Zillow data on rents.
Second, we describe the sources and construction of our residence and workplace 
MW measures.
We describe in detail the trends, timing, and geographic patterns of the 
statutory MW changes that give raise to rich variation in our MW variables.
We explore how the sample of ZIP codes available in Zillow compares to the US 
sample of ZIP codes.
Third, we detail sources of other data used in our analysis.
Finally, we guide the reader through the construction of the baseline sample we
use in estimation.

%%%%%%%%%%%%%%%%%%%%%%%%%%%%%%%%%%%%%%%%%%%%%%%%%%%%%%%%%%%%%%%%%%%%%%%%%%%%%%%%
\subsection{Rents Data from Zillow}

One of the main challenges to estimate the effects of any policy on the rental
housing market is to obtain adequate data.
Recent papers have used the Small Area Fair Market Rents (hereafter SAFMRs) 
series from \textcite{hudSAFMR}, available at the USPS ZIP code and year level 
\parencite{Tidemann2018, Yamagishi2019}.
We, on the other hand, leverage data from Zillow at the ZIP code and month 
levels.
The higher frequency of the Zillow data is an advantage since it allows us to 
explore the effects of MW changes on rents exploiting their precise timing.

Zillow is the leading online real estate and rental platform in the US, hosting
more than 110 million homes and 170 million unique monthly users in 2019 
\parencite{ZillowFacts}.
Zillow provides the median rental and sale price among homes listed on the 
platform for different house types and at geographic and time aggregation 
levels \parencite{ZillowData}.%
\footnote{The availability of different time series changed over time, so most
data used in this paper is not available for download.
See \textcite{ZillowDataArchive} for a snapshot of the website as of 
February 2020.
We downloaded the data on January 2020, a few months before Zillow removed it 
from its website.} 
We collect the USPS ZIP code level monthly time series.
The time span of the data varies at the ZIP code level, and units with a small 
number of listings are omitted.%
\footnote{Two related notes:
(i) once a ZIP code enters our panel, it remains until the final month of our 
data (December 2019);
(ii) the threshold used by Zillow to censor the data is not made public.}
As explained below, we construct a balanced panel to address the changing 
composition of the sample.

We focus our primary analysis on a single housing category:
\textit{single-family} houses and \textit{condominium and cooperative} units (SFCC).
This is the series with the largest number of non-missing ZIP codes, as it 
covers the most common US rental house types.
In fact, roughly a third of the nation's 47.2 million rental units in 2018 fit 
the category of single-family homes \parencite{Fernald2020}.
To account for systematic differences in house size across locations we focus 
on \textit{per square foot} rents.
Our main outcome variable represents the median rental price per square foot in 
the SFCC category among units listed in the platform for a given ZIP code and 
month.
However, we show results using median rents per square foot in other rental 
categories available in the data as well.

Zillow data has several limitations.
First, Zillow's market penetration dictates the sample of ZIP codes available.
Appendix Figure \ref{fig:map_zillow_sample} shows that the sample of ZIP codes
we observe in Zillow typically coincides with areas of high population density.
Second, we only observe the median per-square-foot rental value among listings.
For example, we do not know other percentiles of the rent distribution, which
would permit focusing on the left tail of the housing market.
We also do not observe the number of units listed for rent in a given month,
which would allow us to assess the market conditions in the housing market
around MW changes.

To ensure that our data captures trends in the overall US rental market in 
urban areas, we compare Zillow's median rental price in the SFCC category with 
three SAFMRs series for houses with a different number of bedrooms (2, 3, and 
4 or more).
SAFMRs are calculated for ZIP codes within metropolitan areas at a yearly level, 
and generally correspond to the 40th percentile of the distribution of rents.
Appendix Figure \ref{fig:trend_zillow_safmr} shows that these series evolve
very similarly over time.

%%%%%%%%%%%%%%%%%%%%%%%%%%%%%%%%%%%%%%%%%%%%%%%%%%%%%%%%%%%%%%%%%%%%%%%%%%%%%%%%
\subsection{Minimum Wage}\label{sec:mw_construction}

\subsubsection{The statutory minimum wage}

We collect data on federal-, state-, county-, and city-level statutory MW levels 
from \textcite{VaghulZipperer2016}.
We supplement their data, available up to 2016, with data from 
\textcite{BerkeleyLaborCenter} and from official sub-national government offices 
for the years 2016--2019.%
\footnote{Some states and cities issue different MW levels for small businesses
(usually identified by having less than 25 employees).
In these cases, we select the general MW level as the prevalent one.
In addition, there may be different (lower) MW levels for tipped employees.
We do not account for them because employers are typically required to make up 
for the difference between tipped MW plus tips and actual MW.}
%
% Backing up claim on tipped MW: https://www.dol.gov/general/topic/wages/wagestips
%
Most ZIP codes are contained in a jurisdiction, and for them the statutory MW
is simply the maximum of the federal, state, and local levels.
Some ZIP codes cross jurisdictions, and so are bound to multiple statutory MW 
levels.
In these cases we assign a weighted average of the statutory MW levels in its
constituent census blocks, exploiting as explained an original correspondence 
table between census blocks and USPS ZIP codes, detailed in Appendix 
\ref{sec:blocks_to_uspszip}.
More details on the construction of the ZIP code-level statutory MW panel 
can be found in Appendix \ref{sec:assigning_mw_levels}.

When restricting 
to the sample of ZIP codes available in Zillow, and 
to our sample period, our data contain
$\ZIPMWeventsUnbal$ statutory MW changes at the ZIP code by month level.
These, in turn, arise from 
$\StateMWeventsUnbal$ state-level and 
$\CityCountyMWeventsUnbal$ county- and city-level changes.
Figure \ref{fig:mw_changes_dist} shows the distribution of positive increases in
our statutory MW variable among all ZIP codes available in the Zillow data.%
\footnote{There are a handful of cases of decreases in the MW arising from 
judicial decisions overthrowing local MW ordinances.
For expository reasons, they are not shown in the figure.
However, they are used in estimations throughout the paper.}

Panel (a) shows the distribution of the intensity of the MW changes. 
The average percent change among Zillow ZIP codes is $\AvgPctChange$.
Our estimation strategy exploits the intensity of MW changes.
On the other hand, panel (b) shows the timing of those changes between 2010 and 
2019.
Most changes occur in either January or July, and the majority of them take 
place later in the panel, where our rents data is more abundant.
Statutory MW changes have also been concentrated geographically.
Appendix Figure \ref{fig:map_mw_perc_changes} shows the percentage change 
in the statutory MW levels from January 2010 to December 2019.
There exist many areas across and within state borders that have differential 
MW changes,
which will be central to distinguishing the effect of the two MW-based measures
proposed in Section \ref{sec:model}.
We describe these measures in the next subsection. 

\subsubsection{The residence and workplace minimum wage measures}

In this subsection we define the minimum wage variables we use in our analysis,
which follow Proposition \ref{prop:representation}.
With our MW panel at hand, computing the residence MW is straightforward.
We define it as
\begin{equation*}
    \mw^{\res}_{it} = \ln \MW_{it} \ .
\end{equation*}

We also construct the workplace MW, which captures the spillover effects of
statutory MW policies across locations.
To construct this measure we need to know, for each ZIP code, where workers 
residing in that location work.
We obtain this information from the Longitudinal Employer-Household 
Dynamics Origin-Destination Employment Statistics \parencite[LODES;][]{CensusLODES}
for the years 2009 through 2018.
We collected the datasets for ``All Jobs.''
The raw data is originally aggregated at the census block level. 
We further aggregate it to ZIP codes using the original correspondence between 
census blocks and USPS ZIP codes described in Appendix 
\ref{sec:blocks_to_uspszip}.
This results in ZIP code residence-workplace matrices that, for each location 
and year, indicate the number of jobs of residents in every other location.

We then use the 2017 ZIP code residence-workplace matrix to build exposure 
weights.
Let $\Z(i)$ be the set of ZIP codes in which $i$'s residents work 
(including $i$).
We construct the set of weights $\{\omega_{iz}\}_{z\in\Z(i)}$ as 
$$
\omega_{iz} = \frac{N_{iz}}{N_i} ,
$$
where 
$N_{iz}$ is the number of workers who reside in $i$ and work in $z$, 
and $N_i$ is the total working population of $i$.
Appendix Table \ref{tab:robustness} shows how our results change when we 
use young workers and low-income workers to construct the weights.%
\footnote{The LODES data additionally reports origin-destination matrices for 
the number of workers 29 years old and younger, and the number of workers 
earning less than \$1,251 per month.
The resulting workplace MW measures with any set of weights are highly correlated 
among each other ($\rho>0.99$ for every pair).}
%%
%% MG: Documented in descriptive/events_count.
%%
We define the workplace minimum wage measure as
\begin{equation}
    \mw^{\wkp}_{it} = \sum_{z\in\Z(i)} \omega_{iz} \ln \MW_{zt} \ .
\end{equation}

Figure \ref{fig:map_mw_chicago_jul2019} illustrates the difference in these 
measures by plotting the change in the residence and workplace MW measures 
in the metropolitan area of Chicago in July 2019.
Starting on the first of that month, both Cook County and the city of Chicago 
increased the statutory MW from \$12 to \$13.
We observe how the increase affects ZIP codes far beyond the limits of the 
county, suggesting that rents may be affected there as well.
For completeness, Appendix Figure \ref{fig:map_rents_chicago_jul2019} shows
the changes in our main median rents variable around the same date.


%%%%%%%%%%%%%%%%%%%%%%%%%%%%%%%%%%%%%%%%%%%%%%%%%%%%%%%%%%%%%%%%%%%%%%%%%%%%%%%%
\subsection{Other Data Sources}\label{sec:data_other}

\subsubsection{Time-varying data}
\label{sec:data_other_timevarying}

We complement the origin-destination LODES matrices with block-level aggregates 
on residence and workplace area characteristics from LODES 
\parencite{CensusLODES} for the years 2009 through 2018.
We aggregate these data to the USPS ZIP code level using the correspondence
table discussed in Appendix \ref{sec:blocks_to_uspszip}.
While in principle these data can be constructed from aggregating the 
origin-destination matrices, in practice they contain counts of workers broken
by more detailed categories, such as NAICS industrial aggregates and 
schooling levels.

To proxy for local economic activity we collect data from the 
Quarterly Census of Employment and Wages \parencite[QCEW;][]{QCEW} 
at the county-quarter and county-month levels for several industrial divisions 
and from 2010 to 2019.%
\footnote{The QCEW covers the following industrial aggregates: 
``Natural resources and mining,'' ``Construction,'' ``Manufacturing,'' 
``Trade, transportation, and utilities,'' ``Information,'' 
``Financial activities'' (including insurance and real state), 
``Professional and business services,'' ``Education and health services,'' 
``Leisure and hospitality,'' ``Other services,'' ``Public Administration,''
and ``Unclassified.''}
For each county-quarter-industry, we observe the number of establishments and 
the average weekly wage.
For each county-month-industry cell, we additionally observe the number of 
employed people.
We use this data for descriptive purposes and as controls for the state of 
the local economy in our regression models.

We collect Individual Income Tax Statistics aggregated at the USPS ZIP code 
level for the period 2010--2019 \parencite{IRS}.
For each ZIP code and year we observe the number of households, population, 
adjusted gross income, total wage bill, total business income, number of 
households that receive a wage, number of households that have business income, 
and the number of households with farm income.
We use this data in our counterfactual exercises.

\subsubsection{ZIP code characteristics}
\label{sec:data_other_cross}

The full sample of ZIP codes we use in this paper consists of those that are 
matched to some census block following the procedure described in Appendix \ref{sec:blocks_to_uspszip}.
While our MW assignment recognizes that many of these ZIP codes cross census 
geographies, we assign to each ZIP code a unique geography based on where the 
largest share of its houses fall.
We do this for descriptive purposes and also to use geographical indicator 
variables in our empirical models.

In order to describe our sample of ZIP codes we collect data from the 
2010 US Census \parencite{CensusDecennial}, 
the 5-year 2007-2011 American Community Survey \parencite[ACS;][]{CensusACS}, and 
from Small Area Fair Market Rents \parencite[SAFMR;][]{hudSAFMR}.
We collect most of these data at the block level, and aggregate it to ZIP codes
using the correspondence in Appendix \ref{sec:blocks_to_uspszip}.

%%%%%%%%%%%%%%%%%%%%%%%%%%%%%%%%%%%%%%%%%%%%%%%%%%%%%%%%%%%%%%%%%%%%%%%%%%%%%%%%
\subsection{Estimation Samples}\label{sec:data_final_panel}

We put together an unbalanced panel dataset at the ZIP code and monthly date 
levels from February 2010 to December 2019.
This panel contains $\ZIPMWeventsUnbal$ MW changes at the ZIP code level, 
which arise from $\StateMWeventsUnbal$ state-level changes and 
$\CityCountyMWeventsUnbal$ county- and city-level changes.
Given that ZIP codes enter the Zillow data progressively over time affecting 
the composition of the sample,
%%
%% SH: We could add here the figure that tracks the numbers of ZIP codes in the data.
%%
we construct our baseline estimating panel by keeping in the sample those ZIP 
codes with valid rents data as of July 2015.
This partially balanced panel contains $\ZIPMWeventsBase$ MW changes
at the level of the ZIP code.
We also construct a fully balanced panel by dropping dates before July 2015 
from our baseline panel, which contains $\ZIPMWeventsFullbal$ MW changes at the 
ZIP code level.

Table \ref{tab:stats_zip_samples} compares the Zillow sample and our baseline 
panel to the population of ZIP codes along several demographic dimensions. 
The first and second columns report data for the whole universe of US ZIP codes 
and for the set of urban ZIP codes, respectively.
The third column shows the set of ZIP codes in the Zillow data, i.e., those 
that have some non-missing value of rents per square foot in the SFCC category.
Finally, the fourth column shows descriptive statistics for our baseline 
estimating sample described above.
Throughout the paper we refer to this set of ZIP codes as the baseline sample.

While our baseline sample contains only 11.8\% of all urban ZIP codes, it covers
25.0\% of their population and 25.7\% of their households.
With respect to demographic characteristics, ZIP codes in the baseline sample 
tend to be more populated, richer, with a higher share of Black and Hispanic 
inhabitants, and with a higher share of renter households than both the average
US ZIP code and the average urban ZIP code.
This is the case because Zillow is present in almost every large urban market, 
but it does not operate as often in small urban or rural areas.
In an attempt to capture the treatment effect for the average US urban ZIP code 
we conduct an estimation exercise where we re-weight our sample to match the 
average of a handful of characteristics of those.

Finally, Appendix Table \ref{tab:stats_est_panel} shows some sample statistics 
of our baseline estimating panel.
As suggested in the table, the distribution of the residence and workplace MW 
measures is similar.
However, we show in the next section that they do show independent variation
in our model.
We also show summary statistics of median rents in the SFCC category.
The average monthly median rent is \$1,665 in absolute values and \$1.23 
per square foot, although these variables show a great deal of variation.
Finally, we show average weekly wage, employment, and establishment count 
for the QCEW industries we use as controls in some models.

For some estimations we construct analogous panels where the units of 
observation are the county by month and ZIP code by year.
In the county-by-month panel we define the statutory MW in an analogous fashion 
as for ZIP codes, and we use Zillow data that is already aggregated at this 
level.
We also define a baseline sample keeping counties with Zillow data as of July
2015.
In the ZIP-code-by-year panel we compute the monthly difference in the log rents 
and MW measures and compute their yearly averages.
