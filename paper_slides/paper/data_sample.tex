%%%%%%%%%%%%%%%%%%%%%%%%%%%%%%%%%%%%%%%%%%%%%%%%%%%%%%%%%%%%%%%%%%%%%%%%%%%%%%%%%
%%%%%                             DATA SAMPLE                                %%%%
%%%%%%%%%%%%%%%%%%%%%%%%%%%%%%%%%%%%%%%%%%%%%%%%%%%%%%%%%%%%%%%%%%%%%%%%%%%%%%%%%

In this section, we explain in detail what are our sources of data and the steps that 
we take to put them together in a USPS ZIP code by month panel. We use data from February 
2010 to December 2019.\footnote{The Zillow rental data starts in February 2010.} 
First, we describe the sources and construction of our residence and workplace MW measures. 
We describe in detail the trends, timing, and geographic patterns of the statutory MW changes 
that give raise to rich variation in our MW variables. Second, we describe our data on rents 
coming from Zillow. We explore how the sample of ZIP codes available at Zillow compares to the 
U.S. sample of ZIP codes. Third, we detail our other sources of data. Finally, we guide 
the reader through the construction of our baseline sample using the described data sources.

%%%%%%%%%%%%%%%%%%%%%%%%%%%%%%%%%%%%%%%%%%%%%%%%%%%%%%%%%%%%%%%%%%%%%%%%%%%%%%%%%
\subsection{The Residence and Workplace Minimum Wage}\label{sec:mw_construction}


\subsubsection{The Residence Minimum Wage}\label{data_res_mw}

We collect data on federal, state, county, and city-level statutory MWs from 
\textcite{VaghulZipperer2016}. We supplement their data, available up to mid-2016, 
for the years 2016 to 2019 from \textcite{BerkeleyLaborCenter}. 
Because we are interested in studying rental dynamics at the USPS ZIP code level using Zillow, 
we assign MW levels to USPS ZIP codes by taking the following steps. First, we assign to each USPS 
ZIP code to a single ZIP Code Tabulation Area (ZCTA) using a crosswalk from \ref{UDSMapper}.\footnote{
	ZCTA is a geographic unit reported by the US Census Bureau}. 
Then, based on ZCTA we merge data from the \ref{MissouriCDC} to assign to each USPS ZIP code 
all possible place-county combinations. For each USPS ZIP code we keep only the place-county 
combination that number of housing the highest number of housing units in the 2010 census. 
Based on place and county codes, we are able to assign a unique state and local level MW to each 
USPS ZIP code. We define our \textit{residence} MW variable as the maximum between the 
federal, state, county, and city statutory MWs.\footnote{Some states 
	and cities issue different MW levels for small businesses (usually identified by having 
	less than 25 employees). In these cases, we select the general MW level as the prevalent 
	one. In addition, there may be different (lower) MW levels for tipped employees. We do not 
	account for them because employers are typically required to make up for the difference 
	between tipped MW plus tips and actual MW.}
% Backing up claim on tipped MW: https://www.dol.gov/general/topic/wages/wagestips
% DGP: Should we add this claim?
As a result, we only use MW changes that are binding, meaning that they actually modify 
that maximum.

When restricting to the sample of ZIP codes available in Zillow, and to our sample period, 
our data reports 18,689 residence MW changes at the ZIP code-month level. These, in turn, arise 
from 151 state-level and 182 county and city-level changes. Figure \ref{fig:mw_changes_dist} 
shows the distribution of positive increases in our statutory MW variable among all ZIP 
codes available in the Zillow data.\footnote{There are a few cases of decrease in the 
	MW arising from judicial decisions overthrowing local MW ordinances. For 
	expository reasons, they are not shown in the figure. However, they are 
	used in estimations throughout the paper.} %DGP: We should make sure this is the case.
Panel (a) shows the distribution of the intensity of our residence MW changes. The average percent 
change among Zillow ZIP codes is 5.5\%. %% From unbalanced panel in derived_large
However, we observe a fair amount of large increases. Our baseline estimation strategy will
exploit the intensity of MW changes. On the other hand, panel (b) shows the timing of 
those changes between 2010 and 2019. Most changes occur in either January or July, 
and the majority of them take place later in the panel. This could be problematic since
the timing of entry of ZIP codes into the panel is also concentrated in these months. We 
construct a balanced sample of ZIP codes to tackle this issue.

When focusing on the entire sample of US ZIP codes, residence MW changes have been concentrated 
geographically. Figure \ref{fig:mw_perc_changes_long_run} shows the percentage change in the 
residential MW levels from January 2010 to December 2019. There are substantial changes 
in all regions but the largest happened along the West and Northeast. There are substantial 
changes in the Midwest, and almost none of them in the South. From figure 
\ref{fig:mw_perc_changes_long_run} we can appreciate that there are many areas across 
and within state borders that have differential MW changes, so although our empirical strategy 
exploits ZIP code and monthly variation in a difference-in-differences fashion there are several 
ways to define treatment and control groups in a sensible way or and it could even be attractive 
to exploit borders in a regression discontinuity approach.
	\footnote{In our baseline specification, we use year-month and ZIP code fixed effects, but in 
	appendix XXX we show that our results are robust to interacting the year-month fixed effects
    with metropolitan area codes. This latter yields ZIP code comparisons only within metropolitan areas.
    We also report an alternative stacked model (see section XXX).} 


\subsubsection{The Workplace Minimum Wage}

As an attempt to capture the spillover effects of MW policies beyond residence ZIP codes 
we construct an alternative MW measure: the \textit{workplace} MW. 
This measure aims to account for the fact that workplace location 
often differs from the residence one, and therefore, the MW that may be of interest for 
a given local rental market could be the one experienced by the workers living in it by taking
into account not only the prevalent residence MW but also the access to other MW levels through
their commuting possibilities to other nearby ZIP codes. To construct this measure we need to 
know, for each ZIP code, where workers residing in that ZIP code work. We obtain this 
information from the 2017 Longitudinal Employer-Household Dynamics Origin-Destination 
Employment Statistics (LODES) \ref{LODES}. In particular, we use the origin-destination matrix mapping 
jobs from residence to workplace locations. The data come at the block group level. 
We aggregate it to construct a ZIP code residence-workplace matrix where we observe 
the number of workers for each residence-workplace pair.

We then use the ZIP code residence-workplace matrix to build exposure weights. Denote 
ZIP codes by $i$ and monthly dates by $t$. Let $\Z_i$ be the set of ZIP codes in 
which $i$'s residents work (including $i$). We construct the set of weights 
$\{\omega_{iz}\}_{z \in \Z_i}$ as $$\omega_{iz} = \frac{N_{iz}}{N_i} , $$ where 
$N_{iz}$ is the number of workers who reside in ZIP code $i$ and work in $z$, and $N_i$ 
is the total working-age population of ZIP code $i$.\footnote{The LODES data additionally 
	reports origin-destination matrices for the number of workers 29 years old and younger,  
	and the number of workers earning less than \$1,251 per month. We compute weights based 
	on both these sub-groups as well. However, the resulting workplace MW measures with
	any set of weights are highly correlated among each other ($\rho>0.99$ for every pair).
	Thus, we use working population weights throughout the paper.} 
Given that the origins present a large number of destinations with extremely low percentages of 
workers, we trim the number of destination ZIP codes to those making up to 90 percent of the 
workforce.\footnote{Results based on the full distribution are almost identical to those presented
	in the paper.} 
Letting $\underline{w}_{it}$ denote the residence MW in ZIP code $i$ and month $t$, we 
define the workplace minimum wage measure as

\begin{equation}
\underline{w}^{\text{exp}}_{it} = 
\sum_{z \in \Z_i} \omega_{iz} \underline{w}_{zt} \ . 
\end{equation}

The workplace MW of a ZIP code is based on the residence MWs that are binding in 
other nearby ZIP codes where its residents work. Therefore, an increase in a 
city, for example, may not have an impact on the local rental market if most 
residents are not MW workers. It will, instead, affect neighboring ZIP codes 
where MW workers reside. We use this insight all over our analysis. 

%%%%%%%%%%%%%%%%%%%%%%%%%%%%%%%%%%%%%%%%%%%%%%%%%%%%%%%%%%%%%%%%%%%%%%%%%%%%%%%%%
\subsection{Rents Data from Zillow}

One of the main challenges to estimate the effects of any policy on the rental housing market
is to obtain reliable data. Housing rent data has been particularly scant in the 
literature. Recent papers have used Small Area Fair Market Rents (SAFMRs) series from 
\textcite{hud}, available at the ZIP code and year level \parencite{Tidemann2018, 
Yamagishi2019}. We, on the other hand, leverage newly available data from Zillow at the 
ZIP code and month level. The high frequency of the Zillow data is an advantage since it 
allows us to explore the effects of MW changes on rents exploiting the precise timing of 
the enactment of hundreds of MW policy changes.

Zillow is the leader online real estate and rental platform in the U.S., hosting more 
than 110 million homes and 170 million unique monthly users in 2019 
\parencite{ZillowFacts}. Zillow provides the median rental and sale price (both 
total and per square foot) among homes listed on the platform in a given period. Time 
series are provided for different house types and at several geographic and time 
aggregation levels \parencite{ZillowData}.\footnote{The availability of different time 
	series changed over time, so not all series used for the analysis might be still 
	available to download. See \textcite{ZillowData} for more details on the data shared 
	by Zillow.} 
We collect the USPS ZIP code level monthly time series. The time span of the data 
varies at the ZIP code level, and geographical units with a small number of listings
are omitted.\footnote{Two related notes are the following: (i) once a ZIP code enters 
	our panel, it remains until the final month of our data (December 2019); (ii) we do not 
	know the threshold used by Zillow to censor the data.} 
As explained below, we construct a balanced panel to address the changing composition 
of the sample.

Clearly, even within a single ZIP code, there could be a great deal of heterogeneity in 
terms of house sizes and types, threatening the validity of our estimations.
To minimize price variation arising from housing units' characteristics, we focus 
our primary analysis on a single housing category: \textit{single-family} houses, 
\textit{condominium, and cooperative} units (SFCC). This is by far the series with the 
largest number of non-missing ZIP codes, as it covers the most common U.S. rental house 
types. In fact, roughly a third of the nation's 47.2 million rental units in 2018 fit the 
category of single-family homes \parencite{fernald2020americas}.\footnote{As for condominium 
	and cooperative units we do not know what share of the rentals they account for.} 
We want to define a rental price holding constant house size because, conditional on location,
it is likely the most time invariant heterogeneous attribute of a rental unit. Therefore, 
we focus on \textit{per square foot} rents. As a result, our main 
outcome variable represents the median rental price per square foot in the SFCC category 
among units listed in the platform for a given ZIP code and month. 

Zillow data has several limitations. The first one is that we do not observe the 
underlying number of units listed for rent in a given month. Therefore, changes in the 
inventory introduce additional variation in the reported median rental price that we 
are unable to control for. We do observe the number of houses listed \textit{for sale}, 
which we use as a proxy for this variable in robustness analyses.\footnote{We are not 
	aware of a ZIP code-month dataset that provides counts of houses for rent.}
A second limitation is that Zillow's market penetration dictates the sample of ZIP codes 
available. As a result, we observe a selected sample of typically urban ZIP codes.
%DGP: Diremos algo aca acerca de el balancing y de que con distintas samples nos da lo mismo? 
% diremos algo acerca de el ejercicio alternativo donde usabamos el panel con toda la data 
%pero controlamos por entry date?
We describe our sample in more detail later in this section.

To ensure that our data correctly captures the price evolution of the U.S. rental market, 
we compare Zillow's median rental price with 5 SAFMRs series for houses with a different 
number of bedrooms (0, 1, 2, 3, and 4 or more). SAFMRs are calculated for ZIP codes within 
metropolitan areas at a yearly level, and generally correspond to the 40th percentile of 
the rent distribution for that ZIP code.\footnote{For more information on how SAFMRs are 
	calculated, see \textcite[][page 41641]{hudPreamble}.} 
The correlation between Zillow's SFCC and SAMFR's ZIP-code-level time series is 
consistently above 90 percent. Appendix \autoref{fig:trend_zillow_safmrwgt} compares the 
time series variation of the Zillow SFCC series and a weighted average of the SAFMR series 
for different number of bedrooms.\footnote{	\label{foot:zillow_time_series}
	To compute the weighted SAMFR series, we proceed as follows. First, we compute the 
	national yearly average for both the Zillow SFCC and the 5 SAFMR series. Then, for 
	each of the latter, we compute the U.S. share of single family, condo, and cooperative 
	houses with that number of bedrooms using the \textit{American Housing Survey} (AHS). 
	To ensure comparability, we only use the estimated count for rental houses in this 
	step. (Additionally, AHS data is available only for years 2011, 2013, 2015, 2017, and 
	2019. We therefore fill missing years with the previous year's share.) Finally, we 
	weight SAFMR series using the shares mentioned above.} 
The Zillow rental data is always higher in levels. Part of this difference is intuitively 
related to the fact that Zillow reports median rent prices while SAFMRs are based on the 
40th percentile of the rent distribution. However, the two series show similar trends, 
confirming that Zillow does a reasonable job in capturing the overall dynamics of the U.S. 
rental market in metropolitan areas.

%%%%%%%%%%%%%%%%%%%%%%%%%%%%%%%%%%%%%%%%%%%%%%%%%%%%%%%%%%%%%%%%%%%%%%%%%%%%%%%%%
\subsection{Other Data Sources}\label{sec:data/other_data}

To proxy for local economic activity we collect data from the Quarterly Census of 
Employment and Wages (QCEW) \ref{QCEW} at the county-quarter and county-month levels 
for every main industrial division and from 2010 to 2019.\footnote{The QCEW covers the 
	following industrial aggregates: 
	``Agriculture, Forestry, and Fishing'', ``Mining'', ``Construction'', ``Manufacturing'', 
	``Transportation and Public Utilities'', ``Wholesale Trade'', ``Retail Trade'',
	``Financial activities'' (including insurance and real state), ``Services'', and 
	``Public Administration''.}
For each county-quarter-industry, we observe the number of establishments and the 
average weekly wage. For each county-month-industry cell, we additionally observe the number 
of employed people. We use this data for descriptive purposes and as controls for the 
state of the local economy in our regression models. 

We collect socio-demographic information from the 5-years 2007-2011 American Community 
Survey (ACS) \ref{ACS} at the ZCTA level. We assign it to our USPS ZIP code monthly panel using
the crosswalk from using again the crosswalk from \ref{UDSMapper}.
We collect data on the following characteristics: population, number of housing units, 
median income, African-American population, number of unemployed, and number of college 
students. We use this information to display descriptive statistics of our ZIP codes.

We collect Individual Income Tax Statistics at the USPS ZIP code level from the IRS \ref{IRS} 
from 2010 to 2019. For each ZIP code year we observe the number of households, population,
adjusted gross income, total wage bill, total business income, number of households that receive 
a wage, number of households that have business income, number of households with farm income. 
We use this data to describe our ZIP codes but also in heterogeneity and counterfactual exercises.

We collect rental prices at the ZIP code and year level from the US Housing and Urban 
Development (HUD). The data comes from the Small Area Fair Market Rents (SAFMR) \ref{hudSAFMR} 
reports that are a crucial input for designing housing voucher programs. We gather data 
from 2012 to 2016 and we use it for comparison with the Zillow data and for describing 
the ZIP codes in our sample. 

% DGP: We need to add a paragraph here about identifying MW workers in each ZIP code. This would probably
% use LODES but it couls use something else. 

%%%%%%%%%%%%%%%%%%%%%%%%%%%%%%%%%%%%%%%%%%%%%%%%%%%%%%%%%%%%%%%%%%%%%%%%%%%%%%%%%
\subsection{The Resulting Panel}\label{sec:data_final_panel}

Using the data described above, we put together a panel dataset at the ZIP code and monthly 
date levels from February 2010 to December 2019. Given that ZIP codes enter the Zillow data 
progressively over time affecting the composition of the sample, we construct our baseline 
\textit{estimating panel} by keeping in the sample those ZIP codes with valid rents data as 
of July 2015.\footnote{We note that the resulting panel is still unbalanced, in the sense 
	that the time series for some ZIP codes starts before July 2015. However, from July
	2015 onward our data contains no missing values in the main rent variable used in the 
	analysis.} 
This panel contains 5,302 MW increases, which arise from 124 state changes and 99 county 
and local level changes. 4,224 of those changes take place after ZIP codes already entered
the panel, and thus are used in estimation.
%% See analysis/sumstats ==> DGP: this directory doesn't run anymore. I've tried to fix it but 
%% it is so old that in the end I've killed it and we need to re do it as it would be easier.

We stress the fact that our data does not cover the full sample of ZIP codes, but rather 
a selected one. Appendix \autoref{fig:maps} maps the full set of available ZIP codes in 
the Zillow data, together with population density. The Zillow sample seems fairly 
distributed across urban areas, although some important areas have limited coverage. 


\autoref{tab:stats_zip_samples} further compares the Zillow sample to the 
population of ZIP codes along several critical demographic dimensions. Columns 1 
and 2 report data for the whole universe of U.S. ZIP codes and for the set of urban ZIP 
codes, respectively. In column 3 we show the set of ZIP codes in the Zillow data that have 
rental per square feet data in the SFCC category. We refer to this as the Zillow sample. 
Finally, column 4 shows descriptive statistics our baseline estimating sample. Focusing on 
our preferred variable---median rent per square foot in the SFCC category---, we collect rent 
data from Zillow for 3,315 unique ZIP codes, which amount to 8.5 percent of the 38,893 total 
for the entire U.S. and 46.7 percent of the 2010 U.S. population. 

The average median household annual income for those ZIP codes is \$65,475, almost 25 
percent higher than the same figure for the average U.S. ZIP code and 5 percent higher than 
the top 100 metropolitan areas. ZIP codes in the baseline sample are even richer, with an
average household income of \$66,920. Furthermore, both Zillow ZIP codes and those in our 
estimating panel have a higher share of urban population, college students, African-American
and Hispanic population, and houses for rent than the average urban ZIP code. In an attempt 
to capture the treatment effect for the average urban ZIP code we conduct an estimation 
re-weighting our sample to match characteristics of the top 100 CBSA sample of ZIP codes. 
Because our ZIP codes are richer than the average (i.e., arguably less influenced by MW
changes), we expect to find a larger effect in this exercise.

Finally, \autoref{tab:stats_est_panel} shows some basic sample statistics of our 
baseline estimating panel. As suggested in the table, the statutory and experienced MW 
are quite similar. We compare these measures in more detail in \autoref{sec:experienced_mw}.
We also show summary statistics of median rents in the SFCC category. The average of 
monthly median rents is \$1,651 in absolute values and \$1.27 per square foot, although 
these variables show a great deal of variation. Finally, for illustration, we show average 
weekly wage, employment and establishment count for the ``Financial activities'' sector 
from the QCEW. Appendix \autoref{tab:estimating_panel_stats_long} additionally shows 
summary statistics for the experienced MW computed using alternative weights, rents in 
different categories of the Zillow data, and the full set of QCEW industries we use as 
controls in our regressions.

