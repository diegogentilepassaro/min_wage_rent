%%%%%%%%%%%%%%%%%%%%%%%%%%%%%%%%%%%%%%%%%%%%%%%%%%%%%%%%%%%%%%%%%%%%%%%%%%%%%%%%%
%%%%%                            INTRODUCTION                                %%%%
%%%%%%%%%%%%%%%%%%%%%%%%%%%%%%%%%%%%%%%%%%%%%%%%%%%%%%%%%%%%%%%%%%%%%%%%%%%%%%%%%

% MOTIVATION. 
% CHALLENGES.
% THIS PAPER.

Many US jurisdictions have recently enacted minimum wage policies surpassing the 
federal level of \$7.25, creating considerable variation in minimum wage 
(hereafter MW) levels across and even within metropolitan areas.
These policies are inherently \textit{place-based} in that they are tied to 
a location, and workers may live and work in locations under different 
statutory MW levels, suggesting the presence of spatially heterogeneous policy 
effects.
While most research on the effects of the MW has focused on employment and 
wages irrespective of residence and workplace location
\parencite[e.g.,][]{CardKrueger1994, CegnizEtAl2019},
a full account of the welfare effects of MW policies requires an understanding 
of how they affect different markets and how their effects spill over across 
neighborhoods.
In fact, while the MW appears to lower income inequality through the labor 
market \parencite{Lee1999, AutorEtAl2016},
its overall effect on income for low-wage workers may be smaller if there is 
a significant pass-through from MW changes to prices, including housing
\parencite{Macurdy2015}.

In this paper, we study the effect of MW policies on local rental housing 
markets estimating their effects across neighborhoods within a city.
Consider the introduction of a new MW policy in certain neighborhoods of a 
metropolitan region.
Given that low-wage workers are likely to live in particular neighborhoods with 
access to the (now better-paying) low-wage jobs,
their disposable income will rise, causing a boost in housing demand and rental 
prices in residential areas rather than workplace ones.
This effect, arising from the MW at the workplace, could undermine the 
distributional objective of the policy.
Additionally, the MW hike may affect the jurisdiction that enacted the policy, 
for instance by increasing prices of non-tradable consumption.
This effect, operating through the MW at the residence, will affect the 
demand for housing as well, and consequently rental prices.
Thus, commuting patterns become an essential ingredient to understand the 
heterogeneous effects of local MW policies.

To operationalize this insight we collect granular data on commuting patterns 
and construct, for each USPS ZIP code (hereafter ZIP code) and month,
the \textit{workplace MW}, which we define as the log statutory MW where
the average worker of the ZIP code works.
We also define the \textit{residence MW}, which is just the log statutory MW in 
the same ZIP code.
Figure \ref{fig:map_mw_chicago_jul2019} visually represents these MW-based 
measures by illustrating their changes for the Chicago-Naperville-Elgin 
Core-Based Statistical Area (hereafter CBSA) in July 2019,
when the city of Chicago and Cook County increased the MW from \$12 to \$13 and 
from \$11 to \$12, respectively.
Even though the statutory MW only changed in some locations in the CBSA, the 
increase affected the workplace MW of most locations.
We formulate a simple partial-equilibrium model that suggests that these 
measures are sufficient to determine rents in a local housing market.

Studying the within-city spillover effects of the MW requires granular data on 
rents, which is why we employ a novel ZIP code-level panel dataset from Zillow.
Our main rent variable is calculated as the median rental price per square foot 
for listings within a specific ZIP code-month for Single Family houses, 
Condominiums, and Cooperative units (SFCC).%
\footnote{Single family houses are standalone housing units, while condominiums 
    and cooperatives are multi-unit buildings with varying ownership structures 
    \parencite{ZillowTypesOfHomes}.}
This variable captures the posted price of newly available units, 
thereby avoiding tenure biases and more accurately reflecting current market 
conditions \parencite{AmbroseEtAl2015}.
We find that low-wage households are more likely to be renters,
tend to reside in these housing types,
and that rents per square foot are surprisingly uniform across the income 
distribution.
These findings suggest that the Zillow data can feasibly capture any MW effects.
Moreover, the data varies monthly, aligning with the frequency of MW 
changes, thus allowing us to construct an estimation strategy that exploits the 
exact timing of hundreds of policy changes staggered across jurisdictions and 
months.

We develop a difference-in-differences strategy that compares the evolution of 
rents across ZIP codes differentially exposed to MW changes and find a 
significant effect of the workplace MW on rents.
To further illustrate the importance of commuting patterns in the propagation 
of MW shocks, we use our simple model and our main estimated elasticities to 
evaluate two MW policies: 
a federal MW increase and
a local MW increase in the city of Chicago.
We estimate the share of each dollar of extra income (generated by the MW) that 
accrues to landlords both combining all affected areas and in each particular 
location.
We then discuss our results' implications for assessing the distributional 
impact of MW policies.

% MODEL. Summarize the key formal assumptions you will maintain in your analysis.

The paper starts by laying out a motivating partial equilibrium model of a ZIP 
code's rental market, which is part of a larger geography.
The model is populated by workers who demand housing, and the interaction with 
a supply of rental units by absentee landlords determines the equilibrium rental 
price.
Importantly, residents of the ZIP code can commute to work in other ZIP 
codes, possibly under a different MW policy.
Workers' demand for square footage of housing is modelled as a function of 
prices of non-tradable consumption and income, both of which are influenced by 
the MW levels at residence and workplace locations.
The model illustrates that the impact of a change in MW legislation would vary 
across ZIP codes, depending on whether it primarily alters the MW at the
residence or workplace.%
\footnote{Specifically, MW increases in the workplace would cause rents to 
    go up, whereas increases at residence (conditional on a constant 
    workplace MW) would lower rents.}
The model implies the impact of MW changes in certain ZIP codes on rents can 
be summarized by the workplace MW and the residence MW measures,%
\footnote{This result relies on a constant-elasticity assumption, the 
    plausibility of which we discuss within the body of the paper.}
emphasizing the need to control for the residence MW in empirical analysis.

% METHODS. Explain how you take your model to the data and how you overcome the 
% challenges you raised in paragraphs 3-4.

Guided by the theoretical model, we pose an empirical model where log rents in 
a location depend linearly on
leads and lags of the workplace MW,
the residence MW,
ZIP code and time period fixed effects, and 
time-varying controls.
This model conditions on common trends in rents, and compares ZIP codes
that are differentially exposed to the workplace MW but equally affected by the 
residence MW.
The identification assumption of our model is that, within a ZIP code, 
changes in the workplace MW are strictly exogenous with respect to 
unobserved changes in rents after accounting for common trends and 
partialing out the confounding variation of generated by the residence MW.
Given that MW policies are typically not enacted considering their spillover
effects on local rental markets, we argue that this assumption is plausible.
%Pre-trends testing, as well as robustness tests that use different exposure 
%shares to the MW and alternative specifications, support this assumption.
In an appendix, we discuss a general potential outcomes framework following
\textcite{CallawayEtAl2021}.%
%\footnote{Given that our dependent variable has a shift-share structure, our 
%    identification assumptions can also be cast in terms of 
%    \textcite{GoldsmithpinkhamEtAl2020}, where the shares determine the exposure
%    to common minimum wage shocks.}
We demonstrate that, under the assumptions of \textit{parallel trends} and 
\textit{no selection on gains}, 
the effects of the residence and workplace MW are identified from the 
conditional slope of log rents with respect to each MW measure.
%The fact that the conditional relationship of log rents with respect to each MW
%measure is nearly linear suggests that no selection on gains is a plausible
%assumption as well.

% FINDINGS. Describe the key findings. Make sure they connect clearly to the 
% motivation in paragraphs 1-2.

Our preferred specification implies that 
a 10 percent rise in the workplace MW (holding constant the residence MW) 
increases rents by $\BothBetaBaseTen$ percent (SE=$\BothBetaBaseTenSE$).
Failing to control for the residence MW results in an estimated effect of 
$\OnlyWkpBetaBaseTen$ (SE=$\OnlyWkpBetaBaseTenSE$).
The reason this effect is lower is that the residence MW potentially triggers 
changes in unobserved variables that also affect rents.
The coefficient on the residence MW is negative, consistent with the story 
that increases in local non-tradable consumption ameliorate the effect of the
MW on rents, as in the theoretical model.
However, the coefficient is not statistically significant in our baseline,
and the lack of data on prices of non-tradable consumption of a ZIP code's 
residents prevents us from drawing strong conclusions about this effect.
Using a rough approximation to the share of MW workers in each ZIP code, we show 
that the elasticity of rents to the workplace MW is larger in locations 
with more MW residents, consistent with the fact that the effect affects 
the income of low-wage workers.
Likewise, we find a lower elasticity in locations with larger average incomes.
These results imply that MW changes spill over spatially through commuting, 
affecting local housing markets in places beyond the boundary of the 
jurisdiction that originally enacted the policy.

%% ROBUSTNESS

We provide support for our identification assumptions with a battery of 
additional analysis.
First, we test for pre-period coefficients and construct a non-parametric 
analysis of the relationship between log rents and the MW measures.
We find that future MW changes do not predict rents, and the conditional
relationship of log rents with respect to each MW measure is nearly linear,
suggesting that the identification assumptions are plausible.
Second, we estimate our model using a rental index constructed by Zillow that 
controls for variation in the available housing stock at each time.
This variable alleviates concerns that changes in the composition of available 
units, coinciding with MW changes, drive our estimates.
Third, our estimates are robust to using commuting shares for different
years, and they are stronger when we base the shares on jobs below a certain
nominal income threshold or focus on younger workers, both of which are more
likely to be affected by the MW.
This is consistent with the view that identification arises from the ``shares,''
as in \textcite{GoldsmithpinkhamEtAl2020}.
Finally, we construct a ``stacked'' regression model, similar to 
\textcite{CegnizEtAl2019}, that explicitly compares ZIP codes within 
metropolitan areas where some but not all experienced a change in the 
statutory MW.
This helps alleviate concerns that our estimates stem from undesired 
comparisons in difference-in-differences models with staggered treatment 
timing, as highlighted by recent literature 
\parencite{deChaisemartinEtAl2022,RothEtAl2022}.%
\footnote{
    We also estimate a model that includes the lagged first difference of rents 
    as a control, and is estimated via instrumental variables following 
    \textcite{ArellanoBond1991}.}
Pre-period coefficients are once again non-significant, supporting our
assumptions.

Our results remain robust across different sets of controls,
alternative samples of ZIP codes, and reweighing observations to match 
demographics of the population of urban ZIP codes.
We find similar (but noisier) results when we use median rents in 
different housing categories, which are available for smaller samples of 
ZIP codes.

%% COUNTERFACTUAL

In the final part of the paper, we construct a counterfactual exercise to 
capture the incidence of MW policies on landlords.
We compute the share pocketed by landlords in each ZIP code, and also
compute the total incidence summing across locations.
We simulate two counterfactual MW policies in January 2020, keeping all other
MW policies in their 2019 levels.
In the first scenario, we change the federal MW from \$7.25 to \$9.
In the second, we propose a rise in the Chicago City MW from \$13 to \$14.
We estimate that landlords capture $\totIncidenceCentsFedNine$ cents of each 
dollar across locations in affected CBSAs in the former, and 
$\totIncidenceCentsChiFourteen$ cents of each dollar across locations in the 
Chicago-Naperville-Elgin CBSA in the latter.
We find systematic spatial variation in incidence,
with the share pocketed usually being larger in locations that experience an
increase in the workplace MW but not in the residence MW.
Thus, commuting patterns are essential to understand the incidence of MW
policies on landlords.

Our results imply that a share of the extra income that low-wage workers
receive due to the policy is actually captured by landlords.
Viewed through the lens of our theoretical model,
the mechanism behind this is a rise in housing demand in a scenario of a 
finite housing supply elasticity.
In the context of a general equilibrium model, \textcite{KlineMoretti2014} argue
that this mechanism causes place-based policies to be welfare inefficient.
While studying the full welfare effects of MW policies is beyond the scope of 
the paper, our results imply that ignoring the housing channel leads to an 
overstatement of the gains of low-wage workers from MW policies.

%% LIMITATIONS

Our analysis has some important limitations.
A first limitation is that, while our results highlight the importance of 
controlling for the MW of the location that enacted the policy, data 
unavailability prevents us from directly exploring the role of the residence MW
on local prices of non-tradable consumption.
A second limitation is that, while our model is useful to motivate our 
empirical strategy, it does not account for general equilibrium effects such 
as changes in migration and commuting.
The counterfactual estimates should thus be taken as an approximation to the 
consequences of a small change in the MW.
A final limitation is that our exercises do not capture the full welfare 
effect of MW policies.
A full account of the long-run welfare effect of the sub-state MW policies in 
the 2010s requires specifying a general equilibrium model that accounts for 
changes in consumption prices, changes in workplace and residence locations
of workers, and potential employment effects.
However, as low-wage households are more likely to rent and thus to be 
negatively affected by rent changes, our analysis suggest that such computation 
should take into account the homeownership status of households.

%% LITERATURE

Our findings contribute to the literature studying the effects of MW policies 
on the housing market.
To our knowledge, the only papers whose goal is to estimate the effect of the 
MW on rents in the same location are \textcite{Tidemann2018} and 
\citeauthor{Yamagishi2019} (\cite*{Yamagishi2019}, \cite*{Yamagishi2021}).%
\footnote{In the working paper version \parencite{Yamagishi2019}, the author 
	explores this question using data from both the US and Japan.
	In the published version \parencite{Yamagishi2021}, he excludes the analysis 
	of the US case.}
\textcite{AgarwalEtAl2021} show that MW increases lower the probability of 
rental default, and also present estimates of the effect of the MW on rents using 
transactions data between 2000 and 2009.
Our paper also relates to \textcite{Hughes2020}, who studies the effect of 
MW policies on rent-to-income ratios.
The key difference of our paper with this work is that we differentiate 
between residence and workplace MW levels, incorporating spillovers across 
regions.
We highlight that this distinction is essential in a context of within-city
variation in MW policies, such as the recent experience in the US.
A second difference is the research design: we use high-frequency,
high-resolution data that allows clean identification at the level of the 
local housing market.

We also contribute to the understanding of place-based policies and the spatial 
transmission of shocks.
\textcite{KlineMoretti2014} argue that place-based policies may result in 
welfare losses due to finite housing supply elasticites.
\textcite{HsiehMoretti2019} quantify the costs of housing constraints in the US.
In line with this insight, we show that landlords differentially benefit from a 
place-based MW policy depending on their location.
\textcite{AllenEtAl2020} estimate the within-city transmission of expenditure 
shocks in Barcelona.
We, on the other hand, study the within-city transmission of MW shocks.

More broadly, our paper relates to the large literature estimating the effects
of MW policies on employment
(see \cite{Dube2019} and \cite{NeumarkShirley2021} for recent reviews of the 
literature), 
the distribution of income \parencite[e.g.,][]{Lee1999, AutorEtAl2016, 
	Dube2019Income}, 
and the overall welfare effect of the MW \parencite{AhlfeldtEtAl2022,
	BergerHerkenhoffMongey2022}.%
\footnote{Our paper is also related to work studying 
	the effects of local MW policies 
	\parencite[e.g.,][]{DubeLindner2021, JardimEtAl2022seattle}, 
	the effect of MW policies on commuting and migration 
	\parencite[e.g.,]{Cadena2014, Monras2019, PerezPerez2021}, 
	and prices of consumption goods 
	\parencite[e.g.,]{AllegrettoReich2018, Leung2021}.}
Our contributions are to incorporate spillovers across locations 
\parencite[as in the recent work by][]{JardimEtAl2022discontinuity} and to show 
that rent increases erode some income gains of low-wage workers.
We also contribute by developing a novel panel dataset of MW levels at the 
ZIP code level for the entire US.

Finally, our paper relates to work in econometrics that focuses on spillover 
effects across units,
both in the context of MW policies 
\parencite{Kuehn2016, JardimEtAl2022discontinuity}, 
and more generally of any policy that spills over spatially
\parencite{DelgadoFlorax2015, Butts2021}.
Our approach is similar to \textcite{GiroudMueller2019}: we specify a model for 
spillovers across units that allows us to estimate rich effect patterns of the 
MW on rents.

The rest of the paper is organized as follows.
Section \ref{sec:model} introduces a motivating model of the rental market.
In Section \ref{sec:data} we discuss the empirical relationship between income 
and housing and present our estimation data.
In Section \ref{sec:empirical_strategy} we discuss our empirical strategy and
identification assumptions.
In Section \ref{sec:results} we present our estimation results.
Section \ref{sec:counterfactual} discusses counterfactual MW policies, and
Section \ref{sec:conclusion} concludes.
