%%%%%%%%%%%%%%%%%%%%%%%%%%%%%%%%%%%%%%%%%%%%%%%%%%%%%%%%%%%%%%%%%%%%%%%%%%%%%%%%%
%%%%%                            INTRODUCTION                                %%%%
%%%%%%%%%%%%%%%%%%%%%%%%%%%%%%%%%%%%%%%%%%%%%%%%%%%%%%%%%%%%%%%%%%%%%%%%%%%%%%%%%

In recent years, many US jurisdictions have introduced minimum wages (hereafter MW) above the 
federal level of \$7.25.\footnote{As of January 2020, there were 29 states with a MW larger 
	than the federal one, 52 counties that set a MW above the state, and 15 cities with a 
	minimum above the county.}
Following the early work of \textcite{CardKrueger2000}, most research effort has been devoted 
to understanding the effects of MW policies on employment \parencite[e.g.,][]{Neumark2006, 
DubeEtAl2010, MeerWest2016, CegnizEtAl2019} and income inequality \parencite{Lee1999, 
AutorEtAl2016}. This is not surprising, as employment effects are of first order importance 
to determine the welfare implications of MW changes on households, whereas income inequality 
proxies for an important dimension of the welfare implications of these policies. However, the 
\textit{place-based} nature of MW provisions (accentuated by the fact that most recent 
legislation arises from local jurisdictions) makes it natural to expect that such policies 
will affect the welfare of households through other channels, such as the housing market. 
Not accounting for the potential effect of MW changes on rents is tantamount to omitting 
from the analysis one of the main channels trough which the MW may affect welfare and inequality.

Given these remarks, we pose the question: by what extent (if any) are local rents affected 
by the minimum wage? Surprisingly, there is very little research attempting to estimate 
the causal effect of MW policies on the housing market. To the best of our knowledge, the 
only papers aiming at answering this question directly are \textcite{Tidemann2018, Yamagishi2019, 
Yamagishi2020}.\footnote{\textcite{Yamagishi2019} explores this question using data from both 
	the U.S. and Japan. In an updated version of the paper, \textcite{Yamagishi2020} excludes 
	the analysis of the U.S. case.} 
Even though they use the same data at the year-county for the U.S, these papers find opposing 
results. \citeauthor{Tidemann2018}'s (\citeyear{Tidemann2018}) estimates are negative, whereas 
the results of \textcite{Yamagishi2019} point towards a positive effect.\footnote{
	\textcite{Yamagishi2019} attributes this difference to different model 
	specifications, and argues that with proper standard errors clustering the results in 
	\textcite{Tidemann2018} are statistically insignificant.} 
In a related paper, \textcite{AgarwalEtAl2019} shows that minimum wages decrease 
the probability of rental default, suggesting a strengthening of the local labor market.

Provided that MW policies have small disemployment effects, theory suggests that the effect 
on rents will be positive. A canonical version of the Alonso-Muth-Mills model, for example, 
predicts that general wage increases will be fully capitalized by landlords.\footnote{See 
	\textcite{Brueckner1987} for a complete treatment of this model.} 
In the same tradition, \textcite{Yamagishi2020} shows that minimum wage policies increase 
rents if disemployment effects are small, and that rents are a sufficient statistic of welfare 
under free mobility. We begin our paper by constructing a simple model of a zipcode's rental 
and labor markets, and argue that the effect should be positive. We use the model to benchmark 
the magnitude of our empirical estimates. [UNDER CONSTRUCTION] 
%%  WORK ON MODEL AND UPDATE
 
Understanding the effects of MW policies on the housing market is important both from a 
theoretical and policy perspective. As recent literature has shown, individuals 
respond to changes in local prices (and amenities) by migrating, and this fact has important 
implications for welfare and inequality \parencite{Diamond2016, Couture2019}. Several papers 
make a similar point for the case of MW policies, arguing that they influence migration 
decisions and the location of economic activity \parencite{PerezPerez2018, Monras2019}. We 
believe that a reliable estimate of the effect of MW policies on the local housing market 
will inform this literature and can serve as an important input for policy-makers.
 
In this paper, we construct a dataset at the U.S. zipcode and monthly date levels to explore 
the reduced form effects of MW changes on rents. Our main rent variable comes from Zillow, 
the largest online real-estate platform in the U.S. \parencite{realestateagentpdx, investopedia}, 
and corresponds to the median rent price per square foot across Zillow listings 
in the given zipcode-month cell of the category Single Family, Condominiums and Cooperative 
Houses (SFCC). This is the most popular housing category in the US \parencite{JCHS2020}, 
and also the most populated series in Zillow. We collect data on minimum wage changes from 
\textcite{VaghulZipperer2016} for the period from 1974 to 2016, which we update until January 
2020. Using these data, we construct the actual minimum wage in force in each zipcode and 
month. We also collect data from other sources to both validate our empirical model and to 
deploy as controls in our regressions, including the Quarterly Census of Employment and Wages 
(QCEW) and the Building Permits Survey (BPS). Finally, we use data from the U.S. Census and 
from the LEHD Origin-Destination Employment Statistics (LODES) to explore heterogeneity of 
the effect of interest.\footnote{LEHD is short for Longitudinal Employer-Household Dynamics, 
	which corresponds the source of the origin-destination data.}

Estimating the effect of MW policies on rents presents several challenges. A priori, 
it appears plausible that determinants of local level MW changes might correlate with 
geographical- and time-varying factors also affecting the housing market, invalidating naive OLS 
regressions. To account for this, we use difference-in-differences (DiD) panel specifications 
that condition both on monthly date and zipcode fixed effects. We term this two-way 
fixed-effects model estimated in first differences the \textit{static DiD} model. Identification 
comes from exploiting the size and fine timing of hundreds of MW changes staggered across 
different US jurisdictions from 2010 to 2019. As a result, this specification does not 
suffer from the under-identification problem arising when units are treated only once 
\parencite{BorusyakJaravel2017}. As we discuss in the paper, this estimate recovers the true 
causal effect of MW changes on rents assuming that, within a zipcode, MW changes are 
\textit{strictly exogenous} with respect to changes in the error term. We note that 
this assumption allows for unrestricted auto-correlation of the error term, which we also 
cluster at the state-level.

Strict exogeneity imposes the restriction that past and future MW changes must be uncorrelated 
with innovations in unobservables. This assumption may not hold for several reasons. First, there 
might exist dynamic effects of the MW on rents, ruled out by assumption in the static DiD 
model. Second, strict exogeneity amounts to a ``parallel trends'' assumption in the time-path of 
treated and untreated zipcodes which may not hold in practice. Intuitively, if effects are 
driven by some preexisting time-varying unobserved difference between treated and untreated 
zipcodes, we should see that future MW changes have an effect on rental prices. On the other 
hand, if MW changes can be thought as exogenous with respect to the zipcode rental market (as 
assumed by our model), we should see no anticipatory effects. Motivated by this, we estimate a 
\textit{dynamic} model by extending our static model with leads and lags of minimum wage 
changes. This model allows us to test potential dynamics of the effect of interest, and to 
assess the parallel trends assumption. Reassuringly, our models show no effects of future MW 
changes on current rents. They do, however, suggest a short-lived dynamic effect over the first 
couple months following a MW increase.

We validate our empirical strategy by putting our basic model through several tests. First of 
all, we check for the presence of unobservables affecting both rents and MW changes in two ways: 
(i) we allow for zipcode-specific linear and quadratic time trends; and (ii) we include 
controls that proxy for local economic shocks as well as shocks to the housing market that are 
unlikely to be influenced by our minimum wage variable.
%%%% DISCUSS THIS POINT IN PAPER
These specifications should capture a wide-range of potential confounders in our main 
regressions. The fact that our baseline estimates are robust to the inclusion of these 
controls strengthens the case for the strict exogeneity assumption of MW changes. Second, 
our rents variable is constructed as the median rent across available listings in the month, 
with many of them staying in the sample for more than one month. This introduces 
auto-correlation in the dependent variable which, if not accounted for, may bias our estimates. 
For this reason, we estimate alternative models that include the lagged first difference of rents 
as controls, estimated via instrumental variables following \textcite{ArellanoBond1991} and 
related literature. At the cost of imposing a particular auto-correlation structure in the 
error term, this specification has the advantage of allowing for feedback effects from current 
shocks to future minimum wage changes \parencite{ArellanoHonore2001}. The estimates of this 
model are strikingly similar to our baseline results, rendering credibility to our econometric 
assumptions.

Finally, a drawback of the Zillow data is that it includes a subsample of all U.S. zipcodes 
only. This fact brings about two concerns. First, the composition of zipcodes changes over 
time potentially introducing bias in our estimates. We tackle this issue by performing our 
main analysis with a constant set of units with valid data as of 2015.\footnote{Because these 
	zipcodes enter the sample at different moments in time before 2015, our estimating panel 
	is still unbalanced.}
This strategy alleviates concerns arising from the changing composition of zipcodes, but 
significantly lowers the number of observations used in the estimation. For this reason, we 
also estimate a model on the full sample of available zipcodes, obtaining similar results. 
Secondly, we worry that our estimated effect may be particular to our subsample. In order to 
approximate the average treatment effect for the typical urban zipcode, we reweight our data to 
match average demographic characteristics of the top 100 Metropolitan Statistical Areas 
and re-estimate our main models. The effects not only survive this test, but are bigger in 
magnitude and more precisely estimated.

Our results indicate a small % review following benchmarking
 yet robust impact of MW changes on rents. The \textit{static} 
difference-in-differences specification implies that a 10 percent increase in the MW leads to 
an average 0.26 percent increase in the rental price per square foot. When expanding the model 
to account for \textit{dynamic} effects, we find a statistically significant impact in the 
first two months following a MW change. As a result, a 10 percent increase in the minimum 
wage is estimated to rise rents in 0.45 percent. 

In an effort to disentangle who are the ``winners and losers", we perform an heterogeneity 
analysis of the average treatment effect by allowing the coefficients to differ across the 
distribution of zipcode characteristics. The results suggest that the effect of interest is 
indeed heterogeneous. Those zipcodes which are more likely to have minimum wagers as residents 
--i.e., zipcodes with a relatively high shares of unemployed workers, low-income households, 
and African-American population-- experience a pass-through which is almost twice as large. 
Consistently, we show that zipcodes with very low probability of having minimum wage workers 
as residents exhibit no significant effects. On the other hand, we find that the effect is 
constant across zipcodes with different share of MW workers who work there.

Our approach has several differences with respect to previous research on the topic. Both 
\textcite{Tidemann2018} and \textcite{Yamagishi2019} for the U.S. exploit Fair Markets Rents data 
from the Department of Housing and Urban Development (HUD), which is available at the yearly 
level and aggregated at the geographical level of counties.\footnote{\textcite{Yamagishi2019}, 
	updated in \textcite{Yamagishi2020}, also uses data at the year-prefecture level for the 47 
	Japanese prefectures.} 
An important advantage of our approach is that we use the exact timing of the MW change at 
the monthly level. When using variation arising from a yearly frequency some units are 
``partially treated" which will tend to understate the magnitude of the effect. 

%Furthermore, some jurisdictions have MW changes 
%on many subsequent years, making it challenging to estimate the dynamics around changes that are 
%followed by changes in the immediate year. For example, if there is a change in two subsequent 
%years, then the estimated effect of the change in the second year may be due too the effect of the 
%current MW change or to the past MW change or both. We are able to show that raising the MW  
%increases rents significantly only in the first couple of months after implementation.

Another advantage is that we use data at the zipcode- instead of the county-level.\footnote{
	As of 2019 there were 3,142 counties and 39,295 meaningful zipcodes in the US. We exclude 
	military and unique business zipcodes as they are irrelevant for the housing market.} 
We illustrate the importance of having smaller units of analysis with the following example. 
For a given county, suppose that (1) all low-skill jobs are in one particular zipcode; and 
(2) low-skill households prefer to live near their jobs. Further assume that, following a MW 
change, employment effects are near zero.\footnote{This is consistent with the findings of 
	\textcite{CardKrueger2000} and \textcite{CegnizEtAl2019}, among others.} 
One should then expect demand for housing in the zipcode with low-skill jobs to increase and 
demand for housing in the rest of the zipcodes to go down. If we focus on the effects of the 
MW increase on the county we might even find that the rents go down, when in fact the rents in 
the zipcodes where the low skill jobs are located are increasing. Indeed, \textcite{Tidemann2018} 
finds that a \$1 increase in the MW decreases the yearly average of the monthly rent by 1.5 
percentage points.\footnote{As pointed out by \textcite{Tidemann2018}, the sign of this 
	effect 	implies that the labor demand for low skilled workers is elastic. This is at 
	odds with results of null employment effects in the literature.} 

Using a more detailed geography also aids in the empirical estimation. First of all, it means 
that we can exploit MW changes at any jurisdictional level, effectively increasing the number 
of events used for identification. Secondly, it allows us to use more detailed controls, such 
as zipcode fixed effects and zipcode-specific polynomial trends. This is important 
because the dynamics of the rental market plausibly vary across zipcodes within a county 
following trends at the very local level \parencite{AlmagroDominguez2019}. Importantly, these 
controls make the required identification assumptions more credible. Given that the identifying 
variation comes from within-zipcodes, the determinants of these MW changes are unlikely to be 
related to the particular zipcode and, therefore, are less likely to be correlated to the 
unobservable determinants of rent dynamics there.

% Intuitively, this is the case because, for example, out-of-state migration is in principle more 
% costly than out-of-county migration.  therefore, we expect more residential resorting within a state 
% and across counties when a county changes their local MW wage. Our data allows us to study the 
% heterogeneous effects of different MW changes.\footnote{In principle, our data allows us to answer 
% whether the effects of changes at the federal, state, county, and city/town level are different.} 

% we can use the census to compute the level of employment and the distribution of income in each %  
% zipcode, and check if we observe stronger effects on rents in places where there are more MW earners 
% or in places where there is more low skilled employment. 


Beyond the contribution to the very recent literature on the effects of MW changes on rents, 
we contribute to several strands of the literature. First, we contribute to the literature 
studying the effects of minimum wages on the welfare of low-skill households \parencite[][among 
others]{DinardoEtAl1995, Lee1999, CardKrueger2000, Neumark2006, AutorEtAl2016, CegnizEtAl2019}. 
Most of this literature has focused on disemployment effects. We contribute to this strand of 
literature by exploring the effects of minimum wage policies on the housing market.

Our work also relates to the literature that studies the location decision of agents either 
based on income \parencite{Roback1982, Kennan2011, DesmetRossihansberg2013, PerezPerez2018, 
Monras2019} or on spatial rents and amenity differentials \parencite{Diamond2016, 
AlmagroDominguez2019, Couture2019}. We hope to contribute by adapting this framework to the 
case of the MW changes as a means to rationalize through residential location sorting part of 
the observed reduce form effect on rents.

The rest of the paper is organized as follows. Section \ref{sec:model} motivates the paper
with a simple model of the rental market. In section \ref{sec:data}, we present our data 
sources and show the characteristics of our estimating panel. In section 
\ref{sec:empirical_strategy}, we explain our empirical strategy and we discuss our 
identification assumptions. In section \ref{sec:results}, we present our main results. Section 
\ref{sec:discussion} discusses relevant policy implications, and 
section \ref{sec:conclusion} concludes.
