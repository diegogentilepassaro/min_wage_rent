%%%%%%%%%%%%%%%%%%%%%%%%%%%%%%%%%%%%%%%%%%%%%%%%%%%%%%%%%%%%%%%%%%%%%%%%%%%%%%%%%
%%%%%                            INTRODUCTION                                %%%%
%%%%%%%%%%%%%%%%%%%%%%%%%%%%%%%%%%%%%%%%%%%%%%%%%%%%%%%%%%%%%%%%%%%%%%%%%%%%%%%%%

% MOTIVATION. After reading these paragraphs a reader in any field of economics
% should believe that if you answer your research question your paper will make 
% an important contribution.

In recent years, many US jurisdictions have introduced minimum wages above the 
federal level of \$7.25, including states, counties, and cities.
The resulting variation in minimum wage levels within metropolitan areas makes it 
crucial to acknowledge the \textit{place-based} nature of these policies.
While most research on the effects of the minimum wage has focused on employment 
and wages irrespective of location 
\parencite[e.g.,][]{CardKrueger1994, AutorEtAl2016, CegnizEtAl2019}, 
a full account of the welfare consequences of the minimum wage requires an 
understanding of how it affects different markets and how its effects spill over 
across different neighborhoods \parencite[as recently emphasized by][]{DubeLindner2021}.

In this paper, we study the short-run effect of minimum wage (MW) policies across 
local rental housing markets.
One approach to answer this question would be to relate the minimum wage of a region
with the housing rents of people residing in that region.
This approach, which is the one that was taken in the existing literature, appears 
sensible only if all low-wage workers lived and worked in the same place.
However, if workers are subject to different wage floors in their workplace and
residence locations, and residence and workplace minimum wages have different 
effects on housing demand, this approach will yield innacurate results.
For instance, a minimum wage increase in the city center only will likely affect 
rents of low-wage workers commuting there, but residing outside the city (where 
there was no increase).
Figure \ref{fig:shares_own_geo} shows that the share of workers residing and working 
on the same geographical unit is low, especially at the local level.
As a result, one would expect that the effects of minimum wage increases at different 
jurisdictional levels will have spatially heterogeneous effect on rents, and thus 
on welfare of workers and homeowners, depending on the prevailing structure of 
minimum wage levels and commuting patterns.

% CHALLENGES. These paragraphs explain why your research question has not already
% been answered, i.e., what are the central challenges a researcher must tackle to
% answer this question.

There is little research attempting to estimate the causal effect of minimum wage 
policies on the housing market, and none estimating spatially heterogeneous effects.
To the best of our knowledge, the only papers that estimate the causal effect of 
minimum wages on rents in the same location are \textcite{Tidemann2018} and 
\citeauthor{Yamagishi2019} (\citeyear{Yamagishi2019, Yamagishi2021}).%
\footnote{\textcite{Yamagishi2019} explores this question using data from both the
US and Japan. In the published version of the paper, \textcite{Yamagishi2021} 
excludes the analysis of the US case.}
We think that the relative scarcity of research obeys several challenges.
First, the empirical approach needs to recognize the difference between workplace
and residence minimum wage changes, even more so if a large share of workers
commute to different jurisdictions (as we observe in the US data).
Second, estimation of minimum wage effects at the local level requires high-frequency
and spatially dissagregated data on rents.
Using large geographies might result in null or even negative effects on average
even if no one commutes outside of this region and the actual effect (of workplace
MW) is positive.%
\footnote{The reason is that rents in neighborhoods where low-wage workers live 
are likely to increase, whereas rents elsewhere in the region will likely decrease, 
with an unclear effect on average.}
Similarly, studying the effect over long time horizons while exploiting geographical
variation in minimum wage changes across sub-national jurisdictions is problematic.
In that case, the estimated effect will be conflated with other changes, such as
migration.
Finally, one needs to observe plausibly exogenous variation on minimum wage changes,
as these policies are unlikely to be passed at random, and control for variation 
in prices of non-tradable consumption that stem from the MW change.

% THIS PAPER. This paragraph states in a nutshell what the paper accomplishes and how.

We introduce several innovations to tackle these challenges.
First, we theoretically recognize that minimum wage policies will spill over across
housing markets through commuting.
We device a new estimation approach where rents in each local housing market are 
affected by two minimum wage measures, one summarizing the effect of residence 
minimum wage and a second one the effect of workplace minimum wage.
Second, we use a novel panel dataset on rents at the USPS ZIP code level and with a
monthly frequency from Zillow, the largest online rental marketplace in the US.
We coupled those data with an original dataset of binding minimum wages at the USPS
ZIP code level, and a commuting origin-destination matrix.
As a result, we are able to estimate the effect of minimum wage policies on rents 
using variation of hundreds of policy changes staggered across jurisdictions and 
months that generate plausibly exogenous variation of workplace and residence minimum 
wage levels.
Finally, we use our estimated model to predict what would be the short-run impact
of a federal minimum wage increase on the rental housing market.
Using spatially-disaggregated data from the IRS we document that large variations
in the pass-through rate of minimum wage policies to rents.

% MODEL. Summarize the key formal assumptions you will maintain in your analysis.

We start by laying out a partial equlibirum model of a ZIP code's rental market, 
where workers who reside there commute to work in some other ZIP code, potentially 
under a different minimum wage.
In the model workers demand square feet of livable space (housing) as a function
of their workplace and residence MW.
The model illustrates that, under the assumption that the workplace minimum increases
housing demand and the residence minimum lowers it, then one would observe that the 
effect of a change in minimum wage legislation would affect a ZIP code quite 
differently depending on whether it mostly changes the workplace or residence MW 
of its residents.
In particular, an increase in a workplace MW will cause rents to go up as well, 
whereas a residence MW hike will (conditional on workplace MW) lower rents.
We also show that, under the assumption that workers who work in different locations
have a similar elasticity of housing with respect to workplace minimum wage, the effect
of workplace MW on log rents can be summarized in a single measure, which we call the 
\textit{experienced log MW}.
We use this result to motivate our empirical model.

In a utility-maximization framework, the assumptions on the relationship between
housing demand and the different minimum wages can be rationalized as the consequence
of changing incomes and prices of local consumption.
Workplace MW hikes will tend to increase incomes, so if housing is a normal good 
demand should increase as well.
On the other hand, residence MW hikes will tend to increase prices of local consumption,
plausibly resulting in a decline of demand for housing.
The literature on price effects of the minimum wage appears to support this interpretation.

% DATA. Explain where you obtain your data and how you measure the concepts that 
% are central to your study.

We construct a dataset at the USPS ZIP code and monthly levels where we collect
data on rents and minimum wage levels.
Our main rent variable comes from Zillow, the largest online real estate platform 
in the U.S. \parencite{realestateagentpdx, investopedia}, and corresponds to the 
median rent price per square foot across Zillow listings in the given ZIP code-month 
cell of the category Single Family, Condominiums and Cooperative Houses (SFCC).
This is the most popular housing category in the U.S. \parencite{fernald2020americas}, 
and also the most populated series in the Zillow data.
We collect data on minimum wage changes from \textcite{VaghulZipperer2016} for the 
period from 1974 to 2016, which we update until January 2020 using mostly
\textcite{BerkeleyLaborCenter}.
We assign a binding MW to each USPS ZIP code by taking the maximum across all the
MWs that affect that ZIP code (city, state, and federal levels).%
\footnote{To do this we perform two matching between geographical units.
First, we match USPS ZIP codes to census ZIP code tabulation areas (ZCTA) using 
the crosswalk from UDSMapper (cite).
Second, we match ZCTAs to city and states using crosswalk from Missouri (cite).}
We use our MW data coupled with commuting origin-destination matrices obtained from 
the LEHD Origin-Destination Employment Statistics (LODES) database to construct the 
experienced log MW,%
\footnote{LEHD is short for Longitudinal Employer-Household Dynamics, which 
corresponds the source of the origin-destination data.}
defined, for each ZIP code, as a weighted average of the log statutory MW in each 
other ZIP code where its residents work, where the weights are the share of population 
that work in each one of them.

We also collect data 
on regional economic trends from the Quarterly Census of Employment and Wages (QCEW),
defined at the county-quarter levels; 
on local wages and business income from the IRS SOI (CITE) at the ZIP code and 
year levels,
and from the U.S. Census and ACS to construct measures of characteristics of a ZIP 
codes' residents.
We use these data both in our empirical estimations and to construct heterogeneity
analyses of the estimated effects.

% METHODS. Explain how you take your model to the data and how you overcome the 
% challenges you raised in paragraphs 3-4.

We pose an empirical model where log rents in a location depend on both the 
statutory and experienced log minimum wages, ZIP code and time period fixed effects,
and a set of time-varying controls.
As shocks to rents tend to be serially correlated over time, we estimate the model 
in first-differences.
This model recovers the true causal effect of the minimum wage assuming that, 
within a ZIP code, changes in each of our minimum wage variables are 
\textit{strictly exogenous} with respect to changes in the error term, conditional
on the other MW measure.
In other words, variation in the experienced (statutory) MW after partialling out 
the residence (statutory) MW cannot be correlated with innovations in the error 
term.
We argue 

We test our identifying assumption using leads and lags of each minimum wage variable.
Intuitively, assuming no anticipatory effects in the housing market, differential 
trends across treated and untreated ZIP codes would show up as non-zero pre-trends. 
Reassuringly, our models show no effects of future MW changes on current rents. 
Interestingly, we observe no effects of lags of the MW variables either.
We perform several other robustness checks.
First of all, we check for the presence of unobservables affecting both rents and MW 
changes in three ways:
(i) we show that is reasonable to think of supply as fixed around a short window of the 
MW changes;
(ii) we include controls that proxy for the state of the economy and are plausibly not 
affected by the changes in the MW; and (iii) we allow for richer heterogeneity across ZIP 
codes by controlling for ZIP-code-specific linear and quadratic trends.
These specifications should capture a wide range of potential confounders in our main 
regressions.
The fact that our baseline estimates are robust to the inclusion of these controls 
strengthens the case for the strict exogeneity assumption of MW changes.
Second, as rental listings may stay on Zillow for more than one month, one may worry 
about structural auto-correlation in the dependent variable which, if not accounted for, 
may bias our estimates.
For this reason, we estimate alternative models that include the lagged first difference 
of rents as controls, estimated via instrumental variables following 
\textcite{ArellanoBond1991} and related literature.
%At the cost of imposing a particular auto-correlation structure in the error term, this 
%specification has the advantage of allowing %for feedback effects from current rental 
% price shocks to future minimum wage changes \parencite{ArellanoHonore2001}. 
The estimates of this model are strikingly similar to our baseline results, rendering 
credibility to our econometric assumptions.

Finally, a drawback of the Zillow data is that it includes only a sample of U.S. ZIP codes. 
This fact brings about two concerns.
First, the composition of ZIP codes changes over time potentially introducing bias in our 
estimates. We tackle this issue by performing our main analysis with a constant set of 
units with valid data as of 2015.\footnote{Because these ZIP codes enter the sample at 
	different moments in time before 2015, our estimating panel is still unbalanced. 
	However, no ZIP code has missing values after it enters the panel.}
This strategy alleviates concerns arising from the changing composition of ZIP codes, but 
significantly lowers the number of observations used in the estimation. For this reason, 
we also estimate a model on the full sample of available ZIP codes, obtaining similar 
results.
Secondly, we worry that our estimated effect may be particular to our sample. In order 
to approximate the average treatment effect for the typical urban ZIP code, we re-weight 
our data to match average demographic characteristics of the top 100 Metropolitan 
Statistical Areas and re-estimate our main models. The effects not only survive this test 
but are also larger.

% FINDINGS. Describe the key findings. Make sure they connect clearly to the 
% motivation in paragraphs 1-2.

The \textit{static} difference-in-differences specification implies that a 10 percent 
increase in the MW leads to an average 0.26 percent increase in the rental price per 
square foot.
When expanding the model to account for \textit{dynamic} effects, we see that a 10 
percent increase in the minimum wage increases rents by 0.58 percent in a 6-months period.
Estimates of the long-run effect using the panel specification that includes the lagged 
first difference of rents as controls returns are very similar, but more precisely 
estimated.

In an effort to disentangle who are the ``winners and losers", we perform an 
heterogeneity analysis of the average treatment effect by allowing the coefficients to 
differ across quartiles of the distribution of ZIP code characteristics.
The results suggest that the effect of interest is indeed heterogeneous.
ZIP codes with lower shares of college graduates (bottom quartiles), as well as 
ZIP codes with higher shares of residents below 24 years old and African-American 
residents (top quartiles) experience a rent increase which is almost twice as large.
We then use LODES data to identify, for each ZIP code, the share of young, low-income 
workers and residents, respectively.
We argue that these characteristics proxy for MW workers fairly well.
Consistently, we show that ZIP codes with low concentration of MW earners as residents 
exhibit no significant effects.
On the other hand, we find that the effect is roughly constant across ZIP codes with 
different share of MW workers who work there.

Since it is reasonable to expect workplace and residence locations to differ, the use of 
statutory changes is probably not an accurate measure of where the income gains of MW 
changes are located.
In an extreme case, it could be that a ZIP code experiencing a MW increase actually has 
no MW earners as residents, but plenty as workers.
In that case, we expect to observe changes in rents in those nearby ZIP codes where the 
MW workers reside.
To accommodate for this, we use a measure of ``experienced minimum wage" that uses the 
average MW level a ZIP code's workers are exposed to in their workplace location.
Our results show a slightly higher impact of MW changes on rents when we replace our main 
explanatory variable: a 10 percent MW increase leads to a simultaneous 0.31 percent 
increase in rents.
Interestingly, when we include both the statutory and experienced MW measures in our 
model the effect of the latter gets even larger, whereas the former slightly decreases 
rents.
We interpret this result as consistent with the view that MW policies redistribute income 
geographically from ZIP codes where MW workers work to those where MW workers reside.
Given these income shocks, the supply and demand of local rentals determine the new 
equilibrium in the housing market.

In the final part of the paper, we gauge the magnitude of our estimates by computing the 
pass-through of the policy, which is defined as the ratio between the elasticity of rents 
to the MW and the elasticity of wages to the MW.
To do that, we perform two exercises.
First, we compute the elasticity of wages to the MW using county-quarter data from the 
QCEW. These estimates imply a pass-through of 19 percent.
Secondly, we use estimates of the elasticity of wages to the MW taken from the literature 
and combine it with our results. We compute in this case a 23 percent pass-through.
Because the rents elasticity is somewhat larger, estimates of the pass-through using the 
experienced MW are slightly larger as well.


Not only the evidence is scarce, but also inconclusive. For the US, both authors use the 
same data at the county-year level, but they find opposing results.
\citeauthor{Tidemann2018}'s (\citeyear{Tidemann2018}) estimates are 
negative, whereas the results of \textcite{Yamagishi2019} point towards a positive 
effect.%
\footnote{\textcite{Yamagishi2019} attributes this difference to different model 
specifications, and argues that with proper standard errors clustering the results 
in \textcite{Tidemann2018} are statistically insignificant.}
While we provide comparable estimates to these papers, we focus on the heterogeneity
of effects across local labor markets.

In a related paper, \textcite{AgarwalEtAl2019} shows that minimum wages decrease the 
probability of rental default, suggesting a strengthening of the local labor market. 
Lastly, in a closely related paper, \textcite{Hughes2020} leverages MW changes as a 
natural experiment to study the relationship between housing demand and policies 
affecting low-wage households. His approach aims at disentangling the income effect 
on housing demand from the ``general equilibrium'' price effect in the long-run, 
making our work related yet distinct as we focus on prices in the short-run. He shows 
that a 10 percent increase in MW decreases the rent-to-income ratio by 1.4 percent.

Our approach has several differences with respect to previous research on the topic.
Both \textcite{Tidemann2018} and \textcite{Yamagishi2019} for the U.S. exploit Fair 
Markets Rents data from the Department of Housing and Urban Development (HUD), which is 
available at the yearly level and aggregated at the geographical level of 
counties.\footnote{\textcite{Yamagishi2019}, updated in \textcite{Yamagishi2021}, also 
	uses data at the year-prefecture level for the 47 Japanese prefectures.}
An important advantage of our approach is that we use the exact timing of the MW change 
at the monthly level.
% When using variation arising from a yearly frequency some units are 
%``partially treated" which will tend to understate the magnitude of the effect. 
Another advantage is that we use data at the ZIP code- instead of the 
county-level.\footnote{As of 2019 there were 3,142 counties and 39,295 meaningful ZIP 
	codes in the U.S. We exclude military and unique business ZIP codes in this count, 
	since they are irrelevant for the housing market.} 
We illustrate the importance of having smaller units of analysis with the following 
example.
For a given county, suppose that (1) all low-skill jobs are in one particular ZIP code; 
and (2) low-skill households prefer to live near their jobs. Further assume that, 
following a MW change, employment effects are near zero.\footnote{This is consistent with 
findings in the literature \parencite[for a comprehensive review see][]{Dube2019}.} 
One should then expect demand for housing in the ZIP code with low-skill jobs to increase 
and demand for housing in the rest of the ZIP codes to go down.
If we focus on the effects of the MW increase on the county we might even find that the 
rents go down, when in fact the rents in the ZIP codes where the low skill jobs are 
located are increasing.
Indeed, \textcite{Tidemann2018} finds that a \$1 increase in the MW decreases the yearly 
average of the monthly rent by 1.5 percentage points.\footnote{As pointed out by 
	\textcite{Tidemann2018}, the sign of this effect implies that the labor demand for 
	low skilled workers is elastic. This is at odds with results of null employment 
	effects in the literature.} 

Using a more detailed geography also aids in the empirical estimation.
First of all, it means that we can exploit MW changes at any jurisdictional level, 
effectively increasing the number of events used for identification.
Secondly, it allows us to use more detailed controls, such as ZIP code fixed effects, 
local employment and wages for sectors likely unaffected by MW changes, and ZIP 
code-specific polynomial trends.
This is important because the dynamics of the rental market plausibly vary across ZIP 
codes within a county following trends at the very local level 
\parencite{AlmagroDominguez2019}.
Importantly, these local controls make the required identification assumptions more credible. 
Given that the identifying variation comes from within-ZIP codes, the determinants of 
these MW changes are unlikely to be related to the particular ZIP code and, thus, are 
less likely to be correlated to the unobservable determinants of rent dynamics there.

Beyond the contribution to the very recent literature on the effects of MW changes on 
rents, we contribute to several strands of the literature.
First, we contribute to the literature studying the effects of MW policies on the welfare 
of low-skill households \parencite[][among others]{DinardoEtAl1995, Lee1999, 
CardKrueger2000, Neumark2006, AutorEtAl2016, CegnizEtAl2019}.
Most of this literature has focused on employment effects. We contribute to this 
strand of literature by exploring the effects of minimum wage policies on the housing 
market.
To the best of our knowledge, we are the first to compute a local measure of experienced 
MW that takes into account commuting patterns.
Related research focused on the direct impact of MW changes on commuting patterns across 
state and city borders \parencite{Mckinnish2017, PerezPerez2018}.
We, on the other hand, use commuting patterns to keep track in a more precise way of the 
income redistribution generated by MW policies.

We also contribute to the literature that studies the response of local prices to MW 
changes. \textcite{LeungForthcoming} provides evidence of the MW impact on grocery store prices, 
suggesting that MW also affects products demand besides raising labor costs. In his 
analysis, he further shows that the effect is larger in poorer regions. Similarly to our 
study, this points out novel channels of heterogeneity in the pass-through that may have 
non-trivial distributional consequences.
%%%% SH: REVISE // ADD MORE PAPERS HERE

Our work also relates to the literature that studies the location decision of agents 
either based on income \parencite{Roback1982, Kennan2011, DesmetRossihansberg2013, 
PerezPerez2018, Monras2019} or on spatial rents and amenity differentials 
\parencite{Diamond2016, AlmagroDominguez2019, Couture2019}.
We contribute by adapting this framework to the case of MW changes as a means to 
rationalize through residential location sorting part of the observed reduced-form effect 
on rents. 
%%%% DGP: No entiendo bien este parrafo pero residential sorting no diria porque 
%%%% nuestros efectos son short-term y nuestro modelo solo habilita movimientos within zipcode.

The rest of the paper is organized as follows.
In Section \ref{sec:model} we introduce a motivating model of the rental market.
In Section \ref{sec:data} we present our data sources and show the characteristics 
of our estimating panel.
In Section \ref{sec:empirical_strategy} we explain our empirical strategy and
we discuss our identification assumptions.
In \ref{sec:results} we present our main results.
Section \ref{sec:discussion} discusses relevant policy implications, and
Section \ref{sec:conclusion} concludes.
