%%%%%%%%%%%%%%%%%%%%%%%%%%%%%%%%%%%%%%%%%%%%%%%%%%%%%%%%%%%%%%%%%%%%%%%%%%%%%%%%%
%%%%%                            INTRODUCTION                                %%%%
%%%%%%%%%%%%%%%%%%%%%%%%%%%%%%%%%%%%%%%%%%%%%%%%%%%%%%%%%%%%%%%%%%%%%%%%%%%%%%%%%

% MOTIVATION. After reading these paragraphs a reader in any field of economics
% should believe that if you answer your research question your paper will make 
% an important contribution.

In recent years, many US jurisdictions have introduced minimum wages above the 
federal level of \$7.25, resulting in minimum wage levels that vary 
substantially across space and even within metropolitan areas.
Minimum wage policies (hereafter MW) are \textit{place-based} in that they are 
tied to a location, and workers may live and work in locations under different 
statutory MW levels, suggesting the presence of spatially heterogeneous policy 
effects.
While most research on the effects of the MW has focused on employment and 
wages irrespective of residence and workplace location
\parencite[e.g.,][]{CardKrueger1994, CegnizEtAl2019},
a full account of the welfare effects of MW policies requires an understanding 
of how they affect different markets and how their effects spill over across 
neighborhoods.
In fact, while the MW appears to lower income inequality through the labor 
market \parencite{Lee1999, AutorEtAl2016},
its overall effect on income for low-wage workers may be smaller if there is 
a significant pass-through from MW changes to prices, including housing
\parencite{Macurdy2015}.

In this paper, we study the effect of MW policies on local rental housing 
markets estimating their effects on rents and the subsequent pass-through
to landlords.
Consider a new MW policy in some locations within a metropolitan area.
Because low-wage workers tend to reside in specific neighborhoods with access 
to the (now better-paying) low-wage jobs,
one would expect an increase in disposable income and a subsequent rise in demand 
for housing and rental prices in their residence instead of their workplace.
This effect, which operates through the MW at the workplace, 
will undermine the distributional objective of the policy.
Similarly, the MW hike will translate into higher prices of non-tradable 
consumption that use low-wage workers intensively as an input inside the 
jurisdiction that enacted the policy.
As a result, the demand for housing and rental prices will also be affected.
This effect, which operates through the MW at the residence, will have 
distributional consequences as well.
Commuting patterns thus become an essential ingredient to understand the 
heterogeneous effects of local MW policies.

We operationalize this insight constructing, for each USPS ZIP code (hereafter 
ZIP code) and month,
the \textit{workplace MW}, which we define as the log statutory MW where
the average worker of the ZIP code works.
We also define the \textit{residence MW}, which is just the log statutory MW in 
the same ZIP code.
Figure \ref{fig:map_mw_chicago_jul2019} shows the change in the two MW-based 
measures for the Chicago-Naperville-Elgin Core-Based Statistical Area 
(herefater CBSA) in July 2019, 
when the city of Chicago increased the MW from \$12 to \$13 and 
Cook County from \$11 to \$12.
We observe that, even though the statutory MW only changed in some locations in
the metropolitan area, the increase affected the workplace MW of most locations.

% CHALLENGES. These paragraphs explain why your research question has not already
% been answered, i.e., what are the central challenges a researcher must tackle to
% answer this question.

Estimating the effects of MW policies on rents is challenging for several 
reasons.
First, as opposed to assessing effects on pure labor market outcomes where jobs 
and wages are tied to the workplace, when evaluating the housing market it is 
crucial to account for the fact that people may reside and work under different 
MW levels.%
\footnote{However, several papers have highlighted the importance that studies
on the effect of the MW on employment account for potential spillovers that may
``contaminate'' the control group 
\parencite{Kuehn2016, JardimEtAl2022discontinuity}.}
%%
%% DGP: I am not sure I understand what you want to say in this footnote.
%% SH: Challenge is that there are spillovers in the housing market, which 
%%     are not as relevant in the labor market
%%     However, a few papers on the labor market say that spillovers do matter 
%%     there
%%
This is challenging because accounting for changes in the MW where residents
of a location work requires data on commuting patterns at the local level.
Second, successful identification of MW effects at the local level requires 
spatially disaggregated, high-frequency data on rents.
Using large geographies might result in null or even negative effects on average,
even if no one commutes outside this region and the actual effect (of the
workplace MW) on some local housing markets is positive.%
\footnote{Rents in neighborhoods where low-wage workers live are likely to 
increase, whereas elsewhere they are likely not to change or decrease, 
as those residents ``pay'' for the higher MW through higher prices and lower 
profits.}
Even if the effects in such geographies may be of interest, they may mask 
substantial heterogeneity and therefore miss the fact that some people pay 
higher rents due to the policy change.
Similarly, MW policies change at the monthly level, so using variation at a lower
frequency (such as yearly) will not allow a clean identification using the exact 
month of the MW change.
Finally, the effects of the MW on rents may operate through different channels,
such as prices of consumption, income, or changes in migration and commuting.
Studying the contribution of each channel is important to evaluate the incidence 
of the policy over different locations and time horizons.

% THIS PAPER. This paragraph states in a nutshell what the paper accomplishes and how.

We introduce several innovations to tackle these challenges.
First, we develop a tractable model that allows the MW to spill over 
across local housing markets through commuting.
According to the model, rents in each local housing market are affected by 
two MW-based measures:
the residence MW and the workplace MW.
The model maps these measures to the effect of the MW via 
(i) consumption prices in the same location and
(ii) income generated across locations.
Second, we use a novel panel dataset on rents at the ZIP code level and with 
a monthly frequency from Zillow, the largest online rental marketplace in the US.
We couple those data with an original panel dataset of statutory MWs 
at the ZIP code level, and commuting origin-destination matrices constructed
from administrative records.
As a result, we are able to estimate the effect of MW policies on rents using 
hundreds of policy changes staggered across jurisdictions and months that 
generate plausibly exogenous variation of workplace and residence MW levels.
% We show that our results are robust to using commuting data from different years
% and for different groups of workers, suggesting that commuting patterns are 
% stable, at least in the time window around events we consider, and thus unlikely
% to affect the results.
%% 
%% SH: Commented out the above for brevity
%% 
We use our estimated model to evaluate the impact of two MW policies:
a federal MW increase from \$7.25 to \$9 and
a MW increase from \$13 to \$14 in the city of Chicago.
Coupling our estimates with ZIP code-level income data, we estimate the share of 
each dollar of extra income (generated by the MW) that accrues to landlords
both summing all affected areas and in each particular location.
We discuss the implications of our results for assessing the distributional 
impact of MW policies.

% MODEL. Summarize the key formal assumptions you will maintain in your analysis.

We start by laying out a partial equilibrium model of a ZIP code's rental market,
which is embedded in a larger geography.
The model allows residents of this ZIP code to commute to other ZIP codes to 
work, potentially under a different MW policy.
In the model, workers demand square feet of housing as a function of non-tradable 
prices and income which, in turn, depend on the MW levels workers face at residence 
and workplace locations, respectively.
The model imposes fixed commuting patterns alongside fully flexible prices.%
\footnote{We also assume fixed housing quality.}
This assumption, which is motivated by our empirical setting, is also consistent 
with the literature.%
\footnote{Our data varies at the monthly level. 
Thus, we are assuming that the first order effects of MW changes do not affect 
where agents live and work in a window of a few months around the events.
Relatedly, \textcite{PerezPerez2021} finds small effects of the MW on commuting
in a time horizon of several years.}
However, we note that validity of our results in the long-run will depend on the
degree of migration as a result of the policy.
The model illustrates that, if housing is a normal good and is complementary 
with non-tradable consumption, the effect of a change in MW legislation 
would be heterogeneous across ZIP codes depending on whether it mostly changes 
the MW of its residents at their residence or at their workplace locations.%
\footnote{In particular, MW increases in workplace locations will cause rents to go up,
whereas increases at residence will (conditional on a constant workplace MW)
will lower rents.}
The model implies that the effect of changes in the MW at workplaces on log 
rents can be summarized in a single measure: a ZIP code's workplace MW.
This measure is defined as the weighted average of log minimum wage levels 
across a ZIP code's workplaces, using commuting shares as weights.
The effect of changes in the MW at the residence are summarized by the residence 
MW, defined as the log of the statutory MW in the location.
We use this result to motivate our empirical model.

% DATA. Explain where you obtain your data and how you measure the concepts that 
% are central to your study.

We construct a panel at the ZIP code and monthly levels with rental prices, 
statutory MW levels, and our MW-based measures from January 2015 to December 2019.
The main rent variable comes from Zillow, and corresponds to the median 
rental price per square foot across Zillow listings in the given ZIP 
code-month cell of the category Single Family Houses and Condominiums and 
Cooperative units (SFCC).
We show that low-wage households are more likely to rent,
tend to reside in this type of housing units,
and that rents per square foot are surprisingly constant across the household
income distribution.
These facts suggest that any effects of the MW can plausibly be captured 
in the Zillow data.
We collect data on MW levels from \textcite{VaghulZipperer2016} and 
\textcite{BerkeleyLaborCenter}, and commuting origin-destination matrices at the
census block level from the Longitudinal Employer-Household Dynamics panel 
\parencite{CensusLODES}.
We use the commuting data---along with a novel correspondence table between census
blocks and USPS ZIP codes---to construct our workplace MW measure.

% METHODS. Explain how you take your model to the data and how you overcome the 
% challenges you raised in paragraphs 3-4.

Guided by the theoretical model, we pose an empirical model where log rents in 
a location depend linearly on
the residence MW, % defined as the log of the statutory MW at that location
the workplace MW, % defined as the weighted average of log statutory MW in other ZIP codes where weights are commuting shares
ZIP code and time period fixed effects, and 
time-varying controls.
This model recovers the true causal effect of the MW assuming that, 
within a ZIP code, changes in each of our MW variables are 
strictly exogenous with respect to changes in the error term, 
conditional on the other MW measure and the controls.
To mitigate concerns of changes in the composition of our sample of ZIP codes, 
in our baseline analysis we use a balanced panel.%
\footnote{We use all ZIP codes with valid rents data as of January 2015.}
In an appendix we discuss a general potential outcomes framework following
\textcite{CallawayEtAl2021}.
We show that, under the assumptions of \textit{parallel trends} and 
\textit{no selection on gains}, 
the effects of the residence and workplace MW are identified from the 
conditional slope of log rents with respect to each MW measure.
We discuss evidence in favor of these assumptions, both of which are satisfied 
by the linear functional form used as baseline.

% FINDINGS. Describe the key findings. Make sure they connect clearly to the 
% motivation in paragraphs 1-2.

Our preferred specification implies that 
a 10 percent increase in the workplace MW (holding constant the residence MW) 
\textit{increases} rents by $\BothBetaBaseTen$ percent 
(SE=$\BothBetaBaseTenSE$).
A 10 percent increase in the residence MW (holding constant the workplace MW) 
\textit{decreases} rents by $\BothGammaBaseTenAbs$ percent 
(SE=$\BothGammaBaseTenSE$). 
As a result, if both measures increase simultaneously by 10 percent then 
rents would increase by $\BothSumBaseTen$ percent 
(SE=$\BothSumBaseTenSE$).
These results imply that, holding fixed the commuting shares, MW 
changes spill over spatially through commuting, affecting local housing markets 
in places beyond the boundary of the jurisdiction that instituted the policy.
We find that a naive model estimated only on the same-location MW would yield a 
coefficient similar to the sum of the coefficients on our workplace and 
residence MW measures.
However, this model would not account for MW spillovers to other locations.
Using a rough approximation to the share of MW workers in each ZIP code, we show 
that the elasticities to the residence and workplace MW are larger in locations 
with more MW workers.
We also show that using panels at the county by month and ZIP code by year levels 
one fails to detect effects, highlighting the importance of the granularity of 
our ZIP code by month data.

%% ROBUSTNESS

We conduct several robustness checks to assess the validity of our results.
First, we provide support for our identification assumptions by
(1) testing for ``pre-trends'' by estimating an extended model that includes 
leads and lags of the MW measures, and 
(2) developing a non-parametric analysis of the relationship between log rents
and the MW measures.
We find that future MW changes do not predict rents, and that the conditional
relationship of log rents with respect to each MW measure is nearly linear.
Second, we show robustness of our results to different sets of controls,
alternative definitions of commuting shares, alternative samples,
and reweighing observations to match demographics of the population of
urban ZIP codes.
Third, we estimate models for different categories of housing rentals and
find overall consistent results.
Finally, in the appendix we estimate two alternative models.
We construct a ``stacked'' regression model, similar to \textcite{CegnizEtAl2019},
that compares ZIP codes within metropolitan areas where some but not all 
experienced a change in the statutory MW.
This should alleviate concerns that our estimates stem from undesired 
comparisons in difference-in-differences models with staggered treatment timing, 
as highlighted by recent literature 
\parencite{deChaisemartinEtAl2022,RothEtAl2022}.
The second alternative model includes the lagged first difference of rents as 
a control, and is estimated via instrumental variables following 
\textcite{ArellanoBond1991}.
We find consistent results in both exercises.

%% COUNTERFACTUAL

In the final part of the paper, we construct a counterfactual exercise to 
capture the incidence of MW policies on landlords.
We compute the share pocketed by landlords in each ZIP code, and also
compute the total incidence summing across locations.
We posit two counterfactual MW policies in January 2020, keeping all other
MW policies in their 2019 levels.
In the first scenario, we change the federal MW from \$7.25 to \$9.
In the second exercise, we posit an increase in the Chicago City MW 
from \$13 to \$14.
We estimate that landlords capture $\totIncidenceCentsFedNine$ cents of each 
dollar across locations in affected CBSAs in the former, and 
$\totIncidenceCentsChiFourteen$ cents of each dollar across locations in the 
Chicago-Naperville-Elgin CBSA in the latter.
We find systematic spatial variation in incidence,
with the share pocketed usually being larger in locations that experience an
increase in the workplace MW but not in the residence MW.

Our results imply that a share of the extra income that low-wage workers
receive due to the policy is actually captured by landlords.
Viewed through the lens of our theoretical model,
the mechanism behind this is a rise in housing demand in a scenario of a 
finite housing supply elasticity.
In the context of a general equilibrium model, \textcite{KlineMoretti2014} argue
that this mechanism causes place-based policies to be welfare inefficient.
While studying the full welfare effects of MW policies is beyond the scope of 
the paper, our results imply that ignoring rent changes will lead to an 
overstatement of the gains of low-wage workers following a MW increase.

%% LIMITATIONS

Our analysis has some important limitations.
A first limitation is that our derivation of the residence and workplace MW
measures as reflecting changes in non-tradable prices and income relies on 
several constant-elasticity assumptions.
We discuss the plausibility of the required assumptions in the body of the paper.
A second limitation is that we do not account for changes in migration and 
commuting.
While we maintain that this is a plausible assumption to obtain our empirical 
estimates, the long-run incidence of the policy on landlords may differ from our 
computations if the residence and workplace locations of low-wage households 
respond strongly to the MW.
A final limitation is that our exercises do not capture the full welfare 
effect of MW policies.
A full account of the long-run welfare effect of the sub-state MW policies in 
the 2010s requires specifying a general equilibrium model that accounts for 
changes in consumption prices, changes in workplace and residence locations
of workers, and potential employment effects.
However, as low-wage households are more likely to rent and thus to be 
negatively affected by rent changes, our analysis suggest that such computation 
should take into account the homeownership status of households.

%% LITERATURE

Our findings contribute to the literature studying the effects of MW policies 
on the housing market.
To our knowledge, the only papers that estimate the effect of the MW on rents 
in the same location are \textcite{Tidemann2018} and \citeauthor{Yamagishi2019} 
(\cite*{Yamagishi2019}, \cite*{Yamagishi2021}).%
\footnote{In the working paper version \parencite{Yamagishi2019}, the author 
explores this question using data from both the US and Japan.
In the published version \parencite{Yamagishi2021}, he excludes the analysis of 
the US case.}
\textcite{AgarwalEtAl2021} show that MW increases lower the probability of 
rental default, and present estimates of the effect of the MW on rents using 
transactions data between 2000 and 2009.
Our paper also relates to \textcite{Hughes2020}, who studies the effect of 
MW policies on rent-to-income ratios.
The main difference of our paper with this work is that we differentiate 
between residence and workplace MW levels, incorporating spillovers across 
regions.
A second difference is the research design: we use high-frequency,
high-resolution data that allows clean identification at the level of the local 
housing market.

We also contribute to the understanding of place-based policies and the spatial 
transmission of shocks.
\textcite{KlineMoretti2014} argue that place-based policies may result in 
welfare losses due to finite housing supply elasticites.
\textcite{HsiehMoretti2019} quantify the costs of housing constraints in the US.
In line with this insight, we show that landlords may benefit from a place-based 
MW policy.
\textcite{AllenEtAl2020} estimate the within-city transmission of expenditure 
shocks in Barcelona.
We, on the other hand, study the within-city transmission of MW shocks.

More broadly, our paper relates to the large literature estimating the effects
of MW policies on employment
(see \cite{Dube2019} and \cite{NeumarkShirley2021} for recent reviews of the 
literature), 
the distribution of income \parencite[e.g.,][]{Lee1999, AutorEtAl2016, 
Dube2019Income}, 
and the overall welfare effect of the MW \parencite{AhlfeldtEtAl2022,
BergerHerkenhoffMongey2022}.%
\footnote{Our paper is also related to work studying 
the effects of local MW policies 
\parencite[e.g.,][]{DubeLindner2021, JardimEtAl2022seattle}, 
the effect of MW policies on commuting and migration 
\parencite[e.g.,]{Cadena2014, Monras2019, PerezPerez2021}, 
and prices of consumption goods 
\parencite[e.g.,]{AllegrettoReich2018, Leung2021}.}
Our contributions are to incorporate spillovers across locations 
\parencite[as in the recent work by][]{JardimEtAl2022discontinuity} and to show 
that rent increases erode some income gains of low-wage workers.
We also contribute by developing a novel panel dataset of MW levels at the 
ZIP code level for the entire US.

Finally, our paper relates to work in econometrics that focuses on spillover 
effects across units,
both in the context of MW policies 
\parencite{Kuehn2016, JardimEtAl2022discontinuity}, 
and more generally of any policy that spills over spatially
\parencite{DelgadoFlorax2015, Butts2021}.
Our approach is similar to \textcite{GiroudMueller2019}: we specify a model for 
spillovers across units that allows us to estimate rich effect patterns of the 
MW on rents.

The rest of the paper is organized as follows.
Section \ref{sec:model} introduces a motivating model of the rental market.
In Section \ref{sec:data} we discuss the empirical relationship between income 
and housing and present our estimation data.
In Section \ref{sec:empirical_strategy} we discuss our empirical strategy and
identification assumptions.
In Section \ref{sec:results} we present our estimation results.
Section \ref{sec:counterfactual} discusses counterfactual MW policies, and
Section \ref{sec:conclusion} concludes.
