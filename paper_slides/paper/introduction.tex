%%%%%%%%%%%%%%%%%%%%%%%%%%%%%%%%%%%%%%%%%%%%%%%%%%%%%%%%%%%%%%%%%%%%%%%%%%%%%%%%%
%%%%%                            INTRODUCTION                                %%%%
%%%%%%%%%%%%%%%%%%%%%%%%%%%%%%%%%%%%%%%%%%%%%%%%%%%%%%%%%%%%%%%%%%%%%%%%%%%%%%%%%

% MOTIVATION. After reading these paragraphs a reader in any field of economics
% should believe that if you answer your research question your paper will make 
% an important contribution.

In recent years, many US jurisdictions have introduced minimum wages above the 
federal level of \$7.25, resulting in minimum wage levels that vary 
substantially within metropolitan areas.
Minimum wage policies (hereafter MW) are \textit{place-based} in that they are 
tied to a location, and workers may live and work in locations under different 
MW levels, 
implying differential effects of changes in these policies across space.
While most research on the effects of the MW has focused on employment and 
wages irrespective of residence and work location
\parencite[e.g.,][]{CardKrueger1994, CegnizEtAl2019},
a full account of the welfare effects of the MW requires an understanding of 
how it affects different markets and how its effects spill over across 
neighborhoods.
In fact, while the MW appears to lower income inequality through the labor 
market \parencite{Lee1999, AutorEtAl2016},
its effect on disposable income for those on the left tail of the income 
distribution may be smaller if significant pass-throughs from MW changes to 
prices of consumption exist, including housing.

In this paper we take a step in this direction by studying the short-run effect 
of MW policies on local rental housing markets.
Consider a new MW policy in some locations within a metropolitan area.
Because low-wage workers tend to reside in specific neighborhoods with access 
to those (now better-paying) low-wage jobs,
one would expect an increase in disposable income there.
This, in turn, would lead to a higher demand for housing and thus 
a rise in rental prices in their residence neighborhoods,
undermining (at least partially) the distributional objective the policy.
Similarly, the MW hike will translate into higher prices of non-tradable goods 
that use low-wage workers intensively as an input which, in turn, 
would also affect the demand for housing and rental prices.
Commuting patterns thus become an essential ingredient to understand the 
heterogeneous effects of local MW policies on the housing market when there 
is a divergence in the workplace and residence locations of workers.
Figure \ref{fig:map_shares_chicago_2018} illustrates by showing the share 
of low-income workers and residents in each ZIP code in the 
Chicago-Naperville-Elgin CBSA.

% CHALLENGES. These paragraphs explain why your research question has not already
% been answered, i.e., what are the central challenges a researcher must tackle to
% answer this question.

There is little research attempting to estimate the causal effect of minimum 
wage policies on the housing market and none accounting for spatial spillovers.
To the best of our knowledge, the only papers that estimate the causal effect of 
minimum wages on rents in the same location are \textcite{Tidemann2018} and 
\citeauthor{Yamagishi2019} (\citeyear{Yamagishi2019, Yamagishi2021}).%
\footnote{In the working paper version \parencite{Yamagishi2019}, the author 
explores this question using data from both the US and Japan.
In the published version \parencite{Yamagishi2021}, he excludes the analysis of 
the US case.}
Estimating empirically the effects of MW policies on rents is challenging for 
several reasons. 
First, as opposed to assessing effects on pure labor market outcomes where jobs 
and wages are tied to the workplace, when evaluating the housing market it is 
crucial to account for the fact that people may reside and work under different 
MW levels.%
\footnote{However, several papers have highlighted the importance that studies
on the effect of the MW on employment account for potential spillovers that may
``contaminate'' the control group \parencite{Kuehn2016, Huang2020}.}
%%
%% DGP: I am not sure I understand what you want to say in this footnote.
%% SH: Challenge is that there are spillovers in the housing market, which 
%%     are not as relevant in the labor market
%%     However, a few papers on the labor market say that spillovers do matter 
%%     there
%%
This is challenging because accounting for changes in the MW where residents
of a location work requires data on commuting patterns at the local level.
Second, estimation at the local level requires spatially dissagregated data on 
rents.
Using large geographies might result in null or even negative effects on average,
even if no one commutes outside of this region and the actual effect (of workplace
MW) on some local housing markets is positive.%
\footnote{Rents in neighborhoods where low-wage workers live are likely to 
increase, whereas elsewhere they are likely not to change or even decrease, 
as those residents ``pay'' for the higher MW through higher prices and lower 
profits.}
Even if the effects in the large geographies may be of interest, they will mask 
substantial heterogeneity and therefore miss the fact that some people will be 
paying higher rents due to the policy change.
In addition, as MW changes are unlikely to be set considering the dynamics of 
local rental markets, when using small geographic units the exogeneity assumptions 
required for identification appear more plausible.

% THIS PAPER. This paragraph states in a nutshell what the paper accomplishes and how.

We introduce several innovations to tackle these challenges.
First, we theoretically recognize that minimum wage policies will spill over across
housing markets through commuting.
We device a new model-based estimation approach where rents in each local housing 
market are affected by two MW-based measures, one summarizing the effect of 
residence MW and a second one the effect of workplace MW.
Second, we use a novel panel dataset on rents at the USPS ZIP code level and with 
a monthly frequency from Zillow, the largest online rental marketplace in the US.
We couple those data with an original dataset of binding minimum wages at the
ZIP code level, and commuting origin-destination matrices from \textcite{LODES}.
As a result, we are able to estimate the effect of MW policies on rents using 
variation of hundreds of policy changes staggered across small jurisdictions and 
months that generate plausibly exogenous variation of workplace and residence MW
levels.
%% DGP: Maybe we should add a phrase about all we do to make sure that things
%% DGP are nicely identified. Controls, pretrends tests, alternative specifications,
%% alternative panels, changing commuting shares, etc.
%%
%% SH: We do. We mention these exercises in the ROBUSTNESS part
%%     I'd prefer to keep this paragraph concise

We use our estimated model to evaluate the short-run impact of a federal MW 
increase from \$7.25 to \$9 on rents.
Coupling our estimates with ZIP code-level IRS data, we estimate the share on 
each dollar of extra income (caused by the MW) that accrues to landlords in each 
ZIP code.
We discuss the implications of our results for assessing the distributional 
impact of MW policies.
%% DGP: Following Jesse's advise, we should probably make a statement about what
%% those implications are, at least vagely if not quantitatively.
%%
%% SH: I think that should go in the FINDINGS part of the intro?
%%

% MODEL. Summarize the key formal assumptions you will maintain in your analysis.

We start by laying out a partial equilibrium model of a ZIP code's rental market,
which is embedded in a larger geography.
We allow residents of this ZIP code to commute to other ZIP codes to work, 
potentially under a different MW policy.
In the model workers demand square feet of housing as a function of local prices 
and income, which in turn depend on the MW levels workers face at residence and 
workplace locations, respectively.
This short-run model imposes fixed commuting patterns and fully flexible 
prices.%
\footnote{This assumption is motivated by our dataset, which varies at the 
monthly level. Thus, we are assuming that the first order effects of 
MW changes don't affect where agents live and work.
We also believe that this assumption is consistent with the recent literature
finding small effects of MW changes on employment over longer time horizons
\parencite[see][for a review]{Dube2019}.
Relatedly, \textcite{PerezPerez2021} finds small elasticities of commuting to MW 
policies in a time horizon of several years.}
Motivated by the evidence of the effect of MW policies on 
income \parencite{Dube2019Income, CegnizEtAl2019} and 
prices \parencite{AllegrettoReich2018, Leung2021},
we assume that MW hikes at workplace weakly increase disposable income and MW 
hikes at residence increase local prices.
The model illustrates that, if housing is a normal good and is complementary 
with non-tradable consumption, then the effect of a change in MW legislation 
would be heterogeneous across ZIP codes depending on whether it mostly changes 
the MW of its residents at their residence or workplace locations.
%% DGP: We removed the phrase about housing being complement with non-tradables 
%% from the model section. Should we revamp it?
%% SH: Why not. Feel free to do so.
In particular, we show that a MW increase in some workplace will cause rents to 
go up, whereas an increase in the residence will (conditional on a constant 
workplace MW) lower rents.
We also show that, under some homogeneity assumptions on the effect of MWs 
through income, the effect of changes in MW at workplaces on log rents can be 
summarized in a single measure, which we call a ZIP code's workplace MW.
This measure is defined as the weighted average of log minimum wage levels 
across a ZIP code's workplaces, using commuting shares as weights.
We use this result to motivate our empirical model.

% DATA. Explain where you obtain your data and how you measure the concepts that 
% are central to your study.

We construct a panel at the USPS ZIP code and monthly levels with rental prices 
and binding MW levels.
Our main rent variable comes from Zillow, the largest online real estate 
platform in the US \parencite{realestateagentpdx, investopedia}, and corresponds 
to the median rent price per square foot across Zillow listings in the given ZIP 
code-month cell of the category Single Family, Condominiums and Cooperative 
Houses (SFCC).
This is the most popular housing category in the US \parencite{Fernald2020}, 
and also the most populated series in the Zillow data.
We collect data on MW changes from \textcite{VaghulZipperer2016} for the period 
2010--2016, which we update until January 2020 using data from 
\textcite{BerkeleyLaborCenter} and cross-validating with official sources.
We assign a binding MW to each ZIP code by taking the maximum across all the
MWs that affect that ZIP code (city, state, and federal levels).
We use our MW data coupled with commuting origin-destination matrices obtained 
from the Longitudinal Employer-Household Dynamics Origin-Destination Employment 
Statistics \parencite[LODES;][]{LODES} database.
These data provide workplace locations for the residents of all the US census 
blocks, and we use it to construct the workplace MW measure.

We also collect data on 
county-level economic indicators from the Quarterly Census of Employment and 
Wages \textcite[QCEW;][]{QCEW}; 
wage and business income at the ZIP code-year levels from \textcite{IRS}; and 
ZIP code level sociodemographic characteristics from the \textcite{ACS}.

% METHODS. Explain how you take your model to the data and how you overcome the 
% challenges you raised in paragraphs 3-4.

Guided by the theoretical model, we pose an empirical model where log rents in 
a location depend linearly on
(1) residence MW---the log of the statutory MW at that location---,
(2) workplace MW---the weighted average of log statutory MW in other ZIP codes,
where weights are commuting shares---,
(3)  ZIP code and time period fixed effects,
and 
(4) time-varying controls.
As shocks to rents are expected to be serially correlated over time within ZIP 
codes, we estimate the model in first-differences.
As we discuss in the body of the paper, this model recovers the true causal 
effect of the MW assuming that, within a ZIP code, changes in each of our MW 
variables are \textit{strictly exogenous} with respect to changes in the error 
term, conditional on the other MW measure and the controls.
To mitigate concerns of changes in the composition of our sample of ZIP codes 
while keeping as many of them as possible, in our baseline analysis we use a 
partially balanced panel.%
\footnote{We use all ZIP codes with valid rents data as of July 2015.}
Using an argument akin to the recent difference-in-differences literature
\parencite[e.g.,][]{CallawayEtAl2021}, 
in an appendix we unpack our identification argument beyond the residence and
workplace MW. 
We state clearly the conditions required on the commuting shares and the 
unobservable determinants of rents under a MW policy that increases the MW in 
a subset of ZIP codes only.

% FINDINGS. Describe the key findings. Make sure they connect clearly to the 
% motivation in paragraphs 1-2.

Our preferred specification implies that a 10 percent increase in the workplace
MW increases rents by 0.58 percent (SE=0.28).
A 10 percent increase in the residence MW decrease rents by 0.24 percent
(SE=0.18). 
As a result, if both measures increase simultaneously by 10 percent then 
rents would increase by 0.34 percent instead (SE=0.15).
%% SH:
%%    Revise these numbers
%% DGP: Maybe this way of putting it is more clear also for the appendix. 
%% SH: Can you clarify what you mean here?
These results are clear evidence that, holding fixed the commuting shares, MW 
changes spill over spatially through commuting, affecting local housing markets 
in places beyond the boundary of the jurisdiction that instituted the policy.
We estimate our empirical model allowing the commuting share to vary and find 
similar results.
We find that a naive model estimated only on the same-location MW would yield a 
similar coefficient to the sum of our workplace and residence coefficients.
However, this model would predict changes in rents only at residence locations 
and would not account for MW spillovers, which are central to understanding the 
distributional consequences of the rich pattern of changes in rent gradients
generated by this policy.

Heterogeneity analyses show that ...
%% DGP: Reminder comment to complete this paragraph.

%% ROBUSTNESS

We conduct several robustness checks to test the validity of our results.
First, we test our identifying assumption estimating our model adding leads and 
lags of each MW variable.
Reassuringly, we find no effects of future MW changes on current rents.
We also show the robustness of our results by estimating our model with 
different sets of controls that should account for a variety of confounders, 
such as the state of the local economy or local heterogeneity in 
rental dynamics.
Second, in an appendix we show that our results are similar in a ``stacked 
regression'' model that compares ZIP codes within metropolitan areas where some 
but not all experienced a change in the statutory MW.
Third, as rental listings may stay on Zillow for more than a month, one may worry 
about structural auto-correlation in the dependent variable which, if not 
accounted for, may bias our estimates.
In an appendix, we present an alternative model that includes the lagged first 
difference of rents as a control and estimate it via instrumental variables
following \textcite{ArellanoBond1991} and \textcite{MeerWest2016}.
%At the cost of imposing a particular auto-correlation structure in the error term, this 
%specification has the advantage of allowing %for feedback effects from current rental 
% price shocks to future minimum wage changes \parencite{ArellanoHonore2001}. 
Both alternative estimation procedures yield results that are very similar to our 
baseline.
Finally, we estimate variations of our model under a fixed composition of ZIP 
codes; and using an unbalanced panel with full set of ZIP codes and 
``cohort-by-time'' fixed effects.
Our results are robust to these exercises.
Trying to approximate the average treatment effect beyond our selected sample of
ZIP codes we estimate our model using weights constructed to match match key 
moments of the urban distribution of ZIP codes, finding similar results.

%% COUNTERFACTUAL

In the final part of the paper, we develop a simple extension to our baseline 
model to estimate the ZIP code-specific share on each dollar that accrues to 
landlords following a MW increase.
This parameter depends on the change in the total wage bill of a ZIP code, and 
the share of a ZIP code's total income spent in housing.
We posit a model for the wage bill similar to our baseline, and estimate an 
elasticity of wages to the minimum wage that is in line with the literature
\parencite[e.g.,][]{CegnizEtAl2019}.
Due to data constraints, we assume a range of values for the share of housing
expenditure at the ZIP code level.
%%
%% SH: Can you cite a paper justifying the assumed share of expenditure?
%% DGP: See the following table from the consumer expenditure survey
%% https://www.bls.gov/cex/tables/calendar-year/mean/cu-all-multi-year-2013-2020.pdf
%% SH: That's really helpful. We should add a cite to some report like this.
%%
We focus on studying the consequences of a counterfactual increase in the federal 
MW from \$7.25 to \$9 on January 2020.
We find large variation in the estimated resulting rent changes across ZIP codes.
We estimate that, in ZIP codes where both the residence and workplace MWs increase
due to the policy, landlords pocket between 5 and 8 cents on the dollar.
%% DGP: Does it matter how much is the increase of one measure ralative to the other? 
However, in ZIP codes where the residence MW does not change, the share pocketed
by landlords is higher. 
%% DGP: Should we provide an intuition of why?
These results imply that a share of the extra income of the low-wage workers
due to the policy is actually captured by landlords due to increased housing 
demand and a finite elasticity of housing supply.%
\footnote{This result is consistent with the mechanism proposed by 
\textcite{KlineMoretti2014}, whereby place-based policies generate welfare 
losses due to inneficiencies in the housing market.
However, in our model we do not allow for migration responses that may mitigate 
this coefficient in the medium run.}
Ignoring this fact will lead to an overstatemnt of the gains of low-wage workers
following a MW increase.
%% DGP: Should we add a footnote saying that characterising the full welfare effects
%% is beyond our scope and that it would require a full blown general eq model?

%% LITERATURE

This paper is related to several strands of literature.
First, our paper relates to the large literature estimating the effects of 
minimum wage policies on labor market outcomes.
Starting with \citeauthor{CardKrueger1994}'s (\citeyear{CardKrueger1994}) 
classical study, many papers have explored the effect of these policies on 
employment \parencite[some recent examples include][]{MeerWest2016,
CegnizEtAl2019}.%
\footnote{See \textcite{Neumark2006} for an earlier review of this literature, 
and \textcite{Dube2019, NeumarkShirley2021} for more recent reviews.}
Similarly, several papers study the consequences of minimum wage policies on 
income inequality \parencite{Lee1999, AutorEtAl2016}.
There is also a growing literature studying the effects of local minimum wage 
changes \parencite{DubeLindner2021}.
We contribute to this literature by focusing on a relatively less studied 
channel through which minimum wage policies at subnational jurisdictional 
levels may affect welfare: the housing market.

Second, this paper is related to the literature studying the effects of MW 
policies on behaviors beyond the labor market.
We already mentioned the scant literature estimating the effects of MW policies
on rental housing prices \parencite{Tidemann2018, Yamagishi2021}.
We innovate in several ways relative to these papers.
First, while these papers estimate the effect of same-location MW on rents, we 
differentiate between residence and workplace MW levels, fully incorporating
spillovers across regions.
Second, we use data at a more detailed geography and higher frequency.%
\footnote{Both \textcite{Tidemann2018} and \textcite{Yamagishi2019} for the US 
exploit Fair Markets Rents data from the US Department of Housing and Urban 
Development (HUD), which is available at the yearly level and aggregated at the 
geographical level of counties.}
Both of these facts enrich our understanding of the estimated effects and make 
the required identification assumptions more plausible.
Our paper also relates to \textcite{Hughes2020} who uses a triple difference 
strategy to study the effect of MW policies on rent-to-income ratios. Like us, 
the author explicitly mentions disentangling general equilibrium effects from 
effects on rental markets as a motivation for his approach.
%% DGP: Can we get rent to income ratios as data for housing expenditure shares? 
\footnote{Another related paper is \textcite{AgarwalEtAl2021} who show that MW 
increases lower the probability of rental default.}
Our work is also related to work studying the effects of MW policies on 
commuting and migration \parencite{Cadena2014, Monras2019, PerezPerez2021}, and 
prices of consumption goods \parencite{AllegrettoReich2018, Leung2021}

Third, we also contribute to the literature on place-based policies.
\parencite{KlineMoretti2014} presents a review of place-based policies, and 
argues that these policies result in inefficiencies due to finite housing supply 
elasticites in different locations.
Relatedly, \textcite{HsiehMoretti2019} quantify the aggregate cost of housing 
constraints.
In line with this insight, we show in our counterfactual analysis that landlords
may benefit from a MW increase, eroding some of the rise in low-wage workers' 
income generated by the policy. 

Finally, our paper relates to the literature on the econometric issues arising 
from the presence of spillover effects across units,
both in the context of minimum wage policies \parencite{Kuehn2016, Huang2020}, 
and more generally of any policy that spills over spatially
\parencite{DelgadoFlorax2015, Butts2021}.
In our setting we exploit knowledge of commuting patterns to specify the 
exposure of each unit to treatment in other units.
Under this functional form assumption we are able to account for spatial 
spillovers of MW policies on rents, allowing us to estimate rich effect patterns 
on the rent gradient.

The rest of the paper is organized as follows.
In Section \ref{sec:model} we introduce a motivating model of the rental market.
In Section \ref{sec:data} we present our data.
In Section \ref{sec:empirical_strategy} we discuss our empirical strategy and
we discuss our identification assumptions.
In Section \ref{sec:results} we present our results.
Section \ref{sec:counterfactual} discusses a counterfactual minimum wage policy, and
Section \ref{sec:conclusion} concludes.
