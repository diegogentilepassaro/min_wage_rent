%%%%%%%%%%%%%%%%%%%%%%%%%%%%%%%%%%%%%%%%%%%%%%%%%%%%%%%%%%%%%%%%%%%%%%%%%%%%%%%%%
%%%%%                            INTRODUCTION                                %%%%
%%%%%%%%%%%%%%%%%%%%%%%%%%%%%%%%%%%%%%%%%%%%%%%%%%%%%%%%%%%%%%%%%%%%%%%%%%%%%%%%%

% MOTIVATION. After reading these paragraphs a reader in any field of economics
% should believe that if you answer your research question your paper will make 
% an important contribution.

In recent years, many US jurisdictions have introduced minimum wages above the 
federal level of \$7.25, resulting in minimum wage levels that vary 
substantially across space and even within metropolitan areas.
Minimum wage policies (hereafter MW) are \textit{place-based} in that they are 
tied to a location, and workers may live and work in locations under different 
statutory MW levels, suggesting potentially heterogeneous effects of these 
policies over space.
While most research on the effects of the MW has focused on employment and 
wages irrespective of residence and workplace location
\parencite[e.g.,][]{CardKrueger1994, CegnizEtAl2019},
a full account of the welfare effects of the MW requires an understanding of 
how it affects different markets and how its effects spill over across 
neighborhoods.
In fact, while the MW appears to lower income inequality through the labor 
market \parencite{Lee1999, AutorEtAl2016},
its overall effect on income for low-wage workers may be smaller if there is 
a significant pass-through from MW changes to prices, including housing
\parencite{Macurdy2015}.

In this paper, we study the effect of MW policies on local rental housing 
markets.
Consider a new MW policy in some locations within a metropolitan area.
Because low-wage workers tend to reside in specific neighborhoods with access 
to the (now better-paying) low-wage jobs,
one would expect an increase in disposable income and a subsequent rise in demand 
for housing and rental prices in their residence instead of their workplace.
This effect, which operates through the MW at the workplace, 
will undermine (at least partially) the distributional objective of the policy.
Similarly, the MW hike will translate into higher prices of non-tradable 
consumption that use low-wage workers intensively as an input inside the 
jurisdiction that passed the new policy.
As a result, the demand for housing and rental prices will also be affected.
This effect, which operates through the MW at the residence, will have 
distributional consequences as well.
Commuting patterns thus become an essential ingredient to understand the 
heterogeneous effects of local MW policies on the housing market when there 
is a divergence in the workplace and residence locations of workers.
In Figure \ref{fig:map_shares_chicago_2018} we display, as an example, the 
geographical distribution of low-income workers by residence and workplace in 
the Chicago-Naperville-Elgin CBSA.
We observe a clear divergence between the most common residence and workplace 
locations for these workers.
This pattern is ubiquitous in our data. 

% CHALLENGES. These paragraphs explain why your research question has not already
% been answered, i.e., what are the central challenges a researcher must tackle to
% answer this question.

There is little research attempting to estimate the causal effect of minimum 
wage policies on the housing market and none accounting for spatial spillovers
through commuting.
To the best of our knowledge, the only papers that estimate the causal effect of 
minimum wages on rents in the same location are \textcite{Tidemann2018}, 
\citeauthor{Yamagishi2019} (\cite*{Yamagishi2019}, \cite*{Yamagishi2021}),%
\footnote{In the working paper version \parencite{Yamagishi2019}, the author 
explores this question using data from both the US and Japan.
In the published version \parencite{Yamagishi2021}, he excludes the analysis of 
the US case.}
and \textcite{AgarwalEtAl2021}.%
\footnote{While the main goal of \textcite{AgarwalEtAl2021} is to study the 
effect of the MW on eviction risk, the author also presents estimates of the
MW on rents using individual-level transactions.}
Estimating the effects of MW policies on rents is challenging for several 
reasons. 
First, as opposed to assessing effects on pure labor market outcomes where jobs 
and wages are tied to the workplace, when evaluating the housing market it is 
crucial to account for the fact that people may reside and work under different 
MW levels.%
\footnote{However, several papers have highlighted the importance that studies
on the effect of the MW on employment account for potential spillovers that may
``contaminate'' the control group \parencite{Kuehn2016, Huang2020}.}
%%
%% DGP: I am not sure I understand what you want to say in this footnote.
%% SH: Challenge is that there are spillovers in the housing market, which 
%%     are not as relevant in the labor market
%%     However, a few papers on the labor market say that spillovers do matter 
%%     there
%%
This is challenging because accounting for changes in the MW where residents
of a location work requires data on commuting patterns at the local level.
Second, estimation at the local level requires spatially disaggregated data on 
rents.
Using large geographies might result in null or even negative effects on average,
even if no one commutes outside of this region and the actual effect (of 
workplace MW) on some local housing markets is positive.%
\footnote{Rents in neighborhoods where low-wage workers live are likely to 
increase, whereas elsewhere they are likely not to change or even decrease, 
as those residents ``pay'' for the higher MW through higher prices and lower 
profits.}
Even if the effects in the large geographies may be of interest, they may mask 
substantial heterogeneity and therefore miss the fact that some people may be 
paying higher rents due to the policy change.
In addition, as MW changes are unlikely to be set considering the dynamics of 
local rental markets, when using small geographic units the exogeneity assumptions 
required for identification appear more plausible.
High-frequency data would also be valuable for identification since it allows 
testing whether rents change exactly after the policy change.
Finally, the effects of the MW on rents may operate through different channels,
such as prices of consumption, income, or changes in migration and commuting.
Studying the contribution of each channel separately is important to evaluate
the impact of the policy in different locations and time horizons.

% THIS PAPER. This paragraph states in a nutshell what the paper accomplishes and how.

We introduce several innovations to tackle these challenges.
First, we theoretically recognize that minimum wage policies will spill over 
across local housing markets through commuting.
We devise a new estimation approach in which rents in each local
housing market are affected by two MW-based measures:
the residence MW, which summarizes the effect of residence MW, and 
the workplace MW, which summarizes effect of MW changes at workplace locations.
We motivate this approach with a partial-equilibrium model
that maps these two measures to the effect of the MW via 
(i) consumption prices in the same location and
(ii) income generated across locations.
Second, we use a novel panel dataset on rents at the USPS ZIP code level and with 
a monthly frequency from Zillow, the largest online rental marketplace in the US.
We couple those data with an original panel dataset of statutory MWs 
at the ZIP code level, and commuting origin-destination matrices constructed
from administrative records.
As a result, assuming no changes in commuting patterns in the short run, 
we are able to estimate the effect of MW policies on rents using 
variation from hundreds of policy changes staggered across jurisdictions and 
months that generate plausibly exogenous variation of workplace and residence 
MW levels.
We show that our results are robust to using commuting data from different years
and for different groups of workers, suggesting that commuting patterns are 
stable, at least in the time window around events we consider, and thus unlikely
to affect the results.

We use our estimated model to evaluate the impact of a federal MW increase 
from \$7.25 to \$9 on rents.
Coupling our estimates with ZIP code-level income data, we estimate the share of 
each dollar of extra income (caused by the MW) that accrues to landlords in each 
ZIP code.
We discuss the implications of our results for assessing the distributional 
impact of MW policies.
%% DGP: Following Jesse's advise, we should probably make a statement about what
%% those implications are, at least vagely if not quantitatively.
%%
%% SH: I agree! However, I think that should go in the FINDINGS part of the intro?
%%

% MODEL. Summarize the key formal assumptions you will maintain in your analysis.

We start by laying out a partial equilibrium model of a ZIP code's rental market,
which is embedded in a larger geography.
The model allows residents of this ZIP code to commute to other ZIP codes to 
work, potentially under a different MW policy.
In the model workers demand square feet of housing as a function of local prices 
and income which, in turn, depend on the MW levels workers face at residence and 
workplace locations, respectively.
The model imposes fixed commuting patterns alongside fully flexible prices.%
\footnote{In the model we also assume fixed housing quality.}
This assumption, which is motivated by our empirical setting, is also consistent 
with the literature.
In fact, several recent papers find null or small effects of MW policies on 
employment \parencite{CegnizEtAl2019, DustmannEtAl2022}, and 
small elasticities of commuting to MW policies in a time horizon of several 
years \parencite{PerezPerez2021}.%
\footnote{This assumption is also motivated by our dataset, which varies at the 
monthly level. Thus, we are assuming that the first order effects of 
MW changes do not affect where agents live and work in a window of a few months
around MW changes.}
However, we note that validity of our results in the long-run will depend on the
degree of migration as a result of the policy.
Motivated by the evidence of the effect of MW policies on 
income \parencite[e.g.,][]{CegnizEtAl2019, Dube2019Income} and 
prices \parencite[e.g.,][]{AllegrettoReich2018, Leung2021},
we assume that MW hikes at workplace weakly increase disposable income and MW 
hikes at residence increase local prices.
The model illustrates that, if housing is a normal good and is complementary 
with non-tradable consumption, then the effect of a change in MW legislation 
would be heterogeneous across ZIP codes depending on whether it mostly changes 
the MW of its residents at their residence or at their workplace locations.
%% DGP: We removed the phrase about housing being complement with non-tradables 
%% from the model section. Should we revamp it?
%% SH: Why not. Feel free to do so.
In particular, we show that a MW increase in some workplace will cause rents to 
go up, whereas an increase in the residence will (conditional on a constant 
workplace MW) lower rents.
We also show that, assuming that the elasticities of per-capita housing demand 
to income and of income to MW levels do not vary by workplace, the effect of 
changes in MW at workplaces on log rents can be summarized in a single measure, 
which we call a ZIP code's workplace MW.
This measure is defined as the weighted average of log minimum wage levels 
across a ZIP code's workplaces, using commuting shares as weights.%
\footnote{We discuss the plausibility of the required constant-elasticity 
assumptions in the body of the paper.}
The effect of changes in the MW at the residence can be summarized by
a single measure as well, the log of the statutory MW in the location.
We use this result to motivate our empirical model.

% DATA. Explain where you obtain your data and how you measure the concepts that 
% are central to your study.

We construct a panel at the ZIP code and monthly levels with rental prices 
and statutory MW levels.
Our main rent variable comes from Zillow and corresponds to the median 
rental price per square foot across Zillow listings in the given ZIP 
code-month cell of the category Single Family Houses and Condominiums and 
Cooperative units (SFCC).
We collect data on MW changes from \textcite{VaghulZipperer2016} for the period 
2010--2016, which we update until December 2019 using data from 
\textcite{BerkeleyLaborCenter} and cross-validating with official sources.
We use our MW data coupled with commuting origin-destination matrices obtained 
from the Longitudinal Employer-Household Dynamics Origin-Destination Employment 
Statistics \parencite[LODES;][]{CensusLODES} database.
These data provide workplace locations for the residents of all the US census 
blocks, and we use it---along with a novel correspondence table between blocks 
and USPS ZIP codes---to construct our workplace MW measure.
We also use data on 
geographical economic indicators at the county-quarter level, 
income aggregates at the ZIP code-year level,
housing assistance programs by the \citeauthor{hudHousing}, and 
ZIP code-level sociodemographic characteristics.

% METHODS. Explain how you take your model to the data and how you overcome the 
% challenges you raised in paragraphs 3-4.

Guided by the theoretical model, we pose an empirical model where log rents in 
a location depend linearly on
(1) the residence MW, defined as the log of the statutory MW at that location;
(2) the workplace MW, defined as the weighted average of log statutory MW in other 
ZIP codes where weights are commuting shares;
(3) ZIP code and time period fixed effects;
and 
(4) time-varying controls.
As shocks to rents are expected to be serially correlated over time within ZIP 
codes, we estimate the model in first differences.
As we discuss in the body of the paper, this model recovers the true causal 
effect of the MW assuming that, within a ZIP code, changes in each of our MW 
variables are \textit{strictly exogenous} with respect to changes in the error 
term, conditional on the other MW measure and the controls.
To mitigate concerns of changes in the composition of our sample of ZIP codes 
while keeping as many of them as possible, in our baseline analysis we use a 
partially balanced panel.%
\footnote{We use all ZIP codes with valid rents data as of July 2015.}
Using an argument akin to the recent difference-in-differences literature
\parencite[e.g.,][]{CallawayEtAl2021}, 
in an appendix we unpack our identification beyond the residence and
workplace MW measures. 
We state clearly the conditions required on the commuting shares and the 
unobservable determinants of rents under a MW policy that increases the MW in 
a subset of ZIP codes only.

% FINDINGS. Describe the key findings. Make sure they connect clearly to the 
% motivation in paragraphs 1-2.

Our preferred specification implies that 
a 10 percent increase in the workplace MW (holding constant the residence MW) 
\textit{increases} rents by $\BothBetaBaseTen$ percent 
(SE=$\BothBetaBaseTenSE$).
A 10 percent increase in the residence MW (holding constant the workplace MW) 
\textit{decreases} rents by $\BothGammaBaseTenAbs$ percent 
(SE=$\BothGammaBaseTenSE$). 
As a result, if both measures increase simultaneously by 10 percent then 
rents would increase by $\BothSumBaseTen$ percent 
(SE=$\BothSumBaseTenSE$).
These results are clear evidence that, holding fixed the commuting shares, MW 
changes spill over spatially through commuting, affecting local housing markets 
in places beyond the boundary of the jurisdiction that instituted the policy.
We find that a naive model estimated only on the same-location MW would yield a 
coefficient similar to the sum of the coefficients on our workplace and 
residence MW measures.
However, this model would predict changes in rents only at residence locations 
and would not account for MW spillovers, which are central to understanding the 
distributional consequences of the rich pattern of changes in rents
generated by this policy.

Exploring the heterogeneity of our effects we conclude two things.
First, although differences are not statistically significant, we observe 
stronger results where one would expect according to the theoretical model.
Using data from LODES, we show that the effect of the residence MW measure is 
more negative in ZIP codes that are likely to host a high share of MW jobs.
This is consistent with the idea that the residence MW will cause higher price 
increases (and thus lower rent increases) in locations that use a large share 
of MW workers.
At the same time, the effect of the workplace MW measure is larger for ZIP codes 
that are likely to host a high share of residents that earn close to the MW.
This matches the view that those locations would experience a larger increase
in disposable income, and so a higher increase in rents.
Second, we show that ZIP codes that have any public housing units 
experience much larger coefficients (in absolute value) for both the 
residence and the workplace MW measures.
This may be driven by reverse causation, since locations with public housing
units are likely the residence of MW earners, who are more affected by the 
policy.

%% ROBUSTNESS

We conduct several robustness checks to test the validity of our results.
First, we evaluate our identification assumption estimating our model adding 
leads and lags of each MW measure.
Reassuringly, we find no effects of future MW changes on current rent changes.
Furthermore, the statistical significance of our estimates increases in 
this case.
We show that using panels at the county by month and ZIP code by year levels 
one fails to detect any effects, highlighting the importance of the granularity 
of our ZIP code by month data.
We also show the robustness of our results by estimating our model with 
different sets of controls that should account for a variety of confounders, 
such as the state of the local economy or local heterogeneity in 
rental dynamics.
Second, we find very similar results when computing the workplace MW with 
commuting shares for different years and worker categories.
Furthermore, in a specification we allow the commuting shares to vary by year, 
the frequency with which we observe them in the data.
The fact that results are very similar should alleviate concerns that commuting 
patterns change as response to MW changes, biasing our results.
Third, we estimate our models for different categories of housing rentals and
find similar results for several of them as well (although estimates are noisier).
Fourth, we estimate variations of our model under a fixed composition of ZIP 
codes and using an unbalanced panel with all ZIP codes available in the Zillow
data and ``cohort-by-time'' fixed effects.
The results are quantitatively similar to our baseline.
Trying to approximate the average treatment effect beyond our selected sample 
of ZIP codes, we estimate our model using weights constructed to match key 
moments of the distribution of urban ZIP codes, finding similar results as well.

Finally, in the appendix we estimate two alternative models.
We construct a ``stacked'' regression model that compares ZIP codes within 
metropolitan areas where some but not all experienced a change in the 
statutory MW.
Results are very similar but also noisier, as one should expect given
that this model contains more fixed effects and uses fewer observations.
This should alleviate concerns that our estimates actually stem from undesired 
comparisons across ZIP codes, as highlighted by recent literature 
\parencite{deChaisemartinEtAl2022,RothEtAl2022}.
The second alternative model includes the lagged first difference of rents as 
a control, and is estimated via instrumental variables following 
\textcite{ArellanoBond1991, MeerWest2016}.
This model is valid under a weaker identification assumption, and also yields 
similar but less precise results.

%% COUNTERFACTUAL

In the final part of the paper, we develop a simple extension to our baseline 
framework to estimate the ZIP code-specific share of each dollar that accrues to 
landlords following a MW increase---the ``share pocketed by landlords.''
This parameter depends on the change in the total wage of a ZIP code generated
by the MW, and also on the share of a ZIP code's total income spent in housing.
We posit a model for total wages similar to our baseline, and estimate an 
elasticity of wages to the minimum wage that is in line with the literature.
Due to data constraints, we assume a range of values for the share of housing
expenditure at the ZIP code level.
%%
%% SH: Can you cite a paper justifying the assumed share of expenditure?
%% DGP: See the following table from the consumer expenditure survey
%% https://www.bls.gov/cex/tables/calendar-year/mean/cu-all-multi-year-2013-2020.pdf
%% SH: That's really helpful. We should add a cite to some report like this.
%%
We focus on studying the consequences of a counterfactual increase in the federal 
MW from \$7.25 to \$9 in January 2020, although in an appendix we develop 
counterfactuals for larger and smaller increases in the federal MW.
We find large variation in the estimated resulting rent changes across ZIP codes.
For an assumed share of housing expenditure of $0.35$, we estimate a median
share pocketed of 0.108, implying that around 11 cents out of each dollar will 
be captured by homeowners.
In ZIP codes where both the residence and workplace MW measures increase due to 
the policy, landlords pocket on average 10.5 cents on the dollar.
In ZIP codes where the residence MW does not change, landlords pocket on average
17.6 cents on the dollar.
However, as only a share of workers commute to areas where the new MW is 
binding, the nominal increases in rents and income are smaller in this case.
These results imply that a share of the extra income that low-wage workers
receive due to the policy is actually captured by landlords.
Viewed through the lens of our theoretical model,
the mechanism behind this is a rise in housing demand in a scenario of a 
finite housing supply elasticity.
In the context of a general equilibrium model, \textcite{KlineMoretti2014} argue
that this mechanism causes place-based policies to be welfare inefficient.
While studying the full welfare effects of MW policies is beyond the scope of 
the paper, our results imply that ignoring rent changes will lead to an 
overstatement of the gains of low-wage workers following a MW increase.

%% LITERATURE

This paper is related to several strands of literature.
First, our paper relates to the large literature estimating the effects of 
minimum wage policies on employment
\parencite[e.g.,][]{CardKrueger1994, NeumarkWascher2007, MeerWest2016, CegnizEtAl2019},%
\footnote{See \textcite{Dube2019, NeumarkShirley2021} for recent reviews of the 
literature.} 
and the distribution of income 
\parencite[e.g.,]{Lee1999, AutorEtAl2016, Dube2019Income}.
There is also a growing literature studying the effects of local MW 
policies \parencite[e.g.,]{DubeNaiduReich2007, SchmittRosnick2011, DubeLindner2021}.
We contribute to this literature by focusing on a relatively less studied 
channel through which MW policies at subnational jurisdictions may affect 
welfare: the housing market.
We point out that this channel may erode some of the income gains of low-wage 
workers.
We also contribute by developing a novel panel dataset of MW levels at the 
ZIP code level for the entire US.

Second, this paper is related to the literature studying the effects of MW 
policies beyond the labor market.
We already mentioned the scant literature estimating the effects of MW policies
on rental housing prices \parencite{Tidemann2018, Yamagishi2021}.
We innovate in several ways relative to these papers.
First, while these papers estimate the effect of same-location MW on rents, we 
differentiate between residence and workplace MW levels, fully incorporating
spillovers across regions.
Second, we use data at a more detailed geography and a higher frequency, which
enriches our understanding of the spatial effects on rents and makes the 
required identification assumptions more plausible.%
\footnote{Both \textcite{Tidemann2018} and \textcite{Yamagishi2019} for the US 
use Fair Markets Rents data from the US Department of Housing and Urban 
Development, which is available at the yearly level and aggregated at the 
county level.}
Our paper also relates to \textcite{Hughes2020} and \textcite{AgarwalEtAl2021}.
\textcite{Hughes2020} study the effect of MW policies on rent-to-income ratios.
Like us, the author explicitly mentions disentangling general equilibrium 
effects from effects on rental markets as a motivation for his approach.
\textcite{AgarwalEtAl2021} show that MW increases lower the probability of 
rental default, and present complementary estimates of the effect of the MW 
on rents.%
\footnote{Our paper is also related to work studying the effects of MW policies 
on commuting and migration \parencite{Cadena2014, Monras2019, PerezPerez2021}, and 
prices of consumption goods \parencite{AllegrettoReich2018, Leung2021}.}

Third, we also contribute to the urban economics literature on place-based 
policies and on the spatial transmission of shocks.
\textcite{KlineMoretti2014} present a review of place-based policies, and 
argue that these policies result in welfare losses.
Relatedly, \textcite{HsiehMoretti2019} quantify the aggregate cost of housing 
constraints.
In line with this insight, we show in our counterfactual analysis that landlords
may benefit from a MW increase, eroding part of the rise in low-wage workers' 
income generated by the policy.
Our paper also relates to \textcite{AllenEtAl2020}, who estimate the 
within-city transmission of expenditure shocks using tourism in Barcelona.
We, on the other hand, study the within-city transmission of minimum wage shocks.

Finally, our paper relates to the literature on the econometric issues arising 
from the presence of spillover effects across units,
both in the context of minimum wage policies \parencite{Kuehn2016, Huang2020}, 
and more generally of any policy that spills over spatially
\parencite{DelgadoFlorax2015, Butts2021}.
In our setting we exploit knowledge of commuting patterns to specify the 
exposure of each unit to treatment in other units.
Under this functional form assumption we are able to account for spatial 
spillovers of MW policies on rents, allowing us to estimate rich effect patterns 
on the rent gradient.

The rest of the paper is organized as follows.
In Section \ref{sec:model} we introduce a motivating model of the rental market.
In Section \ref{sec:data} we present our data.
In Section \ref{sec:empirical_strategy} we discuss our empirical strategy and
we discuss our identification assumptions.
In Section \ref{sec:results} we present our results.
Section \ref{sec:counterfactual} discusses a counterfactual minimum wage policy, and
Section \ref{sec:conclusion} concludes.
