%%%%%%%%%%%%%%%%%%%%%%%%%%%%%%%%%%%%%%%%%%%%%%%%%%%%%%%%%%%%%%%%%%%%%%%%%%%%%%%%%
%%%%%                            INTRODUCTION                                %%%%
%%%%%%%%%%%%%%%%%%%%%%%%%%%%%%%%%%%%%%%%%%%%%%%%%%%%%%%%%%%%%%%%%%%%%%%%%%%%%%%%%

% MOTIVATION. After reading these paragraphs a reader in any field of economics
% should believe that if you answer your research question your paper will make 
% an important contribution.

In recent years, many US jurisdictions have introduced minimum wages above the 
federal level of \$7.25, which resulted in minimum wage (hereafter MW) levels that
vary substantially within metropolitan areas. 
%%% DGP: what do you think if we display the figure of percentage changes in MW's 
%%%        from January 2010 to Decemeber 2019 that we build for MW long run?
%%% SH: Interesting idea.
%%%     Because the intro is already long and the figure it's not crucially related
%%%     to the variation we use I'd prefer not to show it
MW policies are \textit{place-based} in that they are tied to a location, and
workers may work and live in locations under different MW policies, crucially 
impacting the effects of changes in these policies across space.
While most research on the effects of the minimum wage has focused on employment 
and wages irrespective of location 
\parencite[e.g.,][]{CardKrueger1994, AutorEtAl2016, CegnizEtAl2019}, 
a full account of the welfare consequences of the MW requires an understanding of 
how it affects different markets and how its effects spill over across different 
neighborhoods \parencite[as recently emphasized by][]{DubeLindner2021}.

In this paper, we study the short-run effect of MW policies across local rental 
housing markets.
One approach to answer this question would be to relate the MW of a region with 
some measure of housing rents of people residing in that region.
This approach, which is the one that was taken in the existing literature, appears 
sensible only if all low-wage workers live and worke in the same place.
However, if workers are subject to different wage floors in their workplace and
residence locations, and residence and workplace minimum wages have different 
effects on housing demand, this approach will yield confounded results.
For instance, if the MW increases in the city center, it will likely affect the 
rents of low-wage workers that commute there from other neighborhoods.
Figure \ref{fig:shares_own_geo} shows that the share of workers residing and 
working on the same geographical unit is low, especially at the local level.
As a result, we expect the effects of a local MW increase on rents---and thus on 
welfare of workers and homeowners---to be heterogeneous across space, both in 
places within the jurisdiction that passed the new legislation and in affecting
other neighbordhoods depending on prevailing commuting patterns.

% CHALLENGES. These paragraphs explain why your research question has not already
% been answered, i.e., what are the central challenges a researcher must tackle to
% answer this question.

There is little research attempting to estimate the causal effect of minimum wage 
policies on the housing market and none accounting for spatial spillovers.
To the best of our knowledge, the only papers that estimate the causal effect of 
minimum wages on rents in the same location are \textcite{Tidemann2018} and 
\citeauthor{Yamagishi2019} (\citeyear{Yamagishi2019, Yamagishi2021}).%
\footnote{\textcite{Yamagishi2019} explores this question using data from both the
US and Japan. In the published version of the paper, \textcite{Yamagishi2021} 
excludes the analysis of the US case.}
Estimating empirically the effects of MW policies on rents is challenging. 
First, as opposed to assessing effects on the labor market where jobs and wages 
are tied to the workplace, when evaluating the housing market it is crucial to 
account for the fact that people may reside and work under different MW levels. 
This is challenging because it requires to define an appropriate measure for 
workplace MW, which in turn requires data on commuting patterns at the local level.
Second, estimation at the local level requires spatially dissagregated data on rents.
Using large geographies might result in null or even negative effects on average,
even if no one commutes outside of this region and the actual effect (of workplace
MW) is positive.%
\footnote{Rents in neighborhoods where low-wage workers live are likely to 
increase, whereas elsewhere they are likely not to change or even decrease, as 
those residents ``pay'' for the higher MW through higher prices and lower profits.
The sign of the resulting effect in the larger geography is ambiguous.}
In addition, as MW changes are unlikely to be set considering the dynamics of 
local rental markets, when using small geographic units the exogeneity assumptions 
required for identification appear more plausible.
Third, even if using local variation in MW changes, the effects estimated over 
longer time horizons may conflate changes in migration, housing demand, and 
housing supply.

% THIS PAPER. This paragraph states in a nutshell what the paper accomplishes and how.

We introduce several innovations to tackle these challenges.
First, we theoretically recognize that minimum wage policies will spill over across
housing markets through commuting.
We device a new model-based estimation approach where rents in each local housing 
market are affected by two MW-based measures, one summarizing the effect of 
residence MW and a second one the effect of workplace MW.
Second, we use a novel panel dataset on rents at the USPS ZIP code level and with 
a monthly frequency from Zillow, the largest online rental marketplace in the US.
We couple those data with an original dataset of binding minimum wages at the
ZIP code level, and commuting origin-destination matrices from \textcite{LODES}.
As a result, we are able to estimate the effect of MW policies on rents using 
variation of hundreds of policy changes staggered across small jurisdictions and 
months that generate plausibly exogenous variation of workplace and residence MW
levels.
We exploit the monthly frequency of our data to focus on effects immediately
around the month of MW changes.
Finally, we use our estimated model to evaluate the short-run impact of a federal 
MW increase from \$7.25 to \$9 on rents.
Coupling our estimates with IRS data, we approximate the ZIP code-specific share on 
each dollar of income change that accrues to landlords.

% MODEL. Summarize the key formal assumptions you will maintain in your analysis.

We start by laying out a partial equilibrium model of a ZIP code's rental market, 
where workers who reside there can commute to work into ZIP codes elsewhere
within the metropolitan potentially facing a different MW level.
In the model workers demand square feet of housing as a function
of local prices and their income, which in turn depends on residence and workplace MW 
levels, respectively.
This short-run model imposes fixed residence-commuting patterns and fully flexible 
prices.%
\footnote{This assumption is motivated by our dataset, which varies at the monthly
level as then we expect the first order effects of MW changes to be such that 
agents do not adjust where they live and where they work.
However, we believe that this assumption is consistent with the recent MW literature
finding small effects of MW changes on employment over longer time horizones
\parencite[see][for a review]{Dube2019}.
Relatedly, \textcite{PerezPerez2021} finds small elasticities of commuting to MW 
policies in a time horizon of several years, so we think that this is even more 
plausible in the short run.}
Motivated by the evidence of the effect of MW policies on income and prices 
\parencite{AllegrettoReich2018,Leung2021}, we assume that MW hikes in the workplace 
increase disposable income and MW hikes in the residence (weakly) increase local 
prices.%
\footnote{We also assume that housing is a normal good so that the demand is 
increasing in income and decreasing in local consumption prices.}
The model illustrates that, under these conditions, then the effect of a change 
in MW legislation would be heterogeneous across ZIP codes depending on whether it 
mostly changes the workplace or residence MW of its residents.
In particular, we show that an increase in some workplace MW will cause rents to 
go up, whereas a residence MW hike will (conditional on workplace MW) will lower 
rents.
We also show that, under the assumption that workers who work in different 
locations have a similar elasticity of housing demand with respect to income, the 
effect of changes in MW at workplaces on log rents can be summarized in a single 
measure, which we call a ZIP code's \textit{workplace MW}.
We use this result to motivate our empirical model.

% DATA. Explain where you obtain your data and how you measure the concepts that 
% are central to your study.

We construct a panel at the USPS ZIP code and monthly levels with rental prices 
and binding MW policies.
Our main rent variable comes from Zillow, the largest online real estate platform 
in the US \parencite{realestateagentpdx, investopedia}, and corresponds to the 
median rent price per square foot across Zillow listings in the given ZIP code-month 
cell of the category Single Family, Condominiums and Cooperative Houses (SFCC). 
%DGP: Maybe say that this category according to Zillow is all listings. Although 
% this is contrary to the next sentence. We should probably group both somehow.
This is the most popular housing category in the US \parencite{Fernald2020}, 
and also the most populated series in the Zillow data.
We collect data on MW changes from \textcite{VaghulZipperer2016} for the 
period from 2010 to 2016, which we update until January 2020 using mostly data from
\textcite{BerkeleyLaborCenter}.
We assign a binding MW to each USPS ZIP code by taking the maximum across all the
MWs that affect that ZIP code (city, state, and federal levels).%
\footnote{To do this we perform two matching procedures between geographical units.
First, we match USPS ZIP codes to census ZIP code tabulation areas (ZCTA) using 
the crosswalk from \textcite{UDSMapper}.
Second, we match ZCTAs to city and states using crosswalk from \textcite{MissouriCenter}.}
We use our MW data coupled with commuting origin-destination matrices obtained from 
the Longitudinal Employer-Household Dynamics Origin-Destination Employment 
Statistics \textcite[LODES;][]{LODES} database to construct the workplace MW for 
each ZIP code, defined as a weighted average of the log statutory MW in each other 
ZIP code where its residents work, where the weights are the share of population 
that work in each one of them.

We also collect data on regional economic trends from the Quarterly Census of 
Employment and Wages (QCEW) \textcite{qcew}, defined at the county-quarter levels;
data on local wages and business income from the \textcite{IRS} at the ZIP code 
and year levels, and from the US Census (YYYY) and ACS (YYY) to construct measures 
of characteristics of a ZIP codes' residents.
%%%%
%%  SH: Add cites to US Census and ACS
%%%%

% METHODS. Explain how you take your model to the data and how you overcome the 
% challenges you raised in paragraphs 3-4.

Guided by the theoretical model, we pose an empirical model where log rents in 
a location depend on
(1) residence MW---the log of the same location statutory MW---,
(2) workplace MW---the weighted average of log statutory MW in other ZIP codes,
where weights are commuting shares---,
(3)  ZIP code and time period fixed effects,
and 
(4) time-varying controls.
As shocks to rents are expected to be serially correlated over time within ZIP codes, 
we estimate the model in first-differences.
This model recovers the true causal effect of the minimum wage assuming that, 
within a ZIP code, changes in each of our minimum wage variables are 
\textit{strictly exogenous} with respect to changes in the error term, conditional
on the other MW measure.
We show that our results are robust to alternative assumptions.
To mitigate concerns of sample composition while keeping as many ZIP codes in the 
sample as possible, in our baseline analysis we use a partially balanced panel.%
\footnote{We use all ZIP codes with valid rents data as of July 2015. 
In February 2010, when our Zillow data starts there are only 9 ZIP codes.}
Using an argument akin to the recent difference-in-differences literature
\parencite[e.g.,][]{CallawaySantAnna2021,CallawayEtAl2021}, 
in an appendix we tease out the comparisons that pin down the coefficients of our 
model, and how they relate to the treatment effects of interest.

% FINDINGS. Describe the key findings. Make sure they connect clearly to the 
% motivation in paragraphs 1-2.

Our preferred specification implies that a 10 percent increase in the workplace MW
only increases rents by XX percent (SE=XX).
If the residence MW also increases, then rents would increase XX percent (SE=XX).
%% THIS STANDARD ERROR SHOULD BE THE ONE OF THE SUM
The reason is that the residence MW is estimated to have a negative partial effect
on rents.
These results are clear evidence that, holding fixed commuting shares, MW changes 
spillover spatially through commuting, affecting local housing markets in places
beyond the boundary of the jurisdiction that instituted the policy.
A naive model estimated only on the same-location MW would yield a similar coefficient
to the sum of our workplace and residence coefficients.
However, this model would predict changes in rents only at residence locations and 
would not account for MW spillovers which are central to characterizing the rich pattern
that these policy changes generate and their distributional consequences. 
% DGP: It would be cool to have some concrete example here, but we can also defer to the counterfactual part.

Heterogeneity analyses show that ...

%% ROBUSTNESS

We conduct several robustness checks to test the validity of our results.
First, we test our identifying assumption using leads and lags of each MW variable.
Reassuringly, our models show no effects of future MW changes on current rents.
We also show the robustness of our results by estimating our model with different 
sets of controls that should account for a variety of confounders, such as the state
of the local economy or local heterogeneity in rental dynamics.
Second, in an appendix we show that our results are similar in a ``stacked regression'' 
model using policy changes with treated and control ZIP codes only within the same 
metropolitan area where time fixed effects are estimated for each set of comparisons 
following \parencite[similar to][]{CegnizEtAl2019}.
Third, as rental listings may stay on Zillow for more than one month, one may worry 
about structural auto-correlation in the dependent variable which, if not accounted for, 
may bias our estimates.
In an appendix, we discuss and estimate alternative models that include the 
lagged first difference of rents as a control, that we estimate via instrumental variables
following \textcite{ArellanoBond1991} and \textcite{MeerWest2016}.
%At the cost of imposing a particular auto-correlation structure in the error term, this 
%specification has the advantage of allowing %for feedback effects from current rental 
% price shocks to future minimum wage changes \parencite{ArellanoHonore2001}. 
Both alternative estimation procedures yield results that are very similar to our 
baseline.
Finally, we estimate variations of our model under a fixed composition of ZIP codes;
using the full set of ZIP codes and ``cohort-by-time'' fixed effects; and re-weighting
the data to match key moments of the distribution of US urban ZIP codes.
Our results are robust to these exercises.

% Finally, a drawback of the Zillow data is that it includes only a sample of US 
% ZIP codes. This fact brings about two concerns.
% First, the composition of ZIP codes changes over time potentially introducing bias in our 
% estimates. We tackle this issue by performing our main analysis with a constant set of 
% units with valid data as of July 2015.%
% \footnote{Because these ZIP codes enter the sample at different moments in time 
% before 2015, our estimating panel is still unbalanced.}
% This strategy alleviates concerns arising from the changing composition of ZIP 
% codes, but significantly lowers the number of observations used in the estimation.
% We also estimate a model on the full sample of available ZIP codes, obtaining 
% similar results.
% Secondly, we worry that our estimated effect may be particular to our sample.
% In order to approximate the average treatment effect for the typical urban ZIP 
% code, we re-weight our data to match average demographic characteristics of the 
% top 100 Metropolitan Statistical Areas and re-estimate our main models 
% \parencite{Hainmueller2012}.
% The effects not only survive this test but are also larger.

%% COUNTERFACTUAL

In the final part of the paper, we develop a simple extension to our baseline model
to estimate the ZIP code-specific share on each dollar that accrues to landlords
following a MW increase.
This parameter depends on the change in the total wage bill of a ZIP code, and the 
share of a ZIP code's earnings spent in housing.
We posit a model for the wage bill similar to our baseline, and estimate an elasticity
of wages to the minimum wage that is in line with the literature. %DGP: probably add a cite or two here.
Due to data constraints, we assume a range of values for the share of housing
expenditure at the ZIP code level following the literature (CITE, DIAMOND?).
We focus on studying the consequences of a counterfactual increase 
in the federal MW from \$7.25 to \$9.
We find large variations in the effects of this policy on rents across ZIP codes.
We estimate that, in ZIP codes where both the residence and workplace MWs increase
due to the policy, landlords pocket between 5 and 8 cents on the dollar.
However, in ZIP codes where the residence MW does not change, the share pocketed
by landlords is higher, as the pass through of MW to prices (other than rents)
is likely to be small.

%% LITERATURE

This paper is related to several strands of literature.
First, our paper relates to the large literature estimating the effects of 
minimum wage policies on labor market outcomes.
Starting with \citeauthor{CardKrueger1994}'s (\citeyear{CardKrueger1994}) classical 
study, many papers have explored the effect of these policies on employment
\parencite[some recent examples include][]{MeerWest2016,CegnizEtAl2019}.%
\footnote{See \textcite{Neumark2006} for an earlier review of this literature,
and \textcite{Dube2019, NeumarkShirley2021} for more recent reviews.}
Similarly, several papers study the consequences of minimum wage policies on the 
distribution of income and inequality \parencite{Lee1999, AutorEtAl2016}.
We contribute to this literature by focusing on a relatively less studied channel
through which minimum wage policies may affect welfare: the housing market.

This paper is similarly related to the literature estimating the effects of MW 
policies on housing markets.
We already mentioned the scant literature estimating the effects of MW policies
on rental housing prices \parencite{Tidemann2018, Yamagishi2021}.
We innovate in several ways when compared to these papers.
First, while these papers estimate the effect of same-location MW on rents, we 
differentiate between residence and workplace MW levels, fully incorporating
spillovers across regions.
Second, we use data at a more detailed geography and higher frequency.%
\footnote{Both \textcite{Tidemann2018} and \textcite{Yamagishi2019} for the US 
exploit Fair Markets Rents data from the Department of Housing and Urban 
Development (HUD), which is available at the yearly level and aggregated at the
geographical level of counties.}
Both of these facts enrich our understanding of the estimated effects and make 
the required identification assumption more plausible.
Our paper also relates to \textcite{Hughes2020} who uses a triple 
difference-in-differences design to study the effect of MW policies 
on rent-to-income ratios. Like us, the author explicitly mentions 
disentangling general equilibrium effects from effects on rental markets as 
a motivation for his approach. 
\footnote{Another related paper is \textcite{AgarwalEtAl2019} who shows that MWs
decrease the probability of rental default.}
Our work is also related to work studying the effects of MW policies on commuting
and migration \parencite{Cadena2014,Monras2019,PerezPerez2021}, and prices of 
consumption goods \parencite{AllegrettoReich2018,Leung2021}

Third, we also contribute to the literature on place-based policies.
\parencite{KlineMoretti2014} presents a review of place-based policies, and argues
that these policies result in inefficiencies due to finite housing supply 
elasticites in different locations.
Relatedly, \textcite{HsiehMoretti2019} quantify the aggregate cost of housing 
constraints.
In line with these insight, we show in our counterfactual MW increase that 
landlords may benefit from this policy eroding some of the extra disposable 
income of low wage workers. 

% Interesting recent place-based policy paper: 
%  https://www.researchgate.net/publication/345323579_The_effects_of_a_place-based_tax_cut_and_minimum_wage_increase_on_labor_market_outcomes


Finally, our paper relates to the literature on the econometric issues of the 
presence of spillover effects across units,
both in the context of minimum wage policies \parencite{Kuehn2016, Huang2020}, and
more generally of any policy that spills-over spatially
\parencite{DelgadoFlorax2015, Butts2021}. 
% Should we also cite AronowSamii2013 ???
In our setting we exploit knowledge of commuting patterns to specify the exposure
of each unit to treatment in other units.
Under simple and intuitive functional form assumptions we are able to account for 
spatial spillovers of MW policies on rents, allowing us
to estimate rich effect patterns on the rent gradient. 

The rest of the paper is organized as follows.
In Section \ref{sec:model} we introduce a motivating model of the rental market.
In Section \ref{sec:data} we present our data.
In Section \ref{sec:empirical_strategy} we discuss our empirical strategy and
we discuss our identification assumptions.
In \ref{sec:results} we present our main results.
Section \ref{sec:counterfactual} discusses a counterfactual minimum wage policy, and
Section \ref{sec:conclusion} concludes.
