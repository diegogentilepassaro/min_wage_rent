%%%%%%%%%%%%%%%%%%%%%%%%%%%%%%%%%%%%%%%%%%%%%%%%%%%%%%%%%%%%%%%%%%%%%%%%%%%%%%%%%
%%%%%                                MODEL                                   %%%%
%%%%%%%%%%%%%%%%%%%%%%%%%%%%%%%%%%%%%%%%%%%%%%%%%%%%%%%%%%%%%%%%%%%%%%%%%%%%%%%%%

We build a simple partial-equilibrium model of the rental market in a zipcode that illustrates 
the main mechanism we believe will drive our results. Later, in \autoref{sec:benchmark} we 
use the model to benchmark our empirical results.

BRIEFLY DESCRIBE MODEL. BRIEFLY COMPARE WITH MODEL IN \textcite{Yamagishi2020}.


%%%%%%%%%%%%%%%%%%%%%%%%%%%%%%%%%%%%%%%%%%%%%%%%%%%%%%%%%%%%%%%%%%%%%%%%%%%%%%%%%
\subsection{Model set-up}

We focus on the supply and demand of housing in a given zipcode. Consider an environment with 
an exogenously given continuum of households in each zipcode divided in two groups: minimum wage 
and non-minimum wage households (HH). The former are fully affected by the MW, whereas the latter 
are not affected at all.

On the supply side, we denote by $H$ the continuous measure of housing units available for rent 
in the zipcode. We assume that units are homogeneous, and can be rented at the a rent of $r$. The 
supply of housing $H(r)$ is assumed to be increasing in rents $r$, so that $H'(r) > 0$.

Let us move to the demand side. Households receive monthly a income, which we denote by 
$\underline{w}$ and $w$ for MW HH and non-MW households, respectively. Demand for housing is given 
by $\underline{H}(r, \underline{w})$ and $\overline{H}(r, w)$ for each household type. We make two 
standard assumptions on these objects: (i) the demand of housing is downward sloping (i.e., 
$\underline{H}_r(r, \underline{w}) < 0$ and $\overline{H}_r(r, w) < 0$); and (ii) the demand for 
housing is increasing in income (i.e., $\underline{H}_w(r, \underline{w}) > 0$ and $\overline{H}_w(r, 
w) > 0$)


%%%%%%%%%%%%%%%%%%%%%%%%%%%%%%%%%%%%%%%%%%%%%%%%%%%%%%%%%%%%%%%%%%%%%%%%%%%%%%%%%
\subsection{Equilibrium and the elasticity of rents to the minimum wage}

Equilibrium rents $r^*$ are such that local housing supply is equated to local housing demand. 
Formally,

\begin{equation*}\label{eq:model-eq}
	H(r) =  \underline{H}(r, \underline{w}) + \overline{H}(r, w) \ .
\end{equation*}

We are interested in the elasticity of equilibrium rents $r^*$ to the minimum wage $\underline{w}$, 
which we denote by $\rho$. The implicit function theorem applied on the above equation yields

\begin{equation}\label{eq:model-elasticity}
	\rho := \frac{d \ln r^*}{d \ln \underline{w}} 
		  = \frac{\underline{w} \ \underline{H}_w}
		  		 {r\  H'(r) - r \ \underline{H}_r - r \ \overline{H}_r} \ ,
\end{equation}
where we denote partial derivatives with sub-indexes.

Note that, since $\underline{H}_r < 0$ and $\overline{H}_r < 0$, the above expression is always 
positive. When the MW increases the local housing market moves to a new equilibrium with higher 
rents. The magnitude of the elasticity is driven by the relative magnitudes of the earnings of 
minimum wage workers ($\overline{w}$) and rents ($r$), and the slopes of the different 
functions in equilibrium. For instance, a higher response of housing demand to the minimum wage 
change ($\underline{H}_w$) would result in a higher elasticity.


%%%%%%%%%%%%%%%%%%%%%%%%%%%%%%%%%%%%%%%%%%%%%%%%%%%%%%%%%%%%%%%%%%%%%%%%%%%%%%%%%
\subsection{Extensions}

Above, we assumed above that the measure of each type of households is exogenously given. However, 
people could move across zipcodes in response to a minimum wage change.\footnote{We are not 
	extremely concerned about this possibility because most MW changes arise from the state in our 
	sample. Therefore, they tend to be uniform across connected zipcodes.} 
Allowing the measure of households to be endogenous would not alter the conclusions as long as 
the overall demand for housing increases after a MW hike. In this case, however, the overall 
effect will arise from MW households moving into the zipcode, and potentially some none-MW 
households leaving (as they face higher rents). 

.... \textbf{IT WOULD BE NICE TO HAVE AN EXPRESSION FOR THIS CASE (WHICH COULD GO IN APPENDIX)}

Another extension would involve houses of different quality, since it's possible that MW households 
rent houses of lower quality on average.... \textbf{THINK ABOUT THIS}

