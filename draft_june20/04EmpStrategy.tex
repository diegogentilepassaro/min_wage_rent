
In this section, we present two empirical strategies we have used to estimate the impact of MW changes over house rents and housing prices. To do this, we outline the main features of the specifications used along with the identifying assumptions used to reach identification of the parameters of interest. We further describe some of the complications we are facing and how we plan to address them in the future. We first present a standard event-study framework with level outcomes. We then discuss the issues arising in the presence of multiple events over the same unit, and what limitations they impose on the estimation. Finally, we discuss an alternative approach based on  that we believe alleviates some of the concerns.

\subsection{Event-study specifications}\label{subsec:empirical_strategy/event-study}

    Let $y_{jct}$ be an outcome of the housing market in zipcode $j$, county $c$, period $t$, such as the median rent or the average housing price. In practice, we will leverage on various measures capturing different aspects of the local housing market.\footnote{As discussed in the Data section, Zillow provides rent values for houses of varying characteristics (e.g. single-family houses, condos, or 2 bedroom houses).} As introduced in \autoref{sec:intro}, to estimate the effect of MW changes on $y_{jct}$, we use a panel event-study methodology exploiting the fine timing of hundreds of events in the 2010 decade staggered across zipcodes and months. These high frequency data should allow for simultaneous identification of relative dynamics around the event and a very rich set of fixed effects.
    
    Ideally, we would like to estimate the following dynamic model
    \begin{equation}\label{eq:ideal-event-study}
        \begin{split}
            y_{jct} = \gamma_{j} + \alpha_{ct} + \boldsymbol{\beta} \boldsymbol{X}_{jct} + \sum_{k = -w}^{w}\delta_{t + k} D_{jct}^k + \epsilon_{jct} 
        \end{split}   
    \end{equation}
    where $j$, $c$, $t$, index zipcodes, counties, and time; $\gamma_{j}$ is a zipcode fixed effect, $\alpha_{ct}$ represents flexible time trends for each county, $\boldsymbol{X}_{jct}$ represent time-varying controls, $D_{jct}^k$ is an indicator for a MW change taking place in zipcode $j$ of county $c$ $k$ periods relative to time $t$, and $w$ is a constant. The coefficients of interest are $\{\delta_{t + k}\}_{k=-w}^w$, which show the conditional dynamics of rents around ``salient'' MW increases.\footnote{We talk about MW ``increases'' because most MW changes tend to be positive, with a handful of exemptions were a local minimum wage was imposed and later overthrown by the state court. By ``salient'' we mean MW increases of at least 50 cents.} Furthermore, under some assumptions on the behavior of the unobservable we would claim that these coefficients reveal the ``causal'' effect of a minimum wage increase on rents.

\subsubsection{Mulitiple events per unit}

    The main issues we have faced with this specification are of \textit{technical} nature. Usually, event-studies are constructed with one event per unit that is implemented at different times. They are implemented by excluding the pre-period dummy and adding ``off window'' dummies that saturate the model, so as to identify the relative time effects with respect to the pre-period. Given that we usually have multiple events per zipcode, it is not clear how to construct the off window indicators in our case. Furthermore, for a given window size $w$ it is common to see events with overlapping windows, so the relative time coefficients may be contaminated by dynamics from other events. 
    
    Figure \ref{fig:multiple-events-example} illustrates these problems. Consider initially panel (a), which depicts a hypothetical zipcode $j$ in a state that increases the minimum wage every 12 months. Initially, the zipcode is bound by the state minimum wage, but after 6 months of the state change, a higher local MW is introduced. Consider a specification with $w = 6$. As shown in the picture, the periods post-event are also the pre-event periods of the following increase, rendering the identification of the coefficients of interest potentially confounded if precise controls are not included in the regression.
    
    Panel (b) depicts a situation where the event windows do not overlap. In this case, the overlapping-windows problem does not arise. However, the implementation is still unclear. How should one construct the off window dummies that saturate the model? If we just define those indicators as one anytime a time period is outside the window of an event, then they will turn on when other events take place confounding the relative time coefficients.
    
    \begin{figure}[t!]
        \begin{subfigure}{1\textwidth} \centering
            \begin{tikzpicture}
                
                %Timeline
                \draw[thick, -Triangle] (0,0) -- (13.5,0) node[font=\scriptsize,below left=3pt and -8pt]{Periods};
                    
                %Ticks and event labels
                \foreach \x in {0,0.5,...,13}
                \draw (\x cm,3pt) -- (\x cm,-3pt);
                
                \foreach \x/\descr in {4.5/\begin{tabular}{c} Event 1 \\ (state) \end{tabular}, 7.5/\begin{tabular}{c} Event 2 \\ (local) \end{tabular}, 10.5/\begin{tabular}{c} State increase \\ (not binding) \end{tabular}}
                \node[font=\scriptsize, text height=5ex, text depth=.5ex] at (\x,.7) {$\descr$};
                
                \draw[very thick] (4.5 cm,3.5pt) -- (4.5 cm,-3.5pt);
                \draw[very thick] (7.5 cm,3.5pt) -- (7.5 cm,-3.5pt);
                \draw[very thick] (10.5 cm,3.5pt) -- (10.5 cm,-3.5pt);
                
                %event windows
                \draw (1.5, -.4) -- (7.5, -.4);
                \draw (4.5, -.7) -- (10.5, -.7);
                
                \foreach \x in {1.5,2,...,7.5}
                \draw (\x cm,-.35) -- (\x cm,-.45);
                \foreach \x in {4.5,5,...,10.5}
                \draw (\x cm,-.65) -- (\x cm,-.75);
                    
                %overlap area
                \draw[lightgray!75!red, line width=5.5pt] (4.5,-.55) -- +(3,0);
                    
            \end{tikzpicture}
            \caption{Overlapping windows}
        \end{subfigure}\\
        \begin{subfigure}{1\textwidth} \centering
            \begin{tikzpicture}
                
                %Timeline
                \draw[thick, -Triangle] (0,0) -- (13.5,0) node[font=\scriptsize,below left=3pt and -8pt]{Periods};
                    
                %Ticks and event labels
                \foreach \x in {0,0.5,...,13}
                \draw (\x cm,3pt) -- (\x cm,-3pt);
                \foreach \x/\descr in {3.5/\begin{tabular}{c} Event 1 \\ (state) \end{tabular}, 10/\begin{tabular}{c} Event 2 \\ (local) \end{tabular}}
                \node[font=\scriptsize, text height=5ex,
                    text depth=.5ex, text width=1.5cm] at (\x,.7) {$\descr$};
    
                \draw[very thick] (3.5 cm,3.5pt) -- (3.5 cm,-3.5pt);
                \draw[very thick] (10 cm,3.5pt) -- (10 cm,-3.5pt);
                
                %event windows
                \draw (.5, -.4) -- (6.5, -.4);
                \draw (7, -.7) -- (13, -.7);
                
                \foreach \x in {0.5,1,...,6.5}
                \draw (\x cm,-.35) -- (\x cm,-.45);
                \foreach \x in {7,7.5,...,13}
                \draw (\x cm,-.65) -- (\x cm,-.75);
                    
            \end{tikzpicture}
            \caption{Undefined off window dummies}
        \end{subfigure}
        \caption{An illustration of the multiple events problem in event-studies}
        \label{fig:multiple-events-example}
    \end{figure}
    
    We are currently working on two different ways to address these issues. Firstly, we are considering a model with ``infinite'' relative time dummies, in the spirit of \textcite[][, equation 1]{borusyak2017revisiting}. Secondly, we are working on a ``stacked'' event study framework, similar to \textcite{cengiz2019effect}, in which one selects as a sample a time window around each event a constructs a panel coded by event-relative time. In the first case the ``off window'' dummies do not need to be constructed, whereas in the second there are well-defined pre- and post-periods for each event. Thus, both are ways to solve the implementation problems. To control for overlapping-windows, we will add dummies for other MW changes that take place within each of the main events used for the analysis.
    
    We are working on both of the before-mentioned alternatives. As a temporary workaroudn, we deliberately selected one event per zipcode to run a traditional event-study. This brings its own complications, as discussed below. But first, let us introduce the framework that we used for the results shown in this draft.

\subsubsection{An event-study with one event per zipcode}

    One way to sort out the difficulties that arise from the multiple-events problem is simply selecting one event per unit to run a tradition two-way fixed effects specification. In this initial draft, we take this approach.  Specifically, we select the last MW increase within each unit. We do this because Zillow adds zipcodes over time, and so closer to the end of panel the more units with valid rents data we have.\footnote{We also impose the restriction that the event must have a complete window of $w$ months after it.}
    
    The estimating equation is
    \begin{equation}\label{eq:last-event-study}
        \begin{split}
            y_{jct} = & \gamma_{j} + \alpha_{ct} + \boldsymbol{\beta} \boldsymbol{X}_{jct} \\
            & + \delta^{-} D_{jct}^{-} + \sum\limits_{k = -w}^{-2}\delta_{t + k}D_{jct}^k + \sum\limits_{k = 0}^{w}\delta_{t + k} D_{jct}^k + \delta^{+} D_{jct}^{+} + \epsilon_{jct} 
        \end{split}   
    \end{equation}
    where, as before, $\gamma_{j}$ and $\alpha_{ct}$ are fixed effects; $D_{jct}^k$ is an indicator for a minimum wage taking place $k$ periods relative to $t$; and $\{D_{jct}^{-}, D_{jct}^{+}\}$ are indicators for pre- and post-event --note that we omit the indicator for the month prior to the event--. The coefficients of interest are $\{\delta_{t-w}, ..., \delta_{t-2}$, $\delta_t$, $\delta_{t+1}, ...$, $\delta_{t+w}\}$, which show the dynamics of the outcome variable around selected MW events. This equation is not free of identification problems, of course. We will discuss some of those in the next subsection.
    
    Before moving to discussing identification, two important notes about the regression equation \eqref{eq:last-event-study}. First of all, the fact that different zipcodes have a different number of accumulated MW increases means that they may also differ in the level of rents and, potentially, different dynamics. We partially account for this by including indicators for different values of the cumulative sum of unused events in the controls $\boldsymbol{X}_{jct}$.\footnote{For example, suppose a zipcode has two MW increases before the last event, at periods $\tau_1$ and $\tau_2$. In that case, we would include indicators for the period before any event $\mathds{1}\left(t \leq \tau_1\right)$, the period between the first and the second $\mathds{1}\left(\tau_1 <  t \leq \tau_2 \right)$, and an indicator after the second event $\mathds{1}\left(t > \tau_2\right)$. In here, $\mathds{1} \left( \cdot \right)$ is the indicator function.} Secondly, the set of ``last MW events'' is obviously a selected sample of all the events. For example, these events are more likely to be local. This qualifies the interpretation of the results.

\subsubsection{Identification issues}
    
    \paragraph{The ideal model}
    
    Suppose we overcome the implementation issues discussed above, and so the model in equation \eqref{eq:ideal-event-study} can be estimated. Those results will be interpretable as the ``causal'' effect of the minimum wage on rents if the following identifying assumption holds: $$E \left[ \epsilon_{jct} D_{jct}^k \big| \gamma_j, \alpha_{ct}, \boldsymbol{\beta} \boldsymbol{X}_{jct}\right]  = 0 \ \ \ \ \forall k\in\{-w, ..., 0, 1, ..., w\}. $$ This means that unobservable determinants of the outcome are mean independent from the relative timing of the MW event conditional on the set of fixed effects and controls. We can also interpret this assumption as ``parallel-trends'', meaning that, in the absence of some minimum wage change, the counterfactual outcome between units treated at different times within a given county would have not differed systematically.
    
    Given that equation \eqref{eq:ideal-event-study} utilizes within zipcode variation to pin-down the effect, this assumption seems much more plausible than the assumptions needed when using within county variation and a much less flexible set of fixed effects. Conditional on county-specific time fixed effects and county-specific calendar months,\footnote{Note that these fixed effects are a very high dimensional object.} the unobserved within zipcode variation is unlikely to be correlated with the determinants of the timing of the MW change, as these are not likely to be determined at the zipcode level. One important reason for this is that MW changes are usually enacted through federal or state law, or through local ordinances that are subject to higher level court blocks and revisions, and that may follow from ballot initiative. Therefore, the timing of the enactment of a MW change in a given zipcode could be thought as random. Furthermore, we control for flexible local business cycles, alleviating concerns about differential trends when introducing a minimum wage.\footnote{Suppose political units, such as states and counties, introduce a minimum wage at the same time when they are experiencing an economic boom. Estimation of this model will exploit variation within county, after controlling for the business cycle. These kind of concerns are thus misplaced in this context.}
    
    \paragraph{The last event specification}
    
    Restricting the pool of events used for estimating the treatment effect to one per unit, however, introduces different identification issues as pointed out by \textcite{borusyak2017revisiting}. Specifically, the identification of a dynamic causal effect is hampered by the impossibility of separately estimate relative time and period fixed effects, as each of them occur only once per unit, making it impossible to pin it down up to a linear trend. One possible solution is to remove the unit fixed effects at the expenses of a stronger identification assumption and less power. We perform this by running an alternative specification of \eqref{eq:last-event-study} where we replace zipcode-level fixed effects with county-level ones. Under this new framework, the identifying assumption becomes $E \left( \epsilon_{jt} D_{j(t+k)} | \alpha_c, \alpha_{ct}, \boldsymbol{\beta} \boldsymbol{X}_{jt}\right)  = 0  \ \ \ \forall k\in[-w, w]$. Altought we include a third-degree polynomial for time trend as additional controls, note that we are still allowing for the possibility of having unobserved time varying trends correlated both with MW changes and with prices. 
    
    In addition to this, we then introduce a further specification with a control group made of zipcode that do not experience changes in MW in our sample [ADD EQUATION FOR CLARITY?]. This allows us to separately pin down year effects from the causal treatment effect of MW changes by the very fact that such units are indeed never treated. Note that, in order to test the strength of our results, we can add county-level trends since such geographical unit does not perfectly overlap with treated and control group. \\   

\subsection{First-difference specifications}\label{subsec:empirical_strategy/first-difference}

\subsubsection{Static model}

    Following \textcite{meer2016effects} : 
    
    \begin{equation}\label{eq:diff_main}
        \Delta \tilde{y}_{jt} = \sum\limits_{k = -5}^{5} \delta_{t+k} \Delta \tilde{D}_{j(t+k)} + \alpha_{t} + \boldsymbol{\beta} \boldsymbol{X}_{jt} + \nu_{jt}
    \end{equation}
    
    where $\Delta \tilde{y}_{jt}$ and $\Delta \tilde{D}_{j(t+k)}$ are the first difference of log outcome and MW respectively. Note that the first difference specification already take care of the unit fixed effect, hence not included. Identification in the current setting is reached assuming changes in the outcome variable that are invariant up to a linear transformation \parencite{borusyak2017revisiting}. In [RESULTS] we show how the county-level FE version of \autoref{eq:main_ziplevel} indeed reveals a sudden jump in the outcome vairable at treatment time along with a fairly linear dynamic hence justifying the validity of this latter approach. 
    
    We will discuss the no pre-trends assumption in the next section.


\subsubsection{Dynamic model}

    Lala
    

\subsection{Towards the effect of wages on rents}

    An object that is also of interest is the effect of wages in a local neighborhood on the rents paid by individuals living there, going beyond the minimum wage. In the future, we would like to estimate this parameter via a TSLS system, where we instrument wages with the minimum wage.
