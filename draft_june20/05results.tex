
    In this section we outline the estimated impact of MW changes on rent and house prices. We start by showing the results of our event-study specification with one event per zipcode. As discussed extensively in section \autoref{sec:empirical_strategy}, this model suffers from underidentification issues. We show several ways in which we address those. Later, we present results for the first difference specification, which we deem more reliable. Finally, we discuss the magnitude of the estimates.


\subsection{Event-study specification}\label{subsec:results/event-study}

    We being discussing the results by presenting our estimation of the last event specification using median rents per square foot as our main dependent variable. In particular, we focus on the median rent per square foot for single family houses and condos (SFCC) as this is the most populated time series in the Zillow rent data. We show the robustness of our results by using alternative rent variables in Appendix Figure \ref{appfig:event_study_change_depvar}.
    As described in the data section, we also fix the composition of the panel by using zipcodes with valid rents data as of June 2015.\footnote{It is worth emphasizing that the panel is still not balanced. The reason is that Zillow started to provide data on these zipcodes at different points in the period 2010-2015.} The concern we address here is that a changing composition of the sample may bias the results.
    
    Figure \ref{fig:event_study_main} shows different specifications of the ``last event'' event-study discussed in section \ref{sec:empirical_strategy}. The first row shows results of the basic two-way fixed effect model in equation \eqref{eq:last-event-study}, using only zipcodes that experience some MW change in the period of interest. Panel (a) controls only for unused MW events, whereas panel (b) adds county-specific linear and quadratic trends. Although this latter panel suggests a mild but non-significant increase in rents after the last MW change, overall these plots suggest no effects on rent. However, we do not trust on these results for several reasons. First of all, \textcite{BorusyakJaravel2017} show that the dynamic coefficients in this model are under-identified. Secondly, by dropping untreated units we use a rather small sample, which means that confidence intervals are relatively large. The remaining panels attempt to address these issues.
    
    While using the same sample of untreated units as (a) and (b), panels (c) and (d) tackle the underidentification problem by replacing zipcode with county fixed effects (equation \eqref{eq:last-event-study-countyFE}). The results are in fact very similar to the basic TWFE, showing a mild increase in the dynamic effects once county-specific trends are included. 
    
    Finally, the third row shows the results of fitting equation \eqref{eq:last-event-study-control}. This specifications keep zipcodes fixed effects, but also add add geographical units that were never treated to the estimating sample. Panel (e) shows a strongly significant effect. Monthly median rent per square foot is around \$0.03 higher 6 months after the MW with respect to the month before the implementation (s.e. 0.0081). This implies that a house of 1,000 square feet would increase its rent around \$30.
    % Full regression output: https://github.com/diegogentilepassaro/min_wage_rent/blob/9403cb9f01217ed02c9d5bf8ea3138430227c0f7/analysis/event_study/output/make.log#L762-L814
    Panel (f) aims at controlling for unobserved trends in rents in local markets that may confound the effect, adding a county-specific linear and quadratic trend. The resulting dynamic effects seem to be slightly lower, and confidence intervals sometimes cross 0. 
    
    While both panels (c)-(d) and (e)-(f) should solve the under-identification issues raised by \textcite{BorusyakJaravel2017}, we believe more the results in the bottom panels. The first reason for this is that they maintain zipcode fixed effects, exploiting within zipcode variation in the regression. This seems specially important in light of the conflicting results in \textcite{tidemann2018mw} and \textcite{yamagishi2019minimum}, who use within zipcode variation.
    , 
    because they use all the zipcodes in our rents panel they are the most powered.. As a result, 
    
    The causal claim on these results relies on the assumptions of no pre-trends. Indeed, all estimations show a rather flat and not statistically significant pre-trend %followed by a immediate impact of around ten cents on rent that remains stable in the following months.
    
    \begin{figure}[h!] \centering
        \caption{Dynamic effects around selected minimum wage events}
        \label{fig:event_study_main}
        \begin{subfigure}{0.5\textwidth} \centering
            \includegraphics[width=0.95\linewidth]{analysis/event_study_exploration/output/last_rentpsqft_sfcc_zfe_w6.png}
            \caption{Basic TWFE with treated units only} \label{fig:event_study_treated}
        \end{subfigure}%
        \begin{subfigure}{0.5\textwidth} \centering
            \includegraphics[width=0.95\linewidth]{analysis/event_study_exploration/output/last_rentpsqft_sfcc_zfe_w6_county-trend.png}
            \caption{Basic TWFE and county-specific trend} \label{fig:event_study_treated_county-trends}
        \end{subfigure}\\
        \begin{subfigure}{0.5\textwidth} \centering
            \includegraphics[width=0.95\linewidth]{analysis/event_study_exploration/output/last_rentpsqft_sfcc_cfe_w6.png}
            \caption{Replacing zipcode with county FE} \label{fig:event_study_countyFE_treated}
        \end{subfigure}%
        \begin{subfigure}{0.5\textwidth} \centering
            \includegraphics[width=0.95\linewidth]{analysis/event_study_exploration/output/last_rentpsqft_sfcc_cfe_w6_county-trend.png}
            \caption{\begin{tabular}{c} Replacing zipcode with county FE \\ and county-specific trend \end{tabular}} \label{fig:event_study_countyFE_treated_county-trends}
        \end{subfigure}\\
        \begin{subfigure}{0.5\textwidth} \centering
            \includegraphics[width=0.95\linewidth]{analysis/event_study/output/last_rentpsqft_sfcc_w6.png}
            \caption{Basic TWFE with untreated units} \label{fig:event_study_all}
        \end{subfigure}%
        \begin{subfigure}{0.5\textwidth} \centering
            \includegraphics[width=0.95\linewidth]{analysis/event_study/output/last_rentpsqft_sfccw6_county-trend.png} \label{fig:event_study_all_county-trends}
            \caption{\begin{tabular}{c} Basic TWFE with untreated units \\ and county-specific trend \end{tabular}}
        \end{subfigure}\\
        \begin{minipage}{.95\textwidth} \footnotesize
			\vspace{2mm} 
			\textit{Notes}: The figure shows the results from fitting the ``last event'' event study in different samples and with changing controls. In all cases, we select the last minimum wage event per zipcode, based on the requirement that the increase was of at least \$0.5 and that it took place before June 2019 (so that every event has at least 6 months after it). Furthermore, all models: use median rent per square foot of the SFCC category, and control for calendar time fixed effects (FE) and dummies for different categories of the cumulative sum of unused MW events. Each row is a different specification, with the panels on the right adding as controls county-specific linear and quadratic trend. Panels (a) and (b) fit the under-identified model using zipcode FE excluding never treated units (equation \ref{eq:last-event-study}), $N = 46,119$. Panels (c) and (d) use FE at the level of the county instead of zipcode, keeping the sample of never treated units only (equation \ref{eq:last-event-study-countyFE}), $N = 46,119$. Panel (e) and (f) use zipcode FE and add control units (equation \ref{eq:last-event-study-control}), $N = 113,071$.
		\end{minipage}
    \end{figure}
    
    
    
    
    Second, we check the robustness of this result by changing the prominence of the MW changes. Specifically, In \ref{appfig:event_size_sensitivity} we use MW changes that comply to our sample criterion (explained in section 2) and are of at least \$0.25 and of at least \$0.75 instead of \$0.5.
    
    
    % TRY COUNTY-QUARTER SPECIFICATION
    
\subsection{Panel difference-in-differences specifications}\label{subsec:results/first-differences}

    \subsubsection{Baseline results}
    
    In this section we present the results from models based on section 4.2. Table XXX shows results from the static model in first differences. Column 1 reports results only including two way fixed effects. Column 2 adds a zipcode-specific linear trend, and column 3 a zipcode specific quadratic trend. 
    
    \begin{table}[h!] \centering
        \caption{Static model}
        \label{tab:fd_table}
        \scalebox{0.85}{
        {
\def\sym#1{\ifmmode^{#1}\else\(^{#1}\)\fi}
\begin{tabular}{l*{3}{c}}
\hline\hline
          &\multicolumn{1}{c}{(1)}&\multicolumn{1}{c}{(2)}&\multicolumn{1}{c}{(3)}\\
          &\multicolumn{1}{c}{D.ln\_med\_rent\_psqft}&\multicolumn{1}{c}{D.ln\_med\_rent\_psqft}&\multicolumn{1}{c}{D.ln\_med\_rent\_psqft}\\
\hline
D.ln\_mw   &   0.0260\sym{**} &   0.0257\sym{**} &   0.0255\sym{**} \\
          & (0.0128)         & (0.0120)         & (0.0117)         \\
\hline
Zipcode-specifc linear trend&       No         &      Yes         &      Yes         \\
Zipcode-specific linear and square trend&       No         &       No         &      Yes         \\
R-squared &    0.022         &    0.024         &    0.026         \\
Observations&   112232         &   112232         &   112232         \\
\hline\hline
\end{tabular}
}
}
        \begin{minipage}{.95\textwidth} \footnotesize
			\vspace{3mm} 
			\textit{Notes}: The table shows
		\end{minipage}
    \end{table}
    
    
    In all cases the effects is significant, stable and a 10\% change in the MW implies around a 0.25\% increase in the rent per square foot. 
    
    Next, in Table XXX we show results from a model with 5 leads and lags of the logarithm of the MW. Again, Column 1 reports results of a specification with two way fixed effects. Column 2 adds a zipcode-specific linear trend, and column 3 a zipcode specific quadratic trend. 
    
    \begin{table}[h!] \centering
        \caption{Dynamic model}
        \label{tab:fd_table}
        \scalebox{0.85}{
        {
\def\sym#1{\ifmmode^{#1}\else\(^{#1}\)\fi}
\begin{tabular}{l*{4}{c}}
\hline\hline
          &\multicolumn{1}{c}{(1)}&\multicolumn{1}{c}{(2)}&\multicolumn{1}{c}{(3)}&\multicolumn{1}{c}{(4)}\\
          &\multicolumn{1}{c}{D.ln\_medrentpricepsqft\_sfcc}&\multicolumn{1}{c}{D.ln\_medrentpricepsqft\_sfcc}&\multicolumn{1}{c}{D.ln\_medrentpricepsqft\_sfcc}&\multicolumn{1}{c}{D.ln\_medrentpricepsqft\_sfcc}\\
\hline
F6D.ln\_actual\_mw& -0.00700         & -0.00723         & -0.00803         & -0.00753         \\
          &(0.00788)         &(0.00683)         &(0.00689)         &(0.00702)         \\
[1em]
F5D.ln\_actual\_mw&  -0.0113         &  -0.0119         &  -0.0133         &  -0.0122         \\
          &(0.00973)         & (0.0106)         & (0.0104)         & (0.0108)         \\
[1em]
F4D.ln\_actual\_mw& -0.00451         & -0.00508         & -0.00658         & -0.00546         \\
          & (0.0122)         & (0.0113)         & (0.0113)         & (0.0106)         \\
[1em]
F3D.ln\_actual\_mw&  0.00261         &  0.00200         & 0.000518         &  0.00163         \\
          & (0.0114)         & (0.0114)         & (0.0111)         & (0.0116)         \\
[1em]
F2D.ln\_actual\_mw&  0.00404         &  0.00344         &  0.00194         &  0.00314         \\
          & (0.0131)         & (0.0127)         & (0.0130)         & (0.0126)         \\
[1em]
FD.ln\_actual\_mw& -0.00353         & -0.00404         & -0.00560         & -0.00482         \\
          & (0.0109)         & (0.0130)         & (0.0127)         & (0.0134)         \\
[1em]
D.ln\_actual\_mw&   0.0307\sym{**} &   0.0299\sym{**} &   0.0278\sym{**} &   0.0287\sym{**} \\
          & (0.0138)         & (0.0116)         & (0.0120)         & (0.0119)         \\
[1em]
LD.ln\_actual\_mw&   0.0127\sym{**} &   0.0118\sym{*}  &  0.00977         &   0.0108         \\
          &(0.00565)         &(0.00595)         &(0.00586)         &(0.00665)         \\
[1em]
L2D.ln\_actual\_mw& -0.00782         & -0.00877         &  -0.0108         & -0.00976         \\
          & (0.0136)         & (0.0121)         & (0.0122)         & (0.0123)         \\
[1em]
L3D.ln\_actual\_mw&  0.00680         &  0.00583         &  0.00387         &  0.00491         \\
          &(0.00851)         &(0.00745)         &(0.00794)         &(0.00855)         \\
[1em]
L4D.ln\_actual\_mw&   0.0104         &  0.00951         &  0.00788         &  0.00876         \\
          &(0.00741)         &(0.00725)         &(0.00713)         &(0.00809)         \\
[1em]
L5D.ln\_actual\_mw&  0.00913         &  0.00823         &  0.00664         &  0.00749         \\
          &(0.00672)         &(0.00711)         &(0.00693)         &(0.00705)         \\
[1em]
L6D.ln\_actual\_mw&  0.00378         &  0.00394         &  0.00273         &  0.00311         \\
          & (0.0121)         & (0.0117)         & (0.0119)         & (0.0115)         \\
\hline
Zipcode-specifc linear trend&       No         &      Yes         &      Yes         &      Yes         \\
Zipcode-specific linear and square trend&       No         &       No         &      Yes         &      Yes         \\
Zipcode-specific linear, square and cubic trend&       No         &       No         &       No         &      Yes         \\
R-squared & .0224817         & .0245459         & .0267439         & .0290873         \\
Observations&   102078         &   102078         &   102078         &   102078         \\
\hline\hline
\end{tabular}
}
}
        \begin{minipage}{.95\textwidth} \footnotesize
			\vspace{3mm} 
			\textit{Notes}: The table shows
		\end{minipage}
    \end{table}
    
    We also report the p-value of a test of $\beta_{-5} = \beta_{-4} = ... = \beta_{-1} = 0$. In all 3 cases we comfortably fail to reject that all leads are equal to 0 at the 95\% level. This is evidence that the pre-existing time paths of rent per square foot in zipcodes where the MW changed is not significantly different that the one for zipcodes in which the change didn't happened on that same time period. Given this, we estimate a model in which we assume all the leads are 0, and we include 5 lags of the logarithm of the MW (we call this model the distributed lags model). The coefficients from this estimation, plus the ones from the specification including 5 leads and lags are plotted in Figure XXX along with the path of effects implied by the static and distributed lags model. 
    
    \begin{figure}[h!] \centering
        \caption{Implied effect of different models}
        \label{fig:fd_models}
        \includegraphics[width=0.75\linewidth]{analysis/first_differences/output/fd_models.png}
        \begin{minipage}{.95\textwidth} \footnotesize
			\vspace{2mm} 
			\textit{Notes}: Results 
		\end{minipage}
    \end{figure}
    
    We can see that there is no evidence of pretrends, and that the effects implied by the dynamic models are slightly larger than the ones implied by the static one.  
    
    
    \subsubsection{Heterogeneity by income}
    
    In order to explore whether the effects of MW on rents are mainly driven by "minimum wager" zipcodes we divide the set of zipcodes into median household income quintiles from the 2010 census and we run the following model:
    
    \begin{equation}\label{eq:diff_main}
            \Delta \tilde{y}_{it} = \theta_t + \gamma_i + \sum_{q=1}^5 \beta_q \mathds{1}\{i\in q\} \Delta \tilde{MW}_{it}+ \Delta \epsilon_{it}
    \end{equation}
    
    Were $q$ indexes median household income quintiles, and $\mathds{1}\{i\in q\}$ is 1 if zipcode $i$ belongs to the $q$th quintile of median household income. 
    
    The results are displayed in the following figure. As expected, zipcodes in the lower quintile have an effect that significant and much larger in magnitude the baseline effect on the full sample of zipcodes.
    
    Further heterogeneity in Appendix Figure \ref{appfig:fd_heterogeneity_appendix}.
    
    \begin{figure}[h!] \centering
        \caption{Heterogeneity of effect across zipcode median income quintiles}
        \label{fig:fd_heterogeneity_income}
        \includegraphics[width=0.75\linewidth]{analysis/first_differences/output/fd_static_heter_med_hhinc20105.png}
        \begin{minipage}{.95\textwidth} \footnotesize
			\vspace{2mm} 
			\textit{Notes}: Results 
		\end{minipage}
    \end{figure}
    
\subsection{Assessing magnitude of the effects}\label{subsec:results/magnitude}


