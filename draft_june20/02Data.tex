
For the entire US, we assemble a panel at the US postal service zipcode-month level from January 2010 to December 2019. This panel that comes from five distinct sources. 

First, our data contains prominent MW changes at the federal, state, county, and city level.\footnote{Note that federal level MW changes still induce meaningful variation as it is binding in some zipcodes and not in others, so that identification don't come only from time series variation. In our baseline estimates we exclude county and city level changes because they might have different effects on rents and amenities. We include prominent MW changes at those levels in robustness checks, and in specifications that allow explicitly for heterogeneous effects at the different jurisdiction levels.} Most of these changes come from \textcite{vaghul2016historical} and \textcite{cengiz2019effect}, but we updated this data for the years 2017, 2018, and 2019. For each zipcode we only use MW changes that are binding, and that are prominent. We define a MW change as prominent when it is of at least \$0.5 following closely \textcite{cengiz2019effect}\footnote{We conduct for changes of at least \$0.25 and \$0.75.}. We only use MW event changes that have at least 6 months of data after the month of the event.\footnote{For example, if an event happened in July 2019 we would exclude it from our sample of events.} Finally, to avoid dealing with overlapping events, our baseline specifications uses a 6 months window period around the latest MW change event that satisfies our criteria for each zipcode. In robustness checks we follow the methodology of \textcite{cengiz2019effect} to use all events and results are very similar.

Second, we use rent and house value data from properties listed in Zillow \parencite{zillow} in our sample period. To make sure that our data is representative of the time series of the housing market in the US, we compare it with the Fair Market Rents data coming from the US Department of Housing and Urban Development \parencite{hud}, and with the Case-Shiller 20-City Composite Home Price Index \parencite{case}. To ease comparison across zipcodes, we focus on rent and house prices per square foot for single family houses. For this category the Zillow data provides information on rents for 3316 unique zipcodes that correspond to \%8.43 of the zipcodes and to \%36.6 of the 2015 population. The average median household annual income for those zipcodes in 2015 was \$75209, which is \%26.6 higher than the same figure for all the US zipcodes. As for the information on house values, Zillow has data on 10875 unique zipcode that correspond to \%27.7 of the zipcodes and to \%78.9 of the 2015 population. The average median household annual income for those zipcodes in 2015 was \$69556, which is \%17 higher than the same figure for all the US zipcodes. Therefore, our zipcodes are more populous and slightly higher income than the average zipcode. 

Third, we assign to each USPS zipcode a Zip Code Tabulation Area (ZCTA) and we use the census at that level to assign to each zipcode the number of inhabitants, the number of houses, and the median income. We use this information to classify zipcodes into high or low population (or population density) and into high or low median income. In addition, given that zipcodes can cross county borders, we use the census data and geographic codes to map each zipcode to a county by assigning it to the one that has the highest share of houses from that zipcode among all counties in which the zipcode has houses. We also map to each zipcode to a metropolitan statistical area or a rural town analogously.

Fourth, to proxy for the quality of amenities at each location, we construct zipcode-month level measures from GPS location point-of-interest data by SafeGraph\parencite{safegraph}. We define several amenity measures. For our first measure, we follow closely \textcite{couture2019income} and construct an index for the quality of restaurants available at each zipcode-month. This index is defined as the average of the propensity of high-income individuals to visit a restaurant in a given zipcode controlling for their distance to the establishments. In order to classify a visitor as high-income we use the census tract location of the visitor. Our second measure is the same as the first but for the quality of the visitors of open public spaces in each zipcode-month. For our third measure, we use the point-of-interest data to count the number of restaurants, coffee shops, bars, and gyms per inhabitant of a zipcode-month. 

Finally, we collect data from the Quarterly Census of Employment and Wages at the county-quarter level. For each county-quarter, and for each 5-digit NAICS industry code, we observe the number of establishment, the number of employed people, and the average weekly wage. We merge this data into our zipcode-month panel, based on the county and quarter that they belong, and we classify counties into high or low wage areas, or into high or low employment area, and for building controls on the level of employment and economic activity. 
