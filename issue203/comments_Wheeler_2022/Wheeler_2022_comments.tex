%%%%%%%%%%%%%%%%%%%%%%%%%%%%%%%%%%%%%%%%%%%%%%%%%%%%%%%%%%%%%%%%%%%%%%%%%%%%%%%%%%%%%%%%%%
%                                     SETTINGS                                           %
%%%%%%%%%%%%%%%%%%%%%%%%%%%%%%%%%%%%%%%%%%%%%%%%%%%%%%%%%%%%%%%%%%%%%%%%%%%%%%%%%%%%%%%%%%
\documentclass[aspectratio=169, t]{beamer}

\usepackage{amsmath,amssymb}
\usepackage{graphics}
\usepackage{graphicx}
\usepackage{subcaption}
\usepackage{hyperref}

\usetheme{default}

\setbeamertemplate{itemize items}[circle]
\setbeamertemplate{footline}[frame number]
\setbeamertemplate{navigation symbols}{}


%%%%%%%%%%%%%%%%%%%%%%%%%%%%%%%%%%%%%%%%%%%%%%%%%%%%%%%%%%%%%%%%%%%%%%%%%%%%%%%%%%%%%%%%%%
%                                     TITLE                                              %
%%%%%%%%%%%%%%%%%%%%%%%%%%%%%%%%%%%%%%%%%%%%%%%%%%%%%%%%%%%%%%%%%%%%%%%%%%%%%%%%%%%%%%%%%%
\title{Locally Optimally Place-Based Policies: \\ 
       Early Evidence from Opportunity Zones}
\subtitle{\vspace{3mm} Comments for the 11th European Meeting of the UEA}
\date{April 29, 2022}
\author{Santiago Hermo\vspace{-3mm} }
\institute{Brown University}

%%%%%%%%%%%%%%%%%%%%%%%%%%%%%%%%%%%%%%%%%%%%%%%%%%%%%%%%%%%%%%%%%%%%%%%%%%%%%%%%%%%%%%%%%%
%                                         BODY                                           %
%%%%%%%%%%%%%%%%%%%%%%%%%%%%%%%%%%%%%%%%%%%%%%%%%%%%%%%%%%%%%%%%%%%%%%%%%%%%%%%%%%%%%%%%%%
\begin{document}
\maketitle


%%% Introduction %%%

\begin{frame}
    \frametitle{Overview of the paper}
    
    Opportunity Zones (OZs): Place-based policy that aims to bolster development in disadvantaged neighborhoods.
    \begin{itemize}
        \item Did it cause new development in the city? Where?
        \item How should policy-makers select OZs to maximize their impact?
    \end{itemize}
    
    \pause
    \vspace{4mm}
    Steps to answer these questions:
    \begin{itemize}
        \item Collect data on new developments at granular level.
        \item Causal analysis to estimate the new development response to OZs.
        \item Structural analysis to assess efficiency of policy relative to \textit{locally optimal} benchmark.
    \end{itemize}

\end{frame}

\begin{frame}
    \frametitle{Comments on the causal analysis}
    
    A

\end{frame}

\begin{frame}
    \frametitle{Comments on the structural analysis}
    
    A

\end{frame}

\end{document}