%%%%%%%%%%%%%%%%%%%%%%%%%%%%%%%%%%%%%%%%%%%%%%%%%%%%%%%%%%%%%%%%%%%%%%%%%%%%%%%%%%%%%%%%%%
%                                     SETTINGS                                           %
%%%%%%%%%%%%%%%%%%%%%%%%%%%%%%%%%%%%%%%%%%%%%%%%%%%%%%%%%%%%%%%%%%%%%%%%%%%%%%%%%%%%%%%%%%
\documentclass[aspectratio=169, t]{beamer}

\usepackage{amsmath,amssymb}
\usepackage{graphics}
\usepackage{graphicx}
\usepackage{subcaption}
\usepackage{hyperref}

\usetheme{default}

\setbeamertemplate{itemize items}[circle]
\setbeamertemplate{footline}[frame number]
\setbeamertemplate{navigation symbols}{}


%%%%%%%%%%%%%%%%%%%%%%%%%%%%%%%%%%%%%%%%%%%%%%%%%%%%%%%%%%%%%%%%%%%%%%%%%%%%%%%%%%%%%%%%%%
%                                     TITLE                                              %
%%%%%%%%%%%%%%%%%%%%%%%%%%%%%%%%%%%%%%%%%%%%%%%%%%%%%%%%%%%%%%%%%%%%%%%%%%%%%%%%%%%%%%%%%%
\title{Locally Optimal Place-Based Policy: \\ 
       Early Evidence from Opportunity Zones}
\subtitle{\vspace{3mm} Comments for the 11th European Meeting of the UEA}
\date{April 29, 2022}
\author{Santiago Hermo\vspace{-3mm} }
\institute{Brown University}

%%%%%%%%%%%%%%%%%%%%%%%%%%%%%%%%%%%%%%%%%%%%%%%%%%%%%%%%%%%%%%%%%%%%%%%%%%%%%%%%%%%%%%%%%%
%                                         BODY                                           %
%%%%%%%%%%%%%%%%%%%%%%%%%%%%%%%%%%%%%%%%%%%%%%%%%%%%%%%%%%%%%%%%%%%%%%%%%%%%%%%%%%%%%%%%%%
\begin{document}
\maketitle


%%% Introduction %%%

\begin{frame}
    \frametitle{Overview of the paper}
    
    Opportunity Zones (OZs): Place-based policy that aims to bolster development in disadvantaged neighborhoods.
    \begin{itemize}
        \item Did it cause new development in the city? Where?
        \item How should local policy-makers select OZs to maximize their impact?
    \end{itemize}
    
    \pause
    \vspace{4mm}
    Steps to answer these questions:
    \begin{itemize}
        \item Collect data on new developments at granular level.
        \item Causal analysis to estimate the new development response to OZs.
        \item Structural analysis to assess efficiency of policy relative to \textit{locally optimal} benchmark.
    \end{itemize}

\end{frame}

\begin{frame}
    \frametitle{Comments on the causal analysis}
    
    The author implements many strategies to convince that there is an effect.
    \begin{itemize}
        \item Difference-in-differences, matching, synthetic control, placebos.
    \end{itemize}

    \pause
    \vspace{2mm}
    Comments
    \begin{itemize}
        \vspace{1mm}
        \item Spillovers to close by tracts:
        \begin{itemize}
            \item Not accounted for explicitly in main estimation, yet later they are estimated.
            \item Unified estimation strategy to include spillovers in the design? (Butts 2021)
        \end{itemize}
        \pause
        \vspace{1mm}
        \item Reallocation effects:
        \begin{itemize}
            \item The paper finds no reallocation effects, so total development increases.
            \item Who are funding the new development? Is the tax break enough? Are new developments cheaper?
            %\item City-level design to estimate total effect of program on new development?
        \end{itemize}
        \pause
        \vspace{1mm}
        \item Is housing getting more affordable in OZs?
        %% Paper doesn't look at any price variable
    \end{itemize}
\end{frame}

\begin{frame}
    \frametitle{Comments on the structural analysis}
    
    Structural model of probability of new development in a location.
    \begin{itemize}
        \item Full-information rational-expectations solution concept.
        \item Estimates model selecting most likely equilibrium.
        \item Counterfactual analysis under stationary distribution implied by model.
    \end{itemize}

    \pause
    \vspace{2mm}
    Comments
    \begin{itemize}
        \vspace{1mm}
        \item Model is not micro-founded:
        \begin{itemize}
            \item No supply and demand of new development.
            \item Hard to talk about welfare of people the program tries to help.
        \end{itemize}
        \pause
        \vspace{1mm}
        \item Can OZs shift the equilibrium in the city?
        \pause
        \vspace{1mm}
        \item Reallocation effects again:
        \begin{itemize}
            \item Estimated model implies that total development in the city went up. What are the mechanisms?
            \item Important to understand mechanisms to evaluate plausibility of estimates.
        \end{itemize}
    \end{itemize}
\end{frame}

\end{document}
