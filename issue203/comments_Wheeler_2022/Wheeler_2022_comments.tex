%%%%%%%%%%%%%%%%%%%%%%%%%%%%%%%%%%%%%%%%%%%%%%%%%%%%%%%%%%%%%%%%%%%%%%%%%%%%%%%%%%%%%%%%%%
%                                     SETTINGS                                           %
%%%%%%%%%%%%%%%%%%%%%%%%%%%%%%%%%%%%%%%%%%%%%%%%%%%%%%%%%%%%%%%%%%%%%%%%%%%%%%%%%%%%%%%%%%
\documentclass[aspectratio=169, t]{beamer}

\usepackage{amsmath,amssymb}
\usepackage{graphics}
\usepackage{graphicx}
\usepackage{subcaption}
\usepackage{hyperref}

\usetheme{default}

\setbeamertemplate{itemize items}[circle]
\setbeamertemplate{footline}[frame number]
\setbeamertemplate{navigation symbols}{}


%%%%%%%%%%%%%%%%%%%%%%%%%%%%%%%%%%%%%%%%%%%%%%%%%%%%%%%%%%%%%%%%%%%%%%%%%%%%%%%%%%%%%%%%%%
%                                     TITLE                                              %
%%%%%%%%%%%%%%%%%%%%%%%%%%%%%%%%%%%%%%%%%%%%%%%%%%%%%%%%%%%%%%%%%%%%%%%%%%%%%%%%%%%%%%%%%%
\title{Locally Optimally Place-Based Policies: \\ 
       Early Evidence from Opportunity Zones}
\subtitle{\vspace{3mm} Comments for the 11th European Meeting of the UEA}
\date{April 29, 2022}
\author{Santiago Hermo\vspace{-3mm} }
\institute{Brown University}

%%%%%%%%%%%%%%%%%%%%%%%%%%%%%%%%%%%%%%%%%%%%%%%%%%%%%%%%%%%%%%%%%%%%%%%%%%%%%%%%%%%%%%%%%%
%                                         BODY                                           %
%%%%%%%%%%%%%%%%%%%%%%%%%%%%%%%%%%%%%%%%%%%%%%%%%%%%%%%%%%%%%%%%%%%%%%%%%%%%%%%%%%%%%%%%%%
\begin{document}
\maketitle


%%% Introduction %%%

\begin{frame}
    \frametitle{Overview of the paper}
    
    Opportunity Zones (OZs): Place-based policy that aims to bolster development in disadvantaged neighborhoods.
    \begin{itemize}
        \item Did it cause new development in the city? Where?
        \item How should policy-makers select OZs to maximize their impact?
    \end{itemize}
    
    \pause
    \vspace{4mm}
    Steps to answer these questions:
    \begin{itemize}
        \item Collect data on new developments at granular level.
        \item Causal analysis to estimate the new development response to OZs.
        \item Structural analysis to assess efficiency of policy relative to \textit{locally optimal} benchmark.
    \end{itemize}

\end{frame}

\begin{frame}
    \frametitle{Comments on the causal analysis}
    
    The author does a lot of things to convince that the effect is there.
    \begin{itemize}
        \item The effect seems to always be there.
    \end{itemize}

    \vspace{2mm}
    Comments are conceptual:
    \begin{itemize}
        \pause
        \vspace{1mm}
        \item Spillovers to close by tracts
        \begin{itemize}
            \item Not accounted for explicitly in main estimation, yet later they are estimated.
            \item Would it be better to device a unified estimation strategy to include them from scratch? (Butts 2021)
        \end{itemize}
        \pause
        \vspace{1mm}
        \item Reallocation effects
        \begin{itemize}
            \item The paper finds no reallocation effects, so total development increases.
            \item Who are funding the new development? Is the tax break enough? Are new developments cheaper?
        \end{itemize}
        \pause
        \vspace{1mm}
        \item Organization of section
        \begin{itemize}
            \item I find it more natural to discuss all issues at first, and show results later.
        \end{itemize}
    \end{itemize}

\end{frame}

\begin{frame}
    \frametitle{Comments on the structural analysis}
    
    A

\end{frame}

\end{document}