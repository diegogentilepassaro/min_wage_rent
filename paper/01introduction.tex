%%%%%%%%%%%%%%%%%%%%%%%%%%%%%%%%%%%%%%%%%%%%%%%%%%%%%%%%%%%%%%%%%%%%%%%%%%%%%%%%%
%%%%%                            INTRODUCTION                                %%%%
%%%%%%%%%%%%%%%%%%%%%%%%%%%%%%%%%%%%%%%%%%%%%%%%%%%%%%%%%%%%%%%%%%%%%%%%%%%%%%%%%

In recent years, many US jurisdictions have introduced minimum wages (hereafter MW) above the 
federal level of \$7.25.\footnote{As of January 2020, there were 29 states with a MW larger 
	than the federal one, 52 counties that set a MW above the state, and 15 cities with a 
	minimum above the county.}
Following the early work of \textcite{CardKrueger2000}, most research effort has been devoted 
to understanding the effects of MW policies on employment \parencite[e.g.,][]{Neumark2006, 
DubeEtAl2010, MeerWest2016, CegnizEtAl2019} and income inequality \parencite{Lee1999, 
AutorEtAl2016}. This is not surprising, as employment effects are of first order importance 
to determine the welfare implications of MW changes on households. However, the 
\textit{place-based} nature of MW provisions (accentuated by the fact that most recent 
legislation arise from local jurisdictions) make it natural to expect that such policies 
will affect thewelfare of households through other channels, such as the housing market. 
Not accounting for the potential effect of MW changes on rents is tantamount to omitting 
from the analysis one of the main channels trough which these policies may affect welfare 
and inequality.

Given these remarks, we pose the question: by what extent (if any) are local rents affected 
by minimum wage policies? Surprisingly, there is very little research attempting to estimate 
the causal effect of minimum wages on the housing market. To the best of our knowledge, the 
only papers aiming at answering this question directly are \textcite{Yamagishi2019, 
Yamagishi2020, Tidemann2018}.\footnote{\textcite{Yamagishi2019} explores this question using 
data from both the U.S. and Japan. In an update version of the paper, 
\textcite{Yamagishi2020} excludes the analysis of the U.S. case.} These papers find opposing 
results, even though they use the same year-county data for the U.S. \textcite{Yamagishi2019} 
finds a small positive effect, while \textcite{Tidemann2018} finds a small negative 
effect.\footnote{\textcite{Yamagishi2019} attributes this difference to different model 
	specifications, and argues that with proper standard errors clustering the results in 
	\textcite{Tidemann2018} are statistically insignificant. We will compare our paper to 
	those.} 
In a related paper, \textcite{AgarwalEtAl2019} shows that minimum wages decrease 
the probability of rental default, suggesting a strengthening of the local labor market.
% Add

Provided that MW policies have small disemployment effects, theory suggests that the effect 
on rents will be positive. A canonical version of the Alonso-Muth-Mills model, for example, 
predicts that general wage increases will be fully capitalized by landlords.\footnote{See 
	\textcite{Brueckner1987} for a complete treatment of this model.} 
In the same tradition, \textcite{Yamagishi2020} shows that minimum wage policies increase 
rents if disemployment effects are small, and they are a sufficient statistic of welfare 
under free mobility. We being our paper by constructing a simple model of a zipcode's rental 
market, and argue that the effect should be positive. [UNDER CONSTRUCTION -- WORK MODEL AND 
UPDATE]
 
Understanding the effects of MW policies on the housing market is important both from a 
scientific and policy-making point of view. As recent literature has shown, individuals 
respond to changes in local prices and amenities by migrating, which can have important 
implications in welfare and inequality \parencite{Diamond2016, Couture2019}. Several papers 
make a similar point for the case of MW policies, arguing that they influence migration 
decisions and the location of economic activity\parencite{PerezPerez2018, Monras2019}. A more 
complete understanding of the benefits and disadvantages of MW policies should necessarily 
include the effects on the housing market, informing policy decisions as well.
 
In this paper, we construct a dataset at the U.S. zipcode and monthly date levels to explore the reduced form effects of MW changes on rents. Our main rent variable comes from Zillow, the largest real-estate company in the US, and corresponds to the median rent price per square foot across Zillow listings in the given zipcode-month cel of the category Single Family and Condos (SFCC)l. This is the most popular housing category in the US (CITE NEEDED), and also the most populated series in Zillow. However, a drawback is that only a subsample of all U.S. zipcodes are included. We deal with the issue of representativeness of our sample in the paper. We also collect data on minimum wage changes from \textcite{VaghulZipperer2016} for the period from 1974 to 2016, which we update until January 2020. Using these data, we construct the actual minimum wage in force in each zipcode. We collect data from other sources to both validate our empirical model and to deploy as controls in our regressions, including the Quarterly Census of Employment and Wages (QCEW) and the Building Permits Survey (BPS). Finally, we use data from the U.S. Census and from the  LEHD Origin-Destination Employment Statistics (LODES) to explore heterogeneity of the effect of interest.\footnote{LEHD is short for Longitudinal Employer-Household Dynamics, which corresponds the source of the origin-destination data.}

Estimating the effect of MW policies on rents presents several challenges. First of all, it appears plausible that determinants of local level MW changes might correlate with geographical and time factors also affecting the housing market, invalidating naive OLS regressions. To account for this, we use difference-in-differences (DiD) panel specifications that condition both on monthly date and zipcode fixed effects \parencite[in the same spirit as ][]{MeerWest2016}. Identification comes from exploiting the size and fine timing of hundreds of MW changes staggered across different US jurisdictions from 2010 to 2019. As a result, this specification does not suffer from the underidentification problem arising when units are treated only once \parencite{BorusyakJaravel2017}. This specification yields the true causal effect of MW changes on rents assuming that, within a zipcode, time-varying factors leading to MW changes are not related to unobservable determinants of the rental price dynamics.

This specification, however, rules out dynamic effects. Furthermore, identification relies on a parallel trends assumption in the time-path of treated and untreated zipcodes. Intuitively, if effects are being driven by some preexisting time-varying unobserved difference between treated and untreated zipcodes, we should see that future MW changes have an effect in the rental prices. On the other hand, if MW changes can be thought of exogenous with respect to the zipcode rental market (as assumed by our model), we should see no anticipatory effects.\footnote{} Motivated by this, we extend our static DiD model to include leads and lags of minimum wage changes. This dynamic model allows us to explore the dynamics of the effect of interest and to test the parallel trends assumption. 

We provide several tests for the validity of our identification strategy: first, we test for differential pre-trends between treated and control 
units. We do this using the insight from \textcite{granger1969investigating}: we add leads of the 
MW changes and show that there are not anticipatory effects in treated zipcodes relative to 
untreated ones.  Second, we check for the presence of unobservables affecting both 
rents and MW changes with proxies for local economic shocks as well as shocks to the housing market. 
Third, we allow for unobserved shocks to rent prices to be not iid by including a lagged dependent 
variable in our specification. Our results survive all of those tests. In addition, our identification 
strategy has the advantage of plausibly having more external validity than research based on a few 
case studies as it scrutinizes the dynamics of rental prices around hundreds of MW changes all over 
the US in a long period of time. In addition, we test the degree of sample bias by reweighting our 
data to match demographic characteristics of the average US zipcode. Our effects not only survive 
but are bigger and more precisely estimated. Finally, we make sure that our effects are not driven 
by changes in the composition of zipcodes appearing in our data through estimating our model in 
under both balanced and unbalanced panel data sets. 

Our results reveal a small yet robust impact of MW changes on rents. The \textit{static} 
difference-in-differences specification shows how a 10 percent increase in the MW leads to an average 
0.26 percent increase in the rental price per square foot. When expanding the model to account for 
\textit{dynamic} effects, we find a statistically significant impact in the first two months following 
a MW change: for the average zipcode a 10 percent increase in MW rents increase between 0.25 and 
0.5 percent. In an effort to understand who are the ``winners and losers", we perform an heterogeneity 
analysis of the estimated impact that reveals how results are driven by effects in zipcodes that are 
more likely to have minimum wagers as residents: those that have a highest share of unemployed, lower 
household income, and a larger share of African-American population. The pass-through for these 
zipcodes is around twice as large. Consistently, we show that zipcodes with very low probability of 
having minimum wage workers as residents exhibit no significant effects. On the other hand, we find 
that the effect is constant across zipcodes with different share of MW workers that work there.

Our approach has several differences with respect to previous research on the topic. Both 
\textcite{tidemann2018mw} and \textcite{yamagishi2019minimum} use Fair Markets Rents data which is 
available at the yearly level and aggregated at the geographical level of counties.\footnote{
	\textcite{yamagishi2019minimum} also uses data at the year-prefecture level for the 47 Japanese 
	prefectures.} 
An important advantage of our approach is that we use the exact timing of the MW change at the monthly 
level. When using variation arising from a yearly frequency some units are "partially treated" which 
will tend to understate the magnitude of the effect. Furthermore, some jurisdictions have MW changes 
on many subsequent years, making it challenging to estimate the dynamics around changes that are 
followed by changes in the immediate year. For example, if there is a change in two subsequent years, 
then the estimated effect of the change in the second year may be due too the effect of the current MW 
change or to the past MW change or both. We are able to show that raising the MW increases rents 
significantly only in the first couple of months after implementation.

An important difference is that we use data at the zipcode- instead of the county-level. As of 2019 
there were 3,142 counties and 39,295 meaningful zipcodes in the US.\footnote{We exclude military and 
unique business zipcodes as they are irrelevant for house prices.} We illustrate the importance of 
having smaller units of analysis with the following example. For a given county \textit{a}, suppose 
that (1) all low skill jobs are in one particular zipcode; and (2) low skill households prefer to live 
near their jobs. Further assume that, following a MW change, employment effects are near 
zero.\footnote{This is consistent with the findings of \textcite{card2000minimum} and 
	\textcite{cengiz2019effect}, among others.}
One should then expect demand for housing in the zipcode with the low skill jobs to increase and 
demand for housing in the rest of the zipcodes to go down. If we focus on the effects of the MW 
increase on the county we might even find that the rents go down, when in fact the rents in the 
zipcodes were the low skill jobs are located are increasing. Indeed, \textcite{tidemann2018mw} found 
that a \$1 increase in the MW decreases the yearly average of the monthly rent by 1.5 percentage 
points.\footnote{As pointed out by \textcite{tidemann2018mw}, the sign of this effect implies that 
	the labor demand for low skilled workers is elastic. This is at odds with the results from 
	\textcite{card2000minimum}, \textcite{cengiz2019effect}, and many others.}

% If we add amenities to the example, matters become much more complicated but if we allow for high 
% skill people to value amenities differently than low skill people (like in 
% \textcite{diamond2016determinants}), we may also expect to see residential resorting of high skill 
% people depending on where are the amenities located, whether the amenities respond endogenously to 
% the high-low skill composition of the zipcode, and depending on which are the zipcodes that the low 
% skilled workers are demanding less. This example, illustrates that as the spatial distribution of 
% jobs and amenities varies at the very local level, when tastes are heterogeneous by skill (or some 
% other dimension) focusing on a large geographic area may be misleading. 

A second advantage of having a more detailed geography is that we can also exploit MW changes at any 
jurisdictional level, effectively increasing the number of events used for identification. This is 
interesting because MW changes at different jurisdiction levels may have different local effects on 
rents.
% Intuitively, this is the case because, for example, out-of-state migration is in principle more 
% costly than out-of-county migration.  therefore, we expect more residential resorting within a state 
% and across counties when a county changes their local MW wage. Our data allows us to study the 
% heterogeneous effects of different MW changes.\footnote{In principle, our data allows us to answer 
% whether the effects of changes at the federal, state, county, and city/town level are different.} 

% we can use the census to compute the level of employment and the distribution of income in each %  
% zipcode, and check if we observe stronger effects on rents in places where there are more MW earners 
% or in places where there is more low skilled employment. 

A third difference of our approach is that by exploiting data at the zipcode monthly level we can go 
well beyond two-way county-year fixed-effect models. This is important because the dynamics of the 
rental market plausibly vary across zipcodes within a county following trends at the very local level 
\parencite{almagro2019location}. Our baseline specification has zipcode and monthly date fixed 
effects. Furthermore, in order to allow for heterogeneity in rental dynamics at the zipcode level we 
allow for zipcode-specific linear and quadratic terms. This in turn has two advantages. First, it 
makes our estimates more precise than the previously available ones. Second, and most importantly, it 
makes the required identification assumptions substantially more plausible. Given that the identifying 
variation comes from within-zipcodes, the determinants of these MW changes are unlikely to be related 
to the particular zipcode, and therefore, are less likely to be correlated to the unobservable 
determinants of rent dynamics there.

% A fourth important difference with past work, is that to the best of our knowledge, this is the 
% first study that estimates the effect of MW changes on amenities. As pointed out recently by 
% \textcite{diamond2016determinants, almagro2019location}, taking into account non-pecuniary  
% dimensions of the utility function can change substantively the welfare computations and the 
% incidence of policy relevant changes. We exploit the use of high frequency GPS data to build amenity 
% measures at the zipcode-month level and estimate the effect of MW changes on them. We hope that by 
% incorporating amenities into the picture we can give a more comprehensive assessment of the effects 
% of MW policies.

LIMITATIONS. Zipcodes enter non-randomly in Zillow sample (representativeness of sample). We capture effect on newly rented houses only.

Beyond the contribution to the very recent literature on the effects of MW changes on rents, we 
contribute to several strands of the economic literature. First, we contribute to the literature 
studying the effects of minimum wages on the welfare of low skill households. However, instead of 
focusing on wages and employment \parencite[][among others]{dinardo1995labor, autor2016contribution, 
card2000minimum, neumark2006minimum, jardim2017minimum}, we contribute to this strand of literature 
by taking into account the effects on the housing market.

Second, our work relates to the literature that studies the location decision of agents either based 
on income \parencite{roback1982wages, kennan2011effect, desmet2013urban, perez2018city} or based on 
spatial rent and amenity differentials \parencite{diamond2016determinants, almagro2019location, 
couture2019income, bayer2004equilibrium}. We hope to contribute by adapting this framework to the 
case of the MW changes, so that we can rationalize through residential location sorting part of 
the observed reduce form effect on rents and amenities.

The rest of the paper is organized as follows. Initially, section \ref{sec:model} motivates the paper
with a simple model of the rental market. In section \ref{sec:data}, we present our data sources and 
show the characteristics of our estimating panel. In section \ref{sec:empirical_strategy}, we explain 
our empirical strategy and we discuss our identification assumptions. In section \ref{sec:results}, we 
present our main results. Section \ref{sec:discussion} discusses relevant policy implications, and 
section \ref{sec:conclusion} concludes.
