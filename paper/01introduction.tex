%%%%%%%%%%%%%%%%%%%%%%%%%%%%%%%%%%%%%%%%%%%%%%%%%%%%%%%%%%%%%%%%%%%%%%%%%%%%%%%%%
%%%%%                            INTRODUCTION                                %%%%
%%%%%%%%%%%%%%%%%%%%%%%%%%%%%%%%%%%%%%%%%%%%%%%%%%%%%%%%%%%%%%%%%%%%%%%%%%%%%%%%%

In recent years, many US jurisdictions have introduced minimum wages (hereafter MW) above the 
federal level of \$7.25.\footnote{As of January 2020, there were 29 states that set a minimum 
	wage higher than the federal minimum, 52 counties that set a higher minimum wage than the 
	state, and 15 cities that set a higher minimum wage than the county.}
Despite prominent debates on recent MW policies both at the local and national level, ever since 
\textcite{card2000minimum} most research effort has been devoted to understanding the effects of 
MW on employment \parencite{neumark2006minimum, dube2010minimum, dube2016minimum}. This is not 
surprising, as employment effects are of first order importance to understand the welfare 
implications of MW changes on households. However, the \textit{place-based} nature of MW provisions 
make it natural to expect such policies to affect the welfare of households through markets other 
than the labor one. By far, the most prominent candidate to investigate is the housing market, and 
the channels through which it may fuel growing disparity in income and opportunities across the US. 
How much do local rents react to MW changes? Surprisingly, there is very little research estimating 
such effect,\footnote{To our knowledge the only papers aiming at answering this question are 
	\textcite{yamagishi2019minimum} and \textcite{tidemann2018mw} Both papers found opposing results 
	despite using the same year-county data. \textcite{yamagishi2019minimum} finds a small positive 
	effect, while \textcite{tidemann2018mw} finds a small negative effect. 
	\textcite{yamagishi2019minimum} attributes this difference to different model specifications, 
	and argues that with proper standard errors clustering the results in \textcite{tidemann2018mw} 
	are statistically insignificant. We will soon explain the differences of this paper with those.}
and virtually no research estimating the effects on local amenities.

A canonical version of the Alonso-Muth-Mills monocentric city  model with homogeneous agents 
predicts that wage increases should be fully capitalized by landlords, as workers end up paying 
higher rents in all locations.\footnote{See \textcite{brueckner1987structure} for a complete treatment 
	of this model.} 
However, we lack a clear empirical estimate of how big the pass-through from a MW change to 
rents is. This simple example illustrates how the welfare implications and incidence of MW changes
may very well depend on what happens to rents. Furthermore, if the pass-through to rents is high, 
we may also expect a response of local amenities through residential sorting. As recently emphasized
by \textcite{diamond2016determinants}, accounting for the welfare implications of amenities may be 
important.
 
In this paper, we use data at the zipcode-month level to assess the reduced form effects of 
MW changes on rents. Estimating empirically the pass-through from MW to rents is relevant both 
from a policy and theoretical perspective. As shown by \textcite{agarwal2019minimum}, if landlords 
know that their tenants have more disposable income, raising the rent will have two consequences. 
On one hand, it increases the landlords revenue conditional on receiving the rent payment. On the 
other hand, it increases the probability of tenants defaulting their payment.

Estimating the MW-rent pass-through is empirically challenging as it requires exogenous variation 
in the MW. It appears plausible that determinants of local level MW changes might correlate with 
geographical and time factors also affecting the housing market. To overcome this challenge, we 
follow \textcite{meer2016effects} and use several empirical approaches based on the panel data 
literature. Our main specification builds on a panel difference-in-differences (DiD) strategy 
that exploits the size and the fine timing of hundreds of MW changes across different US jurisdictions 
from 2010 to 2019. However, our approach differs from the usual DiD as we use insights from the 
event-study literature and from \textcite{arellano1991some}: we are able in this way to take into 
account both the potential dynamic effects of MW changes on rents and the persistence of the shock 
to the rental dynamics at the local level. MW changes are staggered across zipodes and dates, and 
this allows us to rely on within zipcode variation around MW changes to estimate the relevant 
pass-through by controlling for zipcode and time fixed effects. Since many zipcodes experience 
multiple MW changes, our specifications do not suffer from the underidentification problem arising 
when units are treated only once \parencite{BorusyakJaravel2017}.

Our baseline specification yields the true causal effect of MW changes on rents assuming that, 
within a zipcode, time-varying factors leading to MW changes are not related to unobservable 
determinants of the rental price dynamics. We provide several tests for the validity of our 
identification strategy: first, we test for differential pre-trends between treated and control 
units. We do this using the insight from \textcite{granger1969investigating}: we add leads of the 
MW changes and show that there are not anticipatory effects in treated zipcodes relative to 
untreated ones. Intuitively, if effects are being driven by some preexisting time-varying 
unobserved difference between treated and untreated zipcodes we should see that future MW changes 
have effect in the rental prices. Second, we check for the presence of unobservables affecting both 
rents and MW changes with proxies for local economic shocks as well as shocks to the housing market. 
Third, we allow for unobserved shocks to rent prices to be not iid by including a lagged dependent 
variable in our specification. Our results survive all of those tests. In addition, our identification 
strategy has the advantage of plausibly having more external validity than research based on a few 
case studies as it scrutinizes the dynamics of rental prices around hundreds of MW changes all over 
the US in a long period of time. In addition, we test the degree of sample bias by reweighting our 
data to match demographic characteristics of the average US zipcode. Our effects not only survive 
but are bigger and more precisely estimated. Finally, we make sure that our effects are not driven 
by changes in the composition of zipcodes appearing in our data through estimating our model in 
under both balanced and unbalanced panel data sets. 

Our results reveal a small yet robust impact of MW changes on rents. The \textit{static} 
difference-in-differences specification shows how a 10 percent increase in the MW leads to an average 
0.26 percent increase in the rental price per square foot. When expanding the model to account for 
\textit{dynamic} effects, we find a statistically significant impact in the first two months following 
a MW change: for the average zipcode a 10 percent increase in MW rents increase between 0.25 and 
0.5 percent. In an effort to understand who are the ``winners and losers", we perform an heterogeneity 
analysis of the estimated impact that reveals how results are driven by effects in zipcodes that are 
more likely to have minimum wagers as residents: those that have a highest share of unemployed, lower 
household income, and a larger share of African-American population. The pass-through for these 
zipcodes is around twice as large. Consistently, we show that zipcodes with very low probability of 
having minimum wage workers as residents exhibit no significant effects. On the other hand, we find 
that the effect is constant across zipcodes with different share of MW workers that work there.

Our approach has several differences with respect to previous research on the topic. Both 
\textcite{tidemann2018mw} and \textcite{yamagishi2019minimum} use Fair Markets Rents data which is 
available at the yearly level and aggregated at the geographical level of counties.\footnote{
	\textcite{yamagishi2019minimum} also uses data at the year-prefecture level for the 47 Japanese 
	prefectures.} 
An important advantage of our approach is that we use the exact timing of the MW change at the monthly 
level. When using variation arising from a yearly frequency some units are "partially treated" which 
will tend to understate the magnitude of the effect. Furthermore, some jurisdictions have MW changes 
on many subsequent years, making it challenging to estimate the dynamics around changes that are 
followed by changes in the immediate year. For example, if there is a change in two subsequent years, 
then the estimated effect of the change in the second year may be due too the effect of the current MW 
change or to the past MW change or both. We are able to show that raising the MW increases rents 
significantly only in the first couple of months after implementation.

An important difference is that we use data at the zipcode- instead of the county-level. As of 2019 
there were 3,142 counties and 39,295 meaningful zipcodes in the US.\footnote{We exclude military and 
unique business zipcodes as they are irrelevant for house prices.} We illustrate the importance of 
having smaller units of analysis with the following example. For a given county \textit{a}, suppose 
that (1) all low skill jobs are in one particular zipcode; and (2) low skill households prefer to live 
near their jobs. Further assume that, following a MW change, employment effects are near 
zero.\footnote{This is consistent with the findings of \textcite{card2000minimum} and 
	\textcite{cengiz2019effect}, among others.}
One should then expect demand for housing in the zipcode with the low skill jobs to increase and 
demand for housing in the rest of the zipcodes to go down. If we focus on the effects of the MW 
increase on the county we might even find that the rents go down, when in fact the rents in the 
zipcodes were the low skill jobs are located are increasing. Indeed, \textcite{tidemann2018mw} found 
that a \$1 increase in the MW decreases the yearly average of the monthly rent by 1.5 percentage 
points.\footnote{As pointed out by \textcite{tidemann2018mw}, the sign of this effect implies that 
	the labor demand for low skilled workers is elastic. This is at odds with the results from 
	\textcite{card2000minimum}, \textcite{cengiz2019effect}, and many others.}

% If we add amenities to the example, matters become much more complicated but if we allow for high 
% skill people to value amenities differently than low skill people (like in 
% \textcite{diamond2016determinants}), we may also expect to see residential resorting of high skill 
% people depending on where are the amenities located, whether the amenities respond endogenously to 
% the high-low skill composition of the zipcode, and depending on which are the zipcodes that the low 
% skilled workers are demanding less. This example, illustrates that as the spatial distribution of 
% jobs and amenities varies at the very local level, when tastes are heterogeneous by skill (or some 
% other dimension) focusing on a large geographic area may be misleading. 

A second advantage of having a more detailed geography is that we can also exploit MW changes at any 
jurisdictional level, effectively increasing the number of events used for identification. This is 
interesting because MW changes at different jurisdiction levels may have different local effects on 
rents.
% Intuitively, this is the case because, for example, out-of-state migration is in principle more 
% costly than out-of-county migration.  therefore, we expect more residential resorting within a state 
% and across counties when a county changes their local MW wage. Our data allows us to study the 
% heterogeneous effects of different MW changes.\footnote{In principle, our data allows us to answer 
% whether the effects of changes at the federal, state, county, and city/town level are different.} 

% we can use the census to compute the level of employment and the distribution of income in each %  
% zipcode, and check if we observe stronger effects on rents in places where there are more MW earners 
% or in places where there is more low skilled employment. 

A third difference of our approach is that by exploiting data at the zipcode monthly level we can go 
well beyond two-way county-year fixed-effect models. This is important because the dynamics of the 
rental market plausibly vary across zipcodes within a county following trends at the very local level 
\parencite{almagro2019location}. Our baseline specification has zipcode and monthly date fixed 
effects. Furthermore, in order to allow for heterogeneity in rental dynamics at the zipcode level we 
allow for zipcode-specific linear and quadratic terms. This in turn has two advantages. First, it 
makes our estimates more precise than the previously available ones. Second, and most importantly, it 
makes the required identification assumptions substantially more plausible. Given that the identifying 
variation comes from within-zipcodes, the determinants of these MW changes are unlikely to be related 
to the particular zipcode, and therefore, are less likely to be correlated to the unobservable 
determinants of rent dynamics there.

% A fourth important difference with past work, is that to the best of our knowledge, this is the 
% first study that estimates the effect of MW changes on amenities. As pointed out recently by 
% \textcite{diamond2016determinants, almagro2019location}, taking into account non-pecuniary  
% dimensions of the utility function can change substantively the welfare computations and the 
% incidence of policy relevant changes. We exploit the use of high frequency GPS data to build amenity 
% measures at the zipcode-month level and estimate the effect of MW changes on them. We hope that by 
% incorporating amenities into the picture we can give a more comprehensive assessment of the effects 
% of MW policies.

Beyond the contribution to the very recent literature on the effects of MW changes on rents, we 
contribute to several strands of the economic literature. First, we contribute to the literature 
studying the effects of minimum wages on the welfare of low skill households. However, instead of 
focusing on wages and employment \parencite[][among others]{dinardo1995labor, autor2016contribution, 
card2000minimum, neumark2006minimum, jardim2017minimum}, we contribute to this strand of literature 
by taking into account the effects on the housing market.

Second, our work relates to the literature that studies the location decision of agents either based 
on income \parencite{roback1982wages, kennan2011effect, desmet2013urban, perez2018city} or based on 
spatial rent and amenity differentials \parencite{diamond2016determinants, almagro2019location, 
couture2019income, bayer2004equilibrium}. We hope to contribute by adapting this framework to the 
case of the MW changes, so that we can rationalize through residential location sorting part of 
the observed reduce form effect on rents and amenities.

The rest of the paper is organized as follows. Initially, section \ref{sec:model} motivates the paper
with a simple model of the rental market. In section \ref{sec:data}, we present our data sources and 
show the characteristics of our estimating panel. In section \ref{sec:empirical_strategy}, we explain 
our empirical strategy and we discuss our identification assumptions. In section \ref{sec:results}, we 
present our main results. Section \ref{sec:discussion} discusses relevant policy implications, and 
section \ref{sec:conclusion} concludes.
